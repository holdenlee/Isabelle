\documentclass[11pt,a4paper]{article}
\usepackage{isabelle,isabellesym}

% further packages required for unusual symbols (see also
% isabellesym.sty), use only when needed

\usepackage{amssymb}
\usepackage{amsmath}
  %for \<leadsto>, \<box>, \<diamond>, \<sqsupset>, \<mho>, \<Join>,
  %\<lhd>, \<lesssim>, \<greatersim>, \<lessapprox>, \<greaterapprox>,
  %\<triangleq>, \<yen>, \<lozenge>

%\usepackage{eurosym}
  %for \<euro>

%\usepackage[only,bigsqcap]{stmaryrd}
  %for \<Sqinter>

%\usepackage{eufrak}
  %for \<AA> ... \<ZZ>, \<aa> ... \<zz> (also included in amssymb)

%\usepackage{textcomp}
  %for \<onequarter>, \<onehalf>, \<threequarters>, \<degree>, \<cent>,
  %\<currency>

% this should be the last package used
\usepackage{pdfsetup}

% urls in roman style, theory text in math-similar italics
\urlstyle{rm}
\isabellestyle{it}

% for uniform font size
%\renewcommand{\isastyle}{\isastyleminor}


\begin{document}

\title{VectorSpace}
\author{Holden Lee\thanks{This work was funded by the Post-Masters Consultancy and the Computer Laboratory at the University of Cambridge.}}
\maketitle

\abstract{
I present a formalisation of basic linear algebra based completely on locales, building off HOL-Algebra. It includes the following:
\begin{enumerate}
\item
basic definitions: linear combinations, span, linear independence
\item
linear transformations
\item
interpretation of function spaces as vector spaces
\item
direct sum of vector spaces, sum of subspaces
\item
the replacement theorem
\item
existence of bases in finite-dimensional vector spaces, definition of dimension
\item
rank-nullity theorem.
\end{enumerate}
Note that some concepts are actually defined and proved for modules as they also apply there.

In the process, I also prove some basic facts about rings, modules, and fields, as well as finite sums in monoids/modules.

Note that infinite-dimensional vector spaces are supported, but dimension is only supported for finite-dimensional vector spaces.

The proofs are standard; the proofs of the replacement theorem and rank-nullity theorem roughly follow the presentation in~\cite{FIS03}. The rank-nullity theorem generalises the existing development in~\cite{AD13} (originally using type classes, now using a mix of type classes and locales).
}

\tableofcontents

% sane default for proof documents
\parindent 0pt\parskip 0.5ex

% generated text of all theories
%
\begin{isabellebody}%
\def\isabellecontext{RingModuleFacts}%
%
\isamarkupheader{Basic facts about rings and modules%
}
\isamarkuptrue%
%
\isadelimtheory
%
\endisadelimtheory
%
\isatagtheory
\isacommand{theory}\isamarkupfalse%
\ RingModuleFacts\isanewline
\isakeyword{imports}\ Main\isanewline
\ \ {\isachardoublequoteopen}{\isachartilde}{\isachartilde}{\isacharslash}src{\isacharslash}HOL{\isacharslash}Algebra{\isacharslash}Module{\isachardoublequoteclose}\isanewline
\ \ {\isachardoublequoteopen}{\isachartilde}{\isachartilde}{\isacharslash}src{\isacharslash}HOL{\isacharslash}Algebra{\isacharslash}Coset{\isachardoublequoteclose}\isanewline
\ \ \isanewline
\isakeyword{begin}%
\endisatagtheory
{\isafoldtheory}%
%
\isadelimtheory
%
\endisadelimtheory
%
\isamarkupsubsection{Basic facts%
}
\isamarkuptrue%
%
\begin{isamarkuptext}%
In a field, every nonzero element has an inverse.%
\end{isamarkuptext}%
\isamarkuptrue%
\isacommand{lemma}\isamarkupfalse%
\ {\isacharparenleft}\isakeyword{in}\ field{\isacharparenright}\ inverse{\isacharunderscore}exists\ {\isacharbrackleft}simp{\isacharcomma}\ intro{\isacharbrackright}{\isacharcolon}\ \isanewline
\ \ \isakeyword{assumes}\ h{\isadigit{1}}{\isacharcolon}\ {\isachardoublequoteopen}a{\isasymin}carrier\ R{\isachardoublequoteclose}\ \ \isakeyword{and}\ h{\isadigit{2}}{\isacharcolon}\ {\isachardoublequoteopen}a{\isasymnoteq}{\isasymzero}\isactrlbsub R\isactrlesub {\isachardoublequoteclose}\isanewline
\ \ \isakeyword{shows}\ {\isachardoublequoteopen}inv\isactrlbsub R\isactrlesub \ a{\isasymin}\ carrier\ R{\isachardoublequoteclose}\isanewline
%
\isadelimproof
%
\endisadelimproof
%
\isatagproof
\isacommand{proof}\isamarkupfalse%
\ {\isacharminus}\ \isanewline
\ \ \isacommand{have}\isamarkupfalse%
\ {\isadigit{1}}{\isacharcolon}\ {\isachardoublequoteopen}Units\ R\ {\isacharequal}\ carrier\ R\ {\isacharminus}\ {\isacharbraceleft}{\isasymzero}\isactrlbsub R\isactrlesub {\isacharbraceright}\ {\isachardoublequoteclose}\ \isacommand{by}\isamarkupfalse%
\ {\isacharparenleft}rule\ field{\isacharunderscore}Units{\isacharparenright}\isanewline
\ \ \isacommand{from}\isamarkupfalse%
\ h{\isadigit{1}}\ h{\isadigit{2}}\ {\isadigit{1}}\ \isacommand{show}\isamarkupfalse%
\ {\isacharquery}thesis\ \isacommand{by}\isamarkupfalse%
\ auto\isanewline
\isacommand{qed}\isamarkupfalse%
%
\endisatagproof
{\isafoldproof}%
%
\isadelimproof
%
\endisadelimproof
%
\begin{isamarkuptext}%
Multiplication by 0 in $R$ gives 0. (Note that this fact encompasses smult-l-null 
as this is for module while that is for algebra, so smult-l-null is redundant.)%
\end{isamarkuptext}%
\isamarkuptrue%
\isacommand{lemma}\isamarkupfalse%
\ {\isacharparenleft}\isakeyword{in}\ module{\isacharparenright}\ lmult{\isacharunderscore}{\isadigit{0}}\ {\isacharbrackleft}simp{\isacharbrackright}{\isacharcolon}\isanewline
\ \ \isakeyword{assumes}\ {\isadigit{1}}{\isacharcolon}\ {\isachardoublequoteopen}m{\isasymin}carrier\ M{\isachardoublequoteclose}\isanewline
\ \ \isakeyword{shows}\ {\isachardoublequoteopen}{\isasymzero}\isactrlbsub R\isactrlesub {\isasymodot}\isactrlbsub M\isactrlesub \ m{\isacharequal}{\isasymzero}\isactrlbsub M\isactrlesub {\isachardoublequoteclose}\isanewline
%
\isadelimproof
%
\endisadelimproof
%
\isatagproof
\isacommand{proof}\isamarkupfalse%
\ {\isacharminus}\ \isanewline
\ \ \isacommand{from}\isamarkupfalse%
\ {\isadigit{1}}\ \isacommand{have}\isamarkupfalse%
\ {\isadigit{0}}{\isacharcolon}\ {\isachardoublequoteopen}{\isasymzero}\isactrlbsub R\isactrlesub {\isasymodot}\isactrlbsub M\isactrlesub \ m{\isasymin}carrier\ M{\isachardoublequoteclose}\ \isacommand{by}\isamarkupfalse%
\ simp\isanewline
\ \ \isacommand{from}\isamarkupfalse%
\ {\isadigit{1}}\ \isacommand{have}\isamarkupfalse%
\ {\isadigit{2}}{\isacharcolon}\ {\isachardoublequoteopen}{\isasymzero}\isactrlbsub R\isactrlesub {\isasymodot}\isactrlbsub M\isactrlesub \ m\ {\isacharequal}\ {\isacharparenleft}{\isasymzero}\isactrlbsub R\isactrlesub \ {\isasymoplus}\isactrlbsub R\isactrlesub \ {\isasymzero}\isactrlbsub R\isactrlesub {\isacharparenright}\ {\isasymodot}\isactrlbsub M\isactrlesub \ m{\isachardoublequoteclose}\ \isacommand{by}\isamarkupfalse%
\ simp\isanewline
\ \ \isacommand{from}\isamarkupfalse%
\ {\isadigit{1}}\ \isacommand{have}\isamarkupfalse%
\ {\isadigit{3}}{\isacharcolon}\ {\isachardoublequoteopen}{\isacharparenleft}{\isasymzero}\isactrlbsub R\isactrlesub \ {\isasymoplus}\isactrlbsub R\isactrlesub \ {\isasymzero}\isactrlbsub R\isactrlesub {\isacharparenright}\ {\isasymodot}\isactrlbsub M\isactrlesub \ m{\isacharequal}{\isacharparenleft}{\isasymzero}\isactrlbsub R\isactrlesub {\isasymodot}\isactrlbsub M\isactrlesub \ m{\isacharparenright}\ {\isasymoplus}\isactrlbsub M\isactrlesub \ {\isacharparenleft}{\isasymzero}\isactrlbsub R\isactrlesub {\isasymodot}\isactrlbsub M\isactrlesub \ m{\isacharparenright}{\isachardoublequoteclose}\ \ \isacommand{using}\isamarkupfalse%
\ {\isacharbrackleft}{\isacharbrackleft}simp{\isacharunderscore}trace{\isacharcomma}\ simp{\isacharunderscore}trace{\isacharunderscore}depth{\isacharunderscore}limit{\isacharequal}{\isadigit{3}}{\isacharbrackright}{\isacharbrackright}\isanewline
\ \ \ \ \isacommand{by}\isamarkupfalse%
\ {\isacharparenleft}simp\ add{\isacharcolon}\ smult{\isacharunderscore}l{\isacharunderscore}distr\ del{\isacharcolon}\ R{\isachardot}add{\isachardot}r{\isacharunderscore}one\ R{\isachardot}add{\isachardot}l{\isacharunderscore}one{\isacharparenright}\isanewline
\ \ \isacommand{from}\isamarkupfalse%
\ {\isadigit{2}}\ {\isadigit{3}}\ \isacommand{have}\isamarkupfalse%
\ {\isadigit{4}}{\isacharcolon}\ {\isachardoublequoteopen}{\isasymzero}\isactrlbsub R\isactrlesub {\isasymodot}\isactrlbsub M\isactrlesub \ m\ {\isacharequal}{\isacharparenleft}{\isasymzero}\isactrlbsub R\isactrlesub {\isasymodot}\isactrlbsub M\isactrlesub \ m{\isacharparenright}\ {\isasymoplus}\isactrlbsub M\isactrlesub \ {\isacharparenleft}{\isasymzero}\isactrlbsub R\isactrlesub {\isasymodot}\isactrlbsub M\isactrlesub \ m{\isacharparenright}{\isachardoublequoteclose}\ \isacommand{by}\isamarkupfalse%
\ auto\isanewline
\ \ \isacommand{from}\isamarkupfalse%
\ {\isadigit{0}}\ {\isadigit{4}}\ \isacommand{show}\isamarkupfalse%
\ {\isacharquery}thesis\ \isacommand{by}\isamarkupfalse%
\ {\isacharparenleft}metis\ {\isadigit{1}}\ M{\isachardot}add{\isachardot}l{\isacharunderscore}cancel\ M{\isachardot}r{\isacharunderscore}zero\ M{\isachardot}zero{\isacharunderscore}closed{\isacharparenright}\isanewline
\isacommand{qed}\isamarkupfalse%
%
\endisatagproof
{\isafoldproof}%
%
\isadelimproof
%
\endisadelimproof
%
\begin{isamarkuptext}%
Multiplication by 0 in $M$ gives 0.%
\end{isamarkuptext}%
\isamarkuptrue%
\isacommand{lemma}\isamarkupfalse%
\ {\isacharparenleft}\isakeyword{in}\ module{\isacharparenright}\ rmult{\isacharunderscore}{\isadigit{0}}\ {\isacharbrackleft}simp{\isacharbrackright}{\isacharcolon}\isanewline
\ \ \isakeyword{assumes}\ {\isadigit{0}}{\isacharcolon}\ {\isachardoublequoteopen}r{\isasymin}carrier\ R{\isachardoublequoteclose}\isanewline
\ \ \isakeyword{shows}\ {\isachardoublequoteopen}r{\isasymodot}\isactrlbsub M\isactrlesub \ {\isasymzero}\isactrlbsub M\isactrlesub {\isacharequal}{\isasymzero}\isactrlbsub M\isactrlesub {\isachardoublequoteclose}\isanewline
%
\isadelimproof
%
\endisadelimproof
%
\isatagproof
\isacommand{by}\isamarkupfalse%
\ {\isacharparenleft}metis\ M{\isachardot}zero{\isacharunderscore}closed\ R{\isachardot}zero{\isacharunderscore}closed\ assms\ lmult{\isacharunderscore}{\isadigit{0}}\ r{\isacharunderscore}null\ smult{\isacharunderscore}assoc{\isadigit{1}}{\isacharparenright}%
\endisatagproof
{\isafoldproof}%
%
\isadelimproof
%
\endisadelimproof
%
\begin{isamarkuptext}%
Multiplication by $-1$ is the same as negation. May be useful as a simp rule.%
\end{isamarkuptext}%
\isamarkuptrue%
\isacommand{lemma}\isamarkupfalse%
\ {\isacharparenleft}\isakeyword{in}\ module{\isacharparenright}\ smult{\isacharunderscore}minus{\isacharunderscore}{\isadigit{1}}{\isacharcolon}\isanewline
\ \ \isakeyword{fixes}\ v\isanewline
\ \ \isakeyword{assumes}\ {\isadigit{0}}{\isacharcolon}{\isachardoublequoteopen}v{\isasymin}carrier\ M{\isachardoublequoteclose}\isanewline
\ \ \isakeyword{shows}\ {\isachardoublequoteopen}{\isacharparenleft}{\isasymominus}\isactrlbsub R\isactrlesub \ {\isasymone}\isactrlbsub R\isactrlesub {\isacharparenright}\ {\isasymodot}\isactrlbsub M\isactrlesub \ v{\isacharequal}\ {\isacharparenleft}{\isasymominus}\isactrlbsub M\isactrlesub \ \ v{\isacharparenright}{\isachardoublequoteclose}\isanewline
%
\isadelimproof
\isanewline
%
\endisadelimproof
%
\isatagproof
\isacommand{proof}\isamarkupfalse%
\ {\isacharminus}\isanewline
\ \ \isacommand{from}\isamarkupfalse%
\ {\isadigit{0}}\ \isacommand{have}\isamarkupfalse%
\ a{\isadigit{0}}{\isacharcolon}\ {\isachardoublequoteopen}{\isasymone}\isactrlbsub R\isactrlesub \ {\isasymodot}\isactrlbsub M\isactrlesub \ v\ {\isacharequal}\ v{\isachardoublequoteclose}\ \isacommand{by}\isamarkupfalse%
\ simp\isanewline
\ \ \isacommand{from}\isamarkupfalse%
\ {\isadigit{0}}\ \isacommand{have}\isamarkupfalse%
\ {\isadigit{1}}{\isacharcolon}\ {\isachardoublequoteopen}{\isacharparenleft}{\isacharparenleft}{\isasymominus}\isactrlbsub R\isactrlesub \ {\isasymone}\isactrlbsub R\isactrlesub {\isacharparenright}{\isasymoplus}\isactrlbsub R\isactrlesub \ {\isasymone}\isactrlbsub R\isactrlesub {\isacharparenright}\ {\isasymodot}\isactrlbsub M\isactrlesub \ v{\isacharequal}{\isasymzero}\isactrlbsub M\isactrlesub {\isachardoublequoteclose}\ \isanewline
\ \ \ \ \isacommand{by}\isamarkupfalse%
\ {\isacharparenleft}simp\ add{\isacharcolon}R{\isachardot}l{\isacharunderscore}neg{\isacharparenright}\isanewline
\ \ \isacommand{from}\isamarkupfalse%
\ {\isadigit{0}}\ \isacommand{have}\isamarkupfalse%
\ {\isadigit{2}}{\isacharcolon}\ {\isachardoublequoteopen}{\isacharparenleft}{\isacharparenleft}{\isasymominus}\isactrlbsub R\isactrlesub \ {\isasymone}\isactrlbsub R\isactrlesub {\isacharparenright}{\isasymoplus}\isactrlbsub R\isactrlesub \ {\isasymone}\isactrlbsub R\isactrlesub {\isacharparenright}\ {\isasymodot}\isactrlbsub M\isactrlesub \ v{\isacharequal}{\isacharparenleft}{\isasymominus}\isactrlbsub R\isactrlesub \ {\isasymone}\isactrlbsub R\isactrlesub {\isacharparenright}\ {\isasymodot}\isactrlbsub M\isactrlesub \ v\ {\isasymoplus}\isactrlbsub M\isactrlesub \ {\isasymone}\isactrlbsub R\isactrlesub {\isasymodot}\isactrlbsub M\isactrlesub \ v{\isachardoublequoteclose}\isanewline
\ \ \ \ \isacommand{by}\isamarkupfalse%
\ {\isacharparenleft}simp\ add{\isacharcolon}\ smult{\isacharunderscore}l{\isacharunderscore}distr{\isacharparenright}\isanewline
\ \ \isacommand{from}\isamarkupfalse%
\ {\isadigit{1}}\ {\isadigit{2}}\ \isacommand{show}\isamarkupfalse%
\ {\isacharquery}thesis\ \isacommand{by}\isamarkupfalse%
\ {\isacharparenleft}metis\ M{\isachardot}minus{\isacharunderscore}equality\ R{\isachardot}add{\isachardot}inv{\isacharunderscore}closed\ \isanewline
\ \ \ \ a{\isadigit{0}}\ assms\ one{\isacharunderscore}closed\ smult{\isacharunderscore}closed{\isacharparenright}\ \isanewline
\isacommand{qed}\isamarkupfalse%
%
\endisatagproof
{\isafoldproof}%
%
\isadelimproof
%
\endisadelimproof
%
\begin{isamarkuptext}%
The version with equality reversed.%
\end{isamarkuptext}%
\isamarkuptrue%
\isacommand{lemmas}\isamarkupfalse%
\ {\isacharparenleft}\isakeyword{in}\ module{\isacharparenright}\ \ smult{\isacharunderscore}minus{\isacharunderscore}{\isadigit{1}}{\isacharunderscore}back\ {\isacharequal}\ smult{\isacharunderscore}minus{\isacharunderscore}{\isadigit{1}}{\isacharbrackleft}THEN\ sym{\isacharbrackright}%
\begin{isamarkuptext}%
-1 is not 0%
\end{isamarkuptext}%
\isamarkuptrue%
\isacommand{lemma}\isamarkupfalse%
\ {\isacharparenleft}\isakeyword{in}\ field{\isacharparenright}\ neg{\isacharunderscore}{\isadigit{1}}{\isacharunderscore}not{\isacharunderscore}{\isadigit{0}}\ {\isacharbrackleft}simp{\isacharbrackright}{\isacharcolon}\ {\isachardoublequoteopen}{\isasymominus}\isactrlbsub R\isactrlesub \ {\isasymone}\isactrlbsub R\isactrlesub \ {\isasymnoteq}\ {\isasymzero}\isactrlbsub R\isactrlesub {\isachardoublequoteclose}\isanewline
%
\isadelimproof
%
\endisadelimproof
%
\isatagproof
\isacommand{by}\isamarkupfalse%
\ {\isacharparenleft}metis\ local{\isachardot}minus{\isacharunderscore}minus\ local{\isachardot}minus{\isacharunderscore}zero\ one{\isacharunderscore}closed\ zero{\isacharunderscore}not{\isacharunderscore}one{\isacharparenright}%
\endisatagproof
{\isafoldproof}%
%
\isadelimproof
%
\endisadelimproof
%
\begin{isamarkuptext}%
Note smult-assoc1 is the wrong way around for simplification.
This is the reverse of smult-assoc1.%
\end{isamarkuptext}%
\isamarkuptrue%
\isacommand{lemma}\isamarkupfalse%
\ {\isacharparenleft}\isakeyword{in}\ module{\isacharparenright}\ smult{\isacharunderscore}assoc{\isacharunderscore}simp{\isacharcolon}\isanewline
{\isachardoublequoteopen}{\isacharbrackleft}{\isacharbar}\ a\ {\isasymin}\ carrier\ R{\isacharsemicolon}\ b\ {\isasymin}\ carrier\ R{\isacharsemicolon}\ x\ {\isasymin}\ carrier\ M\ {\isacharbar}{\isacharbrackright}\ {\isacharequal}{\isacharequal}{\isachargreater}\isanewline
\ \ \ \ \ \ a\ {\isasymodot}\isactrlbsub M\isactrlesub \ {\isacharparenleft}b\ {\isasymodot}\isactrlbsub M\isactrlesub \ x{\isacharparenright}\ {\isacharequal}\ {\isacharparenleft}a\ {\isasymotimes}\ b{\isacharparenright}\ {\isasymodot}\isactrlbsub M\isactrlesub \ x\ {\isachardoublequoteclose}\isanewline
%
\isadelimproof
%
\endisadelimproof
%
\isatagproof
\isacommand{by}\isamarkupfalse%
\ {\isacharparenleft}auto\ simp\ add{\isacharcolon}\ smult{\isacharunderscore}assoc{\isadigit{1}}{\isacharparenright}%
\endisatagproof
{\isafoldproof}%
%
\isadelimproof
\isanewline
%
\endisadelimproof
\ \ \ \ \ \ \ \ \ \ \ \ \ \isanewline
\isanewline
\isacommand{lemma}\isamarkupfalse%
\ {\isacharparenleft}\isakeyword{in}\ group{\isacharparenright}\ show{\isacharunderscore}r{\isacharunderscore}one\ {\isacharbrackleft}simp{\isacharbrackright}{\isacharcolon}\isanewline
\ \ {\isachardoublequoteopen}{\isasymlbrakk}a{\isasymin}\ carrier\ G{\isacharsemicolon}\ b{\isasymin}\ carrier\ G{\isasymrbrakk}{\isasymLongrightarrow}\ {\isacharparenleft}a\ {\isasymotimes}\isactrlbsub G\isactrlesub \ b{\isacharequal}\ a{\isacharparenright}\ {\isacharequal}\ {\isacharparenleft}\ b{\isacharequal}\ {\isasymone}\isactrlbsub G\isactrlesub {\isacharparenright}{\isachardoublequoteclose}\isanewline
%
\isadelimproof
%
\endisadelimproof
%
\isatagproof
\isacommand{by}\isamarkupfalse%
\ {\isacharparenleft}metis\ l{\isacharunderscore}inv\ r{\isacharunderscore}one\ transpose{\isacharunderscore}inv{\isacharparenright}%
\endisatagproof
{\isafoldproof}%
%
\isadelimproof
\isanewline
%
\endisadelimproof
\isanewline
\isanewline
\isacommand{lemma}\isamarkupfalse%
\ {\isacharparenleft}\isakeyword{in}\ group{\isacharparenright}\ show{\isacharunderscore}l{\isacharunderscore}one\ {\isacharbrackleft}simp{\isacharbrackright}{\isacharcolon}\isanewline
\ \ {\isachardoublequoteopen}{\isasymlbrakk}a{\isasymin}\ carrier\ G{\isacharsemicolon}\ b{\isasymin}\ carrier\ G{\isasymrbrakk}{\isasymLongrightarrow}\ {\isacharparenleft}a\ {\isasymotimes}\isactrlbsub G\isactrlesub \ b{\isacharequal}\ b{\isacharparenright}\ {\isacharequal}\ {\isacharparenleft}\ a{\isacharequal}\ {\isasymone}\isactrlbsub G\isactrlesub {\isacharparenright}{\isachardoublequoteclose}\isanewline
%
\isadelimproof
%
\endisadelimproof
%
\isatagproof
\isacommand{by}\isamarkupfalse%
\ {\isacharparenleft}metis\ l{\isacharunderscore}one\ one{\isacharunderscore}closed\ r{\isacharunderscore}cancel{\isacharparenright}%
\endisatagproof
{\isafoldproof}%
%
\isadelimproof
\isanewline
%
\endisadelimproof
\isanewline
\isanewline
\isacommand{lemmas}\isamarkupfalse%
\ {\isacharparenleft}\isakeyword{in}\ abelian{\isacharunderscore}group{\isacharparenright}\ show{\isacharunderscore}r{\isacharunderscore}zero{\isacharequal}add{\isachardot}show{\isacharunderscore}r{\isacharunderscore}one\isanewline
\isacommand{lemmas}\isamarkupfalse%
\ {\isacharparenleft}\isakeyword{in}\ abelian{\isacharunderscore}group{\isacharparenright}\ show{\isacharunderscore}l{\isacharunderscore}zero{\isacharequal}add{\isachardot}show{\isacharunderscore}l{\isacharunderscore}one%
\begin{isamarkuptext}%
A nontrivial ring has $0\neq 1$.%
\end{isamarkuptext}%
\isamarkuptrue%
\isacommand{lemma}\isamarkupfalse%
\ {\isacharparenleft}\isakeyword{in}\ ring{\isacharparenright}\ nontrivial{\isacharunderscore}ring\ {\isacharbrackleft}simp{\isacharbrackright}{\isacharcolon}\isanewline
\ \ \isakeyword{assumes}\ {\isachardoublequoteopen}carrier\ R{\isasymnoteq}{\isacharbraceleft}{\isasymzero}\isactrlbsub R\isactrlesub {\isacharbraceright}{\isachardoublequoteclose}\isanewline
\ \ \isakeyword{shows}\ {\isachardoublequoteopen}{\isasymzero}\isactrlbsub R\isactrlesub {\isasymnoteq}{\isasymone}\isactrlbsub R\isactrlesub {\isachardoublequoteclose}\isanewline
%
\isadelimproof
%
\endisadelimproof
%
\isatagproof
\isacommand{proof}\isamarkupfalse%
\ {\isacharparenleft}rule\ ccontr{\isacharparenright}\isanewline
\ \ \isacommand{assume}\isamarkupfalse%
\ {\isadigit{1}}{\isacharcolon}\ {\isachardoublequoteopen}{\isasymnot}{\isacharparenleft}{\isasymzero}\isactrlbsub R\isactrlesub {\isasymnoteq}{\isasymone}\isactrlbsub R\isactrlesub {\isacharparenright}{\isachardoublequoteclose}\isanewline
\ \ \isacommand{{\isacharbraceleft}}\isamarkupfalse%
\isanewline
\ \ \ \ \isacommand{fix}\isamarkupfalse%
\ r\isanewline
\ \ \ \ \isacommand{assume}\isamarkupfalse%
\ {\isadigit{2}}{\isacharcolon}\ {\isachardoublequoteopen}r{\isasymin}carrier\ R{\isachardoublequoteclose}\isanewline
\ \ \ \ \isacommand{from}\isamarkupfalse%
\ {\isadigit{1}}\ {\isadigit{2}}\ \isacommand{have}\isamarkupfalse%
\ {\isadigit{3}}{\isacharcolon}\ {\isachardoublequoteopen}{\isasymone}\isactrlbsub R\isactrlesub {\isasymotimes}\isactrlbsub R\isactrlesub \ r\ {\isacharequal}\ {\isasymzero}\isactrlbsub R\isactrlesub {\isasymotimes}\isactrlbsub R\isactrlesub \ r{\isachardoublequoteclose}\ \isacommand{by}\isamarkupfalse%
\ auto\isanewline
\ \ \ \ \isacommand{from}\isamarkupfalse%
\ {\isadigit{2}}\ {\isadigit{3}}\ \isacommand{have}\isamarkupfalse%
\ {\isachardoublequoteopen}r\ {\isacharequal}\ {\isasymzero}\isactrlbsub R\isactrlesub {\isachardoublequoteclose}\ \isacommand{by}\isamarkupfalse%
\ auto\isanewline
\ \ \isacommand{{\isacharbraceright}}\isamarkupfalse%
\isanewline
\ \ \isacommand{from}\isamarkupfalse%
\ this\ assms\ \isacommand{show}\isamarkupfalse%
\ False\ \isacommand{by}\isamarkupfalse%
\ auto\isanewline
\isacommand{qed}\isamarkupfalse%
%
\endisatagproof
{\isafoldproof}%
%
\isadelimproof
%
\endisadelimproof
%
\begin{isamarkuptext}%
Use as simp rule. To show $a-b=0$, it suffices to show $a=b$.%
\end{isamarkuptext}%
\isamarkuptrue%
\isacommand{lemma}\isamarkupfalse%
\ {\isacharparenleft}\isakeyword{in}\ abelian{\isacharunderscore}group{\isacharparenright}\ minus{\isacharunderscore}other{\isacharunderscore}side\ {\isacharbrackleft}simp{\isacharbrackright}{\isacharcolon}\isanewline
\ \ {\isachardoublequoteopen}{\isasymlbrakk}a{\isasymin}carrier\ G{\isacharsemicolon}\ b{\isasymin}carrier\ G{\isasymrbrakk}\ {\isasymLongrightarrow}\ {\isacharparenleft}a{\isasymominus}\isactrlbsub G\isactrlesub b\ {\isacharequal}\ {\isasymzero}\isactrlbsub G\isactrlesub {\isacharparenright}\ {\isacharequal}\ {\isacharparenleft}a{\isacharequal}b{\isacharparenright}{\isachardoublequoteclose}\isanewline
%
\isadelimproof
%
\endisadelimproof
%
\isatagproof
\isacommand{by}\isamarkupfalse%
\ {\isacharparenleft}metis\ add{\isachardot}inv{\isacharunderscore}closed\ add{\isachardot}r{\isacharunderscore}cancel\ minus{\isacharunderscore}eq\ r{\isacharunderscore}neg{\isacharparenright}%
\endisatagproof
{\isafoldproof}%
%
\isadelimproof
%
\endisadelimproof
%
\isamarkupsubsection{Units group%
}
\isamarkuptrue%
%
\begin{isamarkuptext}%
Define the units group $R^{\times}$ and show it is actually a group.%
\end{isamarkuptext}%
\isamarkuptrue%
\isacommand{definition}\isamarkupfalse%
\ units{\isacharunderscore}group{\isacharcolon}{\isacharcolon}{\isachardoublequoteopen}{\isacharparenleft}{\isacharprime}a{\isacharcomma}{\isacharprime}b{\isacharparenright}\ ring{\isacharunderscore}scheme\ {\isasymRightarrow}\ {\isacharprime}a\ monoid{\isachardoublequoteclose}\isanewline
\ \ \isakeyword{where}\ {\isachardoublequoteopen}units{\isacharunderscore}group\ R\ {\isacharequal}\ {\isasymlparr}carrier\ {\isacharequal}\ Units\ R{\isacharcomma}\ mult\ {\isacharequal}\ {\isacharparenleft}{\isasymlambda}x\ y{\isachardot}\ x{\isasymotimes}\isactrlbsub R\isactrlesub \ y{\isacharparenright}{\isacharcomma}\ one\ {\isacharequal}\ {\isasymone}\isactrlbsub R\isactrlesub {\isasymrparr}{\isachardoublequoteclose}%
\begin{isamarkuptext}%
The units form a group.%
\end{isamarkuptext}%
\isamarkuptrue%
\isacommand{lemma}\isamarkupfalse%
\ {\isacharparenleft}\isakeyword{in}\ ring{\isacharparenright}\ units{\isacharunderscore}form{\isacharunderscore}group{\isacharcolon}\ {\isachardoublequoteopen}group\ {\isacharparenleft}units{\isacharunderscore}group\ R{\isacharparenright}{\isachardoublequoteclose}\isanewline
%
\isadelimproof
\ \ %
\endisadelimproof
%
\isatagproof
\isacommand{apply}\isamarkupfalse%
\ {\isacharparenleft}intro\ groupI{\isacharparenright}\isanewline
\ \ \isacommand{apply}\isamarkupfalse%
\ {\isacharparenleft}unfold\ units{\isacharunderscore}group{\isacharunderscore}def{\isacharcomma}\ auto{\isacharparenright}\isanewline
\ \ \isacommand{apply}\isamarkupfalse%
\ {\isacharparenleft}intro\ m{\isacharunderscore}assoc{\isacharparenright}\ \isanewline
\ \ \isacommand{apply}\isamarkupfalse%
\ auto\isanewline
\ \ \isacommand{apply}\isamarkupfalse%
\ {\isacharparenleft}unfold\ Units{\isacharunderscore}def{\isacharparenright}\ \isanewline
\ \ \isacommand{apply}\isamarkupfalse%
\ auto\isanewline
\ \ \isacommand{done}\isamarkupfalse%
%
\endisatagproof
{\isafoldproof}%
%
\isadelimproof
%
\endisadelimproof
%
\begin{isamarkuptext}%
The units of a \isa{cring} form a commutative group.%
\end{isamarkuptext}%
\isamarkuptrue%
\isacommand{lemma}\isamarkupfalse%
\ {\isacharparenleft}\isakeyword{in}\ cring{\isacharparenright}\ units{\isacharunderscore}form{\isacharunderscore}cgroup{\isacharcolon}\ {\isachardoublequoteopen}comm{\isacharunderscore}group\ {\isacharparenleft}units{\isacharunderscore}group\ R{\isacharparenright}{\isachardoublequoteclose}\isanewline
%
\isadelimproof
\ \ %
\endisadelimproof
%
\isatagproof
\isacommand{apply}\isamarkupfalse%
\ {\isacharparenleft}intro\ comm{\isacharunderscore}groupI{\isacharparenright}\isanewline
\ \ \isacommand{apply}\isamarkupfalse%
\ {\isacharparenleft}unfold\ units{\isacharunderscore}group{\isacharunderscore}def{\isacharparenright}\ \isacommand{apply}\isamarkupfalse%
\ auto\isanewline
\ \ \isacommand{apply}\isamarkupfalse%
\ {\isacharparenleft}intro\ m{\isacharunderscore}assoc{\isacharparenright}\ \isacommand{apply}\isamarkupfalse%
\ auto\isanewline
\ \ \isacommand{apply}\isamarkupfalse%
\ {\isacharparenleft}unfold\ Units{\isacharunderscore}def{\isacharparenright}\ \isacommand{apply}\isamarkupfalse%
\ auto\isanewline
\ \ \isacommand{apply}\isamarkupfalse%
\ {\isacharparenleft}rule\ m{\isacharunderscore}comm{\isacharparenright}\ \isacommand{apply}\isamarkupfalse%
\ auto\isanewline
\ \ \isacommand{done}\isamarkupfalse%
%
\endisatagproof
{\isafoldproof}%
%
\isadelimproof
\isanewline
%
\endisadelimproof
\isanewline
\isanewline
%
\isadelimtheory
\isanewline
%
\endisadelimtheory
%
\isatagtheory
\isacommand{end}\isamarkupfalse%
%
\endisatagtheory
{\isafoldtheory}%
%
\isadelimtheory
%
\endisadelimtheory
\end{isabellebody}%
%%% Local Variables:
%%% mode: latex
%%% TeX-master: "root"
%%% End:


%
\begin{isabellebody}%
\def\isabellecontext{FunctionLemmas}%
%
\isamarkupheader{Basic lemmas about functions%
}
\isamarkuptrue%
%
\isadelimtheory
%
\endisadelimtheory
%
\isatagtheory
\isacommand{theory}\isamarkupfalse%
\ FunctionLemmas\isanewline
\isanewline
\isakeyword{imports}\ Main\isanewline
\ \ {\isachardoublequoteopen}{\isachartilde}{\isachartilde}{\isacharslash}src{\isacharslash}HOL{\isacharslash}Library{\isacharslash}FuncSet{\isachardoublequoteclose}\isanewline
\isakeyword{begin}%
\endisatagtheory
{\isafoldtheory}%
%
\isadelimtheory
%
\endisadelimtheory
%
\begin{isamarkuptext}%
These are used in simplification. Note that the difference from Pi-mem is that the statement
about the function comes first, so Isabelle can more easily figure out what $S$ is.%
\end{isamarkuptext}%
\isamarkuptrue%
\isacommand{lemma}\isamarkupfalse%
\ PiE{\isacharunderscore}mem{\isadigit{2}}{\isacharcolon}\ {\isachardoublequoteopen}f\ {\isasymin}\ S{\isasymrightarrow}\isactrlsub E\ T\ {\isasymLongrightarrow}\ x\ {\isasymin}\ S\ {\isasymLongrightarrow}\ f\ x\ {\isasymin}\ T{\isachardoublequoteclose}\isanewline
%
\isadelimproof
\ \ %
\endisadelimproof
%
\isatagproof
\isacommand{unfolding}\isamarkupfalse%
\ PiE{\isacharunderscore}def\ \isacommand{by}\isamarkupfalse%
\ auto%
\endisatagproof
{\isafoldproof}%
%
\isadelimproof
\isanewline
%
\endisadelimproof
\isacommand{lemma}\isamarkupfalse%
\ Pi{\isacharunderscore}mem{\isadigit{2}}{\isacharcolon}\ {\isachardoublequoteopen}f\ {\isasymin}\ S{\isasymrightarrow}\ T\ {\isasymLongrightarrow}\ x\ {\isasymin}\ S\ {\isasymLongrightarrow}\ f\ x\ {\isasymin}\ T{\isachardoublequoteclose}\isanewline
%
\isadelimproof
\ \ %
\endisadelimproof
%
\isatagproof
\isacommand{unfolding}\isamarkupfalse%
\ Pi{\isacharunderscore}def\ \isacommand{by}\isamarkupfalse%
\ auto%
\endisatagproof
{\isafoldproof}%
%
\isadelimproof
\isanewline
%
\endisadelimproof
%
\isadelimtheory
\isanewline
%
\endisadelimtheory
%
\isatagtheory
\isacommand{end}\isamarkupfalse%
%
\endisatagtheory
{\isafoldtheory}%
%
\isadelimtheory
%
\endisadelimtheory
\end{isabellebody}%
%%% Local Variables:
%%% mode: latex
%%% TeX-master: "root"
%%% End:


%
\begin{isabellebody}%
\def\isabellecontext{MonoidSums}%
%
\isamarkupheader{Sums in monoids%
}
\isamarkuptrue%
%
\isadelimtheory
%
\endisadelimtheory
%
\isatagtheory
\isacommand{theory}\isamarkupfalse%
\ MonoidSums\isanewline
\isanewline
\isakeyword{imports}\ Main\isanewline
\ \ {\isachardoublequoteopen}{\isachartilde}{\isachartilde}{\isacharslash}src{\isacharslash}HOL{\isacharslash}Algebra{\isacharslash}Module{\isachardoublequoteclose}\isanewline
\ \ RingModuleFacts\isanewline
\ \ FunctionLemmas\isanewline
\isakeyword{begin}%
\endisatagtheory
{\isafoldtheory}%
%
\isadelimtheory
%
\endisadelimtheory
%
\begin{isamarkuptext}%
We build on the finite product simplifications in FiniteProduct.thy and the analogous ones
for finite sums (see "lemmas" in Ring.thy).%
\end{isamarkuptext}%
\isamarkuptrue%
%
\begin{isamarkuptext}%
Use as an intro rule%
\end{isamarkuptext}%
\isamarkuptrue%
\isacommand{lemma}\isamarkupfalse%
\ {\isacharparenleft}\isakeyword{in}\ comm{\isacharunderscore}monoid{\isacharparenright}\ factors{\isacharunderscore}equal{\isacharcolon}\isanewline
\ \ {\isachardoublequoteopen}{\isasymlbrakk}a{\isacharequal}b{\isacharsemicolon}\ c{\isacharequal}d{\isasymrbrakk}\ {\isasymLongrightarrow}\ a{\isasymotimes}\isactrlbsub G\isactrlesub c\ {\isacharequal}\ \ b{\isasymotimes}\isactrlbsub G\isactrlesub d{\isachardoublequoteclose}\isanewline
%
\isadelimproof
\ \ %
\endisadelimproof
%
\isatagproof
\isacommand{by}\isamarkupfalse%
\ simp%
\endisatagproof
{\isafoldproof}%
%
\isadelimproof
\isanewline
%
\endisadelimproof
\isanewline
\isanewline
\isacommand{lemma}\isamarkupfalse%
\ {\isacharparenleft}\isakeyword{in}\ comm{\isacharunderscore}monoid{\isacharparenright}\ extend{\isacharunderscore}prod{\isacharcolon}\isanewline
\ \ \isakeyword{fixes}\ a\ A\ S\isanewline
\ \ \isakeyword{assumes}\ fin{\isacharcolon}\ {\isachardoublequoteopen}finite\ S{\isachardoublequoteclose}\ \isakeyword{and}\ subset{\isacharcolon}\ {\isachardoublequoteopen}A{\isasymsubseteq}S{\isachardoublequoteclose}\ \isakeyword{and}\ a{\isacharcolon}\ {\isachardoublequoteopen}a{\isasymin}A{\isasymrightarrow}carrier\ G{\isachardoublequoteclose}\isanewline
\ \ \isakeyword{shows}\ {\isachardoublequoteopen}{\isacharparenleft}{\isasymOtimes}\isactrlbsub G\isactrlesub \ x{\isasymin}S{\isachardot}\ {\isacharparenleft}if\ x{\isasymin}A\ then\ a\ x\ else\ {\isasymone}\isactrlbsub G\isactrlesub {\isacharparenright}{\isacharparenright}\ {\isacharequal}\ {\isacharparenleft}{\isasymOtimes}\isactrlbsub G\isactrlesub \ x{\isasymin}A{\isachardot}\ a\ x{\isacharparenright}{\isachardoublequoteclose}\isanewline
\ \ {\isacharparenleft}\isakeyword{is}\ {\isachardoublequoteopen}{\isacharparenleft}{\isasymOtimes}\isactrlbsub G\isactrlesub \ x{\isasymin}S{\isachardot}\ {\isacharquery}b\ x{\isacharparenright}\ {\isacharequal}\ {\isacharparenleft}{\isasymOtimes}\isactrlbsub G\isactrlesub \ x{\isasymin}A{\isachardot}\ a\ x{\isacharparenright}{\isachardoublequoteclose}{\isacharparenright}\isanewline
%
\isadelimproof
%
\endisadelimproof
%
\isatagproof
\isacommand{proof}\isamarkupfalse%
\ {\isacharminus}\ \isanewline
\ \ \isacommand{from}\isamarkupfalse%
\ subset\ \isacommand{have}\isamarkupfalse%
\ uni{\isacharcolon}{\isachardoublequoteopen}S\ {\isacharequal}\ A\ {\isasymunion}\ {\isacharparenleft}S{\isacharminus}A{\isacharparenright}{\isachardoublequoteclose}\ \isacommand{by}\isamarkupfalse%
\ auto\isanewline
\ \ \isacommand{from}\isamarkupfalse%
\ assms\ subset\ \isacommand{show}\isamarkupfalse%
\ {\isacharquery}thesis\isanewline
\ \ \ \ \isacommand{apply}\isamarkupfalse%
\ {\isacharparenleft}subst\ uni{\isacharparenright}\isanewline
\ \ \ \ \isacommand{apply}\isamarkupfalse%
\ {\isacharparenleft}subst\ finprod{\isacharunderscore}Un{\isacharunderscore}disjoint{\isacharcomma}\ auto{\isacharparenright}\isanewline
\ \ \ \ \isacommand{by}\isamarkupfalse%
\ {\isacharparenleft}auto\ cong{\isacharcolon}\ finprod{\isacharunderscore}cong\ if{\isacharunderscore}cong\ elim{\isacharcolon}\ finite{\isacharunderscore}subset\ simp\ add{\isacharcolon}Pi{\isacharunderscore}def\ finite{\isacharunderscore}subset{\isacharparenright}\isanewline
\isanewline
\isacommand{qed}\isamarkupfalse%
%
\endisatagproof
{\isafoldproof}%
%
\isadelimproof
%
\endisadelimproof
%
\begin{isamarkuptext}%
Scalar multiplication distributes over scalar multiplication (on left).%
\end{isamarkuptext}%
\isamarkuptrue%
\isacommand{lemma}\isamarkupfalse%
\ {\isacharparenleft}\isakeyword{in}\ module{\isacharparenright}\ finsum{\isacharunderscore}smult{\isacharcolon}\isanewline
\ \ {\isachardoublequoteopen}{\isacharbrackleft}{\isacharbar}\ finite\ A{\isacharsemicolon}\ c{\isasymin}\ carrier\ R{\isacharsemicolon}\ g\ {\isasymin}\ A\ {\isacharminus}{\isachargreater}\ carrier\ M\ {\isacharbar}{\isacharbrackright}\ {\isacharequal}{\isacharequal}{\isachargreater}\isanewline
\ \ \ {\isacharparenleft}c\ {\isasymodot}\isactrlbsub M\isactrlesub \ finsum\ M\ g\ A{\isacharparenright}\ {\isacharequal}\ finsum\ M\ {\isacharparenleft}{\isacharpercent}x{\isachardot}\ c\ {\isasymodot}\isactrlbsub M\isactrlesub \ g\ x{\isacharparenright}\ A\ {\isachardoublequoteclose}\isanewline
%
\isadelimproof
%
\endisadelimproof
%
\isatagproof
\isacommand{proof}\isamarkupfalse%
\ {\isacharparenleft}induct\ set{\isacharcolon}\ finite{\isacharparenright}\isanewline
\ \ \isacommand{case}\isamarkupfalse%
\ empty\ \isanewline
\ \ \isacommand{from}\isamarkupfalse%
\ {\isacharbackquoteopen}c{\isasymin}carrier\ R{\isacharbackquoteclose}\ \isacommand{show}\isamarkupfalse%
\ {\isacharquery}case\ \isanewline
\ \ \ \ \isacommand{by}\isamarkupfalse%
\ simp\isanewline
\isacommand{next}\isamarkupfalse%
\isanewline
\ \ \isacommand{case}\isamarkupfalse%
\ {\isacharparenleft}insert\ a\ A{\isacharparenright}\isanewline
\ \ \isacommand{from}\isamarkupfalse%
\ insert{\isachardot}hyps\ insert{\isachardot}prems\ \isacommand{have}\isamarkupfalse%
\ {\isadigit{1}}{\isacharcolon}\ {\isachardoublequoteopen}finsum\ M\ g\ {\isacharparenleft}insert\ a\ A{\isacharparenright}\ {\isacharequal}\ g\ a\ {\isasymoplus}\isactrlbsub M\isactrlesub \ finsum\ M\ g\ A{\isachardoublequoteclose}\isanewline
\ \ \ \ \isacommand{by}\isamarkupfalse%
\ {\isacharparenleft}intro\ finsum{\isacharunderscore}insert{\isacharcomma}\ auto{\isacharparenright}\isanewline
\ \ \isacommand{from}\isamarkupfalse%
\ insert{\isachardot}hyps\ insert{\isachardot}prems\ \isacommand{have}\isamarkupfalse%
\ {\isadigit{2}}{\isacharcolon}\ {\isachardoublequoteopen}{\isacharparenleft}{\isasymOplus}\isactrlbsub M\isactrlesub x{\isasymin}insert\ a\ A{\isachardot}\ c\ {\isasymodot}\isactrlbsub M\isactrlesub \ g\ x{\isacharparenright}\ {\isacharequal}\ c\ {\isasymodot}\isactrlbsub M\isactrlesub \ g\ a\ {\isasymoplus}\isactrlbsub M\isactrlesub {\isacharparenleft}{\isasymOplus}\isactrlbsub M\isactrlesub x{\isasymin}A{\isachardot}\ c\ {\isasymodot}\isactrlbsub M\isactrlesub \ g\ x{\isacharparenright}{\isachardoublequoteclose}\ \isanewline
\ \ \ \ \isacommand{by}\isamarkupfalse%
\ {\isacharparenleft}intro\ finsum{\isacharunderscore}insert{\isacharcomma}\ auto{\isacharparenright}\isanewline
\ \ \isacommand{from}\isamarkupfalse%
\ insert{\isachardot}hyps\ insert{\isachardot}prems\ \isacommand{show}\isamarkupfalse%
\ {\isacharquery}case\ \isanewline
\ \ \ \ \isacommand{by}\isamarkupfalse%
\ {\isacharparenleft}auto\ simp\ add{\isacharcolon}{\isadigit{1}}\ {\isadigit{2}}\ smult{\isacharunderscore}r{\isacharunderscore}distr\ finsum{\isacharunderscore}closed{\isacharparenright}\isanewline
\isacommand{qed}\isamarkupfalse%
%
\endisatagproof
{\isafoldproof}%
%
\isadelimproof
%
\endisadelimproof
%
\begin{isamarkuptext}%
Scalar multiplication distributes over scalar multiplication (on right).%
\end{isamarkuptext}%
\isamarkuptrue%
\isacommand{lemma}\isamarkupfalse%
\ {\isacharparenleft}\isakeyword{in}\ module{\isacharparenright}\ finsum{\isacharunderscore}smult{\isacharunderscore}r{\isacharcolon}\isanewline
\ \ {\isachardoublequoteopen}{\isacharbrackleft}{\isacharbar}\ finite\ A{\isacharsemicolon}\ v{\isasymin}\ carrier\ M{\isacharsemicolon}\ f\ {\isasymin}\ A\ {\isacharminus}{\isachargreater}\ carrier\ R\ {\isacharbar}{\isacharbrackright}\ {\isacharequal}{\isacharequal}{\isachargreater}\isanewline
\ \ \ {\isacharparenleft}finsum\ R\ f\ A\ {\isasymodot}\isactrlbsub M\isactrlesub \ v{\isacharparenright}\ {\isacharequal}\ finsum\ M\ {\isacharparenleft}{\isacharpercent}x{\isachardot}\ f\ x\ {\isasymodot}\isactrlbsub M\isactrlesub \ v{\isacharparenright}\ A\ {\isachardoublequoteclose}\isanewline
%
\isadelimproof
%
\endisadelimproof
%
\isatagproof
\isacommand{proof}\isamarkupfalse%
\ {\isacharparenleft}induct\ set{\isacharcolon}\ finite{\isacharparenright}\isanewline
\ \ \isacommand{case}\isamarkupfalse%
\ empty\ \isanewline
\ \ \isacommand{from}\isamarkupfalse%
\ {\isacharbackquoteopen}v{\isasymin}carrier\ M{\isacharbackquoteclose}\ \isacommand{show}\isamarkupfalse%
\ {\isacharquery}case\ \isanewline
\ \ \ \ \isacommand{by}\isamarkupfalse%
\ simp\isanewline
\isacommand{next}\isamarkupfalse%
\isanewline
\ \ \isacommand{case}\isamarkupfalse%
\ {\isacharparenleft}insert\ a\ A{\isacharparenright}\isanewline
\ \ \isacommand{from}\isamarkupfalse%
\ insert{\isachardot}hyps\ insert{\isachardot}prems\ \isacommand{have}\isamarkupfalse%
\ {\isadigit{1}}{\isacharcolon}\ {\isachardoublequoteopen}finsum\ R\ f\ {\isacharparenleft}insert\ a\ A{\isacharparenright}\ {\isacharequal}\ f\ a\ {\isasymoplus}\isactrlbsub R\isactrlesub \ finsum\ R\ f\ A{\isachardoublequoteclose}\isanewline
\ \ \ \ \isacommand{by}\isamarkupfalse%
\ {\isacharparenleft}intro\ R{\isachardot}finsum{\isacharunderscore}insert{\isacharcomma}\ auto{\isacharparenright}\isanewline
\ \ \isacommand{from}\isamarkupfalse%
\ insert{\isachardot}hyps\ insert{\isachardot}prems\ \isacommand{have}\isamarkupfalse%
\ {\isadigit{2}}{\isacharcolon}\ {\isachardoublequoteopen}{\isacharparenleft}{\isasymOplus}\isactrlbsub M\isactrlesub x{\isasymin}insert\ a\ A{\isachardot}\ f\ x\ {\isasymodot}\isactrlbsub M\isactrlesub \ v{\isacharparenright}\ {\isacharequal}\ f\ a\ {\isasymodot}\isactrlbsub M\isactrlesub \ v\ {\isasymoplus}\isactrlbsub M\isactrlesub {\isacharparenleft}{\isasymOplus}\isactrlbsub M\isactrlesub x{\isasymin}A{\isachardot}\ f\ x\ {\isasymodot}\isactrlbsub M\isactrlesub \ v{\isacharparenright}{\isachardoublequoteclose}\ \isanewline
\ \ \ \ \isacommand{by}\isamarkupfalse%
\ {\isacharparenleft}intro\ finsum{\isacharunderscore}insert{\isacharcomma}\ auto{\isacharparenright}\isanewline
\ \ \isacommand{from}\isamarkupfalse%
\ insert{\isachardot}hyps\ insert{\isachardot}prems\ \isacommand{show}\isamarkupfalse%
\ {\isacharquery}case\ \isanewline
\ \ \ \ \isacommand{by}\isamarkupfalse%
\ {\isacharparenleft}auto\ simp\ add{\isacharcolon}{\isadigit{1}}\ {\isadigit{2}}\ smult{\isacharunderscore}l{\isacharunderscore}distr\ finsum{\isacharunderscore}closed{\isacharparenright}\isanewline
\isacommand{qed}\isamarkupfalse%
%
\endisatagproof
{\isafoldproof}%
%
\isadelimproof
%
\endisadelimproof
%
\begin{isamarkuptext}%
A sequence of lemmas that shows that the product does not depend on the ambient group. 
Note I had to dig back into the definitions of foldSet to show this.%
\end{isamarkuptext}%
\isamarkuptrue%
\isacommand{lemma}\isamarkupfalse%
\ foldSet{\isacharunderscore}not{\isacharunderscore}depend{\isacharcolon}\isanewline
\ \ \isakeyword{fixes}\ A\ E\ \isanewline
\ \ \isakeyword{assumes}\ h{\isadigit{1}}{\isacharcolon}\ {\isachardoublequoteopen}D{\isasymsubseteq}E{\isachardoublequoteclose}\isanewline
\ \ \isakeyword{shows}\ {\isachardoublequoteopen}foldSetD\ D\ f\ e\ {\isasymsubseteq}foldSetD\ E\ f\ e{\isachardoublequoteclose}\isanewline
%
\isadelimproof
%
\endisadelimproof
%
\isatagproof
\isacommand{proof}\isamarkupfalse%
\ {\isacharminus}\isanewline
\ \ \isacommand{from}\isamarkupfalse%
\ h{\isadigit{1}}\ \isacommand{have}\isamarkupfalse%
\ {\isadigit{1}}{\isacharcolon}\ {\isachardoublequoteopen}{\isasymAnd}x{\isadigit{1}}\ x{\isadigit{2}}{\isachardot}\ {\isacharparenleft}x{\isadigit{1}}{\isacharcomma}x{\isadigit{2}}{\isacharparenright}\ {\isasymin}\ foldSetD\ D\ f\ e\ {\isasymLongrightarrow}\ {\isacharparenleft}x{\isadigit{1}}{\isacharcomma}\ x{\isadigit{2}}{\isacharparenright}\ {\isasymin}\ foldSetD\ E\ f\ e{\isachardoublequoteclose}\ \isanewline
\ \ \isacommand{proof}\isamarkupfalse%
\ {\isacharminus}\isanewline
\ \ \ \ \isacommand{fix}\isamarkupfalse%
\ x{\isadigit{1}}\ x{\isadigit{2}}\ \isanewline
\ \ \ \ \isacommand{assume}\isamarkupfalse%
\ {\isadigit{2}}{\isacharcolon}\ {\isachardoublequoteopen}{\isacharparenleft}x{\isadigit{1}}{\isacharcomma}x{\isadigit{2}}{\isacharparenright}\ {\isasymin}\ foldSetD\ D\ f\ e{\isachardoublequoteclose}\isanewline
\ \ \ \ \isacommand{from}\isamarkupfalse%
\ h{\isadigit{1}}\ {\isadigit{2}}\ \isacommand{show}\isamarkupfalse%
\ {\isachardoublequoteopen}{\isacharquery}thesis\ x{\isadigit{1}}\ x{\isadigit{2}}{\isachardoublequoteclose}\isanewline
\ \ \ \ \isacommand{apply}\isamarkupfalse%
\ {\isacharparenleft}intro\ foldSetD{\isachardot}induct{\isacharbrackleft}\isakeyword{where}\ {\isacharquery}D{\isacharequal}{\isachardoublequoteopen}D{\isachardoublequoteclose}\ \isakeyword{and}\ {\isacharquery}f{\isacharequal}{\isachardoublequoteopen}f{\isachardoublequoteclose}\ \isakeyword{and}\ {\isacharquery}e{\isacharequal}{\isachardoublequoteopen}e{\isachardoublequoteclose}\ \isakeyword{and}\ {\isacharquery}x{\isadigit{1}}{\isachardot}{\isadigit{0}}{\isacharequal}{\isachardoublequoteopen}x{\isadigit{1}}{\isachardoublequoteclose}\ \isakeyword{and}\ {\isacharquery}x{\isadigit{2}}{\isachardot}{\isadigit{0}}{\isacharequal}{\isachardoublequoteopen}x{\isadigit{2}}{\isachardoublequoteclose}\isanewline
\ \ \ \ \ \ \ \ \isakeyword{and}\ {\isacharquery}P\ {\isacharequal}\ {\isachardoublequoteopen}{\isasymlambda}x{\isadigit{1}}\ x{\isadigit{2}}{\isachardot}\ {\isacharparenleft}{\isacharparenleft}x{\isadigit{1}}{\isacharcomma}\ x{\isadigit{2}}{\isacharparenright}{\isasymin}\ foldSetD\ E\ f\ e{\isacharparenright}{\isachardoublequoteclose}{\isacharbrackright}{\isacharparenright}\ \isanewline
\ \ \ \ \ \ \isacommand{apply}\isamarkupfalse%
\ auto\isanewline
\ \ \ \ \ \isacommand{apply}\isamarkupfalse%
\ {\isacharparenleft}intro\ emptyI{\isacharcomma}\ auto{\isacharparenright}\isanewline
\ \ \ \ \isacommand{by}\isamarkupfalse%
\ {\isacharparenleft}intro\ insertI{\isacharcomma}\ auto{\isacharparenright}\isanewline
\ \ \isacommand{qed}\isamarkupfalse%
\isanewline
\ \ \isacommand{from}\isamarkupfalse%
\ {\isadigit{1}}\ \isacommand{show}\isamarkupfalse%
\ {\isacharquery}thesis\ \isacommand{by}\isamarkupfalse%
\ auto\isanewline
\isacommand{qed}\isamarkupfalse%
%
\endisatagproof
{\isafoldproof}%
%
\isadelimproof
\ \isanewline
%
\endisadelimproof
\isanewline
\isacommand{lemma}\isamarkupfalse%
\ foldD{\isacharunderscore}not{\isacharunderscore}depend{\isacharcolon}\isanewline
\ \ \isakeyword{fixes}\ D\ E\ B\ f\ e\ A\isanewline
\ \ \isakeyword{assumes}\ h{\isadigit{1}}{\isacharcolon}\ {\isachardoublequoteopen}LCD\ B\ D\ f{\isachardoublequoteclose}\ \isakeyword{and}\ h{\isadigit{2}}{\isacharcolon}\ {\isachardoublequoteopen}LCD\ B\ E\ f{\isachardoublequoteclose}\ \isakeyword{and}\ h{\isadigit{3}}{\isacharcolon}\ {\isachardoublequoteopen}D{\isasymsubseteq}E{\isachardoublequoteclose}\ \isakeyword{and}\ h{\isadigit{4}}{\isacharcolon}\ {\isachardoublequoteopen}e{\isasymin}D{\isachardoublequoteclose}\ \isakeyword{and}\ h{\isadigit{5}}{\isacharcolon}\ {\isachardoublequoteopen}A{\isasymsubseteq}B{\isachardoublequoteclose}\ \isakeyword{and}\ h{\isadigit{6}}{\isacharcolon}\ {\isachardoublequoteopen}finite\ B{\isachardoublequoteclose}\isanewline
\ \ \isakeyword{shows}\ {\isachardoublequoteopen}foldD\ D\ f\ e\ A\ {\isacharequal}\ foldD\ E\ f\ e\ A{\isachardoublequoteclose}\isanewline
%
\isadelimproof
%
\endisadelimproof
%
\isatagproof
\isacommand{proof}\isamarkupfalse%
\ {\isacharminus}\ \isanewline
\ \ \isacommand{from}\isamarkupfalse%
\ assms\ \isacommand{have}\isamarkupfalse%
\ {\isadigit{1}}{\isacharcolon}\ {\isachardoublequoteopen}{\isasymexists}y{\isachardot}\ {\isacharparenleft}A{\isacharcomma}y{\isacharparenright}{\isasymin}foldSetD\ D\ f\ e{\isachardoublequoteclose}\isanewline
\ \ \ \ \isacommand{apply}\isamarkupfalse%
\ {\isacharparenleft}intro\ finite{\isacharunderscore}imp{\isacharunderscore}foldSetD{\isacharcomma}\ auto{\isacharparenright}\isanewline
\ \ \ \ \ \isacommand{apply}\isamarkupfalse%
\ {\isacharparenleft}metis\ finite{\isacharunderscore}subset{\isacharparenright}\isanewline
\ \ \ \ \isacommand{by}\isamarkupfalse%
\ {\isacharparenleft}unfold\ LCD{\isacharunderscore}def{\isacharcomma}\ auto{\isacharparenright}\isanewline
\ \ \isacommand{from}\isamarkupfalse%
\ {\isadigit{1}}\ \isacommand{obtain}\isamarkupfalse%
\ y\ \isakeyword{where}\ {\isadigit{2}}{\isacharcolon}\ {\isachardoublequoteopen}{\isacharparenleft}A{\isacharcomma}y{\isacharparenright}{\isasymin}foldSetD\ D\ f\ e{\isachardoublequoteclose}\ \isacommand{by}\isamarkupfalse%
\ auto\isanewline
\ \ \isacommand{from}\isamarkupfalse%
\ assms\ {\isadigit{2}}\ \isacommand{have}\isamarkupfalse%
\ {\isadigit{3}}{\isacharcolon}\ {\isachardoublequoteopen}foldD\ D\ f\ e\ A\ {\isacharequal}\ y{\isachardoublequoteclose}\ \isacommand{by}\isamarkupfalse%
\ {\isacharparenleft}intro\ LCD{\isachardot}foldD{\isacharunderscore}equality{\isacharbrackleft}of\ B{\isacharbrackright}{\isacharcomma}\ auto{\isacharparenright}\isanewline
\ \ \isacommand{from}\isamarkupfalse%
\ h{\isadigit{3}}\ \isacommand{have}\isamarkupfalse%
\ {\isadigit{4}}{\isacharcolon}\ {\isachardoublequoteopen}foldSetD\ D\ f\ e\ {\isasymsubseteq}\ foldSetD\ E\ f\ e{\isachardoublequoteclose}\ \isacommand{by}\isamarkupfalse%
\ {\isacharparenleft}rule\ foldSet{\isacharunderscore}not{\isacharunderscore}depend{\isacharparenright}\isanewline
\ \ \isacommand{from}\isamarkupfalse%
\ {\isadigit{2}}\ {\isadigit{4}}\ \isacommand{have}\isamarkupfalse%
\ {\isadigit{5}}{\isacharcolon}\ {\isachardoublequoteopen}{\isacharparenleft}A{\isacharcomma}y{\isacharparenright}{\isasymin}foldSetD\ E\ f\ e{\isachardoublequoteclose}\ \isacommand{by}\isamarkupfalse%
\ auto\isanewline
\ \ \isacommand{from}\isamarkupfalse%
\ assms\ {\isadigit{5}}\ \isacommand{have}\isamarkupfalse%
\ {\isadigit{6}}{\isacharcolon}\ {\isachardoublequoteopen}foldD\ E\ f\ e\ A\ {\isacharequal}\ y{\isachardoublequoteclose}\ \isacommand{by}\isamarkupfalse%
\ {\isacharparenleft}intro\ LCD{\isachardot}foldD{\isacharunderscore}equality{\isacharbrackleft}of\ B{\isacharbrackright}{\isacharcomma}\ auto{\isacharparenright}\isanewline
\isanewline
\ \ \isacommand{from}\isamarkupfalse%
\ {\isadigit{3}}\ {\isadigit{6}}\ \isacommand{show}\isamarkupfalse%
\ {\isacharquery}thesis\ \isacommand{by}\isamarkupfalse%
\ auto\isanewline
\isacommand{qed}\isamarkupfalse%
%
\endisatagproof
{\isafoldproof}%
%
\isadelimproof
\isanewline
%
\endisadelimproof
\isanewline
\isacommand{lemma}\isamarkupfalse%
\ {\isacharparenleft}\isakeyword{in}\ comm{\isacharunderscore}monoid{\isacharparenright}\ finprod{\isacharunderscore}all{\isadigit{1}}{\isacharbrackleft}simp{\isacharbrackright}{\isacharcolon}\isanewline
\ \ \isakeyword{assumes}\ fin{\isacharcolon}\ {\isachardoublequoteopen}finite\ A{\isachardoublequoteclose}\ \isakeyword{and}\ all{\isadigit{1}}{\isacharcolon}{\isachardoublequoteopen}\ {\isasymAnd}a{\isachardot}\ a{\isasymin}A{\isasymLongrightarrow}f\ a\ {\isacharequal}\ {\isasymone}\isactrlbsub G\isactrlesub {\isachardoublequoteclose}\isanewline
\ \ \isakeyword{shows}\ {\isachardoublequoteopen}{\isacharparenleft}{\isasymOtimes}\isactrlbsub G\isactrlesub \ a{\isasymin}A{\isachardot}\ f\ a{\isacharparenright}\ {\isacharequal}\ {\isasymone}\isactrlbsub G\isactrlesub {\isachardoublequoteclose}\isanewline
%
\isadelimproof
\isanewline
%
\endisadelimproof
%
\isatagproof
\isacommand{proof}\isamarkupfalse%
\ {\isacharminus}\ \isanewline
\ \ \isacommand{from}\isamarkupfalse%
\ assms\ \isacommand{show}\isamarkupfalse%
\ {\isacharquery}thesis\isanewline
\ \ \ \ \isacommand{by}\isamarkupfalse%
\ {\isacharparenleft}simp\ cong{\isacharcolon}\ finprod{\isacharunderscore}cong{\isacharparenright}\isanewline
\isacommand{qed}\isamarkupfalse%
%
\endisatagproof
{\isafoldproof}%
%
\isadelimproof
\isanewline
%
\endisadelimproof
\isanewline
\isacommand{context}\isamarkupfalse%
\ abelian{\isacharunderscore}monoid\isanewline
\isakeyword{begin}\isanewline
\isacommand{lemmas}\isamarkupfalse%
\ summands{\isacharunderscore}equal\ {\isacharequal}\ add{\isachardot}factors{\isacharunderscore}equal\isanewline
\isacommand{lemmas}\isamarkupfalse%
\ extend{\isacharunderscore}sum\ {\isacharequal}\ add{\isachardot}extend{\isacharunderscore}prod\isanewline
\isacommand{lemmas}\isamarkupfalse%
\ finsum{\isacharunderscore}all{\isadigit{0}}\ {\isacharequal}\ add{\isachardot}finprod{\isacharunderscore}all{\isadigit{1}}\isanewline
\isacommand{end}\isamarkupfalse%
\isanewline
%
\isadelimtheory
\isanewline
%
\endisadelimtheory
%
\isatagtheory
\isacommand{end}\isamarkupfalse%
%
\endisatagtheory
{\isafoldtheory}%
%
\isadelimtheory
%
\endisadelimtheory
\end{isabellebody}%
%%% Local Variables:
%%% mode: latex
%%% TeX-master: "root"
%%% End:


%
\begin{isabellebody}%
\def\isabellecontext{LinearCombinations}%
%
\isamarkupheader{Linear Combinations%
}
\isamarkuptrue%
%
\isadelimtheory
%
\endisadelimtheory
%
\isatagtheory
\isacommand{theory}\isamarkupfalse%
\ LinearCombinations\isanewline
\isakeyword{imports}\ Main\isanewline
\ \ {\isachardoublequoteopen}{\isachartilde}{\isachartilde}{\isacharslash}src{\isacharslash}HOL{\isacharslash}Algebra{\isacharslash}Module{\isachardoublequoteclose}\isanewline
\ \ {\isachardoublequoteopen}{\isachartilde}{\isachartilde}{\isacharslash}src{\isacharslash}HOL{\isacharslash}Algebra{\isacharslash}Coset{\isachardoublequoteclose}\isanewline
\ \ RingModuleFacts\isanewline
\ \ MonoidSums\isanewline
\ \ FunctionLemmas\isanewline
\isakeyword{begin}%
\endisatagtheory
{\isafoldtheory}%
%
\isadelimtheory
%
\endisadelimtheory
%
\isamarkupsubsection{Lemmas for simplification%
}
\isamarkuptrue%
%
\begin{isamarkuptext}%
The following are helpful in certain simplifications (esp. congruence rules). Warning: arbitrary
use leads to looping.%
\end{isamarkuptext}%
\isamarkuptrue%
\isacommand{lemma}\isamarkupfalse%
\ {\isacharparenleft}\isakeyword{in}\ ring{\isacharparenright}\ coeff{\isacharunderscore}in{\isacharunderscore}ring{\isacharcolon}\isanewline
\ \ {\isachardoublequoteopen}{\isasymlbrakk}a{\isasymin}A{\isasymrightarrow}carrier\ R{\isacharsemicolon}\ x{\isasymin}A{\isasymrbrakk}\ {\isasymLongrightarrow}\ a\ x\ {\isasymin}carrier\ R{\isachardoublequoteclose}\isanewline
%
\isadelimproof
%
\endisadelimproof
%
\isatagproof
\isacommand{by}\isamarkupfalse%
\ {\isacharparenleft}metis\ Pi{\isacharunderscore}mem{\isacharparenright}%
\endisatagproof
{\isafoldproof}%
%
\isadelimproof
\isanewline
%
\endisadelimproof
\isanewline
\isanewline
\isacommand{lemma}\isamarkupfalse%
\ {\isacharparenleft}\isakeyword{in}\ ring{\isacharparenright}\ coeff{\isacharunderscore}in{\isacharunderscore}ring{\isadigit{2}}{\isacharcolon}\isanewline
\ \ {\isachardoublequoteopen}{\isasymlbrakk}\ x{\isasymin}A{\isacharsemicolon}a{\isasymin}A{\isasymrightarrow}carrier\ R{\isasymrbrakk}\ {\isasymLongrightarrow}\ a\ x\ {\isasymin}carrier\ R{\isachardoublequoteclose}\isanewline
%
\isadelimproof
%
\endisadelimproof
%
\isatagproof
\isacommand{by}\isamarkupfalse%
\ {\isacharparenleft}metis\ Pi{\isacharunderscore}mem{\isacharparenright}%
\endisatagproof
{\isafoldproof}%
%
\isadelimproof
\isanewline
%
\endisadelimproof
\isanewline
\isacommand{lemma}\isamarkupfalse%
\ ring{\isacharunderscore}subset{\isacharunderscore}carrier{\isacharcolon}\isanewline
\ \ {\isachardoublequoteopen}{\isasymlbrakk}x\ {\isasymin}A{\isacharsemicolon}\ A{\isasymsubseteq}carrier\ R{\isasymrbrakk}\ {\isasymLongrightarrow}\ x\ {\isasymin}carrier\ R{\isachardoublequoteclose}\isanewline
%
\isadelimproof
%
\endisadelimproof
%
\isatagproof
\isacommand{by}\isamarkupfalse%
\ auto%
\endisatagproof
{\isafoldproof}%
%
\isadelimproof
%
\endisadelimproof
%
\begin{isamarkuptext}%
A hack to not cause an infinite loop with $\to$ simplification.%
\end{isamarkuptext}%
\isamarkuptrue%
\isacommand{definition}\isamarkupfalse%
\ Pi{\isadigit{2}}{\isacharcolon}{\isacharcolon}{\isachardoublequoteopen}{\isacharparenleft}{\isacharprime}a\ set{\isacharparenright}\ {\isasymRightarrow}\ {\isacharparenleft}{\isacharprime}b\ set{\isacharparenright}\ {\isasymRightarrow}\ {\isacharparenleft}{\isacharprime}a{\isasymRightarrow}{\isacharprime}b{\isacharparenright}\ set{\isachardoublequoteclose}\isanewline
\ \ \isakeyword{where}\ {\isachardoublequoteopen}Pi{\isadigit{2}}\ A\ B\ {\isacharequal}\ A\ {\isasymrightarrow}\ B{\isachardoublequoteclose}\isanewline
\isanewline
\isacommand{lemma}\isamarkupfalse%
\ Pi{\isacharunderscore}implies{\isacharunderscore}Pi{\isadigit{2}}{\isacharcolon}\isanewline
\ \ {\isachardoublequoteopen}a\ {\isasymin}\ A{\isasymrightarrow}B\ {\isasymLongrightarrow}\ a\ {\isasymin}\ Pi{\isadigit{2}}\ A\ B{\isachardoublequoteclose}\isanewline
%
\isadelimproof
%
\endisadelimproof
%
\isatagproof
\isacommand{by}\isamarkupfalse%
\ {\isacharparenleft}unfold\ Pi{\isadigit{2}}{\isacharunderscore}def{\isacharcomma}\ auto{\isacharparenright}%
\endisatagproof
{\isafoldproof}%
%
\isadelimproof
\isanewline
%
\endisadelimproof
\isanewline
\isanewline
\isacommand{lemma}\isamarkupfalse%
\ Pi{\isacharunderscore}mem{\isacharunderscore}Pi{\isadigit{2}}{\isacharcolon}\isanewline
\ \ {\isachardoublequoteopen}{\isasymlbrakk}a{\isasymin}Pi{\isadigit{2}}\ S\ T{\isacharsemicolon}\ x{\isasymin}S{\isasymrbrakk}\ {\isasymLongrightarrow}\ a\ x\ {\isasymin}T{\isachardoublequoteclose}\isanewline
%
\isadelimproof
%
\endisadelimproof
%
\isatagproof
\isacommand{by}\isamarkupfalse%
\ {\isacharparenleft}unfold\ Pi{\isadigit{2}}{\isacharunderscore}def{\isacharcomma}\ rule\ Pi{\isacharunderscore}mem{\isadigit{2}}{\isacharparenright}%
\endisatagproof
{\isafoldproof}%
%
\isadelimproof
\isanewline
%
\endisadelimproof
\isanewline
\isacommand{lemma}\isamarkupfalse%
\ Pi{\isacharunderscore}mem{\isacharunderscore}Pi{\isadigit{2}}{\isacharunderscore}sub{\isadigit{1}}{\isacharcolon}\isanewline
\ \ {\isachardoublequoteopen}{\isasymlbrakk}a{\isasymin}Pi{\isadigit{2}}\ S\ T{\isacharsemicolon}\ x{\isasymin}A{\isacharsemicolon}\ A{\isasymsubseteq}S{\isasymrbrakk}\ {\isasymLongrightarrow}\ a\ x\ {\isasymin}T{\isachardoublequoteclose}\isanewline
%
\isadelimproof
%
\endisadelimproof
%
\isatagproof
\isacommand{by}\isamarkupfalse%
\ {\isacharparenleft}unfold\ Pi{\isadigit{2}}{\isacharunderscore}def{\isacharcomma}\ auto\ intro{\isacharcolon}\ Pi{\isacharunderscore}mem{\isadigit{2}}{\isacharparenright}%
\endisatagproof
{\isafoldproof}%
%
\isadelimproof
\isanewline
%
\endisadelimproof
\isanewline
\isacommand{lemma}\isamarkupfalse%
\ Pi{\isacharunderscore}mem{\isacharunderscore}Pi{\isadigit{2}}{\isacharunderscore}sub{\isadigit{2}}{\isacharcolon}\isanewline
\ \ {\isachardoublequoteopen}{\isasymlbrakk}a{\isasymin}Pi{\isadigit{2}}\ S\ T{\isacharsemicolon}\ x{\isasymin}S{\isacharsemicolon}\ T{\isasymsubseteq}U{\isasymrbrakk}\ {\isasymLongrightarrow}\ a\ x\ {\isasymin}U{\isachardoublequoteclose}\isanewline
%
\isadelimproof
%
\endisadelimproof
%
\isatagproof
\isacommand{by}\isamarkupfalse%
\ {\isacharparenleft}unfold\ Pi{\isadigit{2}}{\isacharunderscore}def{\isacharcomma}\ auto\ intro{\isacharcolon}\ Pi{\isacharunderscore}mem{\isadigit{2}}{\isacharparenright}%
\endisatagproof
{\isafoldproof}%
%
\isadelimproof
\isanewline
%
\endisadelimproof
\isanewline
\isacommand{lemma}\isamarkupfalse%
\ disj{\isacharunderscore}if{\isacharcolon}\isanewline
\ \ {\isachardoublequoteopen}{\isasymlbrakk}A{\isasyminter}B{\isacharequal}{\isacharbraceleft}{\isacharbraceright}{\isacharsemicolon}\ x{\isasymin}\ B{\isasymrbrakk}\ {\isasymLongrightarrow}\ {\isacharparenleft}if\ x{\isasymin}A\ then\ f\ x\ else\ g\ x{\isacharparenright}\ {\isacharequal}\ g\ x{\isachardoublequoteclose}\isanewline
%
\isadelimproof
%
\endisadelimproof
%
\isatagproof
\isacommand{by}\isamarkupfalse%
\ auto%
\endisatagproof
{\isafoldproof}%
%
\isadelimproof
\isanewline
%
\endisadelimproof
\isanewline
\isanewline
\isacommand{lemmas}\isamarkupfalse%
\ Pi{\isacharunderscore}simp\ {\isacharequal}\ Pi{\isacharunderscore}mem{\isacharunderscore}Pi{\isadigit{2}}\ Pi{\isacharunderscore}mem{\isacharunderscore}Pi{\isadigit{2}}{\isacharunderscore}sub{\isadigit{1}}\ Pi{\isacharunderscore}mem{\isacharunderscore}Pi{\isadigit{2}}{\isacharunderscore}sub{\isadigit{2}}\isanewline
\isacommand{lemmas}\isamarkupfalse%
\ {\isacharparenleft}\isakeyword{in}\ module{\isacharparenright}\ sum{\isacharunderscore}simp\ {\isacharequal}\ Pi{\isacharunderscore}simp\ ring{\isacharunderscore}subset{\isacharunderscore}carrier%
\isamarkupsubsection{Linear combinations%
}
\isamarkuptrue%
%
\begin{isamarkuptext}%
A linear combination is $\sum_{v\in A} a_v v$. $(a_v)_{v\in S}$ is a function 
$A\to K$, where $A\subseteq K$.%
\end{isamarkuptext}%
\isamarkuptrue%
\isacommand{definition}\isamarkupfalse%
\ {\isacharparenleft}\isakeyword{in}\ module{\isacharparenright}\ lincomb{\isacharcolon}{\isacharcolon}{\isachardoublequoteopen}{\isacharbrackleft}{\isacharprime}c\ {\isasymRightarrow}\ {\isacharprime}a{\isacharcomma}\ {\isacharprime}c\ set{\isacharbrackright}{\isasymRightarrow}\ {\isacharprime}c{\isachardoublequoteclose}\isanewline
\isakeyword{where}\ {\isachardoublequoteopen}lincomb\ a\ A\ {\isacharequal}\ {\isacharparenleft}{\isasymOplus}\isactrlbsub M\isactrlesub \ \ v{\isasymin}A{\isachardot}\ {\isacharparenleft}a\ v\ {\isasymodot}\isactrlbsub M\isactrlesub \ v{\isacharparenright}{\isacharparenright}{\isachardoublequoteclose}\isanewline
\isanewline
\isacommand{lemma}\isamarkupfalse%
\ {\isacharparenleft}\isakeyword{in}\ module{\isacharparenright}\ summands{\isacharunderscore}valid{\isacharcolon}\isanewline
\ \ \isakeyword{fixes}\ A\ a\isanewline
\ \ \isakeyword{assumes}\ h{\isadigit{2}}{\isacharcolon}\ {\isachardoublequoteopen}A{\isasymsubseteq}\ carrier\ M{\isachardoublequoteclose}\ \isakeyword{and}\ h{\isadigit{3}}{\isacharcolon}\ {\isachardoublequoteopen}a{\isasymin}{\isacharparenleft}A{\isasymrightarrow}carrier\ R{\isacharparenright}{\isachardoublequoteclose}\isanewline
\ \ \isakeyword{shows}\ {\isachardoublequoteopen}{\isasymforall}\ v{\isasymin}\ A{\isachardot}\ {\isacharparenleft}{\isacharparenleft}{\isacharparenleft}a\ v{\isacharparenright}\ {\isasymodot}\isactrlbsub M\isactrlesub \ v{\isacharparenright}{\isasymin}\ carrier\ M{\isacharparenright}{\isachardoublequoteclose}\isanewline
%
\isadelimproof
%
\endisadelimproof
%
\isatagproof
\isacommand{proof}\isamarkupfalse%
\ {\isacharminus}\ \isanewline
\ \ \isacommand{from}\isamarkupfalse%
\ assms\ \isacommand{show}\isamarkupfalse%
\ {\isacharquery}thesis\ \isacommand{by}\isamarkupfalse%
\ auto\isanewline
\isacommand{qed}\isamarkupfalse%
%
\endisatagproof
{\isafoldproof}%
%
\isadelimproof
\isanewline
%
\endisadelimproof
\isanewline
\isacommand{lemma}\isamarkupfalse%
\ {\isacharparenleft}\isakeyword{in}\ module{\isacharparenright}\ lincomb{\isacharunderscore}closed\ {\isacharbrackleft}simp{\isacharcomma}\ intro{\isacharbrackright}{\isacharcolon}\isanewline
\ \ \isakeyword{fixes}\ S\ a\isanewline
\ \ \isakeyword{assumes}\ h{\isadigit{1}}{\isacharcolon}\ {\isachardoublequoteopen}finite\ S{\isachardoublequoteclose}\ \isakeyword{and}\ h{\isadigit{2}}{\isacharcolon}\ {\isachardoublequoteopen}S{\isasymsubseteq}\ carrier\ M{\isachardoublequoteclose}\ \isakeyword{and}\ h{\isadigit{3}}{\isacharcolon}\ {\isachardoublequoteopen}a{\isasymin}{\isacharparenleft}S{\isasymrightarrow}carrier\ R{\isacharparenright}{\isachardoublequoteclose}\isanewline
\ \ \isakeyword{shows}\ {\isachardoublequoteopen}lincomb\ a\ S\ {\isasymin}\ carrier\ M{\isachardoublequoteclose}\isanewline
%
\isadelimproof
%
\endisadelimproof
%
\isatagproof
\isacommand{proof}\isamarkupfalse%
\ {\isacharminus}\isanewline
\ \ \isacommand{from}\isamarkupfalse%
\ h{\isadigit{1}}\ h{\isadigit{2}}\ h{\isadigit{3}}\ \isacommand{show}\isamarkupfalse%
\ {\isacharquery}thesis\ \isacommand{by}\isamarkupfalse%
\ {\isacharparenleft}unfold\ lincomb{\isacharunderscore}def{\isacharcomma}\ auto\ intro{\isacharcolon}finsum{\isacharunderscore}closed{\isacharparenright}\isanewline
\isanewline
\isacommand{qed}\isamarkupfalse%
%
\endisatagproof
{\isafoldproof}%
%
\isadelimproof
\isanewline
%
\endisadelimproof
\isanewline
\isacommand{lemma}\isamarkupfalse%
\ {\isacharparenleft}\isakeyword{in}\ comm{\isacharunderscore}monoid{\isacharparenright}\ finprod{\isacharunderscore}cong{\isadigit{2}}{\isacharcolon}\isanewline
\ \ {\isachardoublequoteopen}{\isacharbrackleft}{\isacharbar}\ A\ {\isacharequal}\ B{\isacharsemicolon}\ \isanewline
\ \ \ \ \ \ {\isacharbang}{\isacharbang}i{\isachardot}\ i\ {\isasymin}\ B\ {\isacharequal}{\isacharequal}{\isachargreater}\ f\ i\ {\isacharequal}\ g\ i{\isacharsemicolon}\ f\ {\isasymin}\ B\ {\isacharminus}{\isachargreater}\ carrier\ G{\isacharbar}{\isacharbrackright}\ {\isacharequal}{\isacharequal}{\isachargreater}\ \isanewline
finprod\ G\ f\ A\ {\isacharequal}\ finprod\ G\ g\ B{\isachardoublequoteclose}\isanewline
%
\isadelimproof
%
\endisadelimproof
%
\isatagproof
\isacommand{by}\isamarkupfalse%
\ {\isacharparenleft}intro\ finprod{\isacharunderscore}cong{\isacharcomma}\ auto{\isacharparenright}%
\endisatagproof
{\isafoldproof}%
%
\isadelimproof
\isanewline
%
\endisadelimproof
\isanewline
\isacommand{lemmas}\isamarkupfalse%
\ {\isacharparenleft}\isakeyword{in}\ abelian{\isacharunderscore}monoid{\isacharparenright}\ finsum{\isacharunderscore}cong{\isadigit{2}}\ {\isacharequal}\ add{\isachardot}finprod{\isacharunderscore}cong{\isadigit{2}}\isanewline
\isanewline
\isacommand{lemma}\isamarkupfalse%
\ {\isacharparenleft}\isakeyword{in}\ module{\isacharparenright}\ lincomb{\isacharunderscore}cong{\isacharcolon}\isanewline
\ \ \isakeyword{fixes}\ a\ b\ A\ B\isanewline
\ \ \isakeyword{assumes}\ h{\isadigit{1}}{\isacharcolon}\ {\isachardoublequoteopen}finite\ {\isacharparenleft}A{\isacharparenright}{\isachardoublequoteclose}\ \ \isakeyword{and}\ h{\isadigit{2}}{\isacharcolon}\ {\isachardoublequoteopen}A{\isacharequal}B{\isachardoublequoteclose}\ \isakeyword{and}\ h{\isadigit{3}}{\isacharcolon}\ {\isachardoublequoteopen}A\ {\isasymsubseteq}\ carrier\ M{\isachardoublequoteclose}\ \isanewline
\ \ \ \ \isakeyword{and}\ h{\isadigit{4}}{\isacharcolon}\ {\isachardoublequoteopen}{\isasymAnd}v{\isachardot}\ v{\isasymin}A\ {\isasymLongrightarrow}\ a\ v\ {\isacharequal}\ b\ v{\isachardoublequoteclose}\ \isakeyword{and}\ h{\isadigit{5}}{\isacharcolon}\ {\isachardoublequoteopen}b{\isasymin}\ B{\isasymrightarrow}carrier\ R{\isachardoublequoteclose}\isanewline
\ \ \isakeyword{shows}\ {\isachardoublequoteopen}lincomb\ a\ A\ {\isacharequal}\ lincomb\ b\ B{\isachardoublequoteclose}\isanewline
%
\isadelimproof
%
\endisadelimproof
%
\isatagproof
\isacommand{proof}\isamarkupfalse%
\ {\isacharminus}\ \isanewline
\ \ \isacommand{from}\isamarkupfalse%
\ assms\ \isacommand{show}\isamarkupfalse%
\ {\isacharquery}thesis\isanewline
\ \ \ \ \isacommand{apply}\isamarkupfalse%
\ {\isacharparenleft}unfold\ lincomb{\isacharunderscore}def{\isacharparenright}\isanewline
\ \ \ \ \isacommand{apply}\isamarkupfalse%
\ {\isacharparenleft}drule\ Pi{\isacharunderscore}implies{\isacharunderscore}Pi{\isadigit{2}}{\isacharparenright}{\isacharplus}\isanewline
\ \ \ \ \isacommand{by}\isamarkupfalse%
\ {\isacharparenleft}simp\ cong{\isacharcolon}\ finsum{\isacharunderscore}cong{\isadigit{2}}\ add{\isacharcolon}\ h{\isadigit{2}}\ Pi{\isacharunderscore}simp\ ring{\isacharunderscore}subset{\isacharunderscore}carrier{\isacharparenright}\isanewline
\isacommand{qed}\isamarkupfalse%
%
\endisatagproof
{\isafoldproof}%
%
\isadelimproof
\isanewline
%
\endisadelimproof
\isanewline
\isacommand{lemma}\isamarkupfalse%
\ {\isacharparenleft}\isakeyword{in}\ module{\isacharparenright}\ lincomb{\isacharunderscore}union{\isacharcolon}\isanewline
\ \ \isakeyword{fixes}\ a\ A\ B\ \isanewline
\ \ \isakeyword{assumes}\ h{\isadigit{1}}{\isacharcolon}\ {\isachardoublequoteopen}finite\ {\isacharparenleft}A{\isasymunion}\ B{\isacharparenright}{\isachardoublequoteclose}\ \ \isakeyword{and}\ h{\isadigit{3}}{\isacharcolon}\ {\isachardoublequoteopen}A{\isasymunion}B\ {\isasymsubseteq}\ carrier\ M{\isachardoublequoteclose}\ \isanewline
\ \ \ \ \isakeyword{and}\ h{\isadigit{4}}{\isacharcolon}\ {\isachardoublequoteopen}A{\isasyminter}B{\isacharequal}{\isacharbraceleft}{\isacharbraceright}{\isachardoublequoteclose}\ \isakeyword{and}\ h{\isadigit{5}}{\isacharcolon}\ {\isachardoublequoteopen}a{\isasymin}{\isacharparenleft}A{\isasymunion}B{\isasymrightarrow}carrier\ R{\isacharparenright}{\isachardoublequoteclose}\isanewline
\ \ \isakeyword{shows}\ {\isachardoublequoteopen}lincomb\ a\ {\isacharparenleft}A{\isasymunion}B{\isacharparenright}\ {\isacharequal}\ lincomb\ a\ A\ {\isasymoplus}\isactrlbsub M\isactrlesub \ lincomb\ a\ B{\isachardoublequoteclose}\isanewline
%
\isadelimproof
%
\endisadelimproof
%
\isatagproof
\isacommand{proof}\isamarkupfalse%
\ {\isacharminus}\ \isanewline
\ \ \isacommand{from}\isamarkupfalse%
\ assms\ \isacommand{show}\isamarkupfalse%
\ {\isacharquery}thesis\isanewline
\ \ \ \ \isacommand{apply}\isamarkupfalse%
\ {\isacharparenleft}unfold\ lincomb{\isacharunderscore}def{\isacharparenright}\isanewline
\ \ \ \ \isacommand{apply}\isamarkupfalse%
\ {\isacharparenleft}drule\ Pi{\isacharunderscore}implies{\isacharunderscore}Pi{\isadigit{2}}{\isacharparenright}\isanewline
\ \ \ \ \isacommand{by}\isamarkupfalse%
\ {\isacharparenleft}simp\ cong{\isacharcolon}\ finsum{\isacharunderscore}cong{\isadigit{2}}\ add{\isacharcolon}\ finsum{\isacharunderscore}Un{\isacharunderscore}disjoint\ Pi{\isacharunderscore}simp\ ring{\isacharunderscore}subset{\isacharunderscore}carrier{\isacharparenright}\isanewline
\isacommand{qed}\isamarkupfalse%
%
\endisatagproof
{\isafoldproof}%
%
\isadelimproof
%
\endisadelimproof
%
\begin{isamarkuptext}%
This is useful as a simp rule sometimes, for combining linear combinations.%
\end{isamarkuptext}%
\isamarkuptrue%
\isacommand{lemma}\isamarkupfalse%
\ {\isacharparenleft}\isakeyword{in}\ module{\isacharparenright}\ lincomb{\isacharunderscore}union{\isadigit{2}}{\isacharcolon}\isanewline
\ \ \isakeyword{fixes}\ a\ b\ A\ B\ \isanewline
\ \ \isakeyword{assumes}\ h{\isadigit{1}}{\isacharcolon}\ {\isachardoublequoteopen}finite\ {\isacharparenleft}A{\isasymunion}\ B{\isacharparenright}{\isachardoublequoteclose}\ \ \isakeyword{and}\ h{\isadigit{3}}{\isacharcolon}\ {\isachardoublequoteopen}A{\isasymunion}B\ {\isasymsubseteq}\ carrier\ M{\isachardoublequoteclose}\ \isanewline
\ \ \ \ \isakeyword{and}\ h{\isadigit{4}}{\isacharcolon}\ {\isachardoublequoteopen}A{\isasyminter}B{\isacharequal}{\isacharbraceleft}{\isacharbraceright}{\isachardoublequoteclose}\ \isakeyword{and}\ h{\isadigit{5}}{\isacharcolon}\ {\isachardoublequoteopen}a{\isasymin}A{\isasymrightarrow}carrier\ R{\isachardoublequoteclose}\ \isakeyword{and}\ h{\isadigit{6}}{\isacharcolon}\ {\isachardoublequoteopen}b{\isasymin}B{\isasymrightarrow}carrier\ R{\isachardoublequoteclose}\isanewline
\ \ \isakeyword{shows}\ {\isachardoublequoteopen}lincomb\ a\ A\ {\isasymoplus}\isactrlbsub M\isactrlesub \ lincomb\ b\ B\ {\isacharequal}\ lincomb\ {\isacharparenleft}{\isasymlambda}v{\isachardot}\ if\ {\isacharparenleft}v{\isasymin}A{\isacharparenright}\ then\ a\ v\ else\ b\ v{\isacharparenright}\ {\isacharparenleft}A{\isasymunion}B{\isacharparenright}{\isachardoublequoteclose}\isanewline
\ \ \ \ {\isacharparenleft}\isakeyword{is}\ {\isachardoublequoteopen}lincomb\ a\ A\ {\isasymoplus}\isactrlbsub M\isactrlesub \ lincomb\ b\ B\ {\isacharequal}\ lincomb\ {\isacharquery}c\ {\isacharparenleft}A{\isasymunion}B{\isacharparenright}{\isachardoublequoteclose}{\isacharparenright}\isanewline
%
\isadelimproof
%
\endisadelimproof
%
\isatagproof
\isacommand{proof}\isamarkupfalse%
\ {\isacharminus}\ \isanewline
\ \ \isacommand{from}\isamarkupfalse%
\ assms\ \isacommand{show}\isamarkupfalse%
\ {\isacharquery}thesis\isanewline
\ \ \ \ \isacommand{apply}\isamarkupfalse%
\ {\isacharparenleft}unfold\ lincomb{\isacharunderscore}def{\isacharparenright}\isanewline
\ \ \ \ \isacommand{apply}\isamarkupfalse%
\ {\isacharparenleft}drule\ Pi{\isacharunderscore}implies{\isacharunderscore}Pi{\isadigit{2}}{\isacharparenright}{\isacharplus}\isanewline
\ \ \ \ \isacommand{by}\isamarkupfalse%
\ {\isacharparenleft}simp\ cong{\isacharcolon}\ finsum{\isacharunderscore}cong{\isadigit{2}}\ add{\isacharcolon}\ finsum{\isacharunderscore}Un{\isacharunderscore}disjoint\ Pi{\isacharunderscore}simp\ ring{\isacharunderscore}subset{\isacharunderscore}carrier\ disj{\isacharunderscore}if{\isacharparenright}\isanewline
\isacommand{qed}\isamarkupfalse%
%
\endisatagproof
{\isafoldproof}%
%
\isadelimproof
\isanewline
%
\endisadelimproof
\isanewline
\isacommand{lemma}\isamarkupfalse%
\ {\isacharparenleft}\isakeyword{in}\ module{\isacharparenright}\ lincomb{\isacharunderscore}del{\isadigit{2}}{\isacharcolon}\isanewline
\ \ \isakeyword{fixes}\ S\ a\ v\isanewline
\ \ \isakeyword{assumes}\ h{\isadigit{1}}{\isacharcolon}\ {\isachardoublequoteopen}finite\ S{\isachardoublequoteclose}\ \isakeyword{and}\ h{\isadigit{2}}{\isacharcolon}\ {\isachardoublequoteopen}S{\isasymsubseteq}\ carrier\ M{\isachardoublequoteclose}\ \isakeyword{and}\ h{\isadigit{3}}{\isacharcolon}\ {\isachardoublequoteopen}a{\isasymin}{\isacharparenleft}S{\isasymrightarrow}carrier\ R{\isacharparenright}{\isachardoublequoteclose}\ \isakeyword{and}\ h{\isadigit{4}}{\isacharcolon}{\isachardoublequoteopen}v{\isasymin}S{\isachardoublequoteclose}\isanewline
\ \ \isakeyword{shows}\ {\isachardoublequoteopen}lincomb\ a\ S\ {\isacharequal}\ {\isacharparenleft}{\isacharparenleft}a\ v{\isacharparenright}\ {\isasymodot}\isactrlbsub M\isactrlesub \ v{\isacharparenright}\ {\isasymoplus}\isactrlbsub M\isactrlesub \ lincomb\ a\ {\isacharparenleft}S{\isacharminus}{\isacharbraceleft}v{\isacharbraceright}{\isacharparenright}{\isachardoublequoteclose}\isanewline
%
\isadelimproof
%
\endisadelimproof
%
\isatagproof
\isacommand{proof}\isamarkupfalse%
\ {\isacharminus}\ \isanewline
\ \ \isacommand{from}\isamarkupfalse%
\ h{\isadigit{4}}\ \isacommand{have}\isamarkupfalse%
\ {\isadigit{1}}{\isacharcolon}\ {\isachardoublequoteopen}S{\isacharequal}{\isacharbraceleft}v{\isacharbraceright}{\isasymunion}{\isacharparenleft}S{\isacharminus}{\isacharbraceleft}v{\isacharbraceright}{\isacharparenright}{\isachardoublequoteclose}\ \isacommand{by}\isamarkupfalse%
\ {\isacharparenleft}metis\ insert{\isacharunderscore}Diff\ insert{\isacharunderscore}is{\isacharunderscore}Un{\isacharparenright}\ \isanewline
\ \ \isacommand{from}\isamarkupfalse%
\ assms\ \isacommand{show}\isamarkupfalse%
\ {\isacharquery}thesis\isanewline
\ \ \ \ \isacommand{apply}\isamarkupfalse%
\ {\isacharparenleft}subst\ {\isadigit{1}}{\isacharparenright}\isanewline
\ \ \ \ \isacommand{apply}\isamarkupfalse%
\ {\isacharparenleft}subst\ lincomb{\isacharunderscore}union{\isacharcomma}\ auto{\isacharparenright}\isanewline
\ \ \ \ \isacommand{by}\isamarkupfalse%
\ {\isacharparenleft}unfold\ lincomb{\isacharunderscore}def{\isacharcomma}\ auto\ simp\ add{\isacharcolon}\ coeff{\isacharunderscore}in{\isacharunderscore}ring{\isacharparenright}\isanewline
\isacommand{qed}\isamarkupfalse%
%
\endisatagproof
{\isafoldproof}%
%
\isadelimproof
\isanewline
%
\endisadelimproof
\isanewline
\isanewline
\isacommand{lemma}\isamarkupfalse%
\ {\isacharparenleft}\isakeyword{in}\ module{\isacharparenright}\ lincomb{\isacharunderscore}insert{\isacharcolon}\isanewline
\ \ \isakeyword{fixes}\ S\ a\ v\isanewline
\ \ \isakeyword{assumes}\ h{\isadigit{1}}{\isacharcolon}\ {\isachardoublequoteopen}finite\ S{\isachardoublequoteclose}\ \isakeyword{and}\ h{\isadigit{2}}{\isacharcolon}\ {\isachardoublequoteopen}S{\isasymsubseteq}\ carrier\ M{\isachardoublequoteclose}\ \isakeyword{and}\ h{\isadigit{3}}{\isacharcolon}\ {\isachardoublequoteopen}a{\isasymin}{\isacharparenleft}S{\isasymunion}{\isacharbraceleft}v{\isacharbraceright}{\isasymrightarrow}carrier\ R{\isacharparenright}{\isachardoublequoteclose}\ \isakeyword{and}\ h{\isadigit{4}}{\isacharcolon}{\isachardoublequoteopen}v{\isasymnotin}S{\isachardoublequoteclose}\ \isakeyword{and}\ \isanewline
h{\isadigit{5}}{\isacharcolon}{\isachardoublequoteopen}v{\isasymin}\ carrier\ M{\isachardoublequoteclose}\ \ \isanewline
\ \ \isakeyword{shows}\ {\isachardoublequoteopen}lincomb\ a\ {\isacharparenleft}S{\isasymunion}{\isacharbraceleft}v{\isacharbraceright}{\isacharparenright}\ {\isacharequal}\ {\isacharparenleft}{\isacharparenleft}a\ v{\isacharparenright}\ {\isasymodot}\isactrlbsub M\isactrlesub \ v{\isacharparenright}\ {\isasymoplus}\isactrlbsub M\isactrlesub \ lincomb\ a\ S{\isachardoublequoteclose}\isanewline
%
\isadelimproof
%
\endisadelimproof
%
\isatagproof
\isacommand{proof}\isamarkupfalse%
\ {\isacharminus}\ \isanewline
\ \ \isacommand{have}\isamarkupfalse%
\ {\isadigit{1}}{\isacharcolon}\ {\isachardoublequoteopen}S{\isasymunion}{\isacharbraceleft}v{\isacharbraceright}{\isacharequal}{\isacharbraceleft}v{\isacharbraceright}{\isasymunion}S{\isachardoublequoteclose}\ \isacommand{by}\isamarkupfalse%
\ auto\isanewline
\ \ \isacommand{from}\isamarkupfalse%
\ assms\ \isacommand{show}\isamarkupfalse%
\ {\isacharquery}thesis\isanewline
\ \ \ \ \isacommand{apply}\isamarkupfalse%
\ {\isacharparenleft}subst\ {\isadigit{1}}{\isacharparenright}\isanewline
\ \ \ \ \isacommand{apply}\isamarkupfalse%
\ {\isacharparenleft}unfold\ lincomb{\isacharunderscore}def{\isacharparenright}\isanewline
\ \ \ \ \isacommand{apply}\isamarkupfalse%
\ {\isacharparenleft}drule\ Pi{\isacharunderscore}implies{\isacharunderscore}Pi{\isadigit{2}}{\isacharparenright}{\isacharplus}\isanewline
\ \ \ \ \isacommand{by}\isamarkupfalse%
\ {\isacharparenleft}simp\ cong{\isacharcolon}\ finsum{\isacharunderscore}cong{\isadigit{2}}\ add{\isacharcolon}\ finsum{\isacharunderscore}Un{\isacharunderscore}disjoint\ Pi{\isacharunderscore}simp\ ring{\isacharunderscore}subset{\isacharunderscore}carrier\ disj{\isacharunderscore}if{\isacharparenright}\isanewline
\isacommand{qed}\isamarkupfalse%
%
\endisatagproof
{\isafoldproof}%
%
\isadelimproof
\isanewline
%
\endisadelimproof
\isanewline
\isacommand{lemma}\isamarkupfalse%
\ {\isacharparenleft}\isakeyword{in}\ module{\isacharparenright}\ lincomb{\isacharunderscore}elim{\isacharunderscore}if\ {\isacharbrackleft}simp{\isacharbrackright}{\isacharcolon}\isanewline
\ \ \isakeyword{fixes}\ b\ c\ S\isanewline
\ \ \isakeyword{assumes}\ h{\isadigit{0}}{\isacharcolon}{\isachardoublequoteopen}finite\ S{\isachardoublequoteclose}\ \isakeyword{and}\ h{\isadigit{1}}{\isacharcolon}\ {\isachardoublequoteopen}S\ {\isasymsubseteq}\ carrier\ M{\isachardoublequoteclose}\ \isakeyword{and}\ h{\isadigit{2}}{\isacharcolon}\ {\isachardoublequoteopen}{\isasymAnd}v{\isachardot}\ v{\isasymin}S{\isasymLongrightarrow}\ {\isasymnot}P\ v{\isachardoublequoteclose}\ \isakeyword{and}\ h{\isadigit{3}}{\isacharcolon}\ {\isachardoublequoteopen}c{\isasymin}S{\isasymrightarrow}carrier\ R{\isachardoublequoteclose}\isanewline
\ \ \isakeyword{shows}\ {\isachardoublequoteopen}lincomb\ {\isacharparenleft}{\isasymlambda}w{\isachardot}\ if\ P\ w\ then\ b\ w\ else\ c\ w{\isacharparenright}\ S\ {\isacharequal}\ lincomb\ c\ S{\isachardoublequoteclose}\isanewline
%
\isadelimproof
%
\endisadelimproof
%
\isatagproof
\isacommand{proof}\isamarkupfalse%
\ {\isacharminus}\isanewline
\ \ \isacommand{from}\isamarkupfalse%
\ assms\ \isacommand{show}\isamarkupfalse%
\ {\isacharquery}thesis\isanewline
\ \ \ \ \isacommand{apply}\isamarkupfalse%
\ {\isacharparenleft}unfold\ lincomb{\isacharunderscore}def{\isacharparenright}\isanewline
\ \ \ \ \isacommand{apply}\isamarkupfalse%
\ {\isacharparenleft}drule\ Pi{\isacharunderscore}implies{\isacharunderscore}Pi{\isadigit{2}}{\isacharparenright}{\isacharplus}\isanewline
\ \ \ \ \isacommand{by}\isamarkupfalse%
\ {\isacharparenleft}simp\ cong{\isacharcolon}\ finsum{\isacharunderscore}cong{\isadigit{2}}\ add{\isacharcolon}\ finsum{\isacharunderscore}Un{\isacharunderscore}disjoint\ Pi{\isacharunderscore}simp\ ring{\isacharunderscore}subset{\isacharunderscore}carrier\ disj{\isacharunderscore}if{\isacharparenright}\isanewline
\isacommand{qed}\isamarkupfalse%
%
\endisatagproof
{\isafoldproof}%
%
\isadelimproof
\isanewline
%
\endisadelimproof
\isanewline
\isacommand{lemma}\isamarkupfalse%
\ {\isacharparenleft}\isakeyword{in}\ module{\isacharparenright}\ lincomb{\isacharunderscore}smult{\isacharcolon}\isanewline
\ \ \isakeyword{fixes}\ A\ c\isanewline
\ \ \isakeyword{assumes}\ h{\isadigit{1}}{\isacharcolon}\ {\isachardoublequoteopen}finite\ A{\isachardoublequoteclose}\ \isakeyword{and}\ h{\isadigit{2}}{\isacharcolon}\ {\isachardoublequoteopen}A{\isasymsubseteq}carrier\ M{\isachardoublequoteclose}\ \ \isakeyword{and}\ h{\isadigit{3}}{\isacharcolon}\ {\isachardoublequoteopen}a{\isasymin}A{\isasymrightarrow}carrier\ R{\isachardoublequoteclose}\ \isakeyword{and}\ h{\isadigit{4}}{\isacharcolon}\ {\isachardoublequoteopen}c{\isasymin}carrier\ R{\isachardoublequoteclose}\isanewline
\ \ \isakeyword{shows}\ {\isachardoublequoteopen}lincomb\ {\isacharparenleft}{\isasymlambda}w{\isachardot}\ c{\isasymotimes}\isactrlbsub R\isactrlesub \ a\ w{\isacharparenright}\ A\ {\isacharequal}\ c{\isasymodot}\isactrlbsub M\isactrlesub \ {\isacharparenleft}lincomb\ a\ A{\isacharparenright}{\isachardoublequoteclose}\isanewline
%
\isadelimproof
%
\endisadelimproof
%
\isatagproof
\isacommand{proof}\isamarkupfalse%
\ {\isacharminus}\ \isanewline
\ \ \isacommand{from}\isamarkupfalse%
\ assms\ \isacommand{show}\isamarkupfalse%
\ {\isacharquery}thesis\isanewline
\ \ \ \ \isacommand{apply}\isamarkupfalse%
\ {\isacharparenleft}unfold\ lincomb{\isacharunderscore}def{\isacharparenright}\isanewline
\ \ \ \ \isacommand{apply}\isamarkupfalse%
\ {\isacharparenleft}drule\ Pi{\isacharunderscore}implies{\isacharunderscore}Pi{\isadigit{2}}{\isacharparenright}{\isacharplus}\isanewline
\ \ \ \ \isacommand{by}\isamarkupfalse%
\ {\isacharparenleft}simp\ cong{\isacharcolon}\ finsum{\isacharunderscore}cong{\isadigit{2}}\ add{\isacharcolon}\ finsum{\isacharunderscore}Un{\isacharunderscore}disjoint\ finsum{\isacharunderscore}smult\ Pi{\isacharunderscore}simp\ ring{\isacharunderscore}subset{\isacharunderscore}carrier\ disj{\isacharunderscore}if\isanewline
smult{\isacharunderscore}assoc{\isadigit{1}}\ coeff{\isacharunderscore}in{\isacharunderscore}ring{\isacharparenright}\isanewline
\isacommand{qed}\isamarkupfalse%
%
\endisatagproof
{\isafoldproof}%
%
\isadelimproof
%
\endisadelimproof
%
\isamarkupsubsection{Linear dependence and independence.%
}
\isamarkuptrue%
%
\begin{isamarkuptext}%
A set $S$ in a module/vectorspace is linearly dependent if there is a finite set $A \subseteq S$
 and coefficients $(a_v)_{v\in A}$ such that $sum_{v\in A} a_vv=0$ and for some $v$, $a_v\neq 0$.%
\end{isamarkuptext}%
\isamarkuptrue%
\isacommand{definition}\isamarkupfalse%
\ {\isacharparenleft}\isakeyword{in}\ module{\isacharparenright}\ lin{\isacharunderscore}dep\ \isakeyword{where}\isanewline
\ \ {\isachardoublequoteopen}lin{\isacharunderscore}dep\ S\ {\isacharequal}\ {\isacharparenleft}{\isasymexists}A\ a\ v{\isachardot}\ {\isacharparenleft}finite\ A\ {\isasymand}\ A{\isasymsubseteq}S\ {\isasymand}\ {\isacharparenleft}a{\isasymin}\ {\isacharparenleft}A{\isasymrightarrow}carrier\ R{\isacharparenright}{\isacharparenright}\ {\isasymand}\ {\isacharparenleft}lincomb\ a\ A\ {\isacharequal}\ {\isasymzero}\isactrlbsub M\isactrlesub {\isacharparenright}\ {\isasymand}\ {\isacharparenleft}v{\isasymin}A{\isacharparenright}\ {\isasymand}\ {\isacharparenleft}a\ v{\isasymnoteq}\ {\isasymzero}\isactrlbsub R\isactrlesub {\isacharparenright}{\isacharparenright}{\isacharparenright}{\isachardoublequoteclose}\isanewline
\ \ \isanewline
\isanewline
\isacommand{abbreviation}\isamarkupfalse%
\ {\isacharparenleft}\isakeyword{in}\ module{\isacharparenright}\ lin{\isacharunderscore}indpt{\isacharcolon}{\isacharcolon}{\isachardoublequoteopen}{\isacharprime}c\ set\ {\isasymRightarrow}\ bool{\isachardoublequoteclose}\isanewline
\ \ \isakeyword{where}\ {\isachardoublequoteopen}lin{\isacharunderscore}indpt\ S\ {\isasymequiv}\ {\isasymnot}lin{\isacharunderscore}dep\ S{\isachardoublequoteclose}%
\begin{isamarkuptext}%
In the finite case, we can take $A=S$. This may be more convenient (e.g., when adding two
linear combinations.%
\end{isamarkuptext}%
\isamarkuptrue%
\isacommand{lemma}\isamarkupfalse%
\ {\isacharparenleft}\isakeyword{in}\ module{\isacharparenright}\ finite{\isacharunderscore}lin{\isacharunderscore}dep{\isacharcolon}\ \isanewline
\ \ \isakeyword{fixes}\ S\isanewline
\ \ \isakeyword{assumes}\ finS{\isacharcolon}{\isachardoublequoteopen}finite\ S{\isachardoublequoteclose}\ \isakeyword{and}\ ld{\isacharcolon}\ {\isachardoublequoteopen}lin{\isacharunderscore}dep\ S{\isachardoublequoteclose}\ \isakeyword{and}\ inC{\isacharcolon}\ {\isachardoublequoteopen}S{\isasymsubseteq}carrier\ M{\isachardoublequoteclose}\isanewline
\ \ \isakeyword{shows}\ {\isachardoublequoteopen}{\isasymexists}a\ v{\isachardot}\ {\isacharparenleft}a{\isasymin}\ {\isacharparenleft}S{\isasymrightarrow}carrier\ R{\isacharparenright}{\isacharparenright}\ {\isasymand}\ {\isacharparenleft}lincomb\ a\ S\ {\isacharequal}\ {\isasymzero}\isactrlbsub M\isactrlesub {\isacharparenright}\ {\isasymand}\ {\isacharparenleft}v{\isasymin}S{\isacharparenright}\ {\isasymand}\ {\isacharparenleft}a\ v{\isasymnoteq}\ {\isasymzero}\isactrlbsub R\isactrlesub {\isacharparenright}{\isachardoublequoteclose}\isanewline
%
\isadelimproof
%
\endisadelimproof
%
\isatagproof
\isacommand{proof}\isamarkupfalse%
\ {\isacharminus}\ \isanewline
\ \ \isacommand{from}\isamarkupfalse%
\ ld\ \isacommand{obtain}\isamarkupfalse%
\ A\ a\ v\ \isakeyword{where}\ A{\isacharcolon}\ {\isachardoublequoteopen}{\isacharparenleft}A{\isasymsubseteq}S\ {\isasymand}\ {\isacharparenleft}a{\isasymin}\ {\isacharparenleft}A{\isasymrightarrow}carrier\ R{\isacharparenright}{\isacharparenright}\ {\isasymand}\ {\isacharparenleft}lincomb\ a\ A\ {\isacharequal}\ {\isasymzero}\isactrlbsub M\isactrlesub {\isacharparenright}\ {\isasymand}\ {\isacharparenleft}v{\isasymin}A{\isacharparenright}\ {\isasymand}\ {\isacharparenleft}a\ v{\isasymnoteq}\ {\isasymzero}\isactrlbsub R\isactrlesub {\isacharparenright}{\isacharparenright}{\isachardoublequoteclose}\ \isanewline
\ \ \ \ \isacommand{by}\isamarkupfalse%
\ {\isacharparenleft}unfold\ lin{\isacharunderscore}dep{\isacharunderscore}def{\isacharcomma}\ auto{\isacharparenright}\isanewline
\ \ \isacommand{let}\isamarkupfalse%
\ {\isacharquery}b{\isacharequal}{\isachardoublequoteopen}{\isasymlambda}w{\isachardot}\ if\ w{\isasymin}A\ then\ a\ w\ else\ {\isasymzero}\isactrlbsub R\isactrlesub {\isachardoublequoteclose}\ \isanewline
\ \ \isacommand{from}\isamarkupfalse%
\ finS\ inC\ A\ \isacommand{have}\isamarkupfalse%
\ if{\isacharunderscore}in{\isacharcolon}\ {\isachardoublequoteopen}{\isacharparenleft}{\isasymOplus}\isactrlbsub M\isactrlesub v{\isasymin}S{\isachardot}\ {\isacharparenleft}if\ v\ {\isasymin}\ A\ then\ a\ v\ else\ {\isasymzero}{\isacharparenright}\ {\isasymodot}\isactrlbsub M\isactrlesub \ v{\isacharparenright}\ {\isacharequal}\ {\isacharparenleft}{\isasymOplus}\isactrlbsub M\isactrlesub v{\isasymin}S{\isachardot}\ {\isacharparenleft}if\ v\ {\isasymin}\ A\ then\ a\ v\ {\isasymodot}\isactrlbsub M\isactrlesub \ v\ else\ {\isasymzero}\isactrlbsub M\isactrlesub {\isacharparenright}{\isacharparenright}{\isachardoublequoteclose}\isanewline
\ \ \ \ \isacommand{apply}\isamarkupfalse%
\ auto\isanewline
\ \ \ \ \ \ \isacommand{apply}\isamarkupfalse%
\ {\isacharparenleft}intro\ finsum{\isacharunderscore}cong{\isacharprime}{\isacharparenright}\ \isanewline
\ \ \ \ \isacommand{by}\isamarkupfalse%
\ {\isacharparenleft}auto\ simp\ add{\isacharcolon}\ coeff{\isacharunderscore}in{\isacharunderscore}ring{\isacharparenright}\ \ \isanewline
\ \ \isacommand{from}\isamarkupfalse%
\ finS\ inC\ A\ \isacommand{have}\isamarkupfalse%
\ b{\isacharcolon}\ {\isachardoublequoteopen}lincomb\ {\isacharquery}b\ S\ {\isacharequal}\ {\isasymzero}\isactrlbsub M\isactrlesub {\isachardoublequoteclose}\isanewline
\ \ \ \ \isacommand{apply}\isamarkupfalse%
\ {\isacharparenleft}unfold\ lincomb{\isacharunderscore}def{\isacharparenright}\isanewline
\ \ \ \ \isacommand{apply}\isamarkupfalse%
\ {\isacharparenleft}subst\ if{\isacharunderscore}in{\isacharparenright}\isanewline
\ \ \ \ \isacommand{by}\isamarkupfalse%
\ {\isacharparenleft}subst\ extend{\isacharunderscore}sum{\isacharcomma}\ auto{\isacharparenright}\isanewline
\ \ \isacommand{from}\isamarkupfalse%
\ A\ b\ \isacommand{show}\isamarkupfalse%
\ {\isacharquery}thesis\ \isanewline
\ \ \ \ \isacommand{apply}\isamarkupfalse%
\ {\isacharparenleft}rule{\isacharunderscore}tac\ x{\isacharequal}{\isachardoublequoteopen}{\isacharquery}b{\isachardoublequoteclose}\ \isakeyword{in}\ exI{\isacharparenright}\isanewline
\ \ \ \ \isacommand{apply}\isamarkupfalse%
\ {\isacharparenleft}rule{\isacharunderscore}tac\ x{\isacharequal}{\isachardoublequoteopen}v{\isachardoublequoteclose}\ \isakeyword{in}\ exI{\isacharparenright}\isanewline
\ \ \ \ \isacommand{by}\isamarkupfalse%
\ auto\isanewline
\isacommand{qed}\isamarkupfalse%
%
\endisatagproof
{\isafoldproof}%
%
\isadelimproof
%
\endisadelimproof
%
\begin{isamarkuptext}%
Criteria of linear dependency in a easy format to apply: apply (rule lin-dep-crit)%
\end{isamarkuptext}%
\isamarkuptrue%
\isacommand{lemma}\isamarkupfalse%
\ {\isacharparenleft}\isakeyword{in}\ module{\isacharparenright}\ lin{\isacharunderscore}dep{\isacharunderscore}crit{\isacharcolon}\ \isanewline
\ \ \isakeyword{fixes}\ A\ S\ a\ v\isanewline
\ \ \isakeyword{assumes}\ fin{\isacharcolon}\ {\isachardoublequoteopen}finite\ A{\isachardoublequoteclose}\ \isakeyword{and}\ subset{\isacharcolon}\ {\isachardoublequoteopen}A{\isasymsubseteq}S{\isachardoublequoteclose}\ \isakeyword{and}\ h{\isadigit{1}}{\isacharcolon}\ {\isachardoublequoteopen}{\isacharparenleft}a{\isasymin}\ {\isacharparenleft}A{\isasymrightarrow}carrier\ R{\isacharparenright}{\isacharparenright}{\isachardoublequoteclose}\ \isakeyword{and}\ h{\isadigit{2}}{\isacharcolon}\ {\isachardoublequoteopen}v{\isasymin}\ A{\isachardoublequoteclose}\ \isanewline
\ \ \ \ \isakeyword{and}\ h{\isadigit{3}}{\isacharcolon}{\isachardoublequoteopen}a\ v{\isasymnoteq}\ {\isasymzero}\isactrlbsub R\isactrlesub {\isachardoublequoteclose}\ \isakeyword{and}\ h{\isadigit{4}}{\isacharcolon}\ {\isachardoublequoteopen}{\isacharparenleft}lincomb\ a\ A\ {\isacharequal}\ {\isasymzero}\isactrlbsub M\isactrlesub {\isacharparenright}{\isachardoublequoteclose}\isanewline
\ \ \isakeyword{shows}\ {\isachardoublequoteopen}lin{\isacharunderscore}dep\ S{\isachardoublequoteclose}\isanewline
%
\isadelimproof
%
\endisadelimproof
%
\isatagproof
\isacommand{proof}\isamarkupfalse%
\ {\isacharminus}\ \isanewline
\ \ \isacommand{from}\isamarkupfalse%
\ assms\ \isacommand{show}\isamarkupfalse%
\ {\isacharquery}thesis\isanewline
\ \ \ \ \isacommand{by}\isamarkupfalse%
\ {\isacharparenleft}unfold\ lin{\isacharunderscore}dep{\isacharunderscore}def{\isacharcomma}\ auto{\isacharparenright}\ \isanewline
\isacommand{qed}\isamarkupfalse%
%
\endisatagproof
{\isafoldproof}%
%
\isadelimproof
%
\endisadelimproof
%
\begin{isamarkuptext}%
If $\sum_{v\in A} a_vv=0$ implies $a_v=0$ for all $v\in S$, then $A$ is linearly independent.%
\end{isamarkuptext}%
\isamarkuptrue%
\isacommand{lemma}\isamarkupfalse%
\ {\isacharparenleft}\isakeyword{in}\ module{\isacharparenright}\ finite{\isacharunderscore}lin{\isacharunderscore}indpt{\isadigit{2}}{\isacharcolon}\isanewline
\ \ \isakeyword{fixes}\ A\ \isanewline
\ \ \isakeyword{assumes}\ A{\isacharunderscore}fin{\isacharcolon}\ {\isachardoublequoteopen}finite\ A{\isachardoublequoteclose}\ \isakeyword{and}\ AinC{\isacharcolon}\ {\isachardoublequoteopen}A{\isasymsubseteq}carrier\ M{\isachardoublequoteclose}\ \isakeyword{and}\isanewline
\ \ \ \ lc{\isadigit{0}}{\isacharcolon}\ {\isachardoublequoteopen}{\isasymAnd}a{\isachardot}\ a{\isasymin}\ {\isacharparenleft}A{\isasymrightarrow}carrier\ R{\isacharparenright}\ {\isasymLongrightarrow}\ {\isacharparenleft}lincomb\ a\ A\ {\isacharequal}\ {\isasymzero}\isactrlbsub M\isactrlesub {\isacharparenright}\ {\isasymLongrightarrow}\ {\isacharparenleft}{\isasymforall}\ v{\isasymin}A{\isachardot}\ a\ v{\isacharequal}\ {\isasymzero}\isactrlbsub R\isactrlesub {\isacharparenright}{\isachardoublequoteclose}\isanewline
\ \ \isakeyword{shows}\ {\isachardoublequoteopen}lin{\isacharunderscore}indpt\ A{\isachardoublequoteclose}\isanewline
%
\isadelimproof
%
\endisadelimproof
%
\isatagproof
\isacommand{proof}\isamarkupfalse%
\ {\isacharparenleft}rule\ ccontr{\isacharparenright}\isanewline
\ \ \isacommand{assume}\isamarkupfalse%
\ {\isachardoublequoteopen}{\isasymnot}lin{\isacharunderscore}indpt\ A{\isachardoublequoteclose}\isanewline
\ \ \isacommand{from}\isamarkupfalse%
\ A{\isacharunderscore}fin\ AinC\ this\ \isacommand{obtain}\isamarkupfalse%
\ a\ v\ \isakeyword{where}\ av{\isacharcolon}\isanewline
\ \ \ \ {\isachardoublequoteopen}{\isacharparenleft}a{\isasymin}\ {\isacharparenleft}A{\isasymrightarrow}carrier\ R{\isacharparenright}{\isacharparenright}\ {\isasymand}\ {\isacharparenleft}lincomb\ a\ A\ {\isacharequal}\ {\isasymzero}\isactrlbsub M\isactrlesub {\isacharparenright}\ {\isasymand}\ {\isacharparenleft}v{\isasymin}A{\isacharparenright}\ {\isasymand}\ {\isacharparenleft}a\ v{\isasymnoteq}\ {\isasymzero}\isactrlbsub R\isactrlesub {\isacharparenright}{\isachardoublequoteclose}\isanewline
\ \ \ \ \isacommand{by}\isamarkupfalse%
\ {\isacharparenleft}metis\ finite{\isacharunderscore}lin{\isacharunderscore}dep{\isacharparenright}\isanewline
\ \ \isacommand{from}\isamarkupfalse%
\ av\ lc{\isadigit{0}}\ \isacommand{show}\isamarkupfalse%
\ False\ \isacommand{by}\isamarkupfalse%
\ auto\isanewline
\isacommand{qed}\isamarkupfalse%
%
\endisatagproof
{\isafoldproof}%
%
\isadelimproof
%
\endisadelimproof
%
\begin{isamarkuptext}%
Any set containing 0 is linearly dependent.%
\end{isamarkuptext}%
\isamarkuptrue%
\isacommand{lemma}\isamarkupfalse%
\ {\isacharparenleft}\isakeyword{in}\ module{\isacharparenright}\ zero{\isacharunderscore}lin{\isacharunderscore}dep{\isacharcolon}\ \isanewline
\ \ \isakeyword{assumes}\ {\isadigit{0}}{\isacharcolon}\ {\isachardoublequoteopen}{\isasymzero}\isactrlbsub M\isactrlesub \ {\isasymin}\ S{\isachardoublequoteclose}\ \isakeyword{and}\ nonzero{\isacharcolon}\ {\isachardoublequoteopen}carrier\ R\ {\isasymnoteq}\ {\isacharbraceleft}{\isasymzero}\isactrlbsub R\isactrlesub {\isacharbraceright}{\isachardoublequoteclose}\isanewline
\ \ \isakeyword{shows}\ {\isachardoublequoteopen}lin{\isacharunderscore}dep\ S{\isachardoublequoteclose}\isanewline
%
\isadelimproof
%
\endisadelimproof
%
\isatagproof
\isacommand{proof}\isamarkupfalse%
\ {\isacharminus}\ \isanewline
\ \ \isacommand{from}\isamarkupfalse%
\ nonzero\ \isacommand{have}\isamarkupfalse%
\ zero{\isacharunderscore}not{\isacharunderscore}one{\isacharcolon}\ {\isachardoublequoteopen}{\isasymzero}\isactrlbsub R\isactrlesub \ {\isasymnoteq}\ {\isasymone}\isactrlbsub R\isactrlesub {\isachardoublequoteclose}\ \isacommand{by}\isamarkupfalse%
\ {\isacharparenleft}rule\ nontrivial{\isacharunderscore}ring{\isacharparenright}\isanewline
\ \ \isacommand{from}\isamarkupfalse%
\ {\isadigit{0}}\ zero{\isacharunderscore}not{\isacharunderscore}one\ \isacommand{show}\isamarkupfalse%
\ {\isacharquery}thesis\isanewline
\ \ \ \ \isacommand{apply}\isamarkupfalse%
\ {\isacharparenleft}unfold\ lin{\isacharunderscore}dep{\isacharunderscore}def{\isacharparenright}\isanewline
\ \ \ \ \isacommand{apply}\isamarkupfalse%
\ {\isacharparenleft}rule{\isacharunderscore}tac\ x{\isacharequal}{\isachardoublequoteopen}{\isacharbraceleft}{\isasymzero}\isactrlbsub M\isactrlesub {\isacharbraceright}{\isachardoublequoteclose}\ \isakeyword{in}\ exI{\isacharparenright}\isanewline
\ \ \ \ \isacommand{apply}\isamarkupfalse%
\ {\isacharparenleft}rule{\isacharunderscore}tac\ x{\isacharequal}{\isachardoublequoteopen}{\isacharparenleft}{\isasymlambda}v{\isachardot}\ {\isasymone}\isactrlbsub R\isactrlesub {\isacharparenright}{\isachardoublequoteclose}\ \isakeyword{in}\ exI{\isacharparenright}\isanewline
\ \ \ \ \isacommand{apply}\isamarkupfalse%
\ {\isacharparenleft}rule{\isacharunderscore}tac\ x{\isacharequal}{\isachardoublequoteopen}{\isasymzero}\isactrlbsub M\isactrlesub {\isachardoublequoteclose}\ \isakeyword{in}\ exI{\isacharparenright}\isanewline
\ \ \ \ \isacommand{by}\isamarkupfalse%
\ {\isacharparenleft}unfold\ lincomb{\isacharunderscore}def{\isacharcomma}\ auto{\isacharparenright}\isanewline
\isacommand{qed}\isamarkupfalse%
%
\endisatagproof
{\isafoldproof}%
%
\isadelimproof
\isanewline
%
\endisadelimproof
\isanewline
\isacommand{lemma}\isamarkupfalse%
\ {\isacharparenleft}\isakeyword{in}\ module{\isacharparenright}\ zero{\isacharunderscore}nin{\isacharunderscore}lin{\isacharunderscore}indpt{\isacharcolon}\ \isanewline
\ \ \isakeyword{assumes}\ h{\isadigit{2}}{\isacharcolon}\ {\isachardoublequoteopen}S{\isasymsubseteq}carrier\ M{\isachardoublequoteclose}\ \isakeyword{and}\ li{\isacharcolon}\ {\isachardoublequoteopen}{\isasymnot}{\isacharparenleft}lin{\isacharunderscore}dep\ S{\isacharparenright}{\isachardoublequoteclose}\ \isakeyword{and}\ nonzero{\isacharcolon}\ {\isachardoublequoteopen}carrier\ R\ {\isasymnoteq}\ {\isacharbraceleft}{\isasymzero}\isactrlbsub R\isactrlesub {\isacharbraceright}{\isachardoublequoteclose}\isanewline
\ \ \isakeyword{shows}\ {\isachardoublequoteopen}{\isasymzero}\isactrlbsub M\isactrlesub \ {\isasymnotin}\ S{\isachardoublequoteclose}\isanewline
%
\isadelimproof
%
\endisadelimproof
%
\isatagproof
\isacommand{proof}\isamarkupfalse%
\ {\isacharparenleft}rule\ ccontr{\isacharparenright}\isanewline
\ \ \isacommand{assume}\isamarkupfalse%
\ a{\isadigit{1}}{\isacharcolon}\ {\isachardoublequoteopen}{\isasymnot}{\isacharparenleft}{\isasymzero}\isactrlbsub M\isactrlesub \ {\isasymnotin}\ S{\isacharparenright}{\isachardoublequoteclose}\isanewline
\ \ \isacommand{from}\isamarkupfalse%
\ a{\isadigit{1}}\ \isacommand{have}\isamarkupfalse%
\ a{\isadigit{2}}{\isacharcolon}\ {\isachardoublequoteopen}{\isasymzero}\isactrlbsub M\isactrlesub \ {\isasymin}\ S{\isachardoublequoteclose}\ \isacommand{by}\isamarkupfalse%
\ auto\isanewline
\ \ \isacommand{from}\isamarkupfalse%
\ a{\isadigit{2}}\ nonzero\ \isacommand{have}\isamarkupfalse%
\ ld{\isacharcolon}\ {\isachardoublequoteopen}lin{\isacharunderscore}dep\ S{\isachardoublequoteclose}\ \isacommand{by}\isamarkupfalse%
\ {\isacharparenleft}rule\ zero{\isacharunderscore}lin{\isacharunderscore}dep{\isacharparenright}\isanewline
\ \ \isacommand{from}\isamarkupfalse%
\ li\ ld\ \isacommand{show}\isamarkupfalse%
\ False\ \isacommand{by}\isamarkupfalse%
\ auto\isanewline
\isacommand{qed}\isamarkupfalse%
%
\endisatagproof
{\isafoldproof}%
%
\isadelimproof
%
\endisadelimproof
%
\begin{isamarkuptext}%
The \isa{span} of $S$ is the set of linear combinations with $A \subseteq S$.%
\end{isamarkuptext}%
\isamarkuptrue%
\isacommand{definition}\isamarkupfalse%
\ {\isacharparenleft}\isakeyword{in}\ module{\isacharparenright}\ span{\isacharcolon}{\isacharcolon}{\isachardoublequoteopen}{\isacharprime}c\ set\ {\isasymRightarrow}{\isacharprime}c\ set{\isachardoublequoteclose}\ \isanewline
\ \ \isakeyword{where}\ {\isachardoublequoteopen}span\ S\ {\isacharequal}\ {\isacharbraceleft}lincomb\ a\ A\ {\isacharbar}\ a\ A{\isachardot}\ finite\ A\ {\isasymand}\ A{\isasymsubseteq}S\ {\isasymand}\ a{\isasymin}\ {\isacharparenleft}A{\isasymrightarrow}carrier\ R{\isacharparenright}{\isacharbraceright}{\isachardoublequoteclose}%
\begin{isamarkuptext}%
The \isa{span} interpreted as a module or vectorspace.%
\end{isamarkuptext}%
\isamarkuptrue%
\isacommand{abbreviation}\isamarkupfalse%
\ {\isacharparenleft}\isakeyword{in}\ module{\isacharparenright}\ span{\isacharunderscore}vs{\isacharcolon}{\isacharcolon}{\isachardoublequoteopen}{\isacharprime}c\ set\ {\isasymRightarrow}\ {\isacharparenleft}{\isacharprime}a{\isacharcomma}{\isacharprime}c{\isacharcomma}{\isacharprime}d{\isacharparenright}\ module{\isacharunderscore}scheme{\isachardoublequoteclose}\ \isanewline
\ \ \isakeyword{where}\ {\isachardoublequoteopen}span{\isacharunderscore}vs\ S\ {\isasymequiv}\ M\ {\isasymlparr}carrier\ {\isacharcolon}{\isacharequal}\ span\ S{\isasymrparr}{\isachardoublequoteclose}%
\begin{isamarkuptext}%
In the finite case, we can take $A=S$ without loss of generality.%
\end{isamarkuptext}%
\isamarkuptrue%
\isacommand{lemma}\isamarkupfalse%
\ {\isacharparenleft}\isakeyword{in}\ module{\isacharparenright}\ finite{\isacharunderscore}span{\isacharcolon}\isanewline
\ \ \isakeyword{assumes}\ fin{\isacharcolon}\ {\isachardoublequoteopen}finite\ S{\isachardoublequoteclose}\ \isakeyword{and}\ inC{\isacharcolon}\ {\isachardoublequoteopen}S{\isasymsubseteq}carrier\ M{\isachardoublequoteclose}\isanewline
\ \ \isakeyword{shows}\ {\isachardoublequoteopen}span\ S\ {\isacharequal}\ {\isacharbraceleft}lincomb\ a\ S\ {\isacharbar}\ a{\isachardot}\ a{\isasymin}\ {\isacharparenleft}S{\isasymrightarrow}carrier\ R{\isacharparenright}{\isacharbraceright}{\isachardoublequoteclose}\isanewline
%
\isadelimproof
%
\endisadelimproof
%
\isatagproof
\isacommand{proof}\isamarkupfalse%
\ {\isacharparenleft}rule\ equalityI{\isacharparenright}\ \isanewline
\ \ \isacommand{{\isacharbraceleft}}\isamarkupfalse%
\isanewline
\ \ \ \ \isacommand{fix}\isamarkupfalse%
\ A\ a\isanewline
\ \ \ \ \isacommand{assume}\isamarkupfalse%
\ subset{\isacharcolon}\ {\isachardoublequoteopen}A\ {\isasymsubseteq}\ S{\isachardoublequoteclose}\ \isakeyword{and}\ \ \ a{\isacharcolon}\ {\isachardoublequoteopen}a\ {\isasymin}\ A\ {\isasymrightarrow}\ carrier\ R{\isachardoublequoteclose}\isanewline
\ \ \ \ \isacommand{let}\isamarkupfalse%
\ {\isacharquery}b{\isacharequal}{\isachardoublequoteopen}{\isacharparenleft}{\isasymlambda}v{\isachardot}\ if\ v\ {\isasymin}\ A\ then\ a\ v\ else\ {\isasymzero}{\isacharparenright}{\isachardoublequoteclose}\isanewline
\ \ \ \ \isacommand{from}\isamarkupfalse%
\ fin\ inC\ subset\ a\ \isacommand{have}\isamarkupfalse%
\ if{\isacharunderscore}in{\isacharcolon}\ {\isachardoublequoteopen}{\isacharparenleft}{\isasymOplus}\isactrlbsub M\isactrlesub v{\isasymin}S{\isachardot}\ {\isacharquery}b\ v\ {\isasymodot}\isactrlbsub M\isactrlesub \ v{\isacharparenright}\ {\isacharequal}\ {\isacharparenleft}{\isasymOplus}\isactrlbsub M\isactrlesub v{\isasymin}S{\isachardot}\ {\isacharparenleft}if\ v\ {\isasymin}\ A\ then\ a\ v\ {\isasymodot}\isactrlbsub M\isactrlesub \ v\ else\ {\isasymzero}\isactrlbsub M\isactrlesub {\isacharparenright}{\isacharparenright}{\isachardoublequoteclose}\isanewline
\ \ \ \ \ \ \isacommand{apply}\isamarkupfalse%
\ {\isacharparenleft}intro\ finsum{\isacharunderscore}cong{\isacharprime}{\isacharparenright}\ \isanewline
\ \ \ \ \ \ \ \ \isacommand{by}\isamarkupfalse%
\ {\isacharparenleft}auto\ simp\ add{\isacharcolon}\ coeff{\isacharunderscore}in{\isacharunderscore}ring{\isacharparenright}\ \ \isanewline
\ \ \ \ \isacommand{from}\isamarkupfalse%
\ fin\ inC\ subset\ a\ \isacommand{have}\isamarkupfalse%
\ {\isachardoublequoteopen}{\isasymexists}b{\isachardot}\ lincomb\ a\ A\ {\isacharequal}\ lincomb\ b\ S\ {\isasymand}\ b\ {\isasymin}\ S\ {\isasymrightarrow}\ carrier\ R{\isachardoublequoteclose}\isanewline
\ \ \ \ \ \ \isacommand{apply}\isamarkupfalse%
\ {\isacharparenleft}rule{\isacharunderscore}tac\ x{\isacharequal}{\isachardoublequoteopen}{\isacharquery}b{\isachardoublequoteclose}\ \isakeyword{in}\ exI{\isacharparenright}\isanewline
\ \ \ \ \ \ \isacommand{apply}\isamarkupfalse%
\ {\isacharparenleft}unfold\ lincomb{\isacharunderscore}def{\isacharcomma}\ auto{\isacharparenright}\isanewline
\ \ \ \ \ \ \isacommand{apply}\isamarkupfalse%
\ {\isacharparenleft}subst\ if{\isacharunderscore}in{\isacharparenright}\isanewline
\ \ \ \ \ \ \isacommand{by}\isamarkupfalse%
\ {\isacharparenleft}subst\ extend{\isacharunderscore}sum{\isacharcomma}\ auto{\isacharparenright}\isanewline
\ \ \isacommand{{\isacharbraceright}}\isamarkupfalse%
\isanewline
\ \ \isacommand{from}\isamarkupfalse%
\ this\ \isacommand{show}\isamarkupfalse%
\ {\isachardoublequoteopen}span\ S\ {\isasymsubseteq}\ {\isacharbraceleft}lincomb\ a\ S\ {\isacharbar}a{\isachardot}\ a\ {\isasymin}\ S\ {\isasymrightarrow}\ carrier\ R{\isacharbraceright}{\isachardoublequoteclose}\isanewline
\ \ \ \ \isacommand{by}\isamarkupfalse%
\ {\isacharparenleft}unfold\ span{\isacharunderscore}def{\isacharcomma}\ auto{\isacharparenright}\isanewline
\isacommand{next}\isamarkupfalse%
\isanewline
\ \ \isacommand{from}\isamarkupfalse%
\ fin\ \isacommand{show}\isamarkupfalse%
\ {\isachardoublequoteopen}{\isacharbraceleft}lincomb\ a\ S\ {\isacharbar}a{\isachardot}\ a\ {\isasymin}\ S\ {\isasymrightarrow}\ carrier\ R{\isacharbraceright}\ {\isasymsubseteq}\ span\ S{\isachardoublequoteclose}\isanewline
\ \ \ \ \isacommand{by}\isamarkupfalse%
\ {\isacharparenleft}unfold\ span{\isacharunderscore}def{\isacharcomma}\ auto{\isacharparenright}\isanewline
\isacommand{qed}\isamarkupfalse%
%
\endisatagproof
{\isafoldproof}%
%
\isadelimproof
%
\endisadelimproof
%
\begin{isamarkuptext}%
If $v\in \text{span S}$, then we can find a linear combination. This is in an easy to apply
format (e.g. obtain a A where\ldots)%
\end{isamarkuptext}%
\isamarkuptrue%
\isacommand{lemma}\isamarkupfalse%
\ {\isacharparenleft}\isakeyword{in}\ module{\isacharparenright}\ in{\isacharunderscore}span{\isacharcolon}\isanewline
\ \ \isakeyword{fixes}\ S\ v\isanewline
\ \ \isakeyword{assumes}\ \ h{\isadigit{2}}{\isacharcolon}\ {\isachardoublequoteopen}S{\isasymsubseteq}carrier\ V{\isachardoublequoteclose}\ \isakeyword{and}\ h{\isadigit{3}}{\isacharcolon}\ {\isachardoublequoteopen}v{\isasymin}span\ S{\isachardoublequoteclose}\isanewline
\ \ \isakeyword{shows}\ {\isachardoublequoteopen}{\isasymexists}a\ A{\isachardot}\ {\isacharparenleft}A{\isasymsubseteq}S\ {\isasymand}\ {\isacharparenleft}a{\isasymin}A{\isasymrightarrow}carrier\ R{\isacharparenright}\ {\isasymand}\ {\isacharparenleft}lincomb\ a\ A{\isacharequal}v{\isacharparenright}{\isacharparenright}{\isachardoublequoteclose}\isanewline
%
\isadelimproof
%
\endisadelimproof
%
\isatagproof
\isacommand{proof}\isamarkupfalse%
\ {\isacharminus}\ \isanewline
\ \ \isacommand{from}\isamarkupfalse%
\ h{\isadigit{2}}\ h{\isadigit{3}}\ \isacommand{show}\isamarkupfalse%
\ {\isacharquery}thesis\isanewline
\ \ \ \ \isacommand{apply}\isamarkupfalse%
\ {\isacharparenleft}unfold\ span{\isacharunderscore}def{\isacharparenright}\isanewline
\ \ \ \ \isacommand{by}\isamarkupfalse%
\ auto\isanewline
\isacommand{qed}\isamarkupfalse%
%
\endisatagproof
{\isafoldproof}%
%
\isadelimproof
%
\endisadelimproof
%
\begin{isamarkuptext}%
In the finite case, we can take $A=S$.%
\end{isamarkuptext}%
\isamarkuptrue%
\isacommand{lemma}\isamarkupfalse%
\ {\isacharparenleft}\isakeyword{in}\ module{\isacharparenright}\ finite{\isacharunderscore}in{\isacharunderscore}span{\isacharcolon}\isanewline
\ \ \isakeyword{fixes}\ S\ v\isanewline
\ \ \isakeyword{assumes}\ fin{\isacharcolon}\ {\isachardoublequoteopen}finite\ S{\isachardoublequoteclose}\ \isakeyword{and}\ h{\isadigit{2}}{\isacharcolon}\ {\isachardoublequoteopen}S{\isasymsubseteq}carrier\ M{\isachardoublequoteclose}\ \isakeyword{and}\ h{\isadigit{3}}{\isacharcolon}\ {\isachardoublequoteopen}v{\isasymin}span\ S{\isachardoublequoteclose}\isanewline
\ \ \isakeyword{shows}\ {\isachardoublequoteopen}{\isasymexists}a{\isachardot}\ {\isacharparenleft}a{\isasymin}S{\isasymrightarrow}carrier\ R{\isacharparenright}\ {\isasymand}\ {\isacharparenleft}lincomb\ a\ S{\isacharequal}v{\isacharparenright}{\isachardoublequoteclose}\isanewline
%
\isadelimproof
%
\endisadelimproof
%
\isatagproof
\isacommand{proof}\isamarkupfalse%
\ {\isacharminus}\ \isanewline
\ \ \isacommand{from}\isamarkupfalse%
\ fin\ h{\isadigit{2}}\ \isacommand{have}\isamarkupfalse%
\ fin{\isacharunderscore}span{\isacharcolon}\ {\isachardoublequoteopen}span\ S\ {\isacharequal}\ {\isacharbraceleft}lincomb\ a\ S\ {\isacharbar}a{\isachardot}\ a\ {\isasymin}\ S\ {\isasymrightarrow}\ carrier\ R{\isacharbraceright}{\isachardoublequoteclose}\ \isacommand{by}\isamarkupfalse%
\ {\isacharparenleft}rule\ finite{\isacharunderscore}span{\isacharparenright}\isanewline
\ \ \isacommand{from}\isamarkupfalse%
\ h{\isadigit{3}}\ fin{\isacharunderscore}span\ \isacommand{show}\isamarkupfalse%
\ {\isacharquery}thesis\ \isacommand{by}\isamarkupfalse%
\ auto\isanewline
\isacommand{qed}\isamarkupfalse%
%
\endisatagproof
{\isafoldproof}%
%
\isadelimproof
%
\endisadelimproof
%
\begin{isamarkuptext}%
If a subset is linearly independent, then any linear combination that is 0 must have a 
nonzero coefficient outside that set.%
\end{isamarkuptext}%
\isamarkuptrue%
\isacommand{lemma}\isamarkupfalse%
\ {\isacharparenleft}\isakeyword{in}\ module{\isacharparenright}\ lincomb{\isacharunderscore}must{\isacharunderscore}include{\isacharcolon}\isanewline
\ \ \isakeyword{fixes}\ A\ S\ T\ b\ v\isanewline
\ \ \isakeyword{assumes}\ \ inC{\isacharcolon}\ {\isachardoublequoteopen}T{\isasymsubseteq}carrier\ M{\isachardoublequoteclose}\ \isakeyword{and}\ li{\isacharcolon}\ {\isachardoublequoteopen}lin{\isacharunderscore}indpt\ S{\isachardoublequoteclose}\ \isakeyword{and}\ Ssub{\isacharcolon}\ {\isachardoublequoteopen}S{\isasymsubseteq}T{\isachardoublequoteclose}\ \isakeyword{and}\ Ssub{\isacharcolon}\ {\isachardoublequoteopen}A{\isasymsubseteq}T{\isachardoublequoteclose}\isanewline
\ \ \ \ \isakeyword{and}\ fin{\isacharcolon}\ {\isachardoublequoteopen}finite\ A{\isachardoublequoteclose}\isanewline
\ \ \ \ \isakeyword{and}\ b{\isacharcolon}\ {\isachardoublequoteopen}b{\isasymin}A{\isasymrightarrow}carrier\ R{\isachardoublequoteclose}\ \isakeyword{and}\ lc{\isacharcolon}\ {\isachardoublequoteopen}lincomb\ b\ A{\isacharequal}{\isasymzero}\isactrlbsub M\isactrlesub {\isachardoublequoteclose}\ \isakeyword{and}\ v{\isacharunderscore}in{\isacharcolon}\ {\isachardoublequoteopen}v{\isasymin}A{\isachardoublequoteclose}\isanewline
\ \ \ \ \isakeyword{and}\ nz{\isacharunderscore}coeff{\isacharcolon}\ {\isachardoublequoteopen}b\ v{\isasymnoteq}{\isasymzero}\isactrlbsub R\isactrlesub {\isachardoublequoteclose}\isanewline
\ \ \isakeyword{shows}\ {\isachardoublequoteopen}{\isasymexists}w{\isasymin}A{\isacharminus}S{\isachardot}\ b\ w{\isasymnoteq}{\isasymzero}\isactrlbsub R\isactrlesub {\isachardoublequoteclose}\isanewline
%
\isadelimproof
%
\endisadelimproof
%
\isatagproof
\isacommand{proof}\isamarkupfalse%
\ {\isacharparenleft}rule\ ccontr{\isacharparenright}\ \isanewline
\ \ \isanewline
\ \ \isacommand{assume}\isamarkupfalse%
\ {\isadigit{0}}{\isacharcolon}\ {\isachardoublequoteopen}{\isasymnot}{\isacharparenleft}{\isasymexists}\ w{\isasymin}A{\isacharminus}S{\isachardot}\ b\ w{\isasymnoteq}{\isasymzero}\isactrlbsub R\isactrlesub {\isacharparenright}{\isachardoublequoteclose}\isanewline
\ \ \isacommand{from}\isamarkupfalse%
\ {\isadigit{0}}\ \isacommand{have}\isamarkupfalse%
\ {\isadigit{1}}{\isacharcolon}\ {\isachardoublequoteopen}{\isasymAnd}w{\isachardot}\ w{\isasymin}A{\isacharminus}S{\isasymLongrightarrow}\ b\ w{\isacharequal}{\isasymzero}\isactrlbsub R\isactrlesub {\isachardoublequoteclose}\ \isacommand{by}\isamarkupfalse%
\ auto\isanewline
\ \ \isacommand{have}\isamarkupfalse%
\ Auni{\isacharcolon}\ {\isachardoublequoteopen}A{\isacharequal}{\isacharparenleft}S{\isasyminter}A{\isacharparenright}\ {\isasymunion}{\isacharparenleft}A{\isacharminus}S{\isacharparenright}{\isachardoublequoteclose}\ \isacommand{by}\isamarkupfalse%
\ auto\isanewline
\ \ \isacommand{from}\isamarkupfalse%
\ fin\ b\ Ssub\ inC\ {\isadigit{1}}\ \isacommand{have}\isamarkupfalse%
\ {\isadigit{2}}{\isacharcolon}\ {\isachardoublequoteopen}lincomb\ b\ A\ {\isacharequal}\ lincomb\ b\ {\isacharparenleft}S{\isasyminter}A{\isacharparenright}{\isachardoublequoteclose}\isanewline
\ \ \ \ \isacommand{apply}\isamarkupfalse%
\ {\isacharparenleft}subst\ Auni{\isacharparenright}\ \isanewline
\ \ \ \ \isacommand{apply}\isamarkupfalse%
\ {\isacharparenleft}subst\ lincomb{\isacharunderscore}union{\isacharcomma}\ auto{\isacharparenright}\isanewline
\ \ \ \ \isanewline
\ \ \ \ \isacommand{apply}\isamarkupfalse%
\ {\isacharparenleft}unfold\ lincomb{\isacharunderscore}def{\isacharparenright}\isanewline
\ \ \ \ \isacommand{apply}\isamarkupfalse%
\ {\isacharparenleft}subst\ {\isacharparenleft}{\isadigit{2}}{\isacharparenright}\ finsum{\isacharunderscore}all{\isadigit{0}}{\isacharcomma}\ auto{\isacharparenright}\isanewline
\ \ \ \ \isacommand{by}\isamarkupfalse%
\ {\isacharparenleft}subst\ show{\isacharunderscore}r{\isacharunderscore}zero{\isacharcomma}\ auto\ intro{\isacharbang}{\isacharcolon}\ finsum{\isacharunderscore}closed{\isacharparenright}\isanewline
\ \ \isacommand{from}\isamarkupfalse%
\ {\isadigit{1}}\ {\isadigit{2}}\ assms\ \isacommand{have}\isamarkupfalse%
\ ld{\isacharcolon}\ {\isachardoublequoteopen}lin{\isacharunderscore}dep\ S{\isachardoublequoteclose}\ \isanewline
\ \ \ \ \isacommand{apply}\isamarkupfalse%
\ {\isacharparenleft}intro\ lin{\isacharunderscore}dep{\isacharunderscore}crit{\isacharbrackleft}\isakeyword{where}\ {\isacharquery}A{\isacharequal}{\isachardoublequoteopen}S{\isasyminter}A{\isachardoublequoteclose}\ \isakeyword{and}\ {\isacharquery}a{\isacharequal}{\isachardoublequoteopen}b{\isachardoublequoteclose}\ \isakeyword{and}\ {\isacharquery}v{\isacharequal}{\isachardoublequoteopen}v{\isachardoublequoteclose}{\isacharbrackright}{\isacharparenright}\isanewline
\ \ \ \ \isacommand{by}\isamarkupfalse%
\ auto\isanewline
\ \ \isacommand{from}\isamarkupfalse%
\ ld\ li\ \isacommand{show}\isamarkupfalse%
\ False\ \isacommand{by}\isamarkupfalse%
\ auto\isanewline
\isacommand{qed}\isamarkupfalse%
%
\endisatagproof
{\isafoldproof}%
%
\isadelimproof
%
\endisadelimproof
%
\begin{isamarkuptext}%
A generating set is a set such that the span of $S$ is all of $M$.%
\end{isamarkuptext}%
\isamarkuptrue%
\isacommand{abbreviation}\isamarkupfalse%
\ {\isacharparenleft}\isakeyword{in}\ module{\isacharparenright}\ gen{\isacharunderscore}set{\isacharcolon}{\isacharcolon}{\isachardoublequoteopen}{\isacharprime}c\ set\ {\isasymRightarrow}\ bool{\isachardoublequoteclose}\isanewline
\ \ \isakeyword{where}\ {\isachardoublequoteopen}gen{\isacharunderscore}set\ S\ {\isasymequiv}\ {\isacharparenleft}span\ S\ {\isacharequal}\ carrier\ M{\isacharparenright}{\isachardoublequoteclose}%
\isamarkupsubsection{Submodules%
}
\isamarkuptrue%
\isacommand{lemma}\isamarkupfalse%
\ module{\isacharunderscore}criteria{\isacharcolon}\isanewline
\ \ \isakeyword{fixes}\ R\ \isakeyword{and}\ M\ \isanewline
\ \ \isakeyword{assumes}\ cring{\isacharcolon}\ {\isachardoublequoteopen}cring\ R{\isachardoublequoteclose}\isanewline
\ \ \ \ \ \ \isakeyword{and}\ zero{\isacharcolon}\ {\isachardoublequoteopen}{\isasymzero}\isactrlbsub M\isactrlesub {\isasymin}\ carrier\ M{\isachardoublequoteclose}\ \isanewline
\ \ \ \ \ \ \isakeyword{and}\ add{\isacharcolon}\ {\isachardoublequoteopen}{\isasymforall}v\ w{\isachardot}\ v{\isasymin}carrier\ M\ {\isasymand}\ w{\isasymin}carrier\ M{\isasymlongrightarrow}\ v{\isasymoplus}\isactrlbsub M\isactrlesub \ w{\isasymin}\ carrier\ M{\isachardoublequoteclose}\isanewline
\ \ \ \ \ \ \isakeyword{and}\ neg{\isacharcolon}\ {\isachardoublequoteopen}{\isasymforall}v{\isasymin}carrier\ M{\isachardot}\ {\isacharparenleft}{\isasymexists}neg{\isacharunderscore}v{\isasymin}carrier\ M{\isachardot}\ v{\isasymoplus}\isactrlbsub M\isactrlesub neg{\isacharunderscore}v{\isacharequal}{\isasymzero}\isactrlbsub M\isactrlesub {\isacharparenright}{\isachardoublequoteclose}\isanewline
\ \ \ \ \ \ \isakeyword{and}\ smult{\isacharcolon}\ {\isachardoublequoteopen}{\isasymforall}c\ v{\isachardot}\ c{\isasymin}\ carrier\ R\ {\isasymand}\ v{\isasymin}carrier\ M{\isasymlongrightarrow}\ c{\isasymodot}\isactrlbsub M\isactrlesub \ v\ {\isasymin}\ carrier\ M{\isachardoublequoteclose}\isanewline
\ \ \ \ \ \ \isakeyword{and}\ comm{\isacharcolon}\ {\isachardoublequoteopen}{\isasymforall}v\ w{\isachardot}\ v{\isasymin}carrier\ M\ {\isasymand}\ w{\isasymin}carrier\ M{\isasymlongrightarrow}\ v{\isasymoplus}\isactrlbsub M\isactrlesub \ w{\isacharequal}w{\isasymoplus}\isactrlbsub M\isactrlesub \ v{\isachardoublequoteclose}\isanewline
\ \ \ \ \ \ \isakeyword{and}\ assoc{\isacharcolon}\ {\isachardoublequoteopen}{\isasymforall}v\ w\ x{\isachardot}\ v{\isasymin}carrier\ M\ {\isasymand}\ w{\isasymin}carrier\ M\ {\isasymand}\ x{\isasymin}carrier\ M{\isasymlongrightarrow}\ {\isacharparenleft}v{\isasymoplus}\isactrlbsub M\isactrlesub \ w{\isacharparenright}{\isasymoplus}\isactrlbsub M\isactrlesub \ x{\isacharequal}\ v{\isasymoplus}\isactrlbsub M\isactrlesub \ {\isacharparenleft}w{\isasymoplus}\isactrlbsub M\isactrlesub \ x{\isacharparenright}{\isachardoublequoteclose}\isanewline
\ \ \ \ \ \ \isakeyword{and}\ add{\isacharunderscore}id{\isacharcolon}\ {\isachardoublequoteopen}{\isasymforall}v{\isasymin}carrier\ M{\isachardot}\ {\isacharparenleft}v{\isasymoplus}\isactrlbsub M\isactrlesub \ {\isasymzero}\isactrlbsub M\isactrlesub \ {\isacharequal}v{\isacharparenright}{\isachardoublequoteclose}\isanewline
\ \ \ \ \ \ \isakeyword{and}\ compat{\isacharcolon}\ {\isachardoublequoteopen}{\isasymforall}a\ b\ v{\isachardot}\ a{\isasymin}\ carrier\ R\ {\isasymand}\ b{\isasymin}\ carrier\ R\ {\isasymand}\ v{\isasymin}carrier\ M{\isasymlongrightarrow}\ {\isacharparenleft}a{\isasymotimes}\isactrlbsub R\isactrlesub \ b{\isacharparenright}{\isasymodot}\isactrlbsub M\isactrlesub \ v\ {\isacharequal}a{\isasymodot}\isactrlbsub M\isactrlesub \ {\isacharparenleft}b{\isasymodot}\isactrlbsub M\isactrlesub \ v{\isacharparenright}{\isachardoublequoteclose}\isanewline
\ \ \ \ \ \ \isakeyword{and}\ smult{\isacharunderscore}id{\isacharcolon}\ {\isachardoublequoteopen}{\isasymforall}v{\isasymin}carrier\ M{\isachardot}\ {\isacharparenleft}{\isasymone}\isactrlbsub R\isactrlesub \ {\isasymodot}\isactrlbsub M\isactrlesub \ v\ {\isacharequal}v{\isacharparenright}{\isachardoublequoteclose}\isanewline
\ \ \ \ \ \ \isakeyword{and}\ dist{\isacharunderscore}f{\isacharcolon}\ {\isachardoublequoteopen}{\isasymforall}a\ b\ v{\isachardot}\ a{\isasymin}\ carrier\ R\ {\isasymand}\ b{\isasymin}\ carrier\ R\ {\isasymand}\ v{\isasymin}carrier\ M{\isasymlongrightarrow}\ {\isacharparenleft}a{\isasymoplus}\isactrlbsub R\isactrlesub \ b{\isacharparenright}{\isasymodot}\isactrlbsub M\isactrlesub \ v\ {\isacharequal}{\isacharparenleft}a{\isasymodot}\isactrlbsub M\isactrlesub \ v{\isacharparenright}\ {\isasymoplus}\isactrlbsub M\isactrlesub \ {\isacharparenleft}b{\isasymodot}\isactrlbsub M\isactrlesub \ v{\isacharparenright}{\isachardoublequoteclose}\isanewline
\ \ \ \ \ \ \isakeyword{and}\ dist{\isacharunderscore}add{\isacharcolon}\ {\isachardoublequoteopen}{\isasymforall}a\ v\ w{\isachardot}\ a{\isasymin}\ carrier\ R\ {\isasymand}\ v{\isasymin}carrier\ M\ {\isasymand}\ w{\isasymin}carrier\ M{\isasymlongrightarrow}\ a{\isasymodot}\isactrlbsub M\isactrlesub \ {\isacharparenleft}v{\isasymoplus}\isactrlbsub M\isactrlesub \ w{\isacharparenright}\ {\isacharequal}{\isacharparenleft}a{\isasymodot}\isactrlbsub M\isactrlesub \ v{\isacharparenright}\ {\isasymoplus}\isactrlbsub M\isactrlesub \ {\isacharparenleft}a{\isasymodot}\isactrlbsub M\isactrlesub \ w{\isacharparenright}{\isachardoublequoteclose}\isanewline
\ \ \isakeyword{shows}\ {\isachardoublequoteopen}module\ R\ M{\isachardoublequoteclose}\isanewline
%
\isadelimproof
%
\endisadelimproof
%
\isatagproof
\isacommand{proof}\isamarkupfalse%
\ {\isacharminus}\ \isanewline
\ \ \isacommand{from}\isamarkupfalse%
\ assms\ \isacommand{have}\isamarkupfalse%
\ {\isadigit{2}}{\isacharcolon}\ {\isachardoublequoteopen}abelian{\isacharunderscore}group\ M{\isachardoublequoteclose}\ \isanewline
\ \ \ \ \isacommand{by}\isamarkupfalse%
\ {\isacharparenleft}intro\ abelian{\isacharunderscore}groupI{\isacharcomma}\ auto{\isacharparenright}\isanewline
\ \ \isacommand{from}\isamarkupfalse%
\ assms\ \isacommand{have}\isamarkupfalse%
\ {\isadigit{3}}{\isacharcolon}\ {\isachardoublequoteopen}module{\isacharunderscore}axioms\ R\ M{\isachardoublequoteclose}\isanewline
\ \ \ \ \isacommand{by}\isamarkupfalse%
\ {\isacharparenleft}unfold\ module{\isacharunderscore}axioms{\isacharunderscore}def{\isacharcomma}\ auto{\isacharparenright}\isanewline
\ \ \isacommand{from}\isamarkupfalse%
\ {\isadigit{2}}\ {\isadigit{3}}\ cring\ \isacommand{show}\isamarkupfalse%
\ {\isacharquery}thesis\ \isanewline
\ \ \ \ \isacommand{by}\isamarkupfalse%
\ {\isacharparenleft}unfold\ module{\isacharunderscore}def\ module{\isacharunderscore}def{\isacharcomma}\ auto{\isacharparenright}\isanewline
\isacommand{qed}\isamarkupfalse%
%
\endisatagproof
{\isafoldproof}%
%
\isadelimproof
%
\endisadelimproof
%
\begin{isamarkuptext}%
A submodule is $N\subseteq M$ that is closed under addition and scalar multiplication, and
contains 0 (so is not empty).%
\end{isamarkuptext}%
\isamarkuptrue%
\isacommand{locale}\isamarkupfalse%
\ submodule\ {\isacharequal}\isanewline
\ \ \isakeyword{fixes}\ R\ \isakeyword{and}\ N\ \isakeyword{and}\ M\ {\isacharparenleft}\isakeyword{structure}{\isacharparenright}\isanewline
\ \ \isakeyword{assumes}\ module{\isacharcolon}\ {\isachardoublequoteopen}module\ R\ M{\isachardoublequoteclose}\ \isanewline
\ \ \ \ \isakeyword{and}\ subset{\isacharcolon}\ {\isachardoublequoteopen}N\ {\isasymsubseteq}\ carrier\ M{\isachardoublequoteclose}\isanewline
\ \ \ \ \isakeyword{and}\ m{\isacharunderscore}closed\ {\isacharbrackleft}intro{\isacharcomma}\ simp{\isacharbrackright}{\isacharcolon}\ {\isachardoublequoteopen}{\isasymlbrakk}v\ {\isasymin}\ N{\isacharsemicolon}\ w\ {\isasymin}\ N{\isasymrbrakk}\ {\isasymLongrightarrow}\ v\ {\isasymoplus}\ w\ {\isasymin}\ N{\isachardoublequoteclose}\isanewline
\ \ \ \ \isakeyword{and}\ zero{\isacharunderscore}closed\ {\isacharbrackleft}simp{\isacharbrackright}{\isacharcolon}\ {\isachardoublequoteopen}{\isasymzero}\ {\isasymin}\ N{\isachardoublequoteclose}\ \isanewline
\ \ \ \ \isakeyword{and}\ smult{\isacharunderscore}closed\ {\isacharbrackleft}intro{\isacharcomma}\ simp{\isacharbrackright}{\isacharcolon}\ {\isachardoublequoteopen}{\isasymlbrakk}c\ {\isasymin}\ carrier\ R{\isacharsemicolon}\ v\ {\isasymin}\ N{\isasymrbrakk}\ {\isasymLongrightarrow}\ c{\isasymodot}v\ {\isasymin}\ N{\isachardoublequoteclose}\isanewline
\isanewline
\isacommand{abbreviation}\isamarkupfalse%
\ {\isacharparenleft}\isakeyword{in}\ module{\isacharparenright}\ md{\isacharcolon}{\isacharcolon}{\isachardoublequoteopen}{\isacharprime}c\ set\ {\isasymRightarrow}\ {\isacharparenleft}{\isacharprime}a{\isacharcomma}\ {\isacharprime}c{\isacharcomma}\ {\isacharprime}d{\isacharparenright}\ module{\isacharunderscore}scheme{\isachardoublequoteclose}\isanewline
\ \ \isakeyword{where}\ {\isachardoublequoteopen}md\ N\ {\isasymequiv}\ M{\isasymlparr}carrier\ {\isacharcolon}{\isacharequal}N{\isasymrparr}{\isachardoublequoteclose}\isanewline
\isanewline
\isacommand{lemma}\isamarkupfalse%
\ {\isacharparenleft}\isakeyword{in}\ module{\isacharparenright}\ carrier{\isacharunderscore}vs{\isacharunderscore}is{\isacharunderscore}self\ {\isacharbrackleft}simp{\isacharbrackright}{\isacharcolon}\isanewline
\ \ {\isachardoublequoteopen}carrier\ {\isacharparenleft}md\ N{\isacharparenright}\ {\isacharequal}\ N{\isachardoublequoteclose}\isanewline
%
\isadelimproof
\ \ %
\endisadelimproof
%
\isatagproof
\isacommand{by}\isamarkupfalse%
\ auto%
\endisatagproof
{\isafoldproof}%
%
\isadelimproof
\isanewline
%
\endisadelimproof
\isanewline
\isacommand{lemma}\isamarkupfalse%
\ {\isacharparenleft}\isakeyword{in}\ module{\isacharparenright}\ submodule{\isacharunderscore}is{\isacharunderscore}module{\isacharcolon}\isanewline
\ \ \isakeyword{fixes}\ N{\isacharcolon}{\isacharcolon}{\isachardoublequoteopen}{\isacharprime}c\ set{\isachardoublequoteclose}\isanewline
\ \ \isakeyword{assumes}\ {\isadigit{0}}{\isacharcolon}\ {\isachardoublequoteopen}submodule\ R\ N\ M{\isachardoublequoteclose}\isanewline
\ \ \isakeyword{shows}\ {\isachardoublequoteopen}module\ R\ {\isacharparenleft}md\ N{\isacharparenright}{\isachardoublequoteclose}\isanewline
%
\isadelimproof
%
\endisadelimproof
%
\isatagproof
\isacommand{proof}\isamarkupfalse%
\ \ {\isacharparenleft}unfold\ module{\isacharunderscore}def{\isacharcomma}\ auto{\isacharparenright}\isanewline
\ \ \isacommand{show}\isamarkupfalse%
\ {\isadigit{1}}{\isacharcolon}\ {\isachardoublequoteopen}cring\ R{\isachardoublequoteclose}\isacommand{{\isachardot}{\isachardot}}\isamarkupfalse%
\isanewline
\isacommand{next}\isamarkupfalse%
\isanewline
\ \ \isacommand{from}\isamarkupfalse%
\ assms\ \isacommand{show}\isamarkupfalse%
\ {\isadigit{2}}{\isacharcolon}\ {\isachardoublequoteopen}abelian{\isacharunderscore}group\ {\isacharparenleft}md\ N{\isacharparenright}{\isachardoublequoteclose}\ \isanewline
\ \ \ \ \isacommand{apply}\isamarkupfalse%
\ {\isacharparenleft}unfold\ submodule{\isacharunderscore}def{\isacharparenright}\isanewline
\ \ \ \ \isacommand{apply}\isamarkupfalse%
\ {\isacharparenleft}intro\ abelian{\isacharunderscore}groupI{\isacharcomma}\ auto{\isacharparenright}\isanewline
\ \ \ \ \ \ \isacommand{apply}\isamarkupfalse%
\ {\isacharparenleft}metis\ {\isacharparenleft}no{\isacharunderscore}types{\isacharcomma}\ hide{\isacharunderscore}lams{\isacharparenright}\ M{\isachardot}add{\isachardot}m{\isacharunderscore}assoc\ contra{\isacharunderscore}subsetD{\isacharparenright}\isanewline
\ \ \ \ \ \isacommand{apply}\isamarkupfalse%
\ {\isacharparenleft}metis\ {\isacharparenleft}no{\isacharunderscore}types{\isacharcomma}\ hide{\isacharunderscore}lams{\isacharparenright}\ M{\isachardot}add{\isachardot}m{\isacharunderscore}comm\ contra{\isacharunderscore}subsetD{\isacharparenright}\isanewline
\ \ \ \ \isacommand{apply}\isamarkupfalse%
\ {\isacharparenleft}rename{\isacharunderscore}tac\ v{\isacharparenright}%
\begin{isamarkuptxt}%
The inverse of $v$ under addition is $-v$%
\end{isamarkuptxt}%
\isamarkuptrue%
\ \ \ \ \isacommand{apply}\isamarkupfalse%
\ {\isacharparenleft}rule{\isacharunderscore}tac\ x{\isacharequal}{\isachardoublequoteopen}{\isasymominus}\isactrlbsub M\isactrlesub v{\isachardoublequoteclose}\ \isakeyword{in}\ bexI{\isacharparenright}\isanewline
\ \ \ \ \ \isacommand{apply}\isamarkupfalse%
\ {\isacharparenleft}metis\ M{\isachardot}l{\isacharunderscore}neg\ contra{\isacharunderscore}subsetD{\isacharparenright}\isanewline
\ \ \ \ \isacommand{by}\isamarkupfalse%
\ {\isacharparenleft}metis\ R{\isachardot}add{\isachardot}inv{\isacharunderscore}closed\ one{\isacharunderscore}closed\ smult{\isacharunderscore}minus{\isacharunderscore}{\isadigit{1}}\ subset{\isacharunderscore}iff{\isacharparenright}\isanewline
\isacommand{next}\isamarkupfalse%
\isanewline
\ \ \isacommand{from}\isamarkupfalse%
\ assms\ \isacommand{show}\isamarkupfalse%
\ {\isadigit{3}}{\isacharcolon}\ {\isachardoublequoteopen}module{\isacharunderscore}axioms\ R\ {\isacharparenleft}md\ N{\isacharparenright}{\isachardoublequoteclose}\isanewline
\ \ \ \ \isacommand{apply}\isamarkupfalse%
\ {\isacharparenleft}unfold\ module{\isacharunderscore}axioms{\isacharunderscore}def\ submodule{\isacharunderscore}def{\isacharcomma}\ auto{\isacharparenright}\isanewline
\ \ \ \ \ \ \isacommand{apply}\isamarkupfalse%
\ {\isacharparenleft}metis\ {\isacharparenleft}no{\isacharunderscore}types{\isacharcomma}\ hide{\isacharunderscore}lams{\isacharparenright}\ smult{\isacharunderscore}l{\isacharunderscore}distr\ contra{\isacharunderscore}subsetD{\isacharparenright}\isanewline
\ \ \ \ \ \isacommand{apply}\isamarkupfalse%
\ {\isacharparenleft}metis\ {\isacharparenleft}no{\isacharunderscore}types{\isacharcomma}\ hide{\isacharunderscore}lams{\isacharparenright}\ smult{\isacharunderscore}r{\isacharunderscore}distr\ contra{\isacharunderscore}subsetD{\isacharparenright}\isanewline
\ \ \ \ \isacommand{by}\isamarkupfalse%
\ {\isacharparenleft}metis\ {\isacharparenleft}no{\isacharunderscore}types{\isacharcomma}\ hide{\isacharunderscore}lams{\isacharparenright}\ smult{\isacharunderscore}assoc{\isadigit{1}}\ contra{\isacharunderscore}subsetD{\isacharparenright}\isanewline
\isacommand{qed}\isamarkupfalse%
%
\endisatagproof
{\isafoldproof}%
%
\isadelimproof
%
\endisadelimproof
%
\begin{isamarkuptext}%
$N_1+N_2=\{x+y | x\in N_1,y\in N_2\}$%
\end{isamarkuptext}%
\isamarkuptrue%
\isacommand{definition}\isamarkupfalse%
\ {\isacharparenleft}\isakeyword{in}\ module{\isacharparenright}\ submodule{\isacharunderscore}sum{\isacharcolon}{\isacharcolon}\ {\isachardoublequoteopen}{\isacharbrackleft}{\isacharprime}c\ set{\isacharcomma}\ {\isacharprime}c\ set{\isacharbrackright}\ {\isasymRightarrow}\ {\isacharprime}c\ set{\isachardoublequoteclose}\isanewline
\ \ \isakeyword{where}\ {\isachardoublequoteopen}submodule{\isacharunderscore}sum\ N{\isadigit{1}}\ N{\isadigit{2}}\ {\isacharequal}\ {\isacharparenleft}{\isasymlambda}\ {\isacharparenleft}x{\isacharcomma}y{\isacharparenright}{\isachardot}\ x\ {\isasymoplus}\isactrlbsub M\isactrlesub \ y{\isacharparenright}\ {\isacharbackquote}{\isacharbraceleft}{\isacharparenleft}x{\isacharcomma}y{\isacharparenright}{\isachardot}\ x{\isasymin}\ \ N{\isadigit{1}}\ {\isasymand}\ y{\isasymin}\ N{\isadigit{2}}{\isacharbraceright}{\isachardoublequoteclose}%
\begin{isamarkuptext}%
A module homomorphism $M\to N$ preserves addition and scalar multiplication.%
\end{isamarkuptext}%
\isamarkuptrue%
\isacommand{definition}\isamarkupfalse%
\ module{\isacharunderscore}hom{\isacharcolon}{\isacharcolon}\ {\isachardoublequoteopen}{\isacharbrackleft}{\isacharparenleft}{\isacharprime}a{\isacharcomma}\ {\isacharprime}c{\isadigit{0}}{\isacharparenright}\ ring{\isacharunderscore}scheme{\isacharcomma}\ \isanewline
\ \ {\isacharparenleft}{\isacharprime}a{\isacharcomma}{\isacharprime}b{\isadigit{1}}{\isacharcomma}{\isacharprime}c{\isadigit{1}}{\isacharparenright}\ module{\isacharunderscore}scheme{\isacharcomma}\ {\isacharparenleft}{\isacharprime}a{\isacharcomma}{\isacharprime}b{\isadigit{2}}{\isacharcomma}{\isacharprime}c{\isadigit{2}}{\isacharparenright}\ module{\isacharunderscore}scheme{\isacharbrackright}\ {\isasymRightarrow}{\isacharparenleft}{\isacharprime}b{\isadigit{1}}{\isasymRightarrow}{\isacharprime}b{\isadigit{2}}{\isacharparenright}\ set{\isachardoublequoteclose}\isanewline
\ \ \isakeyword{where}\ {\isachardoublequoteopen}module{\isacharunderscore}hom\ R\ M\ N\ {\isacharequal}\ {\isacharbraceleft}f{\isachardot}\ \isanewline
\ \ \ \ {\isacharparenleft}{\isacharparenleft}f{\isasymin}\ carrier\ M\ {\isasymrightarrow}\ carrier\ N{\isacharparenright}\isanewline
\ \ \ \ {\isasymand}\ {\isacharparenleft}{\isasymforall}m{\isadigit{1}}\ m{\isadigit{2}}{\isachardot}\ m{\isadigit{1}}{\isasymin}carrier\ M{\isasymand}\ m{\isadigit{2}}{\isasymin}carrier\ M\ {\isasymlongrightarrow}\ f\ {\isacharparenleft}m{\isadigit{1}}\ {\isasymoplus}\isactrlbsub M\isactrlesub \ m{\isadigit{2}}{\isacharparenright}\ {\isacharequal}\ {\isacharparenleft}f\ m{\isadigit{1}}{\isacharparenright}\ {\isasymoplus}\isactrlbsub N\isactrlesub \ {\isacharparenleft}f\ m{\isadigit{2}}{\isacharparenright}{\isacharparenright}\isanewline
\ \ \ \ {\isasymand}\ {\isacharparenleft}{\isasymforall}r\ m{\isachardot}\ r{\isasymin}carrier\ R{\isasymand}\ m{\isasymin}carrier\ M\ {\isasymlongrightarrow}f\ {\isacharparenleft}r\ {\isasymodot}\isactrlbsub M\isactrlesub \ m{\isacharparenright}\ {\isacharequal}\ r\ {\isasymodot}\isactrlbsub N\isactrlesub \ {\isacharparenleft}f\ m{\isacharparenright}{\isacharparenright}{\isacharparenright}{\isacharbraceright}{\isachardoublequoteclose}\isanewline
\isanewline
\isacommand{lemma}\isamarkupfalse%
\ module{\isacharunderscore}hom{\isacharunderscore}closed{\isacharcolon}\ {\isachardoublequoteopen}f{\isasymin}\ module{\isacharunderscore}hom\ R\ M\ N\ {\isasymLongrightarrow}\ f{\isasymin}\ carrier\ M\ {\isasymrightarrow}\ carrier\ N{\isachardoublequoteclose}\isanewline
%
\isadelimproof
%
\endisadelimproof
%
\isatagproof
\isacommand{by}\isamarkupfalse%
\ {\isacharparenleft}unfold\ module{\isacharunderscore}hom{\isacharunderscore}def{\isacharcomma}\ auto{\isacharparenright}%
\endisatagproof
{\isafoldproof}%
%
\isadelimproof
\isanewline
%
\endisadelimproof
\isanewline
\isacommand{lemma}\isamarkupfalse%
\ module{\isacharunderscore}hom{\isacharunderscore}add{\isacharcolon}\ {\isachardoublequoteopen}{\isasymlbrakk}f{\isasymin}\ module{\isacharunderscore}hom\ R\ M\ N{\isacharsemicolon}\ m{\isadigit{1}}{\isasymin}carrier\ M{\isacharsemicolon}\ m{\isadigit{2}}{\isasymin}carrier\ M\ {\isasymrbrakk}\ {\isasymLongrightarrow}\ f\ {\isacharparenleft}m{\isadigit{1}}\ {\isasymoplus}\isactrlbsub M\isactrlesub \ m{\isadigit{2}}{\isacharparenright}\ {\isacharequal}\ {\isacharparenleft}f\ m{\isadigit{1}}{\isacharparenright}\ {\isasymoplus}\isactrlbsub N\isactrlesub \ {\isacharparenleft}f\ m{\isadigit{2}}{\isacharparenright}{\isachardoublequoteclose}\isanewline
%
\isadelimproof
%
\endisadelimproof
%
\isatagproof
\isacommand{by}\isamarkupfalse%
\ {\isacharparenleft}unfold\ module{\isacharunderscore}hom{\isacharunderscore}def{\isacharcomma}\ auto{\isacharparenright}%
\endisatagproof
{\isafoldproof}%
%
\isadelimproof
\isanewline
%
\endisadelimproof
\isanewline
\isacommand{lemma}\isamarkupfalse%
\ module{\isacharunderscore}hom{\isacharunderscore}smult{\isacharcolon}\ {\isachardoublequoteopen}{\isasymlbrakk}f{\isasymin}\ module{\isacharunderscore}hom\ R\ M\ N{\isacharsemicolon}\ r{\isasymin}carrier\ R{\isacharsemicolon}\ m{\isasymin}carrier\ M\ {\isasymrbrakk}\ \ {\isasymLongrightarrow}\ f\ {\isacharparenleft}r\ {\isasymodot}\isactrlbsub M\isactrlesub \ m{\isacharparenright}\ {\isacharequal}\ r\ {\isasymodot}\isactrlbsub N\isactrlesub \ {\isacharparenleft}f\ m{\isacharparenright}{\isachardoublequoteclose}\isanewline
%
\isadelimproof
%
\endisadelimproof
%
\isatagproof
\isacommand{by}\isamarkupfalse%
\ {\isacharparenleft}unfold\ module{\isacharunderscore}hom{\isacharunderscore}def{\isacharcomma}\ auto{\isacharparenright}%
\endisatagproof
{\isafoldproof}%
%
\isadelimproof
\isanewline
%
\endisadelimproof
\isanewline
\isacommand{locale}\isamarkupfalse%
\ mod{\isacharunderscore}hom\ {\isacharequal}\ \isanewline
\ \ M{\isacharcolon}\ module\ R\ M\ {\isacharplus}\ N{\isacharcolon}\ module\ R\ N\isanewline
\ \ \ \ \isakeyword{for}\ R\ \isakeyword{and}\ M\ \isakeyword{and}\ N\ {\isacharplus}\isanewline
\ \ \isakeyword{fixes}\ f\isanewline
\ \ \isakeyword{assumes}\ f{\isacharunderscore}hom{\isacharcolon}\ {\isachardoublequoteopen}f\ {\isasymin}\ module{\isacharunderscore}hom\ R\ M\ N{\isachardoublequoteclose}\isanewline
\ \ \isakeyword{notes}\ f{\isacharunderscore}add\ {\isacharbrackleft}simp{\isacharbrackright}\ {\isacharequal}\ module{\isacharunderscore}hom{\isacharunderscore}add\ {\isacharbrackleft}OF\ f{\isacharunderscore}hom{\isacharbrackright}\isanewline
\ \ \ \ \isakeyword{and}\ f{\isacharunderscore}smult\ {\isacharbrackleft}simp{\isacharbrackright}\ {\isacharequal}\ module{\isacharunderscore}hom{\isacharunderscore}smult\ {\isacharbrackleft}OF\ f{\isacharunderscore}hom{\isacharbrackright}%
\begin{isamarkuptext}%
Some basic simplification rules for module homomorphisms.%
\end{isamarkuptext}%
\isamarkuptrue%
\isacommand{context}\isamarkupfalse%
\ mod{\isacharunderscore}hom\isanewline
\isakeyword{begin}\isanewline
\isanewline
\isacommand{lemma}\isamarkupfalse%
\ f{\isacharunderscore}im\ {\isacharbrackleft}simp{\isacharcomma}\ intro{\isacharbrackright}{\isacharcolon}\ \isanewline
\isakeyword{assumes}\ {\isachardoublequoteopen}v\ {\isasymin}\ carrier\ M{\isachardoublequoteclose}\ \isanewline
\isakeyword{shows}\ {\isachardoublequoteopen}f\ v\ {\isasymin}\ carrier\ N{\isachardoublequoteclose}\isanewline
%
\isadelimproof
%
\endisadelimproof
%
\isatagproof
\isacommand{proof}\isamarkupfalse%
\ {\isacharminus}\ \isanewline
\ \ \isacommand{have}\isamarkupfalse%
\ {\isadigit{0}}{\isacharcolon}\ {\isachardoublequoteopen}mod{\isacharunderscore}hom\ R\ M\ N\ f{\isachardoublequoteclose}\isacommand{{\isachardot}{\isachardot}}\isamarkupfalse%
\isanewline
\ \ \isacommand{from}\isamarkupfalse%
\ {\isadigit{0}}\ assms\ \isacommand{show}\isamarkupfalse%
\ {\isacharquery}thesis\isanewline
\ \ \ \ \isacommand{apply}\isamarkupfalse%
\ {\isacharparenleft}unfold\ mod{\isacharunderscore}hom{\isacharunderscore}def\ module{\isacharunderscore}hom{\isacharunderscore}def\ mod{\isacharunderscore}hom{\isacharunderscore}axioms{\isacharunderscore}def\ Pi{\isacharunderscore}def{\isacharparenright}\isanewline
\ \ \ \ \isacommand{by}\isamarkupfalse%
\ auto\isanewline
\isacommand{qed}\isamarkupfalse%
%
\endisatagproof
{\isafoldproof}%
%
\isadelimproof
\isanewline
%
\endisadelimproof
\isanewline
\isacommand{definition}\isamarkupfalse%
\ im{\isacharcolon}{\isacharcolon}\ {\isachardoublequoteopen}{\isacharprime}e\ set{\isachardoublequoteclose}\isanewline
\ \ \isakeyword{where}\ {\isachardoublequoteopen}im\ {\isacharequal}\ f{\isacharbackquote}{\isacharparenleft}carrier\ M{\isacharparenright}{\isachardoublequoteclose}\isanewline
\isanewline
\isacommand{definition}\isamarkupfalse%
\ ker{\isacharcolon}{\isacharcolon}\ {\isachardoublequoteopen}{\isacharprime}c\ set{\isachardoublequoteclose}\isanewline
\ \ \isakeyword{where}\ {\isachardoublequoteopen}ker\ {\isacharequal}\ {\isacharbraceleft}v{\isachardot}\ v\ {\isasymin}\ carrier\ M\ {\isacharampersand}\ f\ v\ {\isacharequal}\ {\isasymzero}\isactrlbsub N\isactrlesub {\isacharbraceright}{\isachardoublequoteclose}\isanewline
\isanewline
\isacommand{lemma}\isamarkupfalse%
\ f{\isadigit{0}}{\isacharunderscore}is{\isacharunderscore}{\isadigit{0}}{\isacharbrackleft}simp{\isacharbrackright}{\isacharcolon}\ {\isachardoublequoteopen}f\ {\isasymzero}\isactrlbsub M\isactrlesub {\isacharequal}{\isasymzero}\isactrlbsub N\isactrlesub {\isachardoublequoteclose}\isanewline
%
\isadelimproof
%
\endisadelimproof
%
\isatagproof
\isacommand{proof}\isamarkupfalse%
\ {\isacharminus}\isanewline
\ \ \isacommand{have}\isamarkupfalse%
\ {\isadigit{1}}{\isacharcolon}\ {\isachardoublequoteopen}f\ {\isasymzero}\isactrlbsub M\isactrlesub \ {\isacharequal}\ f\ {\isacharparenleft}{\isasymzero}\isactrlbsub R\isactrlesub \ {\isasymodot}\isactrlbsub M\isactrlesub \ {\isasymzero}\isactrlbsub M\isactrlesub {\isacharparenright}{\isachardoublequoteclose}\ \isacommand{by}\isamarkupfalse%
\ simp\isanewline
\ \ \isacommand{have}\isamarkupfalse%
\ {\isadigit{2}}{\isacharcolon}\ {\isachardoublequoteopen}f\ {\isacharparenleft}{\isasymzero}\isactrlbsub R\isactrlesub \ {\isasymodot}\isactrlbsub M\isactrlesub \ {\isasymzero}\isactrlbsub M\isactrlesub {\isacharparenright}\ {\isacharequal}\ {\isasymzero}\isactrlbsub N\isactrlesub {\isachardoublequoteclose}\ \isacommand{by}\isamarkupfalse%
\ {\isacharparenleft}simp\ del{\isacharcolon}\ M{\isachardot}lmult{\isacharunderscore}{\isadigit{0}}\ M{\isachardot}rmult{\isacharunderscore}{\isadigit{0}}\ add{\isacharcolon}f{\isacharunderscore}smult\ f{\isacharunderscore}im{\isacharparenright}\isanewline
\ \ \isacommand{from}\isamarkupfalse%
\ {\isadigit{1}}\ {\isadigit{2}}\ \isacommand{show}\isamarkupfalse%
\ {\isacharquery}thesis\ \isacommand{by}\isamarkupfalse%
\ auto\isanewline
\isacommand{qed}\isamarkupfalse%
%
\endisatagproof
{\isafoldproof}%
%
\isadelimproof
\isanewline
%
\endisadelimproof
\isanewline
\isacommand{lemma}\isamarkupfalse%
\ f{\isacharunderscore}neg\ {\isacharbrackleft}simp{\isacharbrackright}{\isacharcolon}\isanewline
\ \ {\isachardoublequoteopen}v{\isasymin}carrier\ M{\isasymLongrightarrow}f\ {\isacharparenleft}{\isasymominus}\isactrlbsub M\isactrlesub \ v{\isacharparenright}\ {\isacharequal}\ {\isasymominus}\isactrlbsub N\isactrlesub \ f\ v{\isachardoublequoteclose}\isanewline
%
\isadelimproof
%
\endisadelimproof
%
\isatagproof
\isacommand{by}\isamarkupfalse%
\ {\isacharparenleft}simp\ add{\isacharcolon}\ M{\isachardot}smult{\isacharunderscore}minus{\isacharunderscore}{\isadigit{1}}{\isacharbrackleft}THEN\ sym{\isacharbrackright}\ N{\isachardot}smult{\isacharunderscore}minus{\isacharunderscore}{\isadigit{1}}{\isacharbrackleft}THEN\ sym{\isacharbrackright}\ f{\isacharunderscore}smult{\isacharparenright}%
\endisatagproof
{\isafoldproof}%
%
\isadelimproof
\isanewline
%
\endisadelimproof
\isanewline
\isacommand{lemma}\isamarkupfalse%
\ f{\isacharunderscore}minus\ {\isacharbrackleft}simp{\isacharbrackright}{\isacharcolon}\isanewline
\ \ {\isachardoublequoteopen}{\isasymlbrakk}v{\isasymin}carrier\ M{\isacharsemicolon}\ w{\isasymin}carrier\ M{\isasymrbrakk}{\isasymLongrightarrow}f\ {\isacharparenleft}v{\isasymominus}\isactrlbsub M\isactrlesub w{\isacharparenright}\ {\isacharequal}\ f\ v\ {\isasymominus}\isactrlbsub N\isactrlesub \ f\ w{\isachardoublequoteclose}\isanewline
%
\isadelimproof
%
\endisadelimproof
%
\isatagproof
\isacommand{by}\isamarkupfalse%
\ {\isacharparenleft}simp\ add{\isacharcolon}\ a{\isacharunderscore}minus{\isacharunderscore}def\ f{\isacharunderscore}neg\ f{\isacharunderscore}add{\isacharparenright}%
\endisatagproof
{\isafoldproof}%
%
\isadelimproof
\isanewline
%
\endisadelimproof
\isanewline
\isacommand{lemma}\isamarkupfalse%
\ ker{\isacharunderscore}is{\isacharunderscore}submodule{\isacharcolon}\ {\isachardoublequoteopen}submodule\ R\ ker\ M{\isachardoublequoteclose}\isanewline
%
\isadelimproof
%
\endisadelimproof
%
\isatagproof
\isacommand{proof}\isamarkupfalse%
\ {\isacharminus}\ \isanewline
\ \ \isacommand{have}\isamarkupfalse%
\ {\isadigit{0}}{\isacharcolon}\ {\isachardoublequoteopen}mod{\isacharunderscore}hom\ R\ M\ N\ f{\isachardoublequoteclose}\isacommand{{\isachardot}{\isachardot}}\isamarkupfalse%
\isanewline
\ \ \isacommand{from}\isamarkupfalse%
\ {\isadigit{0}}\ \isacommand{have}\isamarkupfalse%
\ {\isadigit{1}}{\isacharcolon}\ {\isachardoublequoteopen}module\ R\ M{\isachardoublequoteclose}\ \isacommand{by}\isamarkupfalse%
\ {\isacharparenleft}unfold\ mod{\isacharunderscore}hom{\isacharunderscore}def{\isacharcomma}\ auto{\isacharparenright}\isanewline
\ \ \isacommand{show}\isamarkupfalse%
\ {\isacharquery}thesis\isanewline
\ \ \ \ \isacommand{by}\isamarkupfalse%
\ \ {\isacharparenleft}rule\ submodule{\isachardot}intro{\isacharcomma}\ auto\ simp\ add{\isacharcolon}\ ker{\isacharunderscore}def{\isacharcomma}\ rule\ {\isadigit{1}}{\isacharparenright}\ \isanewline
\isacommand{qed}\isamarkupfalse%
%
\endisatagproof
{\isafoldproof}%
%
\isadelimproof
\isanewline
%
\endisadelimproof
\isanewline
\isacommand{lemma}\isamarkupfalse%
\ im{\isacharunderscore}is{\isacharunderscore}submodule{\isacharcolon}\ {\isachardoublequoteopen}submodule\ R\ im\ N{\isachardoublequoteclose}\isanewline
%
\isadelimproof
%
\endisadelimproof
%
\isatagproof
\isacommand{proof}\isamarkupfalse%
\ {\isacharminus}\ \isanewline
\ \ \isacommand{have}\isamarkupfalse%
\ {\isadigit{1}}{\isacharcolon}\ {\isachardoublequoteopen}im\ {\isasymsubseteq}\ carrier\ N{\isachardoublequoteclose}\ \isacommand{by}\isamarkupfalse%
\ {\isacharparenleft}auto\ simp\ add{\isacharcolon}\ im{\isacharunderscore}def\ image{\isacharunderscore}def\ mod{\isacharunderscore}hom{\isacharunderscore}def\ module{\isacharunderscore}hom{\isacharunderscore}def\ f{\isacharunderscore}im{\isacharparenright}\ \isanewline
\ \ \isacommand{have}\isamarkupfalse%
\ {\isadigit{2}}{\isacharcolon}\ {\isachardoublequoteopen}{\isasymAnd}w{\isadigit{1}}\ w{\isadigit{2}}{\isachardot}{\isasymlbrakk}w{\isadigit{1}}\ {\isasymin}\ im{\isacharsemicolon}\ w{\isadigit{2}}\ {\isasymin}\ im{\isasymrbrakk}\ {\isasymLongrightarrow}\ w{\isadigit{1}}\ {\isasymoplus}\isactrlbsub N\isactrlesub \ w{\isadigit{2}}\ {\isasymin}\ im{\isachardoublequoteclose}\ \isanewline
\ \ \isacommand{proof}\isamarkupfalse%
\ {\isacharminus}\isanewline
\ \ \ \ \isacommand{fix}\isamarkupfalse%
\ w{\isadigit{1}}\ w{\isadigit{2}}\isanewline
\ \ \ \ \isacommand{assume}\isamarkupfalse%
\ w{\isadigit{1}}{\isacharcolon}\ {\isachardoublequoteopen}w{\isadigit{1}}\ {\isasymin}\ im{\isachardoublequoteclose}\ \isakeyword{and}\ w{\isadigit{2}}{\isacharcolon}\ {\isachardoublequoteopen}w{\isadigit{2}}{\isasymin}\ im{\isachardoublequoteclose}\isanewline
\ \ \ \ \isacommand{from}\isamarkupfalse%
\ w{\isadigit{1}}\ \isacommand{obtain}\isamarkupfalse%
\ v{\isadigit{1}}\ \isakeyword{where}\ {\isadigit{3}}{\isacharcolon}\ {\isachardoublequoteopen}v{\isadigit{1}}{\isasymin}\ carrier\ M\ {\isasymand}\ f\ v{\isadigit{1}}\ {\isacharequal}\ w{\isadigit{1}}{\isachardoublequoteclose}\ \isacommand{by}\isamarkupfalse%
\ {\isacharparenleft}unfold\ im{\isacharunderscore}def{\isacharcomma}\ auto{\isacharparenright}\isanewline
\ \ \ \ \isacommand{from}\isamarkupfalse%
\ w{\isadigit{2}}\ \isacommand{obtain}\isamarkupfalse%
\ v{\isadigit{2}}\ \isakeyword{where}\ {\isadigit{4}}{\isacharcolon}\ {\isachardoublequoteopen}v{\isadigit{2}}{\isasymin}\ carrier\ M\ {\isasymand}\ f\ v{\isadigit{2}}\ {\isacharequal}\ w{\isadigit{2}}{\isachardoublequoteclose}\ \isacommand{by}\isamarkupfalse%
\ {\isacharparenleft}unfold\ im{\isacharunderscore}def{\isacharcomma}\ auto{\isacharparenright}\isanewline
\ \ \ \ \isacommand{from}\isamarkupfalse%
\ {\isadigit{3}}\ {\isadigit{4}}\ \isacommand{have}\isamarkupfalse%
\ {\isadigit{5}}{\isacharcolon}\ {\isachardoublequoteopen}f\ {\isacharparenleft}v{\isadigit{1}}{\isasymoplus}\isactrlbsub M\isactrlesub v{\isadigit{2}}{\isacharparenright}\ {\isacharequal}\ w{\isadigit{1}}\ {\isasymoplus}\isactrlbsub N\isactrlesub \ w{\isadigit{2}}{\isachardoublequoteclose}\ \isacommand{by}\isamarkupfalse%
\ simp\isanewline
\ \ \ \ \isacommand{from}\isamarkupfalse%
\ {\isadigit{3}}\ {\isadigit{4}}\ \isacommand{have}\isamarkupfalse%
\ {\isadigit{6}}{\isacharcolon}\ {\isachardoublequoteopen}v{\isadigit{1}}{\isasymoplus}\isactrlbsub M\isactrlesub v{\isadigit{2}}{\isasymin}\ carrier\ M{\isachardoublequoteclose}\ \isacommand{by}\isamarkupfalse%
\ simp\isanewline
\ \ \ \ \isacommand{from}\isamarkupfalse%
\ {\isadigit{5}}\ {\isadigit{6}}\ \isacommand{have}\isamarkupfalse%
\ {\isadigit{7}}{\isacharcolon}\ {\isachardoublequoteopen}{\isasymexists}x{\isasymin}carrier\ M{\isachardot}\ w{\isadigit{1}}\ {\isasymoplus}\isactrlbsub N\isactrlesub \ w{\isadigit{2}}\ {\isacharequal}\ f\ x{\isachardoublequoteclose}\ \isacommand{by}\isamarkupfalse%
\ metis\isanewline
\ \ \ \ \isacommand{from}\isamarkupfalse%
\ {\isadigit{7}}\ \isacommand{show}\isamarkupfalse%
\ {\isachardoublequoteopen}{\isacharquery}thesis\ w{\isadigit{1}}\ w{\isadigit{2}}{\isachardoublequoteclose}\ \isacommand{by}\isamarkupfalse%
\ {\isacharparenleft}unfold\ im{\isacharunderscore}def\ image{\isacharunderscore}def{\isacharcomma}\ auto{\isacharparenright}\isanewline
\ \ \isacommand{qed}\isamarkupfalse%
\isanewline
\ \ \isacommand{have}\isamarkupfalse%
\ {\isadigit{3}}{\isacharcolon}\ {\isachardoublequoteopen}\ {\isasymzero}\isactrlbsub N\isactrlesub \ {\isasymin}\ im{\isachardoublequoteclose}\isanewline
\ \ \isacommand{proof}\isamarkupfalse%
\ {\isacharminus}\isanewline
\ \ \ \ \isacommand{have}\isamarkupfalse%
\ {\isadigit{8}}{\isacharcolon}\ {\isachardoublequoteopen}f\ {\isasymzero}\isactrlbsub M\isactrlesub \ {\isacharequal}\ {\isasymzero}\isactrlbsub N\isactrlesub \ {\isasymand}\ {\isasymzero}\isactrlbsub M\isactrlesub {\isasymin}carrier\ M{\isachardoublequoteclose}\ \isacommand{by}\isamarkupfalse%
\ auto\isanewline
\ \ \ \ \isacommand{from}\isamarkupfalse%
\ {\isadigit{8}}\ \isacommand{have}\isamarkupfalse%
\ {\isadigit{9}}{\isacharcolon}\ {\isachardoublequoteopen}{\isasymexists}x{\isasymin}carrier\ M{\isachardot}\ {\isasymzero}\isactrlbsub N\isactrlesub \ {\isacharequal}\ f\ x{\isachardoublequoteclose}\ \isacommand{by}\isamarkupfalse%
\ metis\isanewline
\ \ \ \ \isacommand{from}\isamarkupfalse%
\ {\isadigit{9}}\ \isacommand{show}\isamarkupfalse%
\ {\isacharquery}thesis\ \isacommand{by}\isamarkupfalse%
\ {\isacharparenleft}unfold\ im{\isacharunderscore}def\ image{\isacharunderscore}def{\isacharcomma}\ auto{\isacharparenright}\isanewline
\ \ \isacommand{qed}\isamarkupfalse%
\isanewline
\ \ \isacommand{have}\isamarkupfalse%
\ {\isadigit{4}}{\isacharcolon}\ {\isachardoublequoteopen}{\isasymAnd}c\ w{\isachardot}\ {\isasymlbrakk}c\ {\isasymin}\ carrier\ R{\isacharsemicolon}\ w\ {\isasymin}\ im{\isasymrbrakk}\ {\isasymLongrightarrow}\ c{\isasymodot}\isactrlbsub N\isactrlesub \ w\ {\isasymin}\ im{\isachardoublequoteclose}\ \isanewline
\ \ \isacommand{proof}\isamarkupfalse%
\ {\isacharminus}\isanewline
\ \ \ \ \isacommand{fix}\isamarkupfalse%
\ c\ w\isanewline
\ \ \ \ \isacommand{assume}\isamarkupfalse%
\ c{\isacharcolon}\ {\isachardoublequoteopen}c\ {\isasymin}\ carrier\ R{\isachardoublequoteclose}\ \isakeyword{and}\ w{\isacharcolon}\ {\isachardoublequoteopen}w\ {\isasymin}\ im{\isachardoublequoteclose}\isanewline
\ \ \ \ \isacommand{from}\isamarkupfalse%
\ w\ \isacommand{obtain}\isamarkupfalse%
\ v\ \isakeyword{where}\ {\isadigit{1}}{\isadigit{0}}{\isacharcolon}\ {\isachardoublequoteopen}v{\isasymin}\ carrier\ M\ {\isasymand}\ f\ v\ {\isacharequal}\ w{\isachardoublequoteclose}\ \isacommand{by}\isamarkupfalse%
\ {\isacharparenleft}unfold\ im{\isacharunderscore}def{\isacharcomma}\ auto{\isacharparenright}\isanewline
\ \ \ \ \isacommand{from}\isamarkupfalse%
\ c\ {\isadigit{1}}{\isadigit{0}}\ \isacommand{have}\isamarkupfalse%
\ {\isadigit{1}}{\isadigit{1}}{\isacharcolon}\ {\isachardoublequoteopen}f\ {\isacharparenleft}c{\isasymodot}\isactrlbsub M\isactrlesub \ v{\isacharparenright}\ {\isacharequal}\ c{\isasymodot}\isactrlbsub N\isactrlesub \ w{\isasymand}\ {\isacharparenleft}c\ {\isasymodot}\isactrlbsub M\isactrlesub \ v{\isasymin}carrier\ M{\isacharparenright}{\isachardoublequoteclose}\ \isacommand{by}\isamarkupfalse%
\ auto\isanewline
\ \ \ \ \isacommand{from}\isamarkupfalse%
\ {\isadigit{1}}{\isadigit{1}}\ \isacommand{have}\isamarkupfalse%
\ {\isadigit{1}}{\isadigit{2}}{\isacharcolon}\ {\isachardoublequoteopen}{\isasymexists}v{\isadigit{1}}{\isasymin}carrier\ M{\isachardot}\ \ c{\isasymodot}\isactrlbsub N\isactrlesub \ w{\isacharequal}f\ v{\isadigit{1}}{\isachardoublequoteclose}\ \isacommand{by}\isamarkupfalse%
\ metis\ \isanewline
\ \ \ \ \isacommand{from}\isamarkupfalse%
\ {\isadigit{1}}{\isadigit{2}}\ \isacommand{show}\isamarkupfalse%
\ {\isachardoublequoteopen}{\isacharquery}thesis\ c\ w{\isachardoublequoteclose}\ \isacommand{by}\isamarkupfalse%
\ {\isacharparenleft}unfold\ im{\isacharunderscore}def\ image{\isacharunderscore}def{\isacharcomma}\ auto{\isacharparenright}\ \isanewline
\ \ \isacommand{qed}\isamarkupfalse%
\isanewline
\ \ \isacommand{from}\isamarkupfalse%
\ {\isadigit{1}}\ {\isadigit{2}}\ {\isadigit{3}}\ {\isadigit{4}}\ \isacommand{show}\isamarkupfalse%
\ {\isacharquery}thesis\ \isacommand{by}\isamarkupfalse%
\ {\isacharparenleft}unfold{\isacharunderscore}locales{\isacharcomma}\ auto{\isacharparenright}\isanewline
\isacommand{qed}\isamarkupfalse%
%
\endisatagproof
{\isafoldproof}%
%
\isadelimproof
\isanewline
%
\endisadelimproof
\isanewline
\isacommand{lemma}\isamarkupfalse%
\ {\isacharparenleft}\isakeyword{in}\ mod{\isacharunderscore}hom{\isacharparenright}\ f{\isacharunderscore}ker{\isacharcolon}\isanewline
\ \ {\isachardoublequoteopen}v{\isasymin}ker\ {\isasymLongrightarrow}\ f\ v{\isacharequal}{\isasymzero}\isactrlbsub N\isactrlesub {\isachardoublequoteclose}\isanewline
%
\isadelimproof
%
\endisadelimproof
%
\isatagproof
\isacommand{by}\isamarkupfalse%
\ {\isacharparenleft}unfold\ ker{\isacharunderscore}def{\isacharcomma}\ auto{\isacharparenright}%
\endisatagproof
{\isafoldproof}%
%
\isadelimproof
\isanewline
%
\endisadelimproof
\isacommand{end}\isamarkupfalse%
%
\begin{isamarkuptext}%
We will show that for any set $S$, the space of functions $S\to K$ forms a vector space.%
\end{isamarkuptext}%
\isamarkuptrue%
\isacommand{definition}\isamarkupfalse%
\ {\isacharparenleft}\isakeyword{in}\ ring{\isacharparenright}\ func{\isacharunderscore}space{\isacharcolon}{\isacharcolon}\ {\isachardoublequoteopen}{\isacharprime}z\ set{\isasymRightarrow}{\isacharparenleft}{\isacharprime}a{\isacharcomma}{\isacharparenleft}{\isacharprime}z\ {\isasymRightarrow}\ {\isacharprime}a{\isacharparenright}{\isacharparenright}\ module{\isachardoublequoteclose}\isanewline
\ \ \isakeyword{where}\ {\isachardoublequoteopen}func{\isacharunderscore}space\ S\ {\isacharequal}\ {\isasymlparr}carrier\ {\isacharequal}\ S{\isasymrightarrow}\isactrlsub Ecarrier\ R{\isacharcomma}\ \isanewline
\ \ \ \ \ \ \ \ \ \ \ \ \ \ \ \ \ \ mult\ {\isacharequal}\ {\isacharparenleft}{\isasymlambda}\ f\ g{\isachardot}\ restrict\ {\isacharparenleft}{\isasymlambda}v{\isachardot}\ {\isasymzero}\isactrlbsub R\isactrlesub {\isacharparenright}\ S{\isacharparenright}{\isacharcomma}\isanewline
\ \ \ \ \ \ \ \ \ \ \ \ \ \ \ \ \ \ one\ {\isacharequal}\ \ restrict\ {\isacharparenleft}{\isasymlambda}v{\isachardot}\ {\isasymzero}\isactrlbsub R\isactrlesub {\isacharparenright}\ S{\isacharcomma}\isanewline
\ \ \ \ \ \ \ \ \ \ \ \ \ \ \ \ \ \ zero\ {\isacharequal}\ restrict\ {\isacharparenleft}{\isasymlambda}v{\isachardot}\ {\isasymzero}\isactrlbsub R\isactrlesub {\isacharparenright}\ S{\isacharcomma}\isanewline
\ \ \ \ \ \ \ \ \ \ \ \ \ \ \ \ \ \ add\ {\isacharequal}\ {\isacharparenleft}{\isasymlambda}\ f\ g{\isachardot}\ restrict\ {\isacharparenleft}{\isasymlambda}v{\isachardot}\ f\ v\ {\isasymoplus}\isactrlbsub R\isactrlesub \ g\ v{\isacharparenright}\ S{\isacharparenright}{\isacharcomma}\isanewline
\ \ \ \ \ \ \ \ \ \ \ \ \ \ \ \ \ \ smult\ {\isacharequal}\ {\isacharparenleft}{\isasymlambda}\ c\ f{\isachardot}\ restrict\ {\isacharparenleft}{\isasymlambda}v{\isachardot}\ c\ {\isasymotimes}\isactrlbsub R\isactrlesub \ f\ v{\isacharparenright}\ S{\isacharparenright}{\isasymrparr}{\isachardoublequoteclose}\isanewline
\isanewline
\isacommand{lemma}\isamarkupfalse%
\ {\isacharparenleft}\isakeyword{in}\ cring{\isacharparenright}\ func{\isacharunderscore}space{\isacharunderscore}is{\isacharunderscore}module{\isacharcolon}\isanewline
\ \ \isakeyword{fixes}\ S\isanewline
\ \ \isakeyword{shows}\ {\isachardoublequoteopen}module\ R\ {\isacharparenleft}func{\isacharunderscore}space\ S{\isacharparenright}{\isachardoublequoteclose}\ \isanewline
%
\isadelimproof
%
\endisadelimproof
%
\isatagproof
\isacommand{proof}\isamarkupfalse%
\ {\isacharminus}\isanewline
\isacommand{have}\isamarkupfalse%
\ {\isadigit{0}}{\isacharcolon}\ {\isachardoublequoteopen}cring\ R{\isachardoublequoteclose}\isacommand{{\isachardot}{\isachardot}}\isamarkupfalse%
\isanewline
\isacommand{from}\isamarkupfalse%
\ {\isadigit{0}}\ \isacommand{show}\isamarkupfalse%
\ {\isacharquery}thesis\isanewline
\ \ \isacommand{apply}\isamarkupfalse%
\ {\isacharparenleft}auto\ intro{\isacharbang}{\isacharcolon}\ module{\isacharunderscore}criteria\ simp\ add{\isacharcolon}\ func{\isacharunderscore}space{\isacharunderscore}def{\isacharparenright}\isanewline
\ \ \ \ \ \ \ \ \ \ \ \isacommand{apply}\isamarkupfalse%
\ {\isacharparenleft}auto\ simp\ add{\isacharcolon}\ module{\isacharunderscore}def{\isacharparenright}\isanewline
\ \ \ \ \ \ \ \ \ \isacommand{apply}\isamarkupfalse%
\ {\isacharparenleft}rename{\isacharunderscore}tac\ f{\isacharparenright}\isanewline
\ \ \ \ \ \ \ \ \ \isacommand{apply}\isamarkupfalse%
\ {\isacharparenleft}rule{\isacharunderscore}tac\ x{\isacharequal}{\isachardoublequoteopen}restrict\ {\isacharparenleft}{\isasymlambda}v{\isacharprime}{\isachardot}\ {\isasymominus}\isactrlbsub R\isactrlesub \ {\isacharparenleft}f\ v{\isacharprime}{\isacharparenright}{\isacharparenright}\ S{\isachardoublequoteclose}\ \isakeyword{in}\ bexI{\isacharparenright}\isanewline
\ \ \ \ \ \ \ \ \ \ \isacommand{apply}\isamarkupfalse%
\ {\isacharparenleft}auto\ simp\ add{\isacharcolon}restrict{\isacharunderscore}def\ cong{\isacharcolon}\ if{\isacharunderscore}cong\ split{\isacharcolon}\ split{\isacharunderscore}if{\isacharunderscore}asm{\isacharcomma}\ auto{\isacharparenright}\isanewline
\ \ \ \ \ \ \ \ \ \isacommand{apply}\isamarkupfalse%
\ {\isacharparenleft}auto\ simp\ add{\isacharcolon}\ a{\isacharunderscore}ac\ PiE{\isacharunderscore}mem{\isadigit{2}}\ r{\isacharunderscore}neg{\isacharparenright}\ \isanewline
\ \ \ \ \ \ \isacommand{apply}\isamarkupfalse%
\ {\isacharparenleft}unfold\ PiE{\isacharunderscore}def\ extensional{\isacharunderscore}def\ Pi{\isacharunderscore}def{\isacharparenright}\isanewline
\ \ \ \ \ \ \isacommand{by}\isamarkupfalse%
\ {\isacharparenleft}auto\ simp\ add{\isacharcolon}\ m{\isacharunderscore}assoc\ l{\isacharunderscore}distr\ r{\isacharunderscore}distr{\isacharparenright}\isanewline
\isacommand{qed}\isamarkupfalse%
%
\endisatagproof
{\isafoldproof}%
%
\isadelimproof
%
\endisadelimproof
%
\begin{isamarkuptext}%
Note: one can define $M^n$ from this.%
\end{isamarkuptext}%
\isamarkuptrue%
%
\begin{isamarkuptext}%
A linear combination is a module homomorphism from the space of coefficients to the module,
 $(a_v)\mapsto \sum_{v\in S} a_vv$.%
\end{isamarkuptext}%
\isamarkuptrue%
\isacommand{lemma}\isamarkupfalse%
\ {\isacharparenleft}\isakeyword{in}\ module{\isacharparenright}\ lincomb{\isacharunderscore}is{\isacharunderscore}mod{\isacharunderscore}hom{\isacharcolon}\isanewline
\ \ \isakeyword{fixes}\ S\isanewline
\ \ \isakeyword{assumes}\ h{\isacharcolon}\ {\isachardoublequoteopen}finite\ S{\isachardoublequoteclose}\ \isakeyword{and}\ h{\isadigit{2}}{\isacharcolon}\ {\isachardoublequoteopen}S{\isasymsubseteq}carrier\ M{\isachardoublequoteclose}\isanewline
\ \ \isakeyword{shows}\ {\isachardoublequoteopen}mod{\isacharunderscore}hom\ R\ {\isacharparenleft}func{\isacharunderscore}space\ S{\isacharparenright}\ M\ {\isacharparenleft}{\isasymlambda}a{\isachardot}\ lincomb\ a\ S{\isacharparenright}{\isachardoublequoteclose}\ \isanewline
%
\isadelimproof
%
\endisadelimproof
%
\isatagproof
\isacommand{proof}\isamarkupfalse%
\ {\isacharminus}\isanewline
\ \ \isacommand{have}\isamarkupfalse%
\ {\isadigit{0}}{\isacharcolon}\ {\isachardoublequoteopen}module\ R\ M{\isachardoublequoteclose}\isacommand{{\isachardot}{\isachardot}}\isamarkupfalse%
\isanewline
\ \ \isacommand{{\isacharbraceleft}}\isamarkupfalse%
\ \isanewline
\ \ \ \ \isacommand{fix}\isamarkupfalse%
\ m{\isadigit{1}}\ m{\isadigit{2}}\isanewline
\ \ \ \ \isacommand{assume}\isamarkupfalse%
\ m{\isadigit{1}}{\isacharcolon}\ {\isachardoublequoteopen}m{\isadigit{1}}\ {\isasymin}\ S\ {\isasymrightarrow}\isactrlsub E\ carrier\ R{\isachardoublequoteclose}\ \isakeyword{and}\ m{\isadigit{2}}{\isacharcolon}\ {\isachardoublequoteopen}m{\isadigit{2}}\ {\isasymin}\ S\ {\isasymrightarrow}\isactrlsub E\ carrier\ R{\isachardoublequoteclose}\isanewline
\ \ \ \ \isacommand{from}\isamarkupfalse%
\ h\ h{\isadigit{2}}\ m{\isadigit{1}}\ m{\isadigit{2}}\ \isacommand{have}\isamarkupfalse%
\ a{\isadigit{1}}{\isacharcolon}\ {\isachardoublequoteopen}{\isacharparenleft}{\isasymOplus}\isactrlbsub M\isactrlesub v{\isasymin}S{\isachardot}\ {\isacharparenleft}{\isasymlambda}v{\isasymin}S{\isachardot}\ m{\isadigit{1}}\ v\ {\isasymoplus}\isactrlbsub R\isactrlesub \ m{\isadigit{2}}\ v{\isacharparenright}\ v\ {\isasymodot}\isactrlbsub M\isactrlesub \ v{\isacharparenright}\ {\isacharequal}\ \isanewline
\ \ \ \ \ \ {\isacharparenleft}{\isasymOplus}\isactrlbsub M\isactrlesub v{\isasymin}S{\isachardot}\ m{\isadigit{1}}\ v\ {\isasymodot}\isactrlbsub M\isactrlesub \ v\ {\isasymoplus}\isactrlbsub M\isactrlesub \ m{\isadigit{2}}\ v\ {\isasymodot}\isactrlbsub M\isactrlesub \ v{\isacharparenright}{\isachardoublequoteclose}\isanewline
\ \ \ \ \ \ \isacommand{by}\isamarkupfalse%
\ {\isacharparenleft}intro\ finsum{\isacharunderscore}cong{\isacharprime}{\isacharcomma}\ auto\ simp\ add{\isacharcolon}\ smult{\isacharunderscore}l{\isacharunderscore}distr\ PiE{\isacharunderscore}mem{\isadigit{2}}{\isacharparenright}\isanewline
\ \ \ \ \isacommand{from}\isamarkupfalse%
\ h\ h{\isadigit{2}}\ m{\isadigit{1}}\ m{\isadigit{2}}\ \isacommand{have}\isamarkupfalse%
\ a{\isadigit{2}}{\isacharcolon}\ {\isachardoublequoteopen}{\isacharparenleft}{\isasymOplus}\isactrlbsub M\isactrlesub v{\isasymin}S{\isachardot}\ m{\isadigit{1}}\ v\ {\isasymodot}\isactrlbsub M\isactrlesub \ v\ {\isasymoplus}\isactrlbsub M\isactrlesub \ m{\isadigit{2}}\ v\ {\isasymodot}\isactrlbsub M\isactrlesub \ v{\isacharparenright}\ {\isacharequal}\ \isanewline
\ \ \ \ \ \ {\isacharparenleft}{\isasymOplus}\isactrlbsub M\isactrlesub v{\isasymin}S{\isachardot}\ m{\isadigit{1}}\ v\ {\isasymodot}\isactrlbsub M\isactrlesub \ v{\isacharparenright}\ {\isasymoplus}\isactrlbsub M\isactrlesub \ {\isacharparenleft}{\isasymOplus}\isactrlbsub M\isactrlesub v{\isasymin}S{\isachardot}\ m{\isadigit{2}}\ v\ {\isasymodot}\isactrlbsub M\isactrlesub \ v{\isacharparenright}{\isachardoublequoteclose}\isanewline
\ \ \ \ \ \ \isacommand{by}\isamarkupfalse%
\ {\isacharparenleft}intro\ finsum{\isacharunderscore}addf{\isacharcomma}\ auto{\isacharparenright}\isanewline
\ \ \ \ \isacommand{from}\isamarkupfalse%
\ a{\isadigit{1}}\ a{\isadigit{2}}\ \isacommand{have}\isamarkupfalse%
\ {\isachardoublequoteopen}{\isacharparenleft}{\isasymOplus}\isactrlbsub M\isactrlesub v{\isasymin}S{\isachardot}\ {\isacharparenleft}{\isasymlambda}v{\isasymin}S{\isachardot}\ m{\isadigit{1}}\ v\ {\isasymoplus}\ m{\isadigit{2}}\ v{\isacharparenright}\ v\ {\isasymodot}\isactrlbsub M\isactrlesub \ v{\isacharparenright}\ {\isacharequal}\isanewline
\ \ \ \ \ \ \ {\isacharparenleft}{\isasymOplus}\isactrlbsub M\isactrlesub v{\isasymin}S{\isachardot}\ m{\isadigit{1}}\ v\ {\isasymodot}\isactrlbsub M\isactrlesub \ v{\isacharparenright}\ {\isasymoplus}\isactrlbsub M\isactrlesub \ {\isacharparenleft}{\isasymOplus}\isactrlbsub M\isactrlesub v{\isasymin}S{\isachardot}\ m{\isadigit{2}}\ v\ {\isasymodot}\isactrlbsub M\isactrlesub \ v{\isacharparenright}{\isachardoublequoteclose}\ \isacommand{by}\isamarkupfalse%
\ auto\isanewline
\ \ \isacommand{{\isacharbraceright}}\isamarkupfalse%
\isanewline
\ \ \isacommand{hence}\isamarkupfalse%
\ {\isadigit{1}}{\isacharcolon}\ {\isachardoublequoteopen}{\isasymAnd}m{\isadigit{1}}\ m{\isadigit{2}}{\isachardot}\isanewline
\ \ \ \ \ \ \ m{\isadigit{1}}\ {\isasymin}\ S\ {\isasymrightarrow}\isactrlsub E\ carrier\ R\ {\isasymLongrightarrow}\isanewline
\ \ \ \ \ \ \ m{\isadigit{2}}\ {\isasymin}\ S\ {\isasymrightarrow}\isactrlsub E\ carrier\ R\ {\isasymLongrightarrow}\ {\isacharparenleft}{\isasymOplus}\isactrlbsub M\isactrlesub v{\isasymin}S{\isachardot}\ {\isacharparenleft}{\isasymlambda}v{\isasymin}S{\isachardot}\ m{\isadigit{1}}\ v\ {\isasymoplus}\ m{\isadigit{2}}\ v{\isacharparenright}\ v\ {\isasymodot}\isactrlbsub M\isactrlesub \ v{\isacharparenright}\ {\isacharequal}\isanewline
\ \ \ \ \ \ \ {\isacharparenleft}{\isasymOplus}\isactrlbsub M\isactrlesub v{\isasymin}S{\isachardot}\ m{\isadigit{1}}\ v\ {\isasymodot}\isactrlbsub M\isactrlesub \ v{\isacharparenright}\ {\isasymoplus}\isactrlbsub M\isactrlesub \ {\isacharparenleft}{\isasymOplus}\isactrlbsub M\isactrlesub v{\isasymin}S{\isachardot}\ m{\isadigit{2}}\ v\ {\isasymodot}\isactrlbsub M\isactrlesub \ v{\isacharparenright}{\isachardoublequoteclose}\ \isacommand{by}\isamarkupfalse%
\ auto\isanewline
\ \ \isacommand{{\isacharbraceleft}}\isamarkupfalse%
\ \isanewline
\ \ \ \ \isacommand{fix}\isamarkupfalse%
\ r\ m\isanewline
\ \ \ \ \isacommand{assume}\isamarkupfalse%
\ r{\isacharcolon}\ {\isachardoublequoteopen}r\ {\isasymin}\ carrier\ R{\isachardoublequoteclose}\ \isakeyword{and}\ m{\isacharcolon}\ {\isachardoublequoteopen}m\ {\isasymin}\ S\ {\isasymrightarrow}\isactrlsub E\ carrier\ R{\isachardoublequoteclose}\isanewline
\ \ \ \ \isacommand{from}\isamarkupfalse%
\ h\ h{\isadigit{2}}\ r\ m\ \isacommand{have}\isamarkupfalse%
\ b{\isadigit{1}}{\isacharcolon}\ {\isachardoublequoteopen}r\ {\isasymodot}\isactrlbsub M\isactrlesub \ {\isacharparenleft}{\isasymOplus}\isactrlbsub M\isactrlesub v{\isasymin}S{\isachardot}\ m\ v\ {\isasymodot}\isactrlbsub M\isactrlesub \ v{\isacharparenright}\ {\isacharequal}\ \ {\isacharparenleft}{\isasymOplus}\isactrlbsub M\isactrlesub v{\isasymin}S{\isachardot}\ r\ {\isasymodot}\isactrlbsub M\isactrlesub {\isacharparenleft}m\ v\ {\isasymodot}\isactrlbsub M\isactrlesub \ v{\isacharparenright}{\isacharparenright}{\isachardoublequoteclose}\isanewline
\ \ \ \ \ \ \isacommand{by}\isamarkupfalse%
\ {\isacharparenleft}intro\ finsum{\isacharunderscore}smult{\isacharcomma}\ auto{\isacharparenright}\ \isanewline
\ \ \ \ \isacommand{from}\isamarkupfalse%
\ h\ h{\isadigit{2}}\ r\ m\ \isacommand{have}\isamarkupfalse%
\ b{\isadigit{2}}{\isacharcolon}\ {\isachardoublequoteopen}{\isacharparenleft}{\isasymOplus}\isactrlbsub M\isactrlesub v{\isasymin}S{\isachardot}\ {\isacharparenleft}{\isasymlambda}v{\isasymin}S{\isachardot}\ r\ {\isasymotimes}\ m\ v{\isacharparenright}\ v\ {\isasymodot}\isactrlbsub M\isactrlesub \ v{\isacharparenright}\ {\isacharequal}\ r\ {\isasymodot}\isactrlbsub M\isactrlesub \ {\isacharparenleft}{\isasymOplus}\isactrlbsub M\isactrlesub v{\isasymin}S{\isachardot}\ m\ v\ {\isasymodot}\isactrlbsub M\isactrlesub \ v{\isacharparenright}{\isachardoublequoteclose}\isanewline
\ \ \ \ \ \ \isacommand{apply}\isamarkupfalse%
\ {\isacharparenleft}subst\ b{\isadigit{1}}{\isacharparenright}\isanewline
\ \ \ \ \ \ \isacommand{apply}\isamarkupfalse%
\ {\isacharparenleft}intro\ finsum{\isacharunderscore}cong{\isacharprime}{\isacharcomma}\ auto{\isacharparenright}\isanewline
\ \ \ \ \ \ \isacommand{by}\isamarkupfalse%
\ {\isacharparenleft}subst\ smult{\isacharunderscore}assoc{\isadigit{1}}{\isacharcomma}\ auto{\isacharparenright}\isanewline
\ \ \isacommand{{\isacharbraceright}}\isamarkupfalse%
\isanewline
\ \ \isacommand{hence}\isamarkupfalse%
\ {\isadigit{2}}{\isacharcolon}\ {\isachardoublequoteopen}{\isasymAnd}r\ m{\isachardot}\ r\ {\isasymin}\ carrier\ R\ {\isasymLongrightarrow}\isanewline
\ \ \ \ \ \ \ \ \ \ \ m\ {\isasymin}\ S\ {\isasymrightarrow}\isactrlsub E\ carrier\ R\ {\isasymLongrightarrow}\ {\isacharparenleft}{\isasymOplus}\isactrlbsub M\isactrlesub v{\isasymin}S{\isachardot}\ {\isacharparenleft}{\isasymlambda}v{\isasymin}S{\isachardot}\ r\ {\isasymotimes}\ m\ v{\isacharparenright}\ v\ {\isasymodot}\isactrlbsub M\isactrlesub \ v{\isacharparenright}\ {\isacharequal}\ r\ {\isasymodot}\isactrlbsub M\isactrlesub \ {\isacharparenleft}{\isasymOplus}\isactrlbsub M\isactrlesub v{\isasymin}S{\isachardot}\ m\ v\ {\isasymodot}\isactrlbsub M\isactrlesub \ v{\isacharparenright}{\isachardoublequoteclose}\ \isanewline
\ \ \ \ \ \ \ \ \ \ \ \ \isacommand{by}\isamarkupfalse%
\ auto\isanewline
\ \ \isacommand{from}\isamarkupfalse%
\ h\ h{\isadigit{2}}\ {\isadigit{0}}\ {\isadigit{1}}\ {\isadigit{2}}\ \isacommand{show}\isamarkupfalse%
\ {\isacharquery}thesis\isanewline
\ \ \ \ \isacommand{apply}\isamarkupfalse%
\ {\isacharparenleft}unfold\ mod{\isacharunderscore}hom{\isacharunderscore}def{\isacharcomma}\ auto{\isacharparenright}\isanewline
\ \ \ \ \ \isacommand{apply}\isamarkupfalse%
\ {\isacharparenleft}rule\ func{\isacharunderscore}space{\isacharunderscore}is{\isacharunderscore}module{\isacharparenright}\isanewline
\ \ \ \ \isacommand{apply}\isamarkupfalse%
\ {\isacharparenleft}unfold\ mod{\isacharunderscore}hom{\isacharunderscore}axioms{\isacharunderscore}def\ module{\isacharunderscore}hom{\isacharunderscore}def{\isacharcomma}\ auto{\isacharparenright}\isanewline
\ \ \ \ \ \ \isacommand{apply}\isamarkupfalse%
\ {\isacharparenleft}rule\ lincomb{\isacharunderscore}closed{\isacharcomma}\ unfold\ func{\isacharunderscore}space{\isacharunderscore}def{\isacharcomma}\ auto{\isacharparenright}\isanewline
\ \ \ \ \ \isacommand{apply}\isamarkupfalse%
\ {\isacharparenleft}unfold\ lincomb{\isacharunderscore}def{\isacharparenright}\isanewline
\ \ \ \ \ \isacommand{by}\isamarkupfalse%
\ auto\isanewline
\isacommand{qed}\isamarkupfalse%
%
\endisatagproof
{\isafoldproof}%
%
\isadelimproof
\isanewline
%
\endisadelimproof
\isanewline
\isanewline
\isacommand{lemma}\isamarkupfalse%
\ {\isacharparenleft}\isakeyword{in}\ module{\isacharparenright}\ lincomb{\isacharunderscore}sum{\isacharcolon}\isanewline
\ \ \isakeyword{assumes}\ A{\isacharunderscore}fin{\isacharcolon}\ {\isachardoublequoteopen}finite\ A{\isachardoublequoteclose}\ \isakeyword{and}\ AinC{\isacharcolon}\ {\isachardoublequoteopen}A{\isasymsubseteq}carrier\ M{\isachardoublequoteclose}\ \isakeyword{and}\ a{\isacharunderscore}fun{\isacharcolon}\ {\isachardoublequoteopen}a{\isasymin}A{\isasymrightarrow}carrier\ R{\isachardoublequoteclose}\ \isakeyword{and}\ \isanewline
\ \ \ \ b{\isacharunderscore}fun{\isacharcolon}\ {\isachardoublequoteopen}b{\isasymin}A{\isasymrightarrow}carrier\ R{\isachardoublequoteclose}\ \isanewline
\ \ \isakeyword{shows}\ {\isachardoublequoteopen}lincomb\ {\isacharparenleft}{\isasymlambda}v{\isachardot}\ a\ v\ {\isasymoplus}\isactrlbsub R\isactrlesub \ b\ v{\isacharparenright}\ A\ {\isacharequal}\ lincomb\ a\ A\ {\isasymoplus}\isactrlbsub M\isactrlesub \ lincomb\ b\ A{\isachardoublequoteclose}\isanewline
%
\isadelimproof
%
\endisadelimproof
%
\isatagproof
\isacommand{proof}\isamarkupfalse%
\ {\isacharminus}\ \isanewline
\ \ \isacommand{from}\isamarkupfalse%
\ A{\isacharunderscore}fin\ AinC\ \isacommand{interpret}\isamarkupfalse%
\ mh{\isacharcolon}\ mod{\isacharunderscore}hom\ R\ {\isachardoublequoteopen}func{\isacharunderscore}space\ A{\isachardoublequoteclose}\ M\ \ {\isachardoublequoteopen}{\isacharparenleft}{\isasymlambda}a{\isachardot}\ lincomb\ a\ A{\isacharparenright}{\isachardoublequoteclose}\ \isacommand{by}\isamarkupfalse%
\ {\isacharparenleft}rule\ \isanewline
\ \ \ \ lincomb{\isacharunderscore}is{\isacharunderscore}mod{\isacharunderscore}hom{\isacharparenright}\isanewline
\ \ \isacommand{let}\isamarkupfalse%
\ {\isacharquery}a{\isacharequal}{\isachardoublequoteopen}restrict\ a\ A{\isachardoublequoteclose}\isanewline
\ \ \isacommand{let}\isamarkupfalse%
\ {\isacharquery}b{\isacharequal}{\isachardoublequoteopen}restrict\ b\ A{\isachardoublequoteclose}\isanewline
\ \ \isacommand{from}\isamarkupfalse%
\ a{\isacharunderscore}fun\ b{\isacharunderscore}fun\ A{\isacharunderscore}fin\ AinC\isanewline
\ \ \isacommand{have}\isamarkupfalse%
\ {\isadigit{1}}{\isacharcolon}\ {\isachardoublequoteopen}LinearCombinations{\isachardot}module{\isachardot}lincomb\ M\ {\isacharparenleft}{\isacharquery}a{\isasymoplus}\isactrlbsub {\isacharparenleft}LinearCombinations{\isachardot}ring{\isachardot}func{\isacharunderscore}space\ R\ A{\isacharparenright}\isactrlesub \ {\isacharquery}b{\isacharparenright}\ A\isanewline
\ \ \ \ {\isacharequal}\ LinearCombinations{\isachardot}module{\isachardot}lincomb\ M\ {\isacharparenleft}{\isasymlambda}x{\isachardot}\ \ a\ x\ {\isasymoplus}\isactrlbsub R\isactrlesub \ b\ x{\isacharparenright}\ A{\isachardoublequoteclose}\isanewline
\ \ \ \ \isacommand{apply}\isamarkupfalse%
\ {\isacharparenleft}unfold\ func{\isacharunderscore}space{\isacharunderscore}def{\isacharcomma}\ auto{\isacharparenright}\isanewline
\ \ \ \ \isacommand{apply}\isamarkupfalse%
\ {\isacharparenleft}drule\ Pi{\isacharunderscore}implies{\isacharunderscore}Pi{\isadigit{2}}{\isacharparenright}{\isacharplus}\ \isanewline
\ \ \ \ \isacommand{by}\isamarkupfalse%
\ {\isacharparenleft}simp{\isacharunderscore}all\ {\isacharparenleft}no{\isacharunderscore}asm{\isacharunderscore}simp{\isacharparenright}\ \ add{\isacharcolon}\ R{\isachardot}minus{\isacharunderscore}closed\ sum{\isacharunderscore}simp\ cong{\isacharcolon}\ lincomb{\isacharunderscore}cong{\isacharparenright}\isanewline
\ \ \isacommand{from}\isamarkupfalse%
\ a{\isacharunderscore}fun\ b{\isacharunderscore}fun\ A{\isacharunderscore}fin\ AinC\isanewline
\ \ \isacommand{have}\isamarkupfalse%
\ {\isadigit{2}}{\isacharcolon}\ {\isachardoublequoteopen}LinearCombinations{\isachardot}module{\isachardot}lincomb\ M\ {\isacharquery}a\ A\ {\isasymoplus}\isactrlbsub M\isactrlesub \ \isanewline
\ \ \ \ \ \ LinearCombinations{\isachardot}module{\isachardot}lincomb\ M\ {\isacharquery}b\ A\ {\isacharequal}\ LinearCombinations{\isachardot}module{\isachardot}lincomb\ M\ a\ A\ {\isasymoplus}\isactrlbsub M\isactrlesub \ \isanewline
\ \ \ \ \ \ LinearCombinations{\isachardot}module{\isachardot}lincomb\ M\ b\ A{\isachardoublequoteclose}\isanewline
\ \ \ \ \isacommand{apply}\isamarkupfalse%
\ {\isacharparenleft}subst\ refl{\isacharparenright}\ \isanewline
\ \ \ \ \isacommand{apply}\isamarkupfalse%
\ {\isacharparenleft}drule\ Pi{\isacharunderscore}implies{\isacharunderscore}Pi{\isadigit{2}}{\isacharparenright}{\isacharplus}\ \isanewline
\ \ \ \ \isacommand{by}\isamarkupfalse%
\ {\isacharparenleft}simp{\isacharunderscore}all\ {\isacharparenleft}no{\isacharunderscore}asm{\isacharunderscore}simp{\isacharparenright}\ add{\isacharcolon}\ sum{\isacharunderscore}simp\ cong{\isacharcolon}\ lincomb{\isacharunderscore}cong{\isacharparenright}\isanewline
\ \ \isacommand{from}\isamarkupfalse%
\ a{\isacharunderscore}fun\ b{\isacharunderscore}fun\ \isacommand{have}\isamarkupfalse%
\ ainC{\isacharcolon}\ {\isachardoublequoteopen}{\isacharquery}a{\isasymin}carrier\ {\isacharparenleft}LinearCombinations{\isachardot}ring{\isachardot}func{\isacharunderscore}space\ R\ A{\isacharparenright}{\isachardoublequoteclose}\ \isanewline
\ \ \ \ \isakeyword{and}\ binC{\isacharcolon}\ {\isachardoublequoteopen}{\isacharquery}b{\isasymin}carrier\ {\isacharparenleft}LinearCombinations{\isachardot}ring{\isachardot}func{\isacharunderscore}space\ R\ A{\isacharparenright}{\isachardoublequoteclose}\ \isacommand{by}\isamarkupfalse%
\ {\isacharparenleft}unfold\ func{\isacharunderscore}space{\isacharunderscore}def{\isacharcomma}\ auto{\isacharparenright}\isanewline
\ \ \isacommand{from}\isamarkupfalse%
\ ainC\ binC\ \isacommand{have}\isamarkupfalse%
\ lc{\isacharunderscore}sum{\isacharcolon}\ {\isachardoublequoteopen}LinearCombinations{\isachardot}module{\isachardot}lincomb\ M\ {\isacharparenleft}{\isacharquery}a{\isasymoplus}\isactrlbsub {\isacharparenleft}LinearCombinations{\isachardot}ring{\isachardot}func{\isacharunderscore}space\ R\ A{\isacharparenright}\isactrlesub \ {\isacharquery}b{\isacharparenright}\ A\isanewline
\ \ \ \ {\isacharequal}\ LinearCombinations{\isachardot}module{\isachardot}lincomb\ M\ {\isacharquery}a\ A\ {\isasymoplus}\isactrlbsub M\isactrlesub \ \isanewline
\ \ \ \ \ \ LinearCombinations{\isachardot}module{\isachardot}lincomb\ M\ {\isacharquery}b\ A{\isachardoublequoteclose}\ \ \isanewline
\ \ \ \ \isacommand{by}\isamarkupfalse%
\ {\isacharparenleft}simp{\isacharunderscore}all\ cong{\isacharcolon}\ lincomb{\isacharunderscore}cong\ add{\isacharcolon}\ mh{\isachardot}f{\isacharunderscore}add\ func{\isacharunderscore}space{\isacharunderscore}def{\isacharparenright}\isanewline
\ \ \isacommand{from}\isamarkupfalse%
\ {\isadigit{1}}\ {\isadigit{2}}\ lc{\isacharunderscore}sum\ \isacommand{show}\isamarkupfalse%
\ {\isacharquery}thesis\ \isacommand{by}\isamarkupfalse%
\ auto\isanewline
\isacommand{qed}\isamarkupfalse%
%
\endisatagproof
{\isafoldproof}%
%
\isadelimproof
%
\endisadelimproof
%
\begin{isamarkuptext}%
The negative of a function is just pointwise negation.%
\end{isamarkuptext}%
\isamarkuptrue%
\isacommand{lemma}\isamarkupfalse%
\ {\isacharparenleft}\isakeyword{in}\ cring{\isacharparenright}\ func{\isacharunderscore}space{\isacharunderscore}neg{\isacharcolon}\ \isanewline
\ \ \isakeyword{fixes}\ f\isanewline
\ \ \isakeyword{assumes}\ {\isachardoublequoteopen}f{\isasymin}\ carrier\ {\isacharparenleft}func{\isacharunderscore}space\ S{\isacharparenright}{\isachardoublequoteclose}\isanewline
\ \ \isakeyword{shows}\ {\isachardoublequoteopen}{\isasymominus}\isactrlbsub func{\isacharunderscore}space\ S\isactrlesub \ f\ {\isacharequal}\ {\isacharparenleft}{\isasymlambda}\ v{\isachardot}\ if\ {\isacharparenleft}v{\isasymin}S{\isacharparenright}\ then\ {\isasymominus}\isactrlbsub R\isactrlesub \ f\ v\ else\ undefined{\isacharparenright}{\isachardoublequoteclose}\isanewline
%
\isadelimproof
%
\endisadelimproof
%
\isatagproof
\isacommand{proof}\isamarkupfalse%
\ {\isacharminus}\ \isanewline
\ \ \isacommand{interpret}\isamarkupfalse%
\ fs{\isacharcolon}\ module\ R\ {\isachardoublequoteopen}func{\isacharunderscore}space\ S{\isachardoublequoteclose}\ \isacommand{by}\isamarkupfalse%
\ {\isacharparenleft}rule\ func{\isacharunderscore}space{\isacharunderscore}is{\isacharunderscore}module{\isacharparenright}\isanewline
\ \ \isacommand{from}\isamarkupfalse%
\ assms\ \isacommand{show}\isamarkupfalse%
\ {\isacharquery}thesis\isanewline
\ \ \ \ \isacommand{apply}\isamarkupfalse%
\ {\isacharparenleft}intro\ fs{\isachardot}minus{\isacharunderscore}equality{\isacharparenright}\isanewline
\ \ \ \ \ \ \isacommand{apply}\isamarkupfalse%
\ {\isacharparenleft}unfold\ func{\isacharunderscore}space{\isacharunderscore}def\ PiE{\isacharunderscore}def\ extensional{\isacharunderscore}def{\isacharparenright}\isanewline
\ \ \ \ \ \ \isacommand{apply}\isamarkupfalse%
\ auto\isanewline
\ \ \ \ \ \isacommand{apply}\isamarkupfalse%
\ {\isacharparenleft}intro\ restrict{\isacharunderscore}ext{\isacharcomma}\ auto{\isacharparenright}\isanewline
\ \ \ \ \isacommand{by}\isamarkupfalse%
\ {\isacharparenleft}simp\ add{\isacharcolon}\ l{\isacharunderscore}neg\ coeff{\isacharunderscore}in{\isacharunderscore}ring{\isacharparenright}\isanewline
\isacommand{qed}\isamarkupfalse%
%
\endisatagproof
{\isafoldproof}%
%
\isadelimproof
%
\endisadelimproof
%
\begin{isamarkuptext}%
Ditto for subtraction. Note the above is really a special case, when a is the 0 function.%
\end{isamarkuptext}%
\isamarkuptrue%
\isacommand{lemma}\isamarkupfalse%
\ {\isacharparenleft}\isakeyword{in}\ module{\isacharparenright}\ lincomb{\isacharunderscore}diff{\isacharcolon}\isanewline
\ \ \isakeyword{assumes}\ A{\isacharunderscore}fin{\isacharcolon}\ {\isachardoublequoteopen}finite\ A{\isachardoublequoteclose}\ \isakeyword{and}\ AinC{\isacharcolon}\ {\isachardoublequoteopen}A{\isasymsubseteq}carrier\ M{\isachardoublequoteclose}\ \isakeyword{and}\ a{\isacharunderscore}fun{\isacharcolon}\ {\isachardoublequoteopen}a{\isasymin}A{\isasymrightarrow}carrier\ R{\isachardoublequoteclose}\ \isakeyword{and}\ \isanewline
\ \ \ \ b{\isacharunderscore}fun{\isacharcolon}\ {\isachardoublequoteopen}b{\isasymin}A{\isasymrightarrow}carrier\ R{\isachardoublequoteclose}\ \isanewline
\ \ \isakeyword{shows}\ {\isachardoublequoteopen}lincomb\ {\isacharparenleft}{\isasymlambda}v{\isachardot}\ a\ v\ {\isasymominus}\isactrlbsub R\isactrlesub \ b\ v{\isacharparenright}\ A\ {\isacharequal}\ lincomb\ a\ A\ {\isasymominus}\isactrlbsub M\isactrlesub \ lincomb\ b\ A{\isachardoublequoteclose}\isanewline
%
\isadelimproof
%
\endisadelimproof
%
\isatagproof
\isacommand{proof}\isamarkupfalse%
\ {\isacharminus}\ \isanewline
\ \ \isacommand{from}\isamarkupfalse%
\ A{\isacharunderscore}fin\ AinC\ \isacommand{interpret}\isamarkupfalse%
\ mh{\isacharcolon}\ mod{\isacharunderscore}hom\ R\ {\isachardoublequoteopen}func{\isacharunderscore}space\ A{\isachardoublequoteclose}\ M\ \ {\isachardoublequoteopen}{\isacharparenleft}{\isasymlambda}a{\isachardot}\ lincomb\ a\ A{\isacharparenright}{\isachardoublequoteclose}\ \isacommand{by}\isamarkupfalse%
\ {\isacharparenleft}rule\ \isanewline
\ \ \ \ lincomb{\isacharunderscore}is{\isacharunderscore}mod{\isacharunderscore}hom{\isacharparenright}\isanewline
\ \ \isacommand{let}\isamarkupfalse%
\ {\isacharquery}a{\isacharequal}{\isachardoublequoteopen}restrict\ a\ A{\isachardoublequoteclose}\isanewline
\ \ \isacommand{let}\isamarkupfalse%
\ {\isacharquery}b{\isacharequal}{\isachardoublequoteopen}restrict\ b\ A{\isachardoublequoteclose}\isanewline
\ \ \isacommand{from}\isamarkupfalse%
\ a{\isacharunderscore}fun\ b{\isacharunderscore}fun\ \isacommand{have}\isamarkupfalse%
\ ainC{\isacharcolon}\ {\isachardoublequoteopen}{\isacharquery}a{\isasymin}carrier\ {\isacharparenleft}LinearCombinations{\isachardot}ring{\isachardot}func{\isacharunderscore}space\ R\ A{\isacharparenright}{\isachardoublequoteclose}\ \isanewline
\ \ \ \ \isakeyword{and}\ binC{\isacharcolon}\ {\isachardoublequoteopen}{\isacharquery}b{\isasymin}carrier\ {\isacharparenleft}LinearCombinations{\isachardot}ring{\isachardot}func{\isacharunderscore}space\ R\ A{\isacharparenright}{\isachardoublequoteclose}\ \isacommand{by}\isamarkupfalse%
\ {\isacharparenleft}unfold\ func{\isacharunderscore}space{\isacharunderscore}def{\isacharcomma}\ auto{\isacharparenright}\isanewline
\ \ \isacommand{from}\isamarkupfalse%
\ a{\isacharunderscore}fun\ b{\isacharunderscore}fun\ ainC\ binC\ A{\isacharunderscore}fin\ AinC\isanewline
\ \ \isacommand{have}\isamarkupfalse%
\ {\isadigit{1}}{\isacharcolon}\ {\isachardoublequoteopen}LinearCombinations{\isachardot}module{\isachardot}lincomb\ M\ {\isacharparenleft}{\isacharquery}a{\isasymominus}\isactrlbsub {\isacharparenleft}func{\isacharunderscore}space\ A{\isacharparenright}\isactrlesub \ {\isacharquery}b{\isacharparenright}\ A\isanewline
\ \ \ \ {\isacharequal}\ LinearCombinations{\isachardot}module{\isachardot}lincomb\ M\ {\isacharparenleft}{\isasymlambda}x{\isachardot}\ \ a\ x\ {\isasymominus}\isactrlbsub R\isactrlesub \ b\ x{\isacharparenright}\ A{\isachardoublequoteclose}\isanewline
\ \ \ \ \isacommand{apply}\isamarkupfalse%
\ {\isacharparenleft}subst\ mh{\isachardot}M{\isachardot}M{\isachardot}minus{\isacharunderscore}eq{\isacharparenright}\isanewline
\ \ \ \ \ \ \isacommand{apply}\isamarkupfalse%
\ {\isacharparenleft}auto\ simp\ del{\isacharcolon}\ mh{\isachardot}f{\isacharunderscore}minus\ mh{\isachardot}f{\isacharunderscore}add\ {\isacharparenright}\isanewline
\ \ \ \ \isacommand{apply}\isamarkupfalse%
\ {\isacharparenleft}intro\ lincomb{\isacharunderscore}cong{\isacharcomma}\ auto{\isacharparenright}\isanewline
\ \ \ \ \isacommand{apply}\isamarkupfalse%
\ {\isacharparenleft}subst\ func{\isacharunderscore}space{\isacharunderscore}neg{\isacharcomma}\ auto{\isacharparenright}\isanewline
\ \ \ \ \isacommand{apply}\isamarkupfalse%
\ {\isacharparenleft}simp\ add{\isacharcolon}\ restrict{\isacharunderscore}def\ func{\isacharunderscore}space{\isacharunderscore}def{\isacharparenright}\isanewline
\ \ \ \ \isacommand{by}\isamarkupfalse%
\ {\isacharparenleft}subst\ R{\isachardot}minus{\isacharunderscore}eq{\isacharcomma}\ auto{\isacharparenright}\isanewline
\ \ \isacommand{from}\isamarkupfalse%
\ a{\isacharunderscore}fun\ b{\isacharunderscore}fun\ A{\isacharunderscore}fin\ AinC\isanewline
\ \ \isacommand{have}\isamarkupfalse%
\ {\isadigit{2}}{\isacharcolon}\ {\isachardoublequoteopen}LinearCombinations{\isachardot}module{\isachardot}lincomb\ M\ {\isacharquery}a\ A\ {\isasymominus}\isactrlbsub M\isactrlesub \ \isanewline
\ \ \ \ \ \ LinearCombinations{\isachardot}module{\isachardot}lincomb\ M\ {\isacharquery}b\ A\ {\isacharequal}\ LinearCombinations{\isachardot}module{\isachardot}lincomb\ M\ a\ A\ {\isasymominus}\isactrlbsub M\isactrlesub \ \isanewline
\ \ \ \ \ \ LinearCombinations{\isachardot}module{\isachardot}lincomb\ M\ b\ A{\isachardoublequoteclose}\isanewline
\ \ \ \ \isacommand{apply}\isamarkupfalse%
\ {\isacharparenleft}subst\ refl{\isacharparenright}\ \isanewline
\ \ \ \ \isacommand{apply}\isamarkupfalse%
\ {\isacharparenleft}drule\ Pi{\isacharunderscore}implies{\isacharunderscore}Pi{\isadigit{2}}{\isacharparenright}{\isacharplus}\ \isanewline
\ \ \ \ \isacommand{by}\isamarkupfalse%
\ {\isacharparenleft}simp{\isacharunderscore}all\ {\isacharparenleft}no{\isacharunderscore}asm{\isacharunderscore}simp{\isacharparenright}\ \isanewline
\ \ \ \ \ \ add{\isacharcolon}\ R{\isachardot}minus{\isacharunderscore}closed\ sum{\isacharunderscore}simp\ cong{\isacharcolon}\ lincomb{\isacharunderscore}cong{\isacharparenright}\isanewline
\ \ \isacommand{from}\isamarkupfalse%
\ ainC\ binC\ \isacommand{have}\isamarkupfalse%
\ lc{\isacharunderscore}sum{\isacharcolon}\ {\isachardoublequoteopen}LinearCombinations{\isachardot}module{\isachardot}lincomb\ M\ {\isacharparenleft}{\isacharquery}a{\isasymominus}\isactrlbsub {\isacharparenleft}LinearCombinations{\isachardot}ring{\isachardot}func{\isacharunderscore}space\ R\ A{\isacharparenright}\isactrlesub \ {\isacharquery}b{\isacharparenright}\ A\isanewline
\ \ \ \ {\isacharequal}\ LinearCombinations{\isachardot}module{\isachardot}lincomb\ M\ {\isacharquery}a\ A\ {\isasymominus}\isactrlbsub M\isactrlesub \ \isanewline
\ \ \ \ \ \ LinearCombinations{\isachardot}module{\isachardot}lincomb\ M\ {\isacharquery}b\ A{\isachardoublequoteclose}\ \ \isanewline
\ \ \ \ \isacommand{by}\isamarkupfalse%
\ {\isacharparenleft}simp{\isacharunderscore}all\ cong{\isacharcolon}\ lincomb{\isacharunderscore}cong\ add{\isacharcolon}\ mh{\isachardot}f{\isacharunderscore}add\ func{\isacharunderscore}space{\isacharunderscore}def{\isacharparenright}\isanewline
\ \ \isacommand{from}\isamarkupfalse%
\ {\isadigit{1}}\ {\isadigit{2}}\ lc{\isacharunderscore}sum\ \isacommand{show}\isamarkupfalse%
\ {\isacharquery}thesis\ \isacommand{by}\isamarkupfalse%
\ auto\isanewline
\isacommand{qed}\isamarkupfalse%
%
\endisatagproof
{\isafoldproof}%
%
\isadelimproof
%
\endisadelimproof
%
\begin{isamarkuptext}%
The union of nested submodules is a submodule. We will use this to show that span of any
set is a submodule.%
\end{isamarkuptext}%
\isamarkuptrue%
\isacommand{lemma}\isamarkupfalse%
\ {\isacharparenleft}\isakeyword{in}\ module{\isacharparenright}\ nested{\isacharunderscore}union{\isacharunderscore}vs{\isacharcolon}\ \isanewline
\ \ \isakeyword{fixes}\ I\ N\ N{\isacharprime}\isanewline
\ \ \isakeyword{assumes}\ subm{\isacharcolon}\ {\isachardoublequoteopen}{\isasymAnd}i{\isachardot}\ i{\isasymin}I{\isasymLongrightarrow}\ submodule\ R\ {\isacharparenleft}N\ i{\isacharparenright}\ M{\isachardoublequoteclose}\isanewline
\ \ \ \ \isakeyword{and}\ max{\isacharunderscore}exists{\isacharcolon}\ {\isachardoublequoteopen}{\isasymAnd}i\ j{\isachardot}\ i{\isasymin}I{\isasymLongrightarrow}j{\isasymin}I{\isasymLongrightarrow}\ {\isacharparenleft}{\isasymexists}k{\isachardot}\ k{\isasymin}I\ {\isasymand}\ N\ i{\isasymsubseteq}N\ k\ {\isasymand}\ N\ j\ {\isasymsubseteq}N\ k{\isacharparenright}{\isachardoublequoteclose}\ \isanewline
\ \ \ \ \isakeyword{and}\ uni{\isacharcolon}\ {\isachardoublequoteopen}N{\isacharprime}{\isacharequal}{\isacharparenleft}{\isasymUnion}\ i{\isasymin}I{\isachardot}\ N\ i{\isacharparenright}{\isachardoublequoteclose}\isanewline
\ \ \ \ \isakeyword{and}\ ne{\isacharcolon}\ {\isachardoublequoteopen}I{\isasymnoteq}{\isacharbraceleft}{\isacharbraceright}{\isachardoublequoteclose}\isanewline
\ \ \isakeyword{shows}\ {\isachardoublequoteopen}submodule\ R\ N{\isacharprime}\ M{\isachardoublequoteclose}\isanewline
%
\isadelimproof
%
\endisadelimproof
%
\isatagproof
\isacommand{proof}\isamarkupfalse%
\ {\isacharminus}\isanewline
\ \ \isacommand{have}\isamarkupfalse%
\ {\isadigit{1}}{\isacharcolon}\ {\isachardoublequoteopen}module\ R\ M{\isachardoublequoteclose}\isacommand{{\isachardot}{\isachardot}}\isamarkupfalse%
\isanewline
\ \ \isacommand{from}\isamarkupfalse%
\ subm\ \isacommand{have}\isamarkupfalse%
\ all{\isacharunderscore}in{\isacharcolon}\ {\isachardoublequoteopen}{\isasymAnd}i{\isachardot}\ i{\isasymin}I\ {\isasymLongrightarrow}\ N\ i\ {\isasymsubseteq}\ carrier\ M{\isachardoublequoteclose}\isanewline
\ \ \ \ \isacommand{by}\isamarkupfalse%
\ {\isacharparenleft}unfold\ submodule{\isacharunderscore}def{\isacharcomma}\ auto{\isacharparenright}\isanewline
\ \ \isacommand{from}\isamarkupfalse%
\ uni\ all{\isacharunderscore}in\ \isacommand{have}\isamarkupfalse%
\ {\isadigit{2}}{\isacharcolon}\ {\isachardoublequoteopen}{\isasymAnd}x{\isachardot}\ x\ {\isasymin}\ N{\isacharprime}\ {\isasymLongrightarrow}\ x\ {\isasymin}\ carrier\ M{\isachardoublequoteclose}\isanewline
\ \ \ \ \isacommand{by}\isamarkupfalse%
\ auto\isanewline
\ \ \isacommand{from}\isamarkupfalse%
\ uni\ \isacommand{have}\isamarkupfalse%
\ {\isadigit{3}}{\isacharcolon}\ {\isachardoublequoteopen}{\isasymAnd}v\ w{\isachardot}\ v\ {\isasymin}\ N{\isacharprime}\ {\isasymLongrightarrow}\ w\ {\isasymin}\ N{\isacharprime}\ {\isasymLongrightarrow}\ v\ {\isasymoplus}\isactrlbsub M\isactrlesub \ w\ {\isasymin}\ N{\isacharprime}{\isachardoublequoteclose}\isanewline
\ \ \isacommand{proof}\isamarkupfalse%
\ {\isacharminus}\ \isanewline
\ \ \ \ \isacommand{fix}\isamarkupfalse%
\ v\ w\isanewline
\ \ \ \ \isacommand{assume}\isamarkupfalse%
\ v{\isacharcolon}\ {\isachardoublequoteopen}v\ {\isasymin}\ N{\isacharprime}{\isachardoublequoteclose}\ \isakeyword{and}\ w{\isacharcolon}\ {\isachardoublequoteopen}w\ {\isasymin}\ N{\isacharprime}{\isachardoublequoteclose}\isanewline
\ \ \ \ \isacommand{from}\isamarkupfalse%
\ uni\ v\ w\ \isacommand{obtain}\isamarkupfalse%
\ i\ j\ \isakeyword{where}\ i{\isacharcolon}\ {\isachardoublequoteopen}i{\isasymin}I{\isasymand}\ v{\isasymin}\ N\ i{\isachardoublequoteclose}\ \isakeyword{and}\ j{\isacharcolon}\ {\isachardoublequoteopen}j{\isasymin}I{\isasymand}\ w{\isasymin}\ N\ j{\isachardoublequoteclose}\ \isacommand{by}\isamarkupfalse%
\ auto\isanewline
\ \ \ \ \isacommand{from}\isamarkupfalse%
\ max{\isacharunderscore}exists\ i\ j\ \isacommand{obtain}\isamarkupfalse%
\ k\ \isakeyword{where}\ k{\isacharcolon}\ {\isachardoublequoteopen}k{\isasymin}I\ {\isasymand}\ N\ i\ {\isasymsubseteq}\ N\ k\ {\isasymand}\ N\ j\ {\isasymsubseteq}\ N\ k{\isachardoublequoteclose}\ \isacommand{by}\isamarkupfalse%
\ presburger\isanewline
\ \ \ \ \isacommand{from}\isamarkupfalse%
\ v\ w\ i\ j\ k\ \isacommand{have}\isamarkupfalse%
\ v{\isadigit{2}}{\isacharcolon}\ {\isachardoublequoteopen}v{\isasymin}N\ k{\isachardoublequoteclose}\ \isakeyword{and}\ w{\isadigit{2}}{\isacharcolon}\ {\isachardoublequoteopen}w{\isasymin}\ N\ k{\isachardoublequoteclose}\ \isacommand{by}\isamarkupfalse%
\ auto\isanewline
\ \ \ \ \isacommand{from}\isamarkupfalse%
\ v{\isadigit{2}}\ w{\isadigit{2}}\ k\ subm{\isacharbrackleft}of\ k{\isacharbrackright}\ \isacommand{have}\isamarkupfalse%
\ vw{\isacharcolon}\ {\isachardoublequoteopen}v\ {\isasymoplus}\isactrlbsub M\isactrlesub \ w\ {\isasymin}\ N\ k{\isachardoublequoteclose}\ \isacommand{apply}\isamarkupfalse%
\ {\isacharparenleft}unfold\ submodule{\isacharunderscore}def{\isacharparenright}\ \isacommand{by}\isamarkupfalse%
\ auto\isanewline
\ \ \ \ \isacommand{from}\isamarkupfalse%
\ k\ vw\ uni\ \isacommand{show}\isamarkupfalse%
\ {\isachardoublequoteopen}{\isacharquery}thesis\ v\ w{\isachardoublequoteclose}\ \ \isacommand{by}\isamarkupfalse%
\ auto\isanewline
\ \ \isacommand{qed}\isamarkupfalse%
\isanewline
\ \ \isacommand{have}\isamarkupfalse%
\ {\isadigit{4}}{\isacharcolon}\ {\isachardoublequoteopen}{\isasymzero}\isactrlbsub M\isactrlesub \ {\isasymin}\ N{\isacharprime}{\isachardoublequoteclose}\isanewline
\ \ \isacommand{proof}\isamarkupfalse%
\ {\isacharminus}\ \isanewline
\ \ \ \ \isacommand{from}\isamarkupfalse%
\ ne\ \isacommand{obtain}\isamarkupfalse%
\ i\ \isakeyword{where}\ i{\isacharcolon}\ {\isachardoublequoteopen}i{\isasymin}I{\isachardoublequoteclose}\ \isacommand{by}\isamarkupfalse%
\ auto\isanewline
\ \ \ \ \isacommand{from}\isamarkupfalse%
\ i\ subm\ \isacommand{have}\isamarkupfalse%
\ zi{\isacharcolon}\ {\isachardoublequoteopen}{\isasymzero}\isactrlbsub M\isactrlesub {\isasymin}N\ i{\isachardoublequoteclose}\ \isacommand{by}\isamarkupfalse%
\ {\isacharparenleft}unfold\ submodule{\isacharunderscore}def{\isacharcomma}\ auto{\isacharparenright}\isanewline
\ \ \ \ \isacommand{from}\isamarkupfalse%
\ i\ zi\ uni\ \isacommand{show}\isamarkupfalse%
\ {\isacharquery}thesis\ \isacommand{by}\isamarkupfalse%
\ auto\isanewline
\ \ \isacommand{qed}\isamarkupfalse%
\isanewline
\ \ \isacommand{from}\isamarkupfalse%
\ uni\ subm\ \isacommand{have}\isamarkupfalse%
\ {\isadigit{5}}{\isacharcolon}\ {\isachardoublequoteopen}{\isasymAnd}c\ v{\isachardot}\ c\ {\isasymin}\ carrier\ R\ {\isasymLongrightarrow}\ v\ {\isasymin}\ N{\isacharprime}\ {\isasymLongrightarrow}\ c\ {\isasymodot}\isactrlbsub M\isactrlesub \ v\ {\isasymin}\ N{\isacharprime}{\isachardoublequoteclose}\isanewline
\ \ \ \ \isacommand{by}\isamarkupfalse%
\ {\isacharparenleft}unfold\ submodule{\isacharunderscore}def{\isacharcomma}\ auto{\isacharparenright}\isanewline
\ \ \isacommand{from}\isamarkupfalse%
\ {\isadigit{1}}\ {\isadigit{2}}\ {\isadigit{3}}\ {\isadigit{4}}\ {\isadigit{5}}\ \isacommand{show}\isamarkupfalse%
\ {\isacharquery}thesis\ \isacommand{by}\isamarkupfalse%
\ {\isacharparenleft}unfold\ submodule{\isacharunderscore}def{\isacharcomma}\ auto{\isacharparenright}\isanewline
\isacommand{qed}\isamarkupfalse%
%
\endisatagproof
{\isafoldproof}%
%
\isadelimproof
\isanewline
%
\endisadelimproof
\isanewline
\isacommand{lemma}\isamarkupfalse%
\ {\isacharparenleft}\isakeyword{in}\ module{\isacharparenright}\ span{\isacharunderscore}is{\isacharunderscore}monotone{\isacharcolon}\isanewline
\ \ \isakeyword{fixes}\ S\ T\isanewline
\ \ \isakeyword{assumes}\ subs{\isacharcolon}\ {\isachardoublequoteopen}S{\isasymsubseteq}T{\isachardoublequoteclose}\isanewline
\ \ \isakeyword{shows}\ {\isachardoublequoteopen}span\ S\ {\isasymsubseteq}\ span\ T{\isachardoublequoteclose}\isanewline
%
\isadelimproof
%
\endisadelimproof
%
\isatagproof
\isacommand{proof}\isamarkupfalse%
\ {\isacharminus}\isanewline
\ \ \isacommand{from}\isamarkupfalse%
\ subs\ \isacommand{show}\isamarkupfalse%
\ {\isacharquery}thesis\ \isanewline
\ \ \ \ \isacommand{by}\isamarkupfalse%
\ {\isacharparenleft}unfold\ span{\isacharunderscore}def{\isacharcomma}\ auto{\isacharparenright}\isanewline
\isacommand{qed}\isamarkupfalse%
%
\endisatagproof
{\isafoldproof}%
%
\isadelimproof
\isanewline
%
\endisadelimproof
\isanewline
\isanewline
\isacommand{lemma}\isamarkupfalse%
\ {\isacharparenleft}\isakeyword{in}\ module{\isacharparenright}\ span{\isacharunderscore}is{\isacharunderscore}submodule{\isacharcolon}\isanewline
\ \ \isakeyword{fixes}\ S\isanewline
\ \ \isakeyword{assumes}\ \ h{\isadigit{2}}{\isacharcolon}\ {\isachardoublequoteopen}S{\isasymsubseteq}carrier\ M{\isachardoublequoteclose}\isanewline
\ \ \isakeyword{shows}\ {\isachardoublequoteopen}submodule\ R\ {\isacharparenleft}span\ S{\isacharparenright}\ M{\isachardoublequoteclose}\isanewline
%
\isadelimproof
%
\endisadelimproof
%
\isatagproof
\isacommand{proof}\isamarkupfalse%
\ {\isacharparenleft}cases\ {\isachardoublequoteopen}S{\isacharequal}{\isacharbraceleft}{\isacharbraceright}{\isachardoublequoteclose}{\isacharparenright}\isanewline
\ \ \isacommand{case}\isamarkupfalse%
\ True\isanewline
\ \ \isacommand{moreover}\isamarkupfalse%
\ \isacommand{have}\isamarkupfalse%
\ {\isachardoublequoteopen}module\ R\ M{\isachardoublequoteclose}\isacommand{{\isachardot}{\isachardot}}\isamarkupfalse%
\isanewline
\ \ \isacommand{ultimately}\isamarkupfalse%
\ \isacommand{show}\isamarkupfalse%
\ {\isacharquery}thesis\ \isacommand{apply}\isamarkupfalse%
\ {\isacharparenleft}unfold\ submodule{\isacharunderscore}def\ span{\isacharunderscore}def\ lincomb{\isacharunderscore}def{\isacharcomma}\ auto{\isacharparenright}\ \isacommand{done}\isamarkupfalse%
\isanewline
\isacommand{next}\isamarkupfalse%
\ \isanewline
\ \ \isacommand{case}\isamarkupfalse%
\ False\isanewline
\ \ \isacommand{show}\isamarkupfalse%
\ {\isacharquery}thesis\isanewline
\ \ \isacommand{proof}\isamarkupfalse%
\ {\isacharparenleft}rule\ nested{\isacharunderscore}union{\isacharunderscore}vs{\isacharbrackleft}\isakeyword{where}\ {\isacharquery}I{\isacharequal}{\isachardoublequoteopen}{\isacharbraceleft}F{\isachardot}\ F{\isasymsubseteq}S\ {\isasymand}\ finite\ F{\isacharbraceright}{\isachardoublequoteclose}\ \isakeyword{and}\ {\isacharquery}N{\isacharequal}{\isachardoublequoteopen}{\isasymlambda}F{\isachardot}\ span\ F{\isachardoublequoteclose}\ \isakeyword{and}\ {\isacharquery}N{\isacharprime}{\isacharequal}{\isachardoublequoteopen}span\ S{\isachardoublequoteclose}{\isacharbrackright}{\isacharparenright}\isanewline
\ \ \ \ \isacommand{show}\isamarkupfalse%
\ {\isachardoublequoteopen}\ {\isasymAnd}F{\isachardot}\ F\ {\isasymin}\ {\isacharbraceleft}F{\isachardot}\ F\ {\isasymsubseteq}\ S\ {\isasymand}\ finite\ F{\isacharbraceright}\ {\isasymLongrightarrow}\ submodule\ R\ {\isacharparenleft}span\ F{\isacharparenright}\ M{\isachardoublequoteclose}\isanewline
\ \ \ \ \isacommand{proof}\isamarkupfalse%
\ {\isacharminus}\ \isanewline
\ \ \ \ \ \ \isacommand{fix}\isamarkupfalse%
\ F\isanewline
\ \ \ \ \ \ \isacommand{assume}\isamarkupfalse%
\ F{\isacharcolon}\ {\isachardoublequoteopen}F\ {\isasymin}\ {\isacharbraceleft}F{\isachardot}\ F\ {\isasymsubseteq}\ S\ {\isasymand}\ finite\ F{\isacharbraceright}{\isachardoublequoteclose}\isanewline
\ \ \ \ \ \ \isacommand{from}\isamarkupfalse%
\ F\ \isacommand{have}\isamarkupfalse%
\ h{\isadigit{1}}{\isacharcolon}\ {\isachardoublequoteopen}finite\ F{\isachardoublequoteclose}\ \isacommand{by}\isamarkupfalse%
\ auto\isanewline
\ \ \ \ \ \ \isacommand{from}\isamarkupfalse%
\ F\ h{\isadigit{2}}\ \isacommand{have}\isamarkupfalse%
\ inC{\isacharcolon}\ {\isachardoublequoteopen}F{\isasymsubseteq}carrier\ M{\isachardoublequoteclose}\ \isacommand{by}\isamarkupfalse%
\ auto\isanewline
\ \ \ \ \ \ \isacommand{from}\isamarkupfalse%
\ h{\isadigit{1}}\ inC\ \isacommand{interpret}\isamarkupfalse%
\ mh{\isacharcolon}\ mod{\isacharunderscore}hom\ R\ {\isachardoublequoteopen}{\isacharparenleft}func{\isacharunderscore}space\ F{\isacharparenright}{\isachardoublequoteclose}\ M\ {\isachardoublequoteopen}{\isacharparenleft}{\isasymlambda}a{\isachardot}\ lincomb\ a\ F{\isacharparenright}{\isachardoublequoteclose}\ \isanewline
\ \ \ \ \ \ \ \ \isacommand{by}\isamarkupfalse%
\ {\isacharparenleft}rule\ lincomb{\isacharunderscore}is{\isacharunderscore}mod{\isacharunderscore}hom{\isacharparenright}\isanewline
\ \ \ \ \ \ \isacommand{from}\isamarkupfalse%
\ h{\isadigit{1}}\ inC\ \isacommand{have}\isamarkupfalse%
\ {\isadigit{1}}{\isacharcolon}\ {\isachardoublequoteopen}mh{\isachardot}im\ {\isacharequal}\ span\ F{\isachardoublequoteclose}\ \isanewline
\ \ \ \ \ \ \ \ \isacommand{apply}\isamarkupfalse%
\ {\isacharparenleft}unfold\ mh{\isachardot}im{\isacharunderscore}def{\isacharparenright}\ \isanewline
\ \ \ \ \ \ \ \ \isacommand{apply}\isamarkupfalse%
\ {\isacharparenleft}unfold\ func{\isacharunderscore}space{\isacharunderscore}def{\isacharcomma}\ simp{\isacharparenright}\ \isanewline
\ \ \ \ \ \ \ \ \isacommand{apply}\isamarkupfalse%
\ {\isacharparenleft}subst\ finite{\isacharunderscore}span{\isacharcomma}\ auto{\isacharparenright}\isanewline
\ \ \ \ \ \ \ \ \isacommand{apply}\isamarkupfalse%
\ {\isacharparenleft}unfold\ image{\isacharunderscore}def{\isacharcomma}\ auto{\isacharparenright}\isanewline
\ \ \ \ \ \ \ \ \isacommand{apply}\isamarkupfalse%
\ {\isacharparenleft}rule{\isacharunderscore}tac\ x{\isacharequal}{\isachardoublequoteopen}restrict\ a\ F{\isachardoublequoteclose}\ \isakeyword{in}\ bexI{\isacharparenright}\isanewline
\ \ \ \ \ \ \ \ \ \isacommand{by}\isamarkupfalse%
\ {\isacharparenleft}auto\ intro{\isacharbang}{\isacharcolon}\ lincomb{\isacharunderscore}cong{\isacharparenright}\isanewline
\ \ \ \ \ \ \isacommand{from}\isamarkupfalse%
\ {\isadigit{1}}\ \isacommand{show}\isamarkupfalse%
\ {\isachardoublequoteopen}submodule\ R\ {\isacharparenleft}span\ F{\isacharparenright}\ M{\isachardoublequoteclose}\ \isacommand{by}\isamarkupfalse%
\ {\isacharparenleft}metis\ mh{\isachardot}im{\isacharunderscore}is{\isacharunderscore}submodule{\isacharparenright}\isanewline
\ \ \ \ \isacommand{qed}\isamarkupfalse%
\isanewline
\ \ \isacommand{next}\isamarkupfalse%
\ \isanewline
\ \ \ \ \isacommand{show}\isamarkupfalse%
\ {\isachardoublequoteopen}{\isasymAnd}i\ j{\isachardot}\ i\ {\isasymin}\ {\isacharbraceleft}F{\isachardot}\ F\ {\isasymsubseteq}\ S\ {\isasymand}\ finite\ F{\isacharbraceright}\ {\isasymLongrightarrow}\isanewline
\ \ \ \ \ \ \ \ \ \ \ j\ {\isasymin}\ {\isacharbraceleft}F{\isachardot}\ F\ {\isasymsubseteq}\ S\ {\isasymand}\ finite\ F{\isacharbraceright}\ {\isasymLongrightarrow}\isanewline
\ \ \ \ \ \ \ \ \ \ \ {\isasymexists}k{\isachardot}\ k\ {\isasymin}\ {\isacharbraceleft}F{\isachardot}\ F\ {\isasymsubseteq}\ S\ {\isasymand}\ finite\ F{\isacharbraceright}\ {\isasymand}\ span\ i\ {\isasymsubseteq}\ span\ k\ {\isasymand}\ span\ j\ {\isasymsubseteq}\ span\ k{\isachardoublequoteclose}\isanewline
\ \ \ \ \isacommand{proof}\isamarkupfalse%
\ {\isacharminus}\isanewline
\ \ \ \ \ \ \isacommand{fix}\isamarkupfalse%
\ i\ j\ \isanewline
\ \ \ \ \ \ \isacommand{assume}\isamarkupfalse%
\ i{\isacharcolon}\ {\isachardoublequoteopen}i\ {\isasymin}\ {\isacharbraceleft}F{\isachardot}\ F\ {\isasymsubseteq}\ S\ {\isasymand}\ finite\ F{\isacharbraceright}{\isachardoublequoteclose}\ \isakeyword{and}\ j{\isacharcolon}\ {\isachardoublequoteopen}j\ {\isasymin}\ {\isacharbraceleft}F{\isachardot}\ F\ {\isasymsubseteq}\ S\ {\isasymand}\ finite\ F{\isacharbraceright}{\isachardoublequoteclose}\isanewline
\ \ \ \ \ \ \isacommand{from}\isamarkupfalse%
\ i\ j\ \isacommand{show}\isamarkupfalse%
\ {\isachardoublequoteopen}{\isacharquery}thesis\ i\ j{\isachardoublequoteclose}\isanewline
\ \ \ \ \ \ \ \ \isacommand{apply}\isamarkupfalse%
\ {\isacharparenleft}rule{\isacharunderscore}tac\ x{\isacharequal}{\isachardoublequoteopen}i{\isasymunion}j{\isachardoublequoteclose}\ \isakeyword{in}\ exI{\isacharparenright}\isanewline
\ \ \ \ \ \ \ \ \isacommand{apply}\isamarkupfalse%
\ {\isacharparenleft}auto\ del{\isacharcolon}\ subsetI{\isacharparenright}\isanewline
\ \ \ \ \ \ \ \ \ \isacommand{by}\isamarkupfalse%
\ {\isacharparenleft}intro\ span{\isacharunderscore}is{\isacharunderscore}monotone{\isacharcomma}\ auto\ del{\isacharcolon}\ subsetI{\isacharparenright}{\isacharplus}\isanewline
\ \ \ \ \isacommand{qed}\isamarkupfalse%
\isanewline
\ \ \isacommand{next}\isamarkupfalse%
\isanewline
\ \ \ \ \isacommand{show}\isamarkupfalse%
\ {\isachardoublequoteopen}span\ S{\isacharequal}{\isacharparenleft}{\isasymUnion}\ i{\isasymin}{\isacharbraceleft}F{\isachardot}\ F\ {\isasymsubseteq}\ S\ {\isasymand}\ finite\ F{\isacharbraceright}{\isachardot}\ span\ i{\isacharparenright}{\isachardoublequoteclose}\isanewline
\ \ \ \ \ \ \isacommand{by}\isamarkupfalse%
\ {\isacharparenleft}unfold\ span{\isacharunderscore}def{\isacharcomma}auto{\isacharparenright}\isanewline
\ \ \isacommand{next}\isamarkupfalse%
\ \isanewline
\ \ \ \ \isacommand{have}\isamarkupfalse%
\ ne{\isacharcolon}\ {\isachardoublequoteopen}S{\isasymnoteq}{\isacharbraceleft}{\isacharbraceright}{\isachardoublequoteclose}\ \isacommand{by}\isamarkupfalse%
\ fact\isanewline
\ \ \ \ \isacommand{from}\isamarkupfalse%
\ ne\ \isacommand{show}\isamarkupfalse%
\ {\isachardoublequoteopen}{\isacharbraceleft}F{\isachardot}\ F\ {\isasymsubseteq}\ S\ {\isasymand}\ finite\ F{\isacharbraceright}\ {\isasymnoteq}\ {\isacharbraceleft}{\isacharbraceright}{\isachardoublequoteclose}\ \isacommand{by}\isamarkupfalse%
\ auto\isanewline
\ \ \isacommand{qed}\isamarkupfalse%
\isanewline
\isacommand{qed}\isamarkupfalse%
%
\endisatagproof
{\isafoldproof}%
%
\isadelimproof
%
\endisadelimproof
%
\begin{isamarkuptext}%
A finite sum does not depend on the ambient module. This can be done for monoid, but 
"submonoid" isn't currently defined. (It can be copied, however, for groups\ldots)
This lemma requires a somewhat annoying lemma foldD-not-depend. Then we show that linear combinations, 
linear independence, span do not depend on the ambient module.%
\end{isamarkuptext}%
\isamarkuptrue%
\isacommand{lemma}\isamarkupfalse%
\ {\isacharparenleft}\isakeyword{in}\ module{\isacharparenright}\ finsum{\isacharunderscore}not{\isacharunderscore}depend{\isacharcolon}\isanewline
\ \ \isakeyword{fixes}\ a\ A\ N\isanewline
\ \ \isakeyword{assumes}\ h{\isadigit{1}}{\isacharcolon}\ {\isachardoublequoteopen}finite\ A{\isachardoublequoteclose}\ \isakeyword{and}\ h{\isadigit{2}}{\isacharcolon}\ {\isachardoublequoteopen}A{\isasymsubseteq}N{\isachardoublequoteclose}\ \isakeyword{and}\ h{\isadigit{3}}{\isacharcolon}\ {\isachardoublequoteopen}submodule\ R\ N\ M{\isachardoublequoteclose}\ \isakeyword{and}\ h{\isadigit{4}}{\isacharcolon}\ {\isachardoublequoteopen}f{\isacharcolon}A{\isasymrightarrow}N{\isachardoublequoteclose}\isanewline
\ \ \isakeyword{shows}\ {\isachardoublequoteopen}{\isacharparenleft}{\isasymOplus}\isactrlbsub {\isacharparenleft}md\ N{\isacharparenright}\isactrlesub \ v{\isasymin}A{\isachardot}\ f\ v{\isacharparenright}\ {\isacharequal}\ {\isacharparenleft}{\isasymOplus}\isactrlbsub M\isactrlesub \ v{\isasymin}A{\isachardot}\ f\ v{\isacharparenright}{\isachardoublequoteclose}\isanewline
%
\isadelimproof
%
\endisadelimproof
%
\isatagproof
\isacommand{proof}\isamarkupfalse%
\ {\isacharminus}\isanewline
\ \ \isacommand{from}\isamarkupfalse%
\ h{\isadigit{1}}\ h{\isadigit{2}}\ h{\isadigit{3}}\ h{\isadigit{4}}\ \isacommand{show}\isamarkupfalse%
\ {\isacharquery}thesis\isanewline
\ \ \ \ \isacommand{apply}\isamarkupfalse%
\ {\isacharparenleft}unfold\ finsum{\isacharunderscore}def\ finprod{\isacharunderscore}def{\isacharparenright}\isanewline
\ \ \ \ \isacommand{apply}\isamarkupfalse%
\ simp\isanewline
\ \ \ \ \isacommand{apply}\isamarkupfalse%
\ {\isacharparenleft}intro\ foldD{\isacharunderscore}not{\isacharunderscore}depend{\isacharbrackleft}\isakeyword{where}\ {\isacharquery}B{\isacharequal}{\isachardoublequoteopen}A{\isachardoublequoteclose}{\isacharbrackright}{\isacharparenright}\isanewline
\ \ \ \ \ \ \ \ \ \isacommand{apply}\isamarkupfalse%
\ {\isacharparenleft}unfold\ submodule{\isacharunderscore}def\ LCD{\isacharunderscore}def{\isacharcomma}\ auto{\isacharparenright}\isanewline
\ \ \ \ \ \isacommand{by}\isamarkupfalse%
\ {\isacharparenleft}drule\ Pi{\isacharunderscore}implies{\isacharunderscore}Pi{\isadigit{2}}{\isacharcomma}\ simp{\isacharunderscore}all\ add{\isacharcolon}\ a{\isacharunderscore}ac\ Pi{\isacharunderscore}mem{\isacharunderscore}Pi{\isadigit{2}}{\isacharunderscore}sub{\isadigit{2}}\ ring{\isacharunderscore}subset{\isacharunderscore}carrier{\isacharparenright}{\isacharplus}\isanewline
\isacommand{qed}\isamarkupfalse%
%
\endisatagproof
{\isafoldproof}%
%
\isadelimproof
\isanewline
%
\endisadelimproof
\isanewline
\isacommand{lemma}\isamarkupfalse%
\ {\isacharparenleft}\isakeyword{in}\ module{\isacharparenright}\ lincomb{\isacharunderscore}not{\isacharunderscore}depend{\isacharcolon}\isanewline
\ \ \isakeyword{fixes}\ a\ A\ N\isanewline
\ \ \isakeyword{assumes}\ h{\isadigit{1}}{\isacharcolon}\ {\isachardoublequoteopen}finite\ A{\isachardoublequoteclose}\ \isakeyword{and}\ h{\isadigit{2}}{\isacharcolon}\ {\isachardoublequoteopen}A{\isasymsubseteq}N{\isachardoublequoteclose}\ \isakeyword{and}\ h{\isadigit{3}}{\isacharcolon}\ {\isachardoublequoteopen}submodule\ R\ N\ M{\isachardoublequoteclose}\ \isakeyword{and}\ h{\isadigit{4}}{\isacharcolon}\ {\isachardoublequoteopen}a{\isacharcolon}A{\isasymrightarrow}carrier\ R{\isachardoublequoteclose}\isanewline
\ \ \isakeyword{shows}\ {\isachardoublequoteopen}lincomb\ a\ A\ {\isacharequal}\ module{\isachardot}lincomb\ {\isacharparenleft}md\ N{\isacharparenright}\ a\ A{\isachardoublequoteclose}\isanewline
%
\isadelimproof
%
\endisadelimproof
%
\isatagproof
\isacommand{proof}\isamarkupfalse%
\ {\isacharminus}\ \isanewline
\ \ \isacommand{from}\isamarkupfalse%
\ h{\isadigit{3}}\ \isacommand{interpret}\isamarkupfalse%
\ N{\isacharcolon}\ module\ R\ {\isachardoublequoteopen}{\isacharparenleft}md\ N{\isacharparenright}{\isachardoublequoteclose}\ \isacommand{by}\isamarkupfalse%
\ {\isacharparenleft}rule\ submodule{\isacharunderscore}is{\isacharunderscore}module{\isacharparenright}\isanewline
\ \ \isacommand{have}\isamarkupfalse%
\ {\isadigit{3}}{\isacharcolon}\ {\isachardoublequoteopen}N{\isacharequal}carrier\ {\isacharparenleft}md\ N{\isacharparenright}{\isachardoublequoteclose}\ \isacommand{by}\isamarkupfalse%
\ auto\isanewline
\ \ \isacommand{have}\isamarkupfalse%
\ {\isadigit{4}}{\isacharcolon}\ {\isachardoublequoteopen}{\isacharparenleft}smult\ M\ {\isacharparenright}\ {\isacharequal}\ {\isacharparenleft}smult\ {\isacharparenleft}md\ N{\isacharparenright}{\isacharparenright}{\isachardoublequoteclose}\ \isacommand{by}\isamarkupfalse%
\ auto\ \isanewline
\ \ \isacommand{from}\isamarkupfalse%
\ h{\isadigit{1}}\ h{\isadigit{2}}\ h{\isadigit{3}}\ h{\isadigit{4}}\ \isacommand{have}\isamarkupfalse%
\ {\isachardoublequoteopen}{\isacharparenleft}{\isasymOplus}\isactrlbsub {\isacharparenleft}md\ N{\isacharparenright}\isactrlesub v{\isasymin}A{\isachardot}\ a\ v\ {\isasymodot}\isactrlbsub M\isactrlesub \ v{\isacharparenright}\ {\isacharequal}\ {\isacharparenleft}{\isasymOplus}\isactrlbsub M\isactrlesub v{\isasymin}A{\isachardot}\ a\ v\ {\isasymodot}\isactrlbsub M\isactrlesub \ v{\isacharparenright}{\isachardoublequoteclose}\isanewline
\ \ \ \ \isacommand{apply}\isamarkupfalse%
\ {\isacharparenleft}intro\ finsum{\isacharunderscore}not{\isacharunderscore}depend{\isacharcomma}\ auto{\isacharparenright}\isanewline
\ \ \ \ \isacommand{apply}\isamarkupfalse%
\ {\isacharparenleft}subst\ {\isadigit{3}}{\isacharparenright}\isanewline
\ \ \ \ \isacommand{apply}\isamarkupfalse%
\ {\isacharparenleft}subst\ {\isadigit{4}}{\isacharparenright}\isanewline
\ \ \ \ \isacommand{apply}\isamarkupfalse%
\ {\isacharparenleft}intro\ N{\isachardot}smult{\isacharunderscore}closed{\isacharparenright}\isanewline
\ \ \ \ \ \isacommand{by}\isamarkupfalse%
\ {\isacharparenleft}drule\ Pi{\isacharunderscore}implies{\isacharunderscore}Pi{\isadigit{2}}{\isacharcomma}\ auto\ simp\ add{\isacharcolon}\ Pi{\isacharunderscore}simp{\isacharparenright}\isanewline
\ \ \isacommand{from}\isamarkupfalse%
\ this\ \isacommand{show}\isamarkupfalse%
\ {\isacharquery}thesis\ \isacommand{by}\isamarkupfalse%
\ {\isacharparenleft}unfold\ lincomb{\isacharunderscore}def\ N{\isachardot}lincomb{\isacharunderscore}def{\isacharcomma}\ simp{\isacharparenright}\isanewline
\isacommand{qed}\isamarkupfalse%
%
\endisatagproof
{\isafoldproof}%
%
\isadelimproof
\isanewline
%
\endisadelimproof
\isanewline
\isacommand{lemma}\isamarkupfalse%
\ {\isacharparenleft}\isakeyword{in}\ module{\isacharparenright}\ span{\isacharunderscore}li{\isacharunderscore}not{\isacharunderscore}depend{\isacharcolon}\isanewline
\ \ \isakeyword{fixes}\ S\ N\isanewline
\ \ \isakeyword{assumes}\ h{\isadigit{2}}{\isacharcolon}\ {\isachardoublequoteopen}S{\isasymsubseteq}N{\isachardoublequoteclose}\ \isakeyword{and}\ \ h{\isadigit{3}}{\isacharcolon}\ {\isachardoublequoteopen}submodule\ R\ N\ M{\isachardoublequoteclose}\isanewline
\ \ \isakeyword{shows}\ {\isachardoublequoteopen}module{\isachardot}span\ R\ {\isacharparenleft}md\ N{\isacharparenright}\ S\ {\isacharequal}\ module{\isachardot}span\ R\ M\ S{\isachardoublequoteclose}\isanewline
\ \ \ \ \isakeyword{and}\ {\isachardoublequoteopen}module{\isachardot}lin{\isacharunderscore}dep\ R\ {\isacharparenleft}md\ N{\isacharparenright}\ S\ {\isacharequal}\ module{\isachardot}lin{\isacharunderscore}dep\ R\ M\ S{\isachardoublequoteclose}\isanewline
%
\isadelimproof
%
\endisadelimproof
%
\isatagproof
\isacommand{proof}\isamarkupfalse%
\ {\isacharminus}\isanewline
\ \ \isacommand{from}\isamarkupfalse%
\ h{\isadigit{3}}\ \isacommand{interpret}\isamarkupfalse%
\ w{\isacharcolon}\ module\ R\ {\isachardoublequoteopen}{\isacharparenleft}md\ N{\isacharparenright}{\isachardoublequoteclose}\ \isacommand{by}\isamarkupfalse%
\ {\isacharparenleft}rule\ submodule{\isacharunderscore}is{\isacharunderscore}module{\isacharparenright}\isanewline
\ \ \isacommand{from}\isamarkupfalse%
\ h{\isadigit{2}}\ \isacommand{have}\isamarkupfalse%
\ {\isadigit{1}}{\isacharcolon}{\isachardoublequoteopen}submodule\ R\ {\isacharparenleft}module{\isachardot}span\ R\ {\isacharparenleft}md\ N{\isacharparenright}\ S{\isacharparenright}\ {\isacharparenleft}md\ N{\isacharparenright}{\isachardoublequoteclose}\ \isanewline
\ \ \ \ \isacommand{by}\isamarkupfalse%
\ {\isacharparenleft}intro\ w{\isachardot}span{\isacharunderscore}is{\isacharunderscore}submodule{\isacharcomma}\ simp{\isacharparenright}\isanewline
\ \ \isacommand{have}\isamarkupfalse%
\ {\isadigit{3}}{\isacharcolon}\ {\isachardoublequoteopen}{\isasymAnd}a\ A{\isachardot}\ {\isacharparenleft}finite\ A\ {\isasymand}\ A{\isasymsubseteq}S\ {\isasymand}\ a\ {\isasymin}\ A\ {\isasymrightarrow}\ carrier\ R\ {\isasymLongrightarrow}\ \isanewline
\ \ \ \ module{\isachardot}lincomb\ M\ a\ A\ {\isacharequal}\ module{\isachardot}lincomb\ {\isacharparenleft}md\ N{\isacharparenright}\ a\ A{\isacharparenright}{\isachardoublequoteclose}\isanewline
\ \ \isacommand{proof}\isamarkupfalse%
\ {\isacharminus}\ \isanewline
\ \ \ \ \isacommand{fix}\isamarkupfalse%
\ a\ A\isanewline
\ \ \ \ \isacommand{assume}\isamarkupfalse%
\ {\isadigit{3}}{\isadigit{1}}{\isacharcolon}\ {\isachardoublequoteopen}finite\ A\ {\isasymand}\ A{\isasymsubseteq}S\ {\isasymand}\ a\ {\isasymin}\ A\ {\isasymrightarrow}\ carrier\ R{\isachardoublequoteclose}\isanewline
\ \ \ \ \isacommand{from}\isamarkupfalse%
\ assms\ {\isadigit{3}}{\isadigit{1}}\ \isacommand{show}\isamarkupfalse%
\ {\isachardoublequoteopen}{\isacharquery}thesis\ a\ A{\isachardoublequoteclose}\isanewline
\ \ \ \ \ \ \isacommand{by}\isamarkupfalse%
\ {\isacharparenleft}intro\ lincomb{\isacharunderscore}not{\isacharunderscore}depend{\isacharcomma}\ auto{\isacharparenright}\isanewline
\ \ \isacommand{qed}\isamarkupfalse%
\isanewline
\ \ \isacommand{from}\isamarkupfalse%
\ {\isadigit{3}}\ \isacommand{show}\isamarkupfalse%
\ {\isadigit{4}}{\isacharcolon}\ {\isachardoublequoteopen}module{\isachardot}span\ R\ {\isacharparenleft}md\ N{\isacharparenright}\ S\ {\isacharequal}\ module{\isachardot}span\ R\ M\ S{\isachardoublequoteclose}\isanewline
\ \ \ \ \isacommand{apply}\isamarkupfalse%
\ {\isacharparenleft}unfold\ span{\isacharunderscore}def\ w{\isachardot}span{\isacharunderscore}def{\isacharparenright}\isanewline
\ \ \ \ \isacommand{apply}\isamarkupfalse%
\ auto\isanewline
\ \ \ \ \isacommand{by}\isamarkupfalse%
\ {\isacharparenleft}metis{\isacharparenright}\isanewline
\ \ \isacommand{have}\isamarkupfalse%
\ zeros{\isacharcolon}\ {\isachardoublequoteopen}{\isasymzero}\isactrlbsub md\ N\isactrlesub {\isacharequal}{\isasymzero}\isactrlbsub M\isactrlesub {\isachardoublequoteclose}\ \isacommand{by}\isamarkupfalse%
\ auto\isanewline
\ \ \isacommand{from}\isamarkupfalse%
\ assms\ {\isadigit{3}}\ \isacommand{show}\isamarkupfalse%
\ {\isadigit{5}}{\isacharcolon}\ {\isachardoublequoteopen}module{\isachardot}lin{\isacharunderscore}dep\ R\ {\isacharparenleft}md\ N{\isacharparenright}\ S\ {\isacharequal}\ module{\isachardot}lin{\isacharunderscore}dep\ R\ M\ S{\isachardoublequoteclose}\isanewline
\ \ \ \ \isacommand{apply}\isamarkupfalse%
\ {\isacharparenleft}unfold\ lin{\isacharunderscore}dep{\isacharunderscore}def\ w{\isachardot}lin{\isacharunderscore}dep{\isacharunderscore}def{\isacharparenright}\isanewline
\ \ \ \ \isacommand{apply}\isamarkupfalse%
\ {\isacharparenleft}subst\ zeros{\isacharparenright}\ \isanewline
\ \ \ \ \isacommand{by}\isamarkupfalse%
\ metis\isanewline
\isacommand{qed}\isamarkupfalse%
%
\endisatagproof
{\isafoldproof}%
%
\isadelimproof
\isanewline
%
\endisadelimproof
\isanewline
\isacommand{lemma}\isamarkupfalse%
\ {\isacharparenleft}\isakeyword{in}\ module{\isacharparenright}\ span{\isacharunderscore}is{\isacharunderscore}subset{\isacharcolon}\ \isanewline
\ \ \isakeyword{fixes}\ S\ N\isanewline
\ \ \isakeyword{assumes}\ h{\isadigit{2}}{\isacharcolon}\ {\isachardoublequoteopen}S{\isasymsubseteq}N{\isachardoublequoteclose}\ \isakeyword{and}\ \ h{\isadigit{3}}{\isacharcolon}\ {\isachardoublequoteopen}submodule\ R\ N\ M{\isachardoublequoteclose}\isanewline
\ \ \isakeyword{shows}\ {\isachardoublequoteopen}span\ S\ {\isasymsubseteq}\ N{\isachardoublequoteclose}\isanewline
%
\isadelimproof
%
\endisadelimproof
%
\isatagproof
\isacommand{proof}\isamarkupfalse%
\ {\isacharminus}\ \ \isanewline
\ \ \isacommand{from}\isamarkupfalse%
\ h{\isadigit{3}}\ \isacommand{interpret}\isamarkupfalse%
\ w{\isacharcolon}\ module\ R\ {\isachardoublequoteopen}{\isacharparenleft}md\ N{\isacharparenright}{\isachardoublequoteclose}\ \isacommand{by}\isamarkupfalse%
\ {\isacharparenleft}rule\ submodule{\isacharunderscore}is{\isacharunderscore}module{\isacharparenright}\isanewline
\ \ \isacommand{from}\isamarkupfalse%
\ h{\isadigit{2}}\ \isacommand{have}\isamarkupfalse%
\ {\isadigit{1}}{\isacharcolon}{\isachardoublequoteopen}submodule\ R\ {\isacharparenleft}module{\isachardot}span\ R\ {\isacharparenleft}md\ N{\isacharparenright}\ S{\isacharparenright}\ {\isacharparenleft}md\ N{\isacharparenright}{\isachardoublequoteclose}\ \isanewline
\ \ \ \ \isacommand{by}\isamarkupfalse%
\ {\isacharparenleft}intro\ w{\isachardot}span{\isacharunderscore}is{\isacharunderscore}submodule{\isacharcomma}\ simp{\isacharparenright}\isanewline
\ \ \isacommand{from}\isamarkupfalse%
\ assms\ \isacommand{have}\isamarkupfalse%
\ {\isadigit{4}}{\isacharcolon}\ {\isachardoublequoteopen}module{\isachardot}span\ R\ {\isacharparenleft}md\ N{\isacharparenright}\ S\ {\isacharequal}\ module{\isachardot}span\ R\ M\ S{\isachardoublequoteclose}\isanewline
\ \ \ \ \ \isacommand{by}\isamarkupfalse%
\ {\isacharparenleft}rule\ span{\isacharunderscore}li{\isacharunderscore}not{\isacharunderscore}depend{\isacharparenright}\isanewline
\ \ \isacommand{from}\isamarkupfalse%
\ {\isadigit{1}}\ {\isadigit{4}}\ \isacommand{have}\isamarkupfalse%
\ {\isadigit{5}}{\isacharcolon}\ {\isachardoublequoteopen}submodule\ R\ {\isacharparenleft}module{\isachardot}span\ R\ M\ S{\isacharparenright}\ {\isacharparenleft}md\ N{\isacharparenright}{\isachardoublequoteclose}\ \isacommand{by}\isamarkupfalse%
\ auto\isanewline
\ \ \isacommand{from}\isamarkupfalse%
\ {\isadigit{5}}\ \isacommand{show}\isamarkupfalse%
\ {\isacharquery}thesis\ \isacommand{by}\isamarkupfalse%
\ {\isacharparenleft}unfold\ submodule{\isacharunderscore}def{\isacharcomma}\ simp{\isacharparenright}\isanewline
\isacommand{qed}\isamarkupfalse%
%
\endisatagproof
{\isafoldproof}%
%
\isadelimproof
\isanewline
%
\endisadelimproof
\isanewline
\isanewline
\isacommand{lemma}\isamarkupfalse%
\ {\isacharparenleft}\isakeyword{in}\ module{\isacharparenright}\ span{\isacharunderscore}is{\isacharunderscore}subset{\isadigit{2}}{\isacharcolon}\isanewline
\ \ \isakeyword{fixes}\ S\isanewline
\ \ \isakeyword{assumes}\ h{\isadigit{2}}{\isacharcolon}\ {\isachardoublequoteopen}S{\isasymsubseteq}carrier\ M{\isachardoublequoteclose}\isanewline
\ \ \isakeyword{shows}\ {\isachardoublequoteopen}span\ S\ {\isasymsubseteq}\ carrier\ M{\isachardoublequoteclose}\isanewline
%
\isadelimproof
%
\endisadelimproof
%
\isatagproof
\isacommand{proof}\isamarkupfalse%
\ {\isacharminus}\ \isanewline
\ \ \isacommand{have}\isamarkupfalse%
\ {\isadigit{0}}{\isacharcolon}\ {\isachardoublequoteopen}module\ R\ M{\isachardoublequoteclose}\isacommand{{\isachardot}{\isachardot}}\isamarkupfalse%
\isanewline
\ \ \isacommand{from}\isamarkupfalse%
\ {\isadigit{0}}\ \isacommand{have}\isamarkupfalse%
\ h{\isadigit{3}}{\isacharcolon}\ {\isachardoublequoteopen}submodule\ R\ {\isacharparenleft}carrier\ M{\isacharparenright}\ M{\isachardoublequoteclose}\ \isacommand{by}\isamarkupfalse%
\ {\isacharparenleft}unfold\ submodule{\isacharunderscore}def{\isacharcomma}\ auto{\isacharparenright}\isanewline
\ \ \isacommand{from}\isamarkupfalse%
\ h{\isadigit{2}}\ h{\isadigit{3}}\ \isacommand{show}\isamarkupfalse%
\ {\isacharquery}thesis\ \isacommand{by}\isamarkupfalse%
\ {\isacharparenleft}rule\ span{\isacharunderscore}is{\isacharunderscore}subset{\isacharparenright}\isanewline
\isacommand{qed}\isamarkupfalse%
%
\endisatagproof
{\isafoldproof}%
%
\isadelimproof
\isanewline
%
\endisadelimproof
\isanewline
\isacommand{lemma}\isamarkupfalse%
\ {\isacharparenleft}\isakeyword{in}\ module{\isacharparenright}\ in{\isacharunderscore}own{\isacharunderscore}span{\isacharcolon}\ \isanewline
\ \ \isakeyword{fixes}\ S\isanewline
\ \ \isakeyword{assumes}\ \ inC{\isacharcolon}{\isachardoublequoteopen}S{\isasymsubseteq}carrier\ M{\isachardoublequoteclose}\isanewline
\ \ \isakeyword{shows}\ {\isachardoublequoteopen}S\ {\isasymsubseteq}\ span\ S{\isachardoublequoteclose}\isanewline
%
\isadelimproof
%
\endisadelimproof
%
\isatagproof
\isacommand{proof}\isamarkupfalse%
\ {\isacharminus}\ \isanewline
\ \ \isacommand{from}\isamarkupfalse%
\ inC\ \isacommand{show}\isamarkupfalse%
\ {\isacharquery}thesis\ \isanewline
\ \ \ \ \isacommand{apply}\isamarkupfalse%
\ {\isacharparenleft}unfold\ span{\isacharunderscore}def{\isacharcomma}\ auto{\isacharparenright}\isanewline
\ \ \ \ \isacommand{apply}\isamarkupfalse%
\ {\isacharparenleft}rename{\isacharunderscore}tac\ v{\isacharparenright}\isanewline
\ \ \ \ \isacommand{apply}\isamarkupfalse%
\ {\isacharparenleft}rule{\isacharunderscore}tac\ x{\isacharequal}{\isachardoublequoteopen}{\isacharparenleft}{\isasymlambda}\ w{\isachardot}\ if\ {\isacharparenleft}w{\isacharequal}v{\isacharparenright}\ then\ {\isasymone}\isactrlbsub R\isactrlesub \ else\ {\isasymzero}\isactrlbsub R\isactrlesub {\isacharparenright}{\isachardoublequoteclose}\ \isakeyword{in}\ exI{\isacharparenright}\isanewline
\ \ \ \ \isacommand{apply}\isamarkupfalse%
\ {\isacharparenleft}rule{\isacharunderscore}tac\ x{\isacharequal}{\isachardoublequoteopen}{\isacharbraceleft}v{\isacharbraceright}{\isachardoublequoteclose}\ \isakeyword{in}\ exI{\isacharparenright}\isanewline
\ \ \ \ \isacommand{apply}\isamarkupfalse%
\ {\isacharparenleft}unfold\ lincomb{\isacharunderscore}def{\isacharparenright}\isanewline
\ \ \ \ \isacommand{by}\isamarkupfalse%
\ {\isacharparenleft}auto\ simp\ add{\isacharcolon}\ finsum{\isacharunderscore}insert{\isacharparenright}\isanewline
\isacommand{qed}\isamarkupfalse%
%
\endisatagproof
{\isafoldproof}%
%
\isadelimproof
\isanewline
%
\endisadelimproof
\isanewline
\isacommand{lemma}\isamarkupfalse%
\ {\isacharparenleft}\isakeyword{in}\ module{\isacharparenright}\ supset{\isacharunderscore}ld{\isacharunderscore}is{\isacharunderscore}ld{\isacharcolon}\isanewline
\ \ \isakeyword{fixes}\ A\ B\isanewline
\ \ \isakeyword{assumes}\ ld{\isacharcolon}\ {\isachardoublequoteopen}lin{\isacharunderscore}dep\ A{\isachardoublequoteclose}\ \isakeyword{and}\ sub{\isacharcolon}\ {\isachardoublequoteopen}A\ {\isasymsubseteq}\ B{\isachardoublequoteclose}\isanewline
\ \ \isakeyword{shows}\ {\isachardoublequoteopen}lin{\isacharunderscore}dep\ B{\isachardoublequoteclose}\isanewline
%
\isadelimproof
%
\endisadelimproof
%
\isatagproof
\isacommand{proof}\isamarkupfalse%
\ {\isacharminus}\ \isanewline
\ \ \isacommand{from}\isamarkupfalse%
\ ld\ \isacommand{obtain}\isamarkupfalse%
\ A{\isacharprime}\ a\ v\ \isakeyword{where}\ {\isadigit{1}}{\isacharcolon}\ {\isachardoublequoteopen}{\isacharparenleft}finite\ A{\isacharprime}\ {\isasymand}\ A{\isacharprime}{\isasymsubseteq}A\ {\isasymand}\ {\isacharparenleft}a{\isasymin}\ {\isacharparenleft}A{\isacharprime}{\isasymrightarrow}carrier\ R{\isacharparenright}{\isacharparenright}\ {\isasymand}\ {\isacharparenleft}lincomb\ a\ A{\isacharprime}\ {\isacharequal}\ {\isasymzero}\isactrlbsub M\isactrlesub {\isacharparenright}\ {\isasymand}\ {\isacharparenleft}v{\isasymin}A{\isacharprime}{\isacharparenright}\ {\isasymand}\ {\isacharparenleft}a\ v{\isasymnoteq}\ {\isasymzero}\isactrlbsub R\isactrlesub {\isacharparenright}{\isacharparenright}{\isachardoublequoteclose}\isanewline
\ \ \ \ \isacommand{by}\isamarkupfalse%
\ {\isacharparenleft}unfold\ lin{\isacharunderscore}dep{\isacharunderscore}def{\isacharcomma}\ auto{\isacharparenright}\isanewline
\ \ \isacommand{from}\isamarkupfalse%
\ {\isadigit{1}}\ sub\ \isacommand{show}\isamarkupfalse%
\ {\isacharquery}thesis\ \isanewline
\ \ \ \ \isacommand{apply}\isamarkupfalse%
\ {\isacharparenleft}unfold\ lin{\isacharunderscore}dep{\isacharunderscore}def{\isacharparenright}\isanewline
\ \ \ \ \isacommand{apply}\isamarkupfalse%
\ {\isacharparenleft}rule{\isacharunderscore}tac\ x{\isacharequal}{\isachardoublequoteopen}A{\isacharprime}{\isachardoublequoteclose}\ \isakeyword{in}\ exI{\isacharparenright}\isanewline
\ \ \ \ \isacommand{apply}\isamarkupfalse%
\ {\isacharparenleft}rule{\isacharunderscore}tac\ x{\isacharequal}{\isachardoublequoteopen}a{\isachardoublequoteclose}\ \isakeyword{in}\ exI{\isacharparenright}\isanewline
\ \ \ \ \isacommand{apply}\isamarkupfalse%
\ {\isacharparenleft}rule{\isacharunderscore}tac\ x{\isacharequal}{\isachardoublequoteopen}v{\isachardoublequoteclose}\ \isakeyword{in}\ exI{\isacharparenright}\isanewline
\ \ \ \ \isacommand{by}\isamarkupfalse%
\ auto\isanewline
\isacommand{qed}\isamarkupfalse%
%
\endisatagproof
{\isafoldproof}%
%
\isadelimproof
\isanewline
%
\endisadelimproof
\isanewline
\isacommand{lemma}\isamarkupfalse%
\ {\isacharparenleft}\isakeyword{in}\ module{\isacharparenright}\ subset{\isacharunderscore}li{\isacharunderscore}is{\isacharunderscore}li{\isacharcolon}\isanewline
\ \ \isakeyword{fixes}\ A\ B\isanewline
\ \ \isakeyword{assumes}\ li{\isacharcolon}\ {\isachardoublequoteopen}lin{\isacharunderscore}indpt\ A{\isachardoublequoteclose}\ \isakeyword{and}\ sub{\isacharcolon}\ {\isachardoublequoteopen}B\ {\isasymsubseteq}\ A{\isachardoublequoteclose}\ \isanewline
\ \ \isakeyword{shows}\ {\isachardoublequoteopen}lin{\isacharunderscore}indpt\ B{\isachardoublequoteclose}\isanewline
%
\isadelimproof
%
\endisadelimproof
%
\isatagproof
\isacommand{proof}\isamarkupfalse%
\ {\isacharparenleft}rule\ ccontr{\isacharparenright}\isanewline
\ \ \isacommand{assume}\isamarkupfalse%
\ ld{\isacharcolon}\ {\isachardoublequoteopen}{\isasymnot}lin{\isacharunderscore}indpt\ B{\isachardoublequoteclose}\isanewline
\ \ \isacommand{from}\isamarkupfalse%
\ ld\ sub\ \isacommand{have}\isamarkupfalse%
\ ldA{\isacharcolon}\ {\isachardoublequoteopen}lin{\isacharunderscore}dep\ A{\isachardoublequoteclose}\ \isacommand{by}\isamarkupfalse%
\ {\isacharparenleft}metis\ supset{\isacharunderscore}ld{\isacharunderscore}is{\isacharunderscore}ld{\isacharparenright}\isanewline
\ \ \isacommand{from}\isamarkupfalse%
\ li\ ldA\ \isacommand{show}\isamarkupfalse%
\ False\ \isacommand{by}\isamarkupfalse%
\ auto\isanewline
\isacommand{qed}\isamarkupfalse%
%
\endisatagproof
{\isafoldproof}%
%
\isadelimproof
\isanewline
%
\endisadelimproof
\isanewline
\isacommand{lemma}\isamarkupfalse%
\ {\isacharparenleft}\isakeyword{in}\ mod{\isacharunderscore}hom{\isacharparenright}\ hom{\isacharunderscore}sum{\isacharcolon}\isanewline
\ \ \isakeyword{fixes}\ A\ B\ g\isanewline
\ \ \isakeyword{assumes}\ h{\isadigit{1}}{\isacharcolon}\ {\isachardoublequoteopen}finite\ A{\isachardoublequoteclose}\ \isakeyword{and}\ h{\isadigit{2}}{\isacharcolon}\ {\isachardoublequoteopen}A{\isasymsubseteq}carrier\ M{\isachardoublequoteclose}\ \isakeyword{and}\ h{\isadigit{3}}{\isacharcolon}\ {\isachardoublequoteopen}g{\isacharcolon}A{\isasymrightarrow}carrier\ M{\isachardoublequoteclose}\isanewline
\ \ \isakeyword{shows}\ {\isachardoublequoteopen}f\ {\isacharparenleft}{\isasymOplus}\isactrlbsub M\isactrlesub \ a{\isasymin}A{\isachardot}\ g\ a{\isacharparenright}\ {\isacharequal}\ {\isacharparenleft}{\isasymOplus}\isactrlbsub N\isactrlesub \ a{\isasymin}A{\isachardot}\ f\ {\isacharparenleft}g\ a{\isacharparenright}{\isacharparenright}{\isachardoublequoteclose}\isanewline
%
\isadelimproof
%
\endisadelimproof
%
\isatagproof
\isacommand{proof}\isamarkupfalse%
\ {\isacharminus}\ \ \ \isanewline
\ \ \isacommand{from}\isamarkupfalse%
\ h{\isadigit{1}}\ h{\isadigit{2}}\ h{\isadigit{3}}\ \isacommand{show}\isamarkupfalse%
\ {\isacharquery}thesis\isanewline
\ \ \isacommand{proof}\isamarkupfalse%
\ {\isacharparenleft}induct\ set{\isacharcolon}\ finite{\isacharparenright}\ \isanewline
\ \ \ \ \isacommand{case}\isamarkupfalse%
\ empty\isanewline
\ \ \ \ \isacommand{show}\isamarkupfalse%
\ {\isacharquery}case\ \isacommand{by}\isamarkupfalse%
\ auto\isanewline
\ \ \isacommand{next}\isamarkupfalse%
\isanewline
\ \ \ \ \isacommand{case}\isamarkupfalse%
\ {\isacharparenleft}insert\ a\ A{\isacharparenright}\isanewline
\ \ \ \ \isacommand{from}\isamarkupfalse%
\ insert{\isachardot}prems\ insert{\isachardot}hyps\ \isacommand{have}\isamarkupfalse%
\ {\isadigit{1}}{\isacharcolon}\ {\isachardoublequoteopen}{\isacharparenleft}{\isasymOplus}\isactrlbsub N\isactrlesub a{\isasymin}insert\ a\ A{\isachardot}\ f\ {\isacharparenleft}g\ a{\isacharparenright}{\isacharparenright}\ {\isacharequal}\ f\ {\isacharparenleft}g\ a{\isacharparenright}\ {\isasymoplus}\isactrlbsub N\isactrlesub \ {\isacharparenleft}{\isasymOplus}\isactrlbsub N\isactrlesub a{\isasymin}A{\isachardot}\ f\ {\isacharparenleft}g\ a{\isacharparenright}{\isacharparenright}{\isachardoublequoteclose}\ \ \isanewline
\ \ \ \ \ \ \isacommand{by}\isamarkupfalse%
\ {\isacharparenleft}intro\ finsum{\isacharunderscore}insert{\isacharcomma}\ auto{\isacharparenright}\isanewline
\ \ \ \ \ \ \isanewline
\ \ \ \ \isacommand{from}\isamarkupfalse%
\ insert{\isachardot}prems\ insert{\isachardot}hyps\ {\isadigit{1}}\ \isacommand{show}\isamarkupfalse%
\ {\isacharquery}case\isanewline
\ \ \ \ \ \ \isacommand{by}\isamarkupfalse%
\ {\isacharparenleft}simp\ add{\isacharcolon}\ finsum{\isacharunderscore}insert{\isacharparenright}\isanewline
\ \ \isacommand{qed}\isamarkupfalse%
\isanewline
\isacommand{qed}\isamarkupfalse%
%
\endisatagproof
{\isafoldproof}%
%
\isadelimproof
\isanewline
%
\endisadelimproof
\isanewline
%
\isadelimtheory
\isanewline
%
\endisadelimtheory
%
\isatagtheory
\isacommand{end}\isamarkupfalse%
%
\endisatagtheory
{\isafoldtheory}%
%
\isadelimtheory
%
\endisadelimtheory
\end{isabellebody}%
%%% Local Variables:
%%% mode: latex
%%% TeX-master: "root"
%%% End:


%
\begin{isabellebody}%
\def\isabellecontext{SumSpaces}%
%
\isamarkupheader{The direct sum of modules.%
}
\isamarkuptrue%
%
\isadelimtheory
%
\endisadelimtheory
%
\isatagtheory
\isacommand{theory}\isamarkupfalse%
\ SumSpaces\isanewline
\isakeyword{imports}\ Main\isanewline
\ \ {\isachardoublequoteopen}{\isachartilde}{\isachartilde}{\isacharslash}src{\isacharslash}HOL{\isacharslash}Algebra{\isacharslash}Module{\isachardoublequoteclose}\isanewline
\ \ {\isachardoublequoteopen}{\isachartilde}{\isachartilde}{\isacharslash}src{\isacharslash}HOL{\isacharslash}Algebra{\isacharslash}Coset{\isachardoublequoteclose}\isanewline
\ \ RingModuleFacts\isanewline
\ \ MonoidSums\isanewline
\ \ FunctionLemmas\isanewline
\ \ LinearCombinations\isanewline
\isakeyword{begin}%
\endisatagtheory
{\isafoldtheory}%
%
\isadelimtheory
%
\endisadelimtheory
%
\begin{isamarkuptext}%
We define the direct sum $M_1\oplus M_2$ of 2 vector spaces as the set $M_1\times M_2$ under
componentwise addition and scalar multiplication.%
\end{isamarkuptext}%
\isamarkuptrue%
\isacommand{definition}\isamarkupfalse%
\ direct{\isacharunderscore}sum{\isacharcolon}{\isacharcolon}\ {\isachardoublequoteopen}{\isacharparenleft}{\isacharprime}a{\isacharcomma}{\isacharprime}b{\isacharcomma}\ {\isacharprime}d{\isacharparenright}\ module{\isacharunderscore}scheme\ {\isasymRightarrow}\ {\isacharparenleft}{\isacharprime}a{\isacharcomma}\ {\isacharprime}c{\isacharcomma}\ {\isacharprime}e{\isacharparenright}\ module{\isacharunderscore}scheme\ {\isasymRightarrow}{\isacharparenleft}{\isacharprime}a{\isacharcomma}\ {\isacharparenleft}{\isacharprime}b{\isasymtimes}{\isacharprime}c{\isacharparenright}{\isacharparenright}\ module{\isachardoublequoteclose}\ \isanewline
\ \ \isakeyword{where}\ \ {\isachardoublequoteopen}direct{\isacharunderscore}sum\ M{\isadigit{1}}\ M{\isadigit{2}}\ {\isacharequal}\ {\isasymlparr}carrier\ {\isacharequal}\ carrier\ M{\isadigit{1}}\ {\isasymtimes}\ carrier\ M{\isadigit{2}}{\isacharcomma}\ \isanewline
\ \ \ \ \ \ \ \ \ \ \ \ \ \ \ \ \ \ mult\ {\isacharequal}\ {\isacharparenleft}{\isasymlambda}\ v\ w{\isachardot}\ {\isacharparenleft}{\isasymzero}\isactrlbsub M{\isadigit{1}}\isactrlesub {\isacharcomma}\ {\isasymzero}\isactrlbsub M{\isadigit{2}}\isactrlesub {\isacharparenright}{\isacharparenright}{\isacharcomma}\isanewline
\ \ \ \ \ \ \ \ \ \ \ \ \ \ \ \ \ \ one\ {\isacharequal}\ \ {\isacharparenleft}{\isasymzero}\isactrlbsub M{\isadigit{1}}\isactrlesub {\isacharcomma}\ {\isasymzero}\isactrlbsub M{\isadigit{2}}\isactrlesub {\isacharparenright}{\isacharcomma}\isanewline
\ \ \ \ \ \ \ \ \ \ \ \ \ \ \ \ \ \ zero\ {\isacharequal}\ {\isacharparenleft}{\isasymzero}\isactrlbsub M{\isadigit{1}}\isactrlesub {\isacharcomma}\ {\isasymzero}\isactrlbsub M{\isadigit{2}}\isactrlesub {\isacharparenright}{\isacharcomma}\isanewline
\ \ \ \ \ \ \ \ \ \ \ \ \ \ \ \ \ \ add\ {\isacharequal}\ {\isacharparenleft}{\isasymlambda}\ v\ w{\isachardot}\ {\isacharparenleft}fst\ v\ {\isasymoplus}\isactrlbsub M{\isadigit{1}}\isactrlesub \ fst\ w{\isacharcomma}\ snd\ v\ {\isasymoplus}\isactrlbsub M{\isadigit{2}}\isactrlesub \ snd\ w{\isacharparenright}{\isacharparenright}{\isacharcomma}\isanewline
\ \ \ \ \ \ \ \ \ \ \ \ \ \ \ \ \ \ smult\ {\isacharequal}\ {\isacharparenleft}{\isasymlambda}\ c\ v{\isachardot}\ {\isacharparenleft}c\ {\isasymodot}\isactrlbsub M{\isadigit{1}}\isactrlesub \ fst\ v{\isacharcomma}\ c\ {\isasymodot}\isactrlbsub M{\isadigit{2}}\isactrlesub \ snd\ v{\isacharparenright}{\isacharparenright}{\isasymrparr}{\isachardoublequoteclose}\isanewline
\isanewline
\isacommand{lemma}\isamarkupfalse%
\ direct{\isacharunderscore}sum{\isacharunderscore}is{\isacharunderscore}module{\isacharcolon}\ \isanewline
\ \ \isakeyword{fixes}\ R\ M{\isadigit{1}}\ M{\isadigit{2}}\isanewline
\ \ \isakeyword{assumes}\ h{\isadigit{1}}{\isacharcolon}\ {\isachardoublequoteopen}module\ R\ M{\isadigit{1}}{\isachardoublequoteclose}\ \isakeyword{and}\ h{\isadigit{2}}{\isacharcolon}\ {\isachardoublequoteopen}module\ R\ M{\isadigit{2}}{\isachardoublequoteclose}\isanewline
\ \ \isakeyword{shows}\ {\isachardoublequoteopen}module\ R\ {\isacharparenleft}direct{\isacharunderscore}sum\ M{\isadigit{1}}\ M{\isadigit{2}}{\isacharparenright}{\isachardoublequoteclose}\isanewline
%
\isadelimproof
%
\endisadelimproof
%
\isatagproof
\isacommand{proof}\isamarkupfalse%
\ {\isacharminus}\isanewline
\ \ \isacommand{from}\isamarkupfalse%
\ h{\isadigit{1}}\ \isacommand{have}\isamarkupfalse%
\ {\isadigit{1}}{\isacharcolon}\ {\isachardoublequoteopen}cring\ R{\isachardoublequoteclose}\ \isacommand{by}\isamarkupfalse%
\ {\isacharparenleft}unfold\ module{\isacharunderscore}def{\isacharcomma}\ auto{\isacharparenright}\isanewline
\ \ \isacommand{from}\isamarkupfalse%
\ h{\isadigit{1}}\ \isacommand{interpret}\isamarkupfalse%
\ v{\isadigit{1}}{\isacharcolon}\ module\ R\ M{\isadigit{1}}\ \isacommand{by}\isamarkupfalse%
\ auto\isanewline
\ \ \isacommand{from}\isamarkupfalse%
\ h{\isadigit{2}}\ \isacommand{interpret}\isamarkupfalse%
\ v{\isadigit{2}}{\isacharcolon}\ module\ R\ M{\isadigit{2}}\ \isacommand{by}\isamarkupfalse%
\ auto\isanewline
\ \ \isacommand{from}\isamarkupfalse%
\ h{\isadigit{1}}\ h{\isadigit{2}}\ \isacommand{have}\isamarkupfalse%
\ {\isadigit{2}}{\isacharcolon}\ {\isachardoublequoteopen}abelian{\isacharunderscore}group\ {\isacharparenleft}direct{\isacharunderscore}sum\ M{\isadigit{1}}\ M{\isadigit{2}}{\isacharparenright}{\isachardoublequoteclose}\ \isanewline
\ \ \ \ \isacommand{apply}\isamarkupfalse%
\ {\isacharparenleft}intro\ abelian{\isacharunderscore}groupI{\isacharcomma}\ auto{\isacharparenright}\isanewline
\ \ \ \ \ \ \ \ \ \isacommand{apply}\isamarkupfalse%
\ {\isacharparenleft}unfold\ direct{\isacharunderscore}sum{\isacharunderscore}def{\isacharcomma}\ auto{\isacharparenright}\isanewline
\ \ \ \ \ \ \ \isacommand{by}\isamarkupfalse%
\ {\isacharparenleft}auto\ simp\ add{\isacharcolon}\ v{\isadigit{1}}{\isachardot}a{\isacharunderscore}ac\ v{\isadigit{2}}{\isachardot}a{\isacharunderscore}ac{\isacharparenright}\isanewline
\ \ \isacommand{from}\isamarkupfalse%
\ h{\isadigit{1}}\ h{\isadigit{2}}\ assms\ \isacommand{have}\isamarkupfalse%
\ {\isadigit{3}}{\isacharcolon}\ {\isachardoublequoteopen}module{\isacharunderscore}axioms\ R\ {\isacharparenleft}direct{\isacharunderscore}sum\ M{\isadigit{1}}\ M{\isadigit{2}}{\isacharparenright}{\isachardoublequoteclose}\isanewline
\ \ \ \ \isacommand{apply}\isamarkupfalse%
\ {\isacharparenleft}unfold\ module{\isacharunderscore}axioms{\isacharunderscore}def{\isacharcomma}\ auto{\isacharparenright}\isanewline
\ \ \ \ \ \ \ \ \isacommand{apply}\isamarkupfalse%
\ {\isacharparenleft}unfold\ direct{\isacharunderscore}sum{\isacharunderscore}def{\isacharcomma}\ auto{\isacharparenright}\isanewline
\ \ \ \ \ \ \ \ \ \isacommand{by}\isamarkupfalse%
\ {\isacharparenleft}auto\ simp\ add{\isacharcolon}\ v{\isadigit{1}}{\isachardot}smult{\isacharunderscore}l{\isacharunderscore}distr\ v{\isadigit{2}}{\isachardot}smult{\isacharunderscore}l{\isacharunderscore}distr\ v{\isadigit{1}}{\isachardot}smult{\isacharunderscore}r{\isacharunderscore}distr\ v{\isadigit{2}}{\isachardot}smult{\isacharunderscore}r{\isacharunderscore}distr\isanewline
\ \ \ \ \ \ v{\isadigit{1}}{\isachardot}smult{\isacharunderscore}assoc{\isadigit{1}}\ v{\isadigit{2}}{\isachardot}smult{\isacharunderscore}assoc{\isadigit{1}}{\isacharparenright}\isanewline
\ \ \isacommand{from}\isamarkupfalse%
\ {\isadigit{1}}\ {\isadigit{2}}\ {\isadigit{3}}\ \ \isacommand{show}\isamarkupfalse%
\ {\isacharquery}thesis\ \isacommand{by}\isamarkupfalse%
\ {\isacharparenleft}unfold\ module{\isacharunderscore}def{\isacharcomma}\ auto{\isacharparenright}\isanewline
\isacommand{qed}\isamarkupfalse%
%
\endisatagproof
{\isafoldproof}%
%
\isadelimproof
\isanewline
%
\endisadelimproof
\isanewline
\isacommand{definition}\isamarkupfalse%
\ inj{\isadigit{1}}{\isacharcolon}{\isacharcolon}\ {\isachardoublequoteopen}{\isacharparenleft}{\isacharprime}a{\isacharcomma}{\isacharprime}b{\isacharparenright}\ module\ {\isasymRightarrow}\ {\isacharparenleft}{\isacharprime}a{\isacharcomma}\ {\isacharprime}c{\isacharparenright}\ module\ {\isasymRightarrow}{\isacharparenleft}{\isacharprime}b{\isasymRightarrow}{\isacharparenleft}{\isacharprime}b{\isasymtimes}{\isacharprime}c{\isacharparenright}{\isacharparenright}{\isachardoublequoteclose}\isanewline
\ \ \isakeyword{where}\ {\isachardoublequoteopen}inj{\isadigit{1}}\ M{\isadigit{1}}\ M{\isadigit{2}}\ {\isacharequal}\ {\isacharparenleft}{\isasymlambda}v{\isachardot}\ {\isacharparenleft}v{\isacharcomma}\ {\isasymzero}\isactrlbsub M{\isadigit{2}}\isactrlesub {\isacharparenright}{\isacharparenright}{\isachardoublequoteclose}\isanewline
\isanewline
\isacommand{definition}\isamarkupfalse%
\ inj{\isadigit{2}}{\isacharcolon}{\isacharcolon}\ {\isachardoublequoteopen}{\isacharparenleft}{\isacharprime}a{\isacharcomma}{\isacharprime}b{\isacharparenright}\ module\ {\isasymRightarrow}\ {\isacharparenleft}{\isacharprime}a{\isacharcomma}\ {\isacharprime}c{\isacharparenright}\ module\ {\isasymRightarrow}{\isacharparenleft}{\isacharprime}c{\isasymRightarrow}{\isacharparenleft}{\isacharprime}b{\isasymtimes}{\isacharprime}c{\isacharparenright}{\isacharparenright}{\isachardoublequoteclose}\isanewline
\ \ \isakeyword{where}\ {\isachardoublequoteopen}inj{\isadigit{2}}\ M{\isadigit{1}}\ M{\isadigit{2}}\ {\isacharequal}\ {\isacharparenleft}{\isasymlambda}v{\isachardot}\ {\isacharparenleft}{\isasymzero}\isactrlbsub M{\isadigit{1}}\isactrlesub {\isacharcomma}\ v{\isacharparenright}{\isacharparenright}{\isachardoublequoteclose}\isanewline
\isanewline
\isacommand{lemma}\isamarkupfalse%
\ inj{\isadigit{1}}{\isacharunderscore}hom{\isacharcolon}\isanewline
\ \ \isakeyword{fixes}\ R\ M{\isadigit{1}}\ M{\isadigit{2}}\isanewline
\ \ \isakeyword{assumes}\ h{\isadigit{1}}{\isacharcolon}\ {\isachardoublequoteopen}module\ R\ M{\isadigit{1}}{\isachardoublequoteclose}\ \isakeyword{and}\ h{\isadigit{2}}{\isacharcolon}\ {\isachardoublequoteopen}module\ R\ M{\isadigit{2}}{\isachardoublequoteclose}\isanewline
\ \ \isakeyword{shows}\ {\isachardoublequoteopen}mod{\isacharunderscore}hom\ R\ M{\isadigit{1}}\ {\isacharparenleft}direct{\isacharunderscore}sum\ M{\isadigit{1}}\ M{\isadigit{2}}{\isacharparenright}\ {\isacharparenleft}inj{\isadigit{1}}\ M{\isadigit{1}}\ M{\isadigit{2}}{\isacharparenright}{\isachardoublequoteclose}\isanewline
%
\isadelimproof
%
\endisadelimproof
%
\isatagproof
\isacommand{proof}\isamarkupfalse%
\ {\isacharminus}\ \isanewline
\ \ \isacommand{from}\isamarkupfalse%
\ h{\isadigit{1}}\ \isacommand{interpret}\isamarkupfalse%
\ v{\isadigit{1}}{\isacharcolon}module\ R\ M{\isadigit{1}}\ \isacommand{by}\isamarkupfalse%
\ auto\isanewline
\ \ \isacommand{from}\isamarkupfalse%
\ h{\isadigit{2}}\ \isacommand{interpret}\isamarkupfalse%
\ v{\isadigit{2}}{\isacharcolon}module\ R\ M{\isadigit{2}}\ \isacommand{by}\isamarkupfalse%
\ auto\isanewline
\ \ \isacommand{from}\isamarkupfalse%
\ h{\isadigit{1}}\ h{\isadigit{2}}\ \isacommand{show}\isamarkupfalse%
\ {\isacharquery}thesis\isanewline
\ \ \ \ \isacommand{apply}\isamarkupfalse%
\ {\isacharparenleft}unfold\ mod{\isacharunderscore}hom{\isacharunderscore}def\ module{\isacharunderscore}hom{\isacharunderscore}def\ mod{\isacharunderscore}hom{\isacharunderscore}axioms{\isacharunderscore}def\ inj{\isadigit{1}}{\isacharunderscore}def{\isacharcomma}\ auto{\isacharparenright}\isanewline
\ \ \ \ \ \ \ \isacommand{apply}\isamarkupfalse%
\ {\isacharparenleft}rule\ direct{\isacharunderscore}sum{\isacharunderscore}is{\isacharunderscore}module{\isacharcomma}\ auto{\isacharparenright}\isanewline
\ \ \ \ \ \ \isacommand{by}\isamarkupfalse%
\ {\isacharparenleft}unfold\ direct{\isacharunderscore}sum{\isacharunderscore}def{\isacharcomma}\ auto{\isacharparenright}\isanewline
\isacommand{qed}\isamarkupfalse%
%
\endisatagproof
{\isafoldproof}%
%
\isadelimproof
\isanewline
%
\endisadelimproof
\isanewline
\isacommand{lemma}\isamarkupfalse%
\ inj{\isadigit{2}}{\isacharunderscore}hom{\isacharcolon}\isanewline
\ \ \isakeyword{fixes}\ R\ M{\isadigit{1}}\ M{\isadigit{2}}\isanewline
\ \ \isakeyword{assumes}\ h{\isadigit{1}}{\isacharcolon}\ {\isachardoublequoteopen}module\ R\ M{\isadigit{1}}{\isachardoublequoteclose}\ \isakeyword{and}\ h{\isadigit{2}}{\isacharcolon}\ {\isachardoublequoteopen}module\ R\ M{\isadigit{2}}{\isachardoublequoteclose}\isanewline
\ \ \isakeyword{shows}\ {\isachardoublequoteopen}mod{\isacharunderscore}hom\ R\ M{\isadigit{2}}\ {\isacharparenleft}direct{\isacharunderscore}sum\ M{\isadigit{1}}\ M{\isadigit{2}}{\isacharparenright}\ {\isacharparenleft}inj{\isadigit{2}}\ M{\isadigit{1}}\ M{\isadigit{2}}{\isacharparenright}{\isachardoublequoteclose}\isanewline
%
\isadelimproof
%
\endisadelimproof
%
\isatagproof
\isacommand{proof}\isamarkupfalse%
\ {\isacharminus}\ \isanewline
\ \ \isacommand{from}\isamarkupfalse%
\ h{\isadigit{1}}\ \isacommand{interpret}\isamarkupfalse%
\ v{\isadigit{1}}{\isacharcolon}module\ R\ M{\isadigit{1}}\ \isacommand{by}\isamarkupfalse%
\ auto\isanewline
\ \ \isacommand{from}\isamarkupfalse%
\ h{\isadigit{2}}\ \isacommand{interpret}\isamarkupfalse%
\ v{\isadigit{2}}{\isacharcolon}module\ R\ M{\isadigit{2}}\ \isacommand{by}\isamarkupfalse%
\ auto\isanewline
\ \ \isacommand{from}\isamarkupfalse%
\ h{\isadigit{1}}\ h{\isadigit{2}}\ \isacommand{show}\isamarkupfalse%
\ {\isacharquery}thesis\isanewline
\ \ \ \ \isacommand{apply}\isamarkupfalse%
\ {\isacharparenleft}unfold\ mod{\isacharunderscore}hom{\isacharunderscore}def\ module{\isacharunderscore}hom{\isacharunderscore}def\ mod{\isacharunderscore}hom{\isacharunderscore}axioms{\isacharunderscore}def\ inj{\isadigit{2}}{\isacharunderscore}def{\isacharcomma}\ auto{\isacharparenright}\isanewline
\ \ \ \ \ \ \ \isacommand{apply}\isamarkupfalse%
\ {\isacharparenleft}rule\ direct{\isacharunderscore}sum{\isacharunderscore}is{\isacharunderscore}module{\isacharcomma}\ auto{\isacharparenright}\isanewline
\ \ \ \ \ \ \isacommand{by}\isamarkupfalse%
\ {\isacharparenleft}unfold\ direct{\isacharunderscore}sum{\isacharunderscore}def{\isacharcomma}\ auto{\isacharparenright}\isanewline
\isacommand{qed}\isamarkupfalse%
%
\endisatagproof
{\isafoldproof}%
%
\isadelimproof
%
\endisadelimproof
%
\begin{isamarkuptext}%
For submodules $M_1,M_2\subseteq M$, the map $M_1\oplus M_2\to M$ given by $(m_1,m_2)\mapsto 
m_1+m_2$ is linear.%
\end{isamarkuptext}%
\isamarkuptrue%
\isacommand{lemma}\isamarkupfalse%
\ {\isacharparenleft}\isakeyword{in}\ module{\isacharparenright}\ sum{\isacharunderscore}map{\isacharunderscore}hom{\isacharcolon}\ \isanewline
\ \ \isakeyword{fixes}\ M{\isadigit{1}}\ M{\isadigit{2}}\isanewline
\ \ \isakeyword{assumes}\ h{\isadigit{1}}{\isacharcolon}\ {\isachardoublequoteopen}submodule\ R\ M{\isadigit{1}}\ M{\isachardoublequoteclose}\ \isakeyword{and}\ h{\isadigit{2}}{\isacharcolon}\ {\isachardoublequoteopen}submodule\ R\ M{\isadigit{2}}\ M{\isachardoublequoteclose}\isanewline
\ \ \isakeyword{shows}\ {\isachardoublequoteopen}mod{\isacharunderscore}hom\ R\ {\isacharparenleft}direct{\isacharunderscore}sum\ {\isacharparenleft}md\ M{\isadigit{1}}{\isacharparenright}\ {\isacharparenleft}md\ M{\isadigit{2}}{\isacharparenright}{\isacharparenright}\ M\ {\isacharparenleft}{\isasymlambda}\ v{\isachardot}\ {\isacharparenleft}fst\ v{\isacharparenright}\ {\isasymoplus}\isactrlbsub M\isactrlesub \ {\isacharparenleft}snd\ v{\isacharparenright}{\isacharparenright}{\isachardoublequoteclose}\isanewline
%
\isadelimproof
%
\endisadelimproof
%
\isatagproof
\isacommand{proof}\isamarkupfalse%
\ {\isacharminus}\ \isanewline
\ \ \isacommand{have}\isamarkupfalse%
\ {\isadigit{0}}{\isacharcolon}\ {\isachardoublequoteopen}module\ R\ M{\isachardoublequoteclose}\isacommand{{\isachardot}{\isachardot}}\isamarkupfalse%
\isanewline
\ \ \isacommand{from}\isamarkupfalse%
\ h{\isadigit{1}}\ \isacommand{have}\isamarkupfalse%
\ {\isadigit{1}}{\isacharcolon}\ {\isachardoublequoteopen}module\ R\ {\isacharparenleft}md\ M{\isadigit{1}}{\isacharparenright}{\isachardoublequoteclose}\ \isacommand{by}\isamarkupfalse%
\ {\isacharparenleft}rule\ submodule{\isacharunderscore}is{\isacharunderscore}module{\isacharparenright}\isanewline
\ \ \isacommand{from}\isamarkupfalse%
\ h{\isadigit{2}}\ \isacommand{have}\isamarkupfalse%
\ {\isadigit{2}}{\isacharcolon}\ {\isachardoublequoteopen}module\ R\ {\isacharparenleft}md\ M{\isadigit{2}}{\isacharparenright}{\isachardoublequoteclose}\ \isacommand{by}\isamarkupfalse%
\ {\isacharparenleft}rule\ submodule{\isacharunderscore}is{\isacharunderscore}module{\isacharparenright}\isanewline
\ \ \isacommand{from}\isamarkupfalse%
\ h{\isadigit{1}}\ \isacommand{interpret}\isamarkupfalse%
\ w{\isadigit{1}}{\isacharcolon}\ module\ R\ {\isachardoublequoteopen}{\isacharparenleft}md\ M{\isadigit{1}}{\isacharparenright}{\isachardoublequoteclose}\ \isacommand{by}\isamarkupfalse%
\ {\isacharparenleft}rule\ submodule{\isacharunderscore}is{\isacharunderscore}module{\isacharparenright}\isanewline
\ \ \isacommand{from}\isamarkupfalse%
\ h{\isadigit{2}}\ \isacommand{interpret}\isamarkupfalse%
\ w{\isadigit{2}}{\isacharcolon}\ module\ R\ {\isachardoublequoteopen}{\isacharparenleft}md\ M{\isadigit{2}}{\isacharparenright}{\isachardoublequoteclose}\ \isacommand{by}\isamarkupfalse%
\ {\isacharparenleft}rule\ submodule{\isacharunderscore}is{\isacharunderscore}module{\isacharparenright}\isanewline
\ \ \isacommand{from}\isamarkupfalse%
\ {\isadigit{0}}\ h{\isadigit{1}}\ h{\isadigit{2}}\ {\isadigit{1}}\ {\isadigit{2}}\ \isacommand{show}\isamarkupfalse%
\ {\isacharquery}thesis\isanewline
\ \ \ \ \isacommand{apply}\isamarkupfalse%
\ {\isacharparenleft}unfold\ mod{\isacharunderscore}hom{\isacharunderscore}def\ mod{\isacharunderscore}hom{\isacharunderscore}axioms{\isacharunderscore}def\ module{\isacharunderscore}hom{\isacharunderscore}def{\isacharcomma}\ auto{\isacharparenright}\isanewline
\ \ \ \ \ \ \ \isacommand{apply}\isamarkupfalse%
\ {\isacharparenleft}rule\ direct{\isacharunderscore}sum{\isacharunderscore}is{\isacharunderscore}module{\isacharcomma}\ auto{\isacharparenright}\isanewline
\ \ \ \ \ \ \isacommand{apply}\isamarkupfalse%
\ {\isacharparenleft}unfold\ direct{\isacharunderscore}sum{\isacharunderscore}def{\isacharcomma}\ auto{\isacharparenright}\isanewline
\ \ \ \ \ \ \isacommand{apply}\isamarkupfalse%
\ {\isacharparenleft}unfold\ submodule{\isacharunderscore}def{\isacharcomma}\ auto{\isacharparenright}\isanewline
\ \ \ \ \ \isacommand{by}\isamarkupfalse%
\ {\isacharparenleft}auto\ simp\ add{\isacharcolon}\ a{\isacharunderscore}ac\ smult{\isacharunderscore}r{\isacharunderscore}distr\ ring{\isacharunderscore}subset{\isacharunderscore}carrier{\isacharparenright}\ \isanewline
\ \ \ \ \ \ \isanewline
\isacommand{qed}\isamarkupfalse%
%
\endisatagproof
{\isafoldproof}%
%
\isadelimproof
\isanewline
%
\endisadelimproof
\isanewline
\isacommand{lemma}\isamarkupfalse%
\ {\isacharparenleft}\isakeyword{in}\ module{\isacharparenright}\ sum{\isacharunderscore}is{\isacharunderscore}submodule{\isacharcolon}\isanewline
\ \ \isakeyword{fixes}\ N{\isadigit{1}}\ N{\isadigit{2}}\isanewline
\ \ \isakeyword{assumes}\ h{\isadigit{1}}{\isacharcolon}\ {\isachardoublequoteopen}submodule\ R\ N{\isadigit{1}}\ M{\isachardoublequoteclose}\ \isakeyword{and}\ h{\isadigit{2}}{\isacharcolon}\ {\isachardoublequoteopen}submodule\ R\ N{\isadigit{2}}\ M{\isachardoublequoteclose}\isanewline
\ \ \isakeyword{shows}\ {\isachardoublequoteopen}submodule\ R\ {\isacharparenleft}submodule{\isacharunderscore}sum\ N{\isadigit{1}}\ N{\isadigit{2}}{\isacharparenright}\ M{\isachardoublequoteclose}\isanewline
%
\isadelimproof
%
\endisadelimproof
%
\isatagproof
\isacommand{proof}\isamarkupfalse%
\ {\isacharminus}\isanewline
\ \ \isacommand{from}\isamarkupfalse%
\ h{\isadigit{1}}\ h{\isadigit{2}}\ \isacommand{interpret}\isamarkupfalse%
\ l{\isacharcolon}\ mod{\isacharunderscore}hom\ R\ {\isachardoublequoteopen}{\isacharparenleft}direct{\isacharunderscore}sum\ {\isacharparenleft}md\ N{\isadigit{1}}{\isacharparenright}\ {\isacharparenleft}md\ N{\isadigit{2}}{\isacharparenright}{\isacharparenright}{\isachardoublequoteclose}\ M\ {\isachardoublequoteopen}{\isacharparenleft}{\isasymlambda}\ v{\isachardot}\ {\isacharparenleft}fst\ v{\isacharparenright}\ {\isasymoplus}\isactrlbsub M\isactrlesub \ {\isacharparenleft}snd\ v{\isacharparenright}{\isacharparenright}{\isachardoublequoteclose}\ \isanewline
\ \ \ \ \isacommand{by}\isamarkupfalse%
\ {\isacharparenleft}rule\ sum{\isacharunderscore}map{\isacharunderscore}hom{\isacharparenright}\isanewline
\ \ \isacommand{have}\isamarkupfalse%
\ {\isadigit{1}}{\isacharcolon}\ {\isachardoublequoteopen}l{\isachardot}im\ {\isacharequal}submodule{\isacharunderscore}sum\ N{\isadigit{1}}\ N{\isadigit{2}}{\isachardoublequoteclose}\isanewline
\ \ \ \ \isacommand{apply}\isamarkupfalse%
\ {\isacharparenleft}unfold\ l{\isachardot}im{\isacharunderscore}def\ submodule{\isacharunderscore}sum{\isacharunderscore}def{\isacharparenright}\isanewline
\ \ \ \ \isacommand{apply}\isamarkupfalse%
\ {\isacharparenleft}unfold\ direct{\isacharunderscore}sum{\isacharunderscore}def{\isacharcomma}\ auto{\isacharparenright}\isanewline
\ \ \ \ \isacommand{by}\isamarkupfalse%
\ {\isacharparenleft}unfold\ image{\isacharunderscore}def{\isacharcomma}\ auto{\isacharparenright}\isanewline
\ \ \isacommand{have}\isamarkupfalse%
\ {\isadigit{2}}{\isacharcolon}\ {\isachardoublequoteopen}submodule\ R\ {\isacharparenleft}l{\isachardot}im{\isacharparenright}\ M{\isachardoublequoteclose}\ \isacommand{by}\isamarkupfalse%
\ {\isacharparenleft}rule\ l{\isachardot}im{\isacharunderscore}is{\isacharunderscore}submodule{\isacharparenright}\isanewline
\ \ \isacommand{from}\isamarkupfalse%
\ {\isadigit{1}}\ {\isadigit{2}}\ \isacommand{show}\isamarkupfalse%
\ {\isacharquery}thesis\ \isacommand{by}\isamarkupfalse%
\ auto\isanewline
\isacommand{qed}\isamarkupfalse%
%
\endisatagproof
{\isafoldproof}%
%
\isadelimproof
\isanewline
%
\endisadelimproof
\isanewline
\isacommand{lemma}\isamarkupfalse%
\ {\isacharparenleft}\isakeyword{in}\ module{\isacharparenright}\ in{\isacharunderscore}sum{\isacharcolon}\isanewline
\ \ \isakeyword{fixes}\ N{\isadigit{1}}\ N{\isadigit{2}}\isanewline
\ \ \isakeyword{assumes}\ h{\isadigit{1}}{\isacharcolon}\ {\isachardoublequoteopen}submodule\ R\ N{\isadigit{1}}\ M{\isachardoublequoteclose}\ \isakeyword{and}\ h{\isadigit{2}}{\isacharcolon}\ {\isachardoublequoteopen}submodule\ R\ N{\isadigit{2}}\ M{\isachardoublequoteclose}\isanewline
\ \ \isakeyword{shows}\ {\isachardoublequoteopen}N{\isadigit{1}}\ {\isasymsubseteq}\ submodule{\isacharunderscore}sum\ N{\isadigit{1}}\ N{\isadigit{2}}{\isachardoublequoteclose}\isanewline
%
\isadelimproof
%
\endisadelimproof
%
\isatagproof
\isacommand{proof}\isamarkupfalse%
\ {\isacharminus}\isanewline
\ \ \isacommand{from}\isamarkupfalse%
\ h{\isadigit{1}}\ h{\isadigit{2}}\ \isacommand{show}\isamarkupfalse%
\ {\isacharquery}thesis\isanewline
\ \ \ \ \isacommand{apply}\isamarkupfalse%
\ auto\isanewline
\ \ \ \ \isacommand{apply}\isamarkupfalse%
\ {\isacharparenleft}unfold\ submodule{\isacharunderscore}sum{\isacharunderscore}def\ image{\isacharunderscore}def{\isacharcomma}\ auto{\isacharparenright}\isanewline
\ \ \ \ \isacommand{apply}\isamarkupfalse%
\ {\isacharparenleft}rename{\isacharunderscore}tac\ v{\isacharparenright}\isanewline
\ \ \ \ \isacommand{apply}\isamarkupfalse%
\ {\isacharparenleft}rule{\isacharunderscore}tac\ x{\isacharequal}{\isachardoublequoteopen}v{\isachardoublequoteclose}\ \isakeyword{in}\ bexI{\isacharparenright}\isanewline
\ \ \ \ \ \isacommand{apply}\isamarkupfalse%
\ {\isacharparenleft}rule{\isacharunderscore}tac\ x{\isacharequal}{\isachardoublequoteopen}{\isasymzero}\isactrlbsub M\isactrlesub {\isachardoublequoteclose}\ \isakeyword{in}\ bexI{\isacharparenright}\isanewline
\ \ \ \ \ \ \isacommand{by}\isamarkupfalse%
\ {\isacharparenleft}unfold\ submodule{\isacharunderscore}def{\isacharcomma}\ auto{\isacharparenright}\isanewline
\isacommand{qed}\isamarkupfalse%
%
\endisatagproof
{\isafoldproof}%
%
\isadelimproof
\isanewline
%
\endisadelimproof
\isanewline
\isacommand{lemma}\isamarkupfalse%
\ {\isacharparenleft}\isakeyword{in}\ module{\isacharparenright}\ msum{\isacharunderscore}comm{\isacharcolon}\isanewline
\ \ \isakeyword{fixes}\ N{\isadigit{1}}\ N{\isadigit{2}}\isanewline
\ \ \isakeyword{assumes}\ h{\isadigit{1}}{\isacharcolon}\ {\isachardoublequoteopen}submodule\ R\ N{\isadigit{1}}\ M{\isachardoublequoteclose}\ \isakeyword{and}\ h{\isadigit{2}}{\isacharcolon}\ {\isachardoublequoteopen}submodule\ R\ N{\isadigit{2}}\ M{\isachardoublequoteclose}\isanewline
\ \ \isakeyword{shows}\ {\isachardoublequoteopen}{\isacharparenleft}submodule{\isacharunderscore}sum\ N{\isadigit{1}}\ N{\isadigit{2}}{\isacharparenright}\ {\isacharequal}\ {\isacharparenleft}submodule{\isacharunderscore}sum\ N{\isadigit{2}}\ N{\isadigit{1}}{\isacharparenright}{\isachardoublequoteclose}\isanewline
%
\isadelimproof
%
\endisadelimproof
%
\isatagproof
\isacommand{proof}\isamarkupfalse%
\ {\isacharminus}\ \isanewline
\ \ \isanewline
\ \ \isacommand{from}\isamarkupfalse%
\ h{\isadigit{1}}\ h{\isadigit{2}}\ \isacommand{show}\isamarkupfalse%
\ {\isacharquery}thesis\isanewline
\ \ \ \ \isacommand{apply}\isamarkupfalse%
\ {\isacharparenleft}unfold\ submodule{\isacharunderscore}sum{\isacharunderscore}def\ image{\isacharunderscore}def{\isacharcomma}\ auto{\isacharparenright}\isanewline
\ \ \ \ \ \isacommand{apply}\isamarkupfalse%
\ {\isacharparenleft}unfold\ submodule{\isacharunderscore}def{\isacharparenright}\isanewline
\ \ \ \ \ \isacommand{apply}\isamarkupfalse%
\ {\isacharparenleft}rename{\isacharunderscore}tac\ v\ w{\isacharparenright}\isanewline
\ \ \ \ \ \isacommand{by}\isamarkupfalse%
\ {\isacharparenleft}metis\ {\isacharparenleft}full{\isacharunderscore}types{\isacharparenright}\ M{\isachardot}add{\isachardot}m{\isacharunderscore}comm\ subsetD{\isacharparenright}{\isacharplus}\isanewline
\ \ \ \ \isanewline
\isacommand{qed}\isamarkupfalse%
%
\endisatagproof
{\isafoldproof}%
%
\isadelimproof
%
\endisadelimproof
%
\begin{isamarkuptext}%
If $M_1,M_2\subseteq M$ are submodules, then $M_1+M_2$ is the minimal subspace such that 
both $M_1\subseteq M$ and $M_2\subseteq M$.%
\end{isamarkuptext}%
\isamarkuptrue%
\isacommand{lemma}\isamarkupfalse%
\ {\isacharparenleft}\isakeyword{in}\ module{\isacharparenright}\ sum{\isacharunderscore}is{\isacharunderscore}minimal{\isacharcolon}\isanewline
\ \ \isakeyword{fixes}\ N\ N{\isadigit{1}}\ N{\isadigit{2}}\isanewline
\ \ \isakeyword{assumes}\ h{\isadigit{1}}{\isacharcolon}\ {\isachardoublequoteopen}submodule\ R\ N{\isadigit{1}}\ M{\isachardoublequoteclose}\ \isakeyword{and}\ h{\isadigit{2}}{\isacharcolon}\ {\isachardoublequoteopen}submodule\ R\ N{\isadigit{2}}\ M{\isachardoublequoteclose}\ \isakeyword{and}\ h{\isadigit{3}}{\isacharcolon}\ {\isachardoublequoteopen}submodule\ R\ N\ M{\isachardoublequoteclose}\isanewline
\ \ \isakeyword{shows}\ {\isachardoublequoteopen}{\isacharparenleft}submodule{\isacharunderscore}sum\ N{\isadigit{1}}\ N{\isadigit{2}}{\isacharparenright}\ {\isasymsubseteq}\ N\ {\isasymlongleftrightarrow}\ N{\isadigit{1}}\ {\isasymsubseteq}\ N\ {\isasymand}\ N{\isadigit{2}}\ {\isasymsubseteq}\ N{\isachardoublequoteclose}\isanewline
%
\isadelimproof
%
\endisadelimproof
%
\isatagproof
\isacommand{proof}\isamarkupfalse%
\ {\isacharminus}\ \isanewline
\ \ \isacommand{have}\isamarkupfalse%
\ {\isadigit{1}}{\isacharcolon}\ {\isachardoublequoteopen}{\isacharparenleft}submodule{\isacharunderscore}sum\ N{\isadigit{1}}\ N{\isadigit{2}}{\isacharparenright}\ {\isasymsubseteq}\ N\ {\isasymLongrightarrow}\ N{\isadigit{1}}\ {\isasymsubseteq}\ N\ {\isasymand}\ N{\isadigit{2}}\ {\isasymsubseteq}\ N{\isachardoublequoteclose}\isanewline
\ \ \isacommand{proof}\isamarkupfalse%
\ {\isacharminus}\isanewline
\ \ \ \ \isacommand{assume}\isamarkupfalse%
\ {\isadigit{1}}{\isadigit{0}}{\isacharcolon}\ {\isachardoublequoteopen}{\isacharparenleft}submodule{\isacharunderscore}sum\ N{\isadigit{1}}\ N{\isadigit{2}}{\isacharparenright}\ {\isasymsubseteq}\ N{\isachardoublequoteclose}\isanewline
\ \ \ \ \isacommand{from}\isamarkupfalse%
\ h{\isadigit{1}}\ h{\isadigit{2}}\ \isacommand{have}\isamarkupfalse%
\ {\isadigit{1}}{\isadigit{1}}{\isacharcolon}\ {\isachardoublequoteopen}N{\isadigit{1}}{\isasymsubseteq}submodule{\isacharunderscore}sum\ N{\isadigit{1}}\ N{\isadigit{2}}{\isachardoublequoteclose}\ \isacommand{by}\isamarkupfalse%
\ {\isacharparenleft}rule\ in{\isacharunderscore}sum{\isacharparenright}\isanewline
\ \ \ \ \isacommand{from}\isamarkupfalse%
\ h{\isadigit{2}}\ h{\isadigit{1}}\ \isacommand{have}\isamarkupfalse%
\ {\isadigit{1}}{\isadigit{2}}{\isacharcolon}\ {\isachardoublequoteopen}N{\isadigit{2}}{\isasymsubseteq}submodule{\isacharunderscore}sum\ N{\isadigit{2}}\ N{\isadigit{1}}{\isachardoublequoteclose}\ \isacommand{by}\isamarkupfalse%
\ {\isacharparenleft}rule\ in{\isacharunderscore}sum{\isacharparenright}\isanewline
\ \ \ \ \isacommand{from}\isamarkupfalse%
\ {\isadigit{1}}{\isadigit{2}}\ h{\isadigit{1}}\ h{\isadigit{2}}\ \isacommand{have}\isamarkupfalse%
\ {\isadigit{1}}{\isadigit{3}}{\isacharcolon}\ {\isachardoublequoteopen}N{\isadigit{2}}{\isasymsubseteq}submodule{\isacharunderscore}sum\ N{\isadigit{1}}\ N{\isadigit{2}}{\isachardoublequoteclose}\ \isacommand{by}\isamarkupfalse%
\ {\isacharparenleft}metis\ msum{\isacharunderscore}comm{\isacharparenright}\isanewline
\ \ \ \ \isacommand{from}\isamarkupfalse%
\ {\isadigit{1}}{\isadigit{0}}\ {\isadigit{1}}{\isadigit{1}}\ {\isadigit{1}}{\isadigit{3}}\ \isacommand{show}\isamarkupfalse%
\ {\isacharquery}thesis\ \isacommand{by}\isamarkupfalse%
\ auto\isanewline
\ \ \isacommand{qed}\isamarkupfalse%
\isanewline
\ \ \isacommand{have}\isamarkupfalse%
\ {\isadigit{2}}{\isacharcolon}\ {\isachardoublequoteopen}N{\isadigit{1}}\ {\isasymsubseteq}\ N\ {\isasymand}\ N{\isadigit{2}}\ {\isasymsubseteq}\ N\ {\isasymLongrightarrow}\ {\isacharparenleft}submodule{\isacharunderscore}sum\ N{\isadigit{1}}\ N{\isadigit{2}}{\isacharparenright}\ {\isasymsubseteq}\ N{\isachardoublequoteclose}\isanewline
\ \ \isacommand{proof}\isamarkupfalse%
\ {\isacharminus}\isanewline
\ \ \ \ \isacommand{assume}\isamarkupfalse%
\ {\isadigit{1}}{\isadigit{9}}{\isacharcolon}\ {\isachardoublequoteopen}N{\isadigit{1}}\ {\isasymsubseteq}\ N\ {\isasymand}\ N{\isadigit{2}}\ {\isasymsubseteq}\ N{\isachardoublequoteclose}\isanewline
\ \ \ \ \isacommand{{\isacharbraceleft}}\isamarkupfalse%
\ \ \isanewline
\ \ \ \ \isacommand{fix}\isamarkupfalse%
\ v\isanewline
\ \ \ \ \isacommand{assume}\isamarkupfalse%
\ {\isadigit{2}}{\isadigit{0}}{\isacharcolon}\ {\isachardoublequoteopen}v{\isasymin}submodule{\isacharunderscore}sum\ N{\isadigit{1}}\ N{\isadigit{2}}{\isachardoublequoteclose}\isanewline
\ \ \ \ \isacommand{from}\isamarkupfalse%
\ {\isadigit{2}}{\isadigit{0}}\ \isacommand{obtain}\isamarkupfalse%
\ w{\isadigit{1}}\ w{\isadigit{2}}\ \isakeyword{where}\ {\isadigit{2}}{\isadigit{1}}{\isacharcolon}\ {\isachardoublequoteopen}w{\isadigit{1}}{\isasymin}N{\isadigit{1}}{\isachardoublequoteclose}\ \isakeyword{and}\ {\isadigit{2}}{\isadigit{2}}{\isacharcolon}\ {\isachardoublequoteopen}w{\isadigit{2}}{\isasymin}N{\isadigit{2}}{\isachardoublequoteclose}\ \isakeyword{and}\ {\isadigit{2}}{\isadigit{3}}{\isacharcolon}\ {\isachardoublequoteopen}v{\isacharequal}w{\isadigit{1}}{\isasymoplus}\isactrlbsub M\isactrlesub \ w{\isadigit{2}}{\isachardoublequoteclose}\ \isanewline
\ \ \ \ \ \ \isacommand{by}\isamarkupfalse%
\ {\isacharparenleft}unfold\ submodule{\isacharunderscore}sum{\isacharunderscore}def\ image{\isacharunderscore}def{\isacharcomma}\ auto{\isacharparenright}\isanewline
\ \ \ \ \isacommand{from}\isamarkupfalse%
\ {\isadigit{1}}{\isadigit{9}}\ {\isadigit{2}}{\isadigit{1}}\ {\isadigit{2}}{\isadigit{2}}\ {\isadigit{2}}{\isadigit{3}}\ h{\isadigit{3}}\ \isacommand{have}\isamarkupfalse%
\ {\isachardoublequoteopen}v\ {\isasymin}\ N{\isachardoublequoteclose}\ \isanewline
\ \ \ \ \ \ \isacommand{apply}\isamarkupfalse%
\ {\isacharparenleft}unfold\ submodule{\isacharunderscore}def{\isacharcomma}\ auto{\isacharparenright}\isanewline
\ \ \ \ \ \ \isacommand{by}\isamarkupfalse%
\ {\isacharparenleft}metis\ {\isacharparenleft}poly{\isacharunderscore}guards{\isacharunderscore}query{\isacharparenright}\ \ contra{\isacharunderscore}subsetD{\isacharparenright}\isanewline
\ \isanewline
\ \ \ \ \isacommand{{\isacharbraceright}}\isamarkupfalse%
\isanewline
\ \ \ \ \isacommand{thus}\isamarkupfalse%
\ {\isacharquery}thesis\isanewline
\ \ \ \ \ \ \isacommand{by}\isamarkupfalse%
\ {\isacharparenleft}metis\ subset{\isacharunderscore}iff{\isacharparenright}\isanewline
\ \ \isacommand{qed}\isamarkupfalse%
\isanewline
\ \ \isacommand{from}\isamarkupfalse%
\ {\isadigit{1}}\ {\isadigit{2}}\ \isacommand{show}\isamarkupfalse%
\ {\isacharquery}thesis\ \isacommand{by}\isamarkupfalse%
\ metis\isanewline
\isacommand{qed}\isamarkupfalse%
%
\endisatagproof
{\isafoldproof}%
%
\isadelimproof
%
\endisadelimproof
%
\begin{isamarkuptext}%
$\text{span} A\cup B = \text{span} A + \text{span} B$%
\end{isamarkuptext}%
\isamarkuptrue%
\isacommand{lemma}\isamarkupfalse%
\ {\isacharparenleft}\isakeyword{in}\ module{\isacharparenright}\ span{\isacharunderscore}union{\isacharunderscore}is{\isacharunderscore}sum{\isacharcolon}\ \isanewline
\ \ \isakeyword{fixes}\ A\ B\isanewline
\ \ \isakeyword{assumes}\ \ h{\isadigit{2}}{\isacharcolon}\ {\isachardoublequoteopen}A{\isasymsubseteq}carrier\ M{\isachardoublequoteclose}\ \isakeyword{and}\ h{\isadigit{3}}{\isacharcolon}\ {\isachardoublequoteopen}B{\isasymsubseteq}carrier\ M{\isachardoublequoteclose}\isanewline
\ \ \isakeyword{shows}\ {\isachardoublequoteopen}span\ {\isacharparenleft}A{\isasymunion}\ B{\isacharparenright}\ {\isacharequal}\ submodule{\isacharunderscore}sum\ {\isacharparenleft}span\ A{\isacharparenright}\ {\isacharparenleft}span\ B{\isacharparenright}{\isachardoublequoteclose}\isanewline
%
\isadelimproof
%
\endisadelimproof
%
\isatagproof
\isacommand{proof}\isamarkupfalse%
{\isacharminus}\isanewline
\ \ \isacommand{let}\isamarkupfalse%
\ {\isacharquery}AplusB{\isacharequal}{\isachardoublequoteopen}submodule{\isacharunderscore}sum\ {\isacharparenleft}span\ A{\isacharparenright}\ {\isacharparenleft}span\ B{\isacharparenright}{\isachardoublequoteclose}\isanewline
\ \ \isacommand{from}\isamarkupfalse%
\ \ h{\isadigit{2}}\ \isacommand{have}\isamarkupfalse%
\ s{\isadigit{0}}{\isacharcolon}\ {\isachardoublequoteopen}submodule\ R\ {\isacharparenleft}span\ A{\isacharparenright}\ M{\isachardoublequoteclose}\ \isacommand{by}\isamarkupfalse%
\ {\isacharparenleft}rule\ span{\isacharunderscore}is{\isacharunderscore}submodule{\isacharparenright}\isanewline
\ \ \isacommand{from}\isamarkupfalse%
\ \ h{\isadigit{3}}\ \isacommand{have}\isamarkupfalse%
\ s{\isadigit{1}}{\isacharcolon}\ {\isachardoublequoteopen}submodule\ R\ {\isacharparenleft}span\ B{\isacharparenright}\ M{\isachardoublequoteclose}\ \isacommand{by}\isamarkupfalse%
\ {\isacharparenleft}rule\ span{\isacharunderscore}is{\isacharunderscore}submodule{\isacharparenright}\isanewline
\ \ \isacommand{from}\isamarkupfalse%
\ s{\isadigit{0}}\ \isacommand{have}\isamarkupfalse%
\ s{\isadigit{0}}{\isacharunderscore}{\isadigit{1}}{\isacharcolon}\ {\isachardoublequoteopen}{\isacharparenleft}span\ A{\isacharparenright}{\isasymsubseteq}carrier\ M{\isachardoublequoteclose}\ \isacommand{by}\isamarkupfalse%
\ {\isacharparenleft}unfold\ submodule{\isacharunderscore}def{\isacharcomma}\ auto{\isacharparenright}\isanewline
\ \ \isacommand{from}\isamarkupfalse%
\ s{\isadigit{1}}\ \isacommand{have}\isamarkupfalse%
\ s{\isadigit{1}}{\isacharunderscore}{\isadigit{1}}{\isacharcolon}\ {\isachardoublequoteopen}{\isacharparenleft}span\ B{\isacharparenright}{\isasymsubseteq}carrier\ M{\isachardoublequoteclose}\ \isacommand{by}\isamarkupfalse%
\ {\isacharparenleft}unfold\ submodule{\isacharunderscore}def{\isacharcomma}\ auto{\isacharparenright}\isanewline
\ \ \isacommand{from}\isamarkupfalse%
\ h{\isadigit{2}}\ h{\isadigit{3}}\ \isacommand{have}\isamarkupfalse%
\ {\isadigit{1}}{\isacharcolon}\ {\isachardoublequoteopen}A{\isasymunion}B{\isasymsubseteq}carrier\ M{\isachardoublequoteclose}\ \isacommand{by}\isamarkupfalse%
\ auto\isanewline
\ \ \isacommand{from}\isamarkupfalse%
\ \ {\isadigit{1}}\ \isacommand{have}\isamarkupfalse%
\ {\isadigit{2}}{\isacharcolon}\ {\isachardoublequoteopen}submodule\ R\ {\isacharparenleft}span\ {\isacharparenleft}A{\isasymunion}B{\isacharparenright}{\isacharparenright}\ M{\isachardoublequoteclose}\ \isacommand{by}\isamarkupfalse%
\ {\isacharparenleft}rule\ span{\isacharunderscore}is{\isacharunderscore}submodule{\isacharparenright}\isanewline
\ \ \isacommand{from}\isamarkupfalse%
\ s{\isadigit{0}}\ s{\isadigit{1}}\ \isacommand{have}\isamarkupfalse%
\ {\isadigit{3}}{\isacharcolon}\ {\isachardoublequoteopen}submodule\ R\ {\isacharquery}AplusB\ M{\isachardoublequoteclose}\ \isacommand{by}\isamarkupfalse%
\ {\isacharparenleft}rule\ sum{\isacharunderscore}is{\isacharunderscore}submodule{\isacharparenright}\isanewline
\ \ \isacommand{have}\isamarkupfalse%
\ c{\isadigit{1}}{\isacharcolon}\ {\isachardoublequoteopen}span\ {\isacharparenleft}A{\isasymunion}B{\isacharparenright}\ {\isasymsubseteq}\ {\isacharquery}AplusB{\isachardoublequoteclose}\isanewline
\isanewline
\ \ \isacommand{proof}\isamarkupfalse%
\ {\isacharminus}\isanewline
\ \ \ \ \isacommand{from}\isamarkupfalse%
\ \ h{\isadigit{2}}\ \isacommand{have}\isamarkupfalse%
\ a{\isadigit{1}}{\isacharcolon}\ {\isachardoublequoteopen}A{\isasymsubseteq}span\ A{\isachardoublequoteclose}\ \isacommand{by}\isamarkupfalse%
\ {\isacharparenleft}rule\ in{\isacharunderscore}own{\isacharunderscore}span{\isacharparenright}\isanewline
\ \ \ \ \isacommand{from}\isamarkupfalse%
\ s{\isadigit{0}}\ s{\isadigit{1}}\ \isacommand{have}\isamarkupfalse%
\ a{\isadigit{2}}{\isacharcolon}\ {\isachardoublequoteopen}span\ A\ {\isasymsubseteq}\ {\isacharquery}AplusB{\isachardoublequoteclose}\ \isacommand{by}\isamarkupfalse%
\ {\isacharparenleft}rule\ in{\isacharunderscore}sum{\isacharparenright}\ \isanewline
\ \ \ \ \isacommand{from}\isamarkupfalse%
\ a{\isadigit{1}}\ a{\isadigit{2}}\ \isacommand{have}\isamarkupfalse%
\ a{\isadigit{3}}{\isacharcolon}\ {\isachardoublequoteopen}A{\isasymsubseteq}\ {\isacharquery}AplusB{\isachardoublequoteclose}\ \isacommand{by}\isamarkupfalse%
\ auto\isanewline
\ \ \ \ \isanewline
\ \ \ \ \isacommand{from}\isamarkupfalse%
\ \ h{\isadigit{3}}\ \isacommand{have}\isamarkupfalse%
\ b{\isadigit{1}}{\isacharcolon}\ {\isachardoublequoteopen}B{\isasymsubseteq}span\ B{\isachardoublequoteclose}\ \isacommand{by}\isamarkupfalse%
\ {\isacharparenleft}rule\ in{\isacharunderscore}own{\isacharunderscore}span{\isacharparenright}\isanewline
\ \ \ \ \isacommand{from}\isamarkupfalse%
\ s{\isadigit{1}}\ s{\isadigit{0}}\ \isacommand{have}\isamarkupfalse%
\ b{\isadigit{2}}{\isacharcolon}\ {\isachardoublequoteopen}span\ B\ {\isasymsubseteq}\ {\isacharquery}AplusB{\isachardoublequoteclose}\ \isacommand{by}\isamarkupfalse%
\ {\isacharparenleft}metis\ in{\isacharunderscore}sum\ msum{\isacharunderscore}comm{\isacharparenright}\ \isanewline
\ \ \ \ \isacommand{from}\isamarkupfalse%
\ b{\isadigit{1}}\ b{\isadigit{2}}\ \isacommand{have}\isamarkupfalse%
\ b{\isadigit{3}}{\isacharcolon}\ {\isachardoublequoteopen}B{\isasymsubseteq}\ {\isacharquery}AplusB{\isachardoublequoteclose}\ \isacommand{by}\isamarkupfalse%
\ auto\isanewline
\ \ \ \ \isacommand{from}\isamarkupfalse%
\ a{\isadigit{3}}\ b{\isadigit{3}}\ \isacommand{have}\isamarkupfalse%
\ {\isadigit{5}}{\isacharcolon}\ {\isachardoublequoteopen}A{\isasymunion}B{\isasymsubseteq}\ {\isacharquery}AplusB{\isachardoublequoteclose}\ \isacommand{by}\isamarkupfalse%
\ auto\isanewline
\ \ \ \ \ \ \isanewline
\ \ \ \ \isacommand{from}\isamarkupfalse%
\ {\isadigit{5}}\ {\isadigit{3}}\ \isacommand{show}\isamarkupfalse%
\ {\isacharquery}thesis\ \isacommand{by}\isamarkupfalse%
\ {\isacharparenleft}rule\ span{\isacharunderscore}is{\isacharunderscore}subset{\isacharparenright}\isanewline
\ \ \isacommand{qed}\isamarkupfalse%
\isanewline
\ \ \isacommand{have}\isamarkupfalse%
\ c{\isadigit{2}}{\isacharcolon}\ {\isachardoublequoteopen}{\isacharquery}AplusB\ {\isasymsubseteq}\ span\ {\isacharparenleft}A{\isasymunion}B{\isacharparenright}{\isachardoublequoteclose}\ \isanewline
\ \ \isacommand{proof}\isamarkupfalse%
\ {\isacharminus}\ \isanewline
\ \ \ \ \isacommand{have}\isamarkupfalse%
\ {\isadigit{1}}{\isadigit{1}}{\isacharcolon}\ {\isachardoublequoteopen}A{\isasymsubseteq}A{\isasymunion}B{\isachardoublequoteclose}\ \isacommand{by}\isamarkupfalse%
\ auto\isanewline
\ \ \ \ \isacommand{have}\isamarkupfalse%
\ {\isadigit{1}}{\isadigit{2}}{\isacharcolon}\ {\isachardoublequoteopen}B{\isasymsubseteq}A{\isasymunion}B{\isachardoublequoteclose}\ \isacommand{by}\isamarkupfalse%
\ auto\isanewline
\ \ \ \ \isacommand{from}\isamarkupfalse%
\ \ {\isadigit{1}}{\isadigit{1}}\ \isacommand{have}\isamarkupfalse%
\ {\isadigit{2}}{\isadigit{1}}{\isacharcolon}{\isachardoublequoteopen}span\ A\ {\isasymsubseteq}span\ {\isacharparenleft}A{\isasymunion}B{\isacharparenright}{\isachardoublequoteclose}\ \isacommand{by}\isamarkupfalse%
\ {\isacharparenleft}rule\ span{\isacharunderscore}is{\isacharunderscore}monotone{\isacharparenright}\isanewline
\ \ \ \ \isacommand{from}\isamarkupfalse%
\ \ {\isadigit{1}}{\isadigit{2}}\ \isacommand{have}\isamarkupfalse%
\ {\isadigit{2}}{\isadigit{2}}{\isacharcolon}{\isachardoublequoteopen}span\ B\ {\isasymsubseteq}span\ {\isacharparenleft}A{\isasymunion}B{\isacharparenright}{\isachardoublequoteclose}\ \isacommand{by}\isamarkupfalse%
\ {\isacharparenleft}rule\ span{\isacharunderscore}is{\isacharunderscore}monotone{\isacharparenright}\isanewline
\ \ \ \ \isacommand{from}\isamarkupfalse%
\ s{\isadigit{0}}\ s{\isadigit{1}}\ {\isadigit{2}}\ {\isadigit{2}}{\isadigit{1}}\ {\isadigit{2}}{\isadigit{2}}\ \isacommand{show}\isamarkupfalse%
\ {\isacharquery}thesis\ \isacommand{by}\isamarkupfalse%
\ {\isacharparenleft}auto\ simp\ add{\isacharcolon}\ sum{\isacharunderscore}is{\isacharunderscore}minimal{\isacharparenright}\isanewline
\ \ \isacommand{qed}\isamarkupfalse%
\isanewline
\ \ \isacommand{from}\isamarkupfalse%
\ c{\isadigit{1}}\ c{\isadigit{2}}\ \isacommand{show}\isamarkupfalse%
\ {\isacharquery}thesis\ \isacommand{by}\isamarkupfalse%
\ auto\isanewline
\isacommand{qed}\isamarkupfalse%
%
\endisatagproof
{\isafoldproof}%
%
\isadelimproof
\isanewline
%
\endisadelimproof
%
\isadelimtheory
\isanewline
%
\endisadelimtheory
%
\isatagtheory
\isacommand{end}\isamarkupfalse%
%
\endisatagtheory
{\isafoldtheory}%
%
\isadelimtheory
%
\endisadelimtheory
\end{isabellebody}%
%%% Local Variables:
%%% mode: latex
%%% TeX-master: "root"
%%% End:


%
\begin{isabellebody}%
\def\isabellecontext{VectorSpace}%
%
\isamarkupheader{Basic theory of vector spaces, using locales%
}
\isamarkuptrue%
%
\isadelimtheory
%
\endisadelimtheory
%
\isatagtheory
\isacommand{theory}\isamarkupfalse%
\ VectorSpace\isanewline
\isakeyword{imports}\ Main\isanewline
\ \ {\isachardoublequoteopen}{\isachartilde}{\isachartilde}{\isacharslash}src{\isacharslash}HOL{\isacharslash}Algebra{\isacharslash}Module{\isachardoublequoteclose}\isanewline
\ \ {\isachardoublequoteopen}{\isachartilde}{\isachartilde}{\isacharslash}src{\isacharslash}HOL{\isacharslash}Algebra{\isacharslash}Coset{\isachardoublequoteclose}\isanewline
\ \ RingModuleFacts\isanewline
\ \ MonoidSums\isanewline
\ \ LinearCombinations\isanewline
\ \ SumSpaces\isanewline
\isakeyword{begin}%
\endisatagtheory
{\isafoldtheory}%
%
\isadelimtheory
%
\endisadelimtheory
%
\isamarkupsubsection{Basic definitions and facts carried over from modules%
}
\isamarkuptrue%
%
\begin{isamarkuptext}%
A \isa{vectorspace} is a module where the ring is a field. 
Note that we switch notation from $(R, M)$ to $(K, V)$.%
\end{isamarkuptext}%
\isamarkuptrue%
\isacommand{locale}\isamarkupfalse%
\ vectorspace\ {\isacharequal}\ \isanewline
\ \ module{\isacharcolon}\ module\ K\ V\ {\isacharplus}\ field{\isacharcolon}\ field\ K\ \ \isanewline
\ \ \isakeyword{for}\ K\ \isakeyword{and}\ V%
\begin{isamarkuptext}%
A \isa{subspace} of a vectorspace is a nonempty subset 
that is closed under addition and scalar multiplication. These properties
have already been defined in submodule. Caution: W is a set, while V is 
a module record. To get W as a vectorspace, write vs W.%
\end{isamarkuptext}%
\isamarkuptrue%
\isacommand{locale}\isamarkupfalse%
\ subspace\ {\isacharequal}\isanewline
\ \ \isakeyword{fixes}\ K\ \isakeyword{and}\ W\ \isakeyword{and}\ V\ {\isacharparenleft}\isakeyword{structure}{\isacharparenright}\isanewline
\ \ \isakeyword{assumes}\ vs{\isacharcolon}\ {\isachardoublequoteopen}vectorspace\ K\ V{\isachardoublequoteclose}\isanewline
\ \ \ \ \ \ \isakeyword{and}\ submod{\isacharcolon}\ {\isachardoublequoteopen}submodule\ K\ W\ V{\isachardoublequoteclose}\isanewline
\isanewline
\isanewline
\isacommand{lemma}\isamarkupfalse%
\ {\isacharparenleft}\isakeyword{in}\ vectorspace{\isacharparenright}\ is{\isacharunderscore}module{\isacharbrackleft}simp{\isacharbrackright}{\isacharcolon}\isanewline
\ \ {\isachardoublequoteopen}subspace\ K\ W\ V{\isasymLongrightarrow}submodule\ K\ W\ V{\isachardoublequoteclose}\isanewline
%
\isadelimproof
%
\endisadelimproof
%
\isatagproof
\isacommand{by}\isamarkupfalse%
\ {\isacharparenleft}unfold\ subspace{\isacharunderscore}def{\isacharcomma}\ auto{\isacharparenright}%
\endisatagproof
{\isafoldproof}%
%
\isadelimproof
%
\endisadelimproof
%
\begin{isamarkuptext}%
We introduce some basic facts and definitions copied from module.
We introduce some abbreviations, to match convention.%
\end{isamarkuptext}%
\isamarkuptrue%
\isacommand{abbreviation}\isamarkupfalse%
\ {\isacharparenleft}\isakeyword{in}\ vectorspace{\isacharparenright}\ vs{\isacharcolon}{\isacharcolon}{\isachardoublequoteopen}{\isacharprime}c\ set\ {\isasymRightarrow}\ {\isacharparenleft}{\isacharprime}a{\isacharcomma}\ {\isacharprime}c{\isacharcomma}\ {\isacharprime}d{\isacharparenright}\ module{\isacharunderscore}scheme{\isachardoublequoteclose}\isanewline
\ \ \isakeyword{where}\ {\isachardoublequoteopen}vs\ W\ {\isasymequiv}\ V{\isasymlparr}carrier\ {\isacharcolon}{\isacharequal}W{\isasymrparr}{\isachardoublequoteclose}\isanewline
\isanewline
\isacommand{lemma}\isamarkupfalse%
\ {\isacharparenleft}\isakeyword{in}\ vectorspace{\isacharparenright}\ carrier{\isacharunderscore}vs{\isacharunderscore}is{\isacharunderscore}self\ {\isacharbrackleft}simp{\isacharbrackright}{\isacharcolon}\isanewline
\ \ {\isachardoublequoteopen}carrier\ {\isacharparenleft}vs\ W{\isacharparenright}\ {\isacharequal}\ W{\isachardoublequoteclose}\isanewline
%
\isadelimproof
\ \ %
\endisadelimproof
%
\isatagproof
\isacommand{by}\isamarkupfalse%
\ auto%
\endisatagproof
{\isafoldproof}%
%
\isadelimproof
\isanewline
%
\endisadelimproof
\isanewline
\isacommand{lemma}\isamarkupfalse%
\ {\isacharparenleft}\isakeyword{in}\ vectorspace{\isacharparenright}\ subspace{\isacharunderscore}is{\isacharunderscore}vs{\isacharcolon}\isanewline
\ \ \isakeyword{fixes}\ W{\isacharcolon}{\isacharcolon}{\isachardoublequoteopen}{\isacharprime}c\ set{\isachardoublequoteclose}\isanewline
\ \ \isakeyword{assumes}\ {\isadigit{0}}{\isacharcolon}\ {\isachardoublequoteopen}subspace\ K\ W\ V{\isachardoublequoteclose}\isanewline
\ \ \isakeyword{shows}\ {\isachardoublequoteopen}vectorspace\ K\ {\isacharparenleft}vs\ W{\isacharparenright}{\isachardoublequoteclose}\isanewline
%
\isadelimproof
%
\endisadelimproof
%
\isatagproof
\isacommand{proof}\isamarkupfalse%
\ {\isacharminus}\isanewline
\ \ \isacommand{from}\isamarkupfalse%
\ {\isadigit{0}}\ \isacommand{show}\isamarkupfalse%
\ {\isacharquery}thesis\isanewline
\ \ \ \ \isacommand{apply}\isamarkupfalse%
\ {\isacharparenleft}unfold\ vectorspace{\isacharunderscore}def\ subspace{\isacharunderscore}def{\isacharcomma}\ auto{\isacharparenright}\isanewline
\ \ \ \ \isacommand{by}\isamarkupfalse%
\ {\isacharparenleft}intro\ submodule{\isacharunderscore}is{\isacharunderscore}module{\isacharcomma}\ auto{\isacharparenright}\isanewline
\isacommand{qed}\isamarkupfalse%
%
\endisatagproof
{\isafoldproof}%
%
\isadelimproof
\isanewline
%
\endisadelimproof
\isanewline
\isacommand{abbreviation}\isamarkupfalse%
\ {\isacharparenleft}\isakeyword{in}\ module{\isacharparenright}\ subspace{\isacharunderscore}sum{\isacharcolon}{\isacharcolon}\ {\isachardoublequoteopen}{\isacharbrackleft}{\isacharprime}c\ set{\isacharcomma}\ {\isacharprime}c\ set{\isacharbrackright}\ {\isasymRightarrow}\ {\isacharprime}c\ set{\isachardoublequoteclose}\isanewline
\ \ \isakeyword{where}\ {\isachardoublequoteopen}subspace{\isacharunderscore}sum\ W{\isadigit{1}}\ W{\isadigit{2}}\ {\isasymequiv}submodule{\isacharunderscore}sum\ W{\isadigit{1}}\ W{\isadigit{2}}{\isachardoublequoteclose}\isanewline
\isanewline
\isacommand{lemma}\isamarkupfalse%
\ {\isacharparenleft}\isakeyword{in}\ vectorspace{\isacharparenright}\ vs{\isacharunderscore}zero{\isacharunderscore}lin{\isacharunderscore}dep{\isacharcolon}\ \isanewline
\ \ \isakeyword{assumes}\ h{\isadigit{2}}{\isacharcolon}\ {\isachardoublequoteopen}S{\isasymsubseteq}carrier\ V{\isachardoublequoteclose}\ \isakeyword{and}\ h{\isadigit{3}}{\isacharcolon}\ {\isachardoublequoteopen}lin{\isacharunderscore}indpt\ S{\isachardoublequoteclose}\isanewline
\ \ \isakeyword{shows}\ {\isachardoublequoteopen}{\isasymzero}\isactrlbsub V\isactrlesub \ {\isasymnotin}\ S{\isachardoublequoteclose}\isanewline
%
\isadelimproof
%
\endisadelimproof
%
\isatagproof
\isacommand{proof}\isamarkupfalse%
\ {\isacharminus}\isanewline
\ \ \isacommand{have}\isamarkupfalse%
\ vs{\isacharcolon}\ {\isachardoublequoteopen}vectorspace\ K\ V{\isachardoublequoteclose}\isacommand{{\isachardot}{\isachardot}}\isamarkupfalse%
\isanewline
\ \ \isacommand{from}\isamarkupfalse%
\ vs\ \isacommand{have}\isamarkupfalse%
\ nonzero{\isacharcolon}\ {\isachardoublequoteopen}carrier\ K\ {\isasymnoteq}{\isacharbraceleft}{\isasymzero}\isactrlbsub K\isactrlesub {\isacharbraceright}{\isachardoublequoteclose}\isanewline
\ \ \ \ \isacommand{by}\isamarkupfalse%
\ {\isacharparenleft}metis\ one{\isacharunderscore}zeroI\ zero{\isacharunderscore}not{\isacharunderscore}one{\isacharparenright}\isanewline
\ \ \isacommand{from}\isamarkupfalse%
\ h{\isadigit{2}}\ h{\isadigit{3}}\ nonzero\ \isacommand{show}\isamarkupfalse%
\ {\isacharquery}thesis\ \isacommand{by}\isamarkupfalse%
\ {\isacharparenleft}rule\ zero{\isacharunderscore}nin{\isacharunderscore}lin{\isacharunderscore}indpt{\isacharparenright}\isanewline
\isacommand{qed}\isamarkupfalse%
%
\endisatagproof
{\isafoldproof}%
%
\isadelimproof
%
\endisadelimproof
%
\begin{isamarkuptext}%
A \isa{linear{\isacharunderscore}map} is a module homomorphism between 2 vectorspaces
over the same field.%
\end{isamarkuptext}%
\isamarkuptrue%
\isacommand{locale}\isamarkupfalse%
\ linear{\isacharunderscore}map\ {\isacharequal}\ \isanewline
\ \ V{\isacharcolon}\ vectorspace\ K\ V\ {\isacharplus}\ W{\isacharcolon}\ vectorspace\ K\ W\isanewline
\ \ {\isacharplus}\ mod{\isacharunderscore}hom{\isacharcolon}\ mod{\isacharunderscore}hom\ K\ V\ W\ T\isanewline
\ \ \ \ \isakeyword{for}\ K\ \isakeyword{and}\ V\ \isakeyword{and}\ W\ \isakeyword{and}\ T\isanewline
\isanewline
\isacommand{context}\isamarkupfalse%
\ linear{\isacharunderscore}map\isanewline
\isakeyword{begin}\isanewline
\isacommand{lemmas}\isamarkupfalse%
\ T{\isacharunderscore}hom\ {\isacharequal}\ f{\isacharunderscore}hom\isanewline
\isacommand{lemmas}\isamarkupfalse%
\ T{\isacharunderscore}add\ {\isacharequal}\ f{\isacharunderscore}add\isanewline
\isacommand{lemmas}\isamarkupfalse%
\ T{\isacharunderscore}smult\ {\isacharequal}\ f{\isacharunderscore}smult\isanewline
\isacommand{lemmas}\isamarkupfalse%
\ T{\isacharunderscore}im\ {\isacharequal}\ f{\isacharunderscore}im\isanewline
\isacommand{lemmas}\isamarkupfalse%
\ T{\isacharunderscore}neg\ {\isacharequal}\ f{\isacharunderscore}neg\isanewline
\isacommand{lemmas}\isamarkupfalse%
\ T{\isacharunderscore}minus\ {\isacharequal}\ f{\isacharunderscore}minus\isanewline
\isacommand{lemmas}\isamarkupfalse%
\ T{\isacharunderscore}ker\ {\isacharequal}\ f{\isacharunderscore}ker\isanewline
\isanewline
\isacommand{abbreviation}\isamarkupfalse%
\ imT{\isacharcolon}{\isacharcolon}\ {\isachardoublequoteopen}{\isacharprime}e\ set{\isachardoublequoteclose}\isanewline
\ \ \isakeyword{where}\ {\isachardoublequoteopen}imT\ {\isasymequiv}\ mod{\isacharunderscore}hom{\isachardot}im{\isachardoublequoteclose}\isanewline
\isanewline
\isacommand{abbreviation}\isamarkupfalse%
\ kerT{\isacharcolon}{\isacharcolon}\ {\isachardoublequoteopen}{\isacharprime}c\ set{\isachardoublequoteclose}\isanewline
\ \ \isakeyword{where}\ {\isachardoublequoteopen}kerT\ {\isasymequiv}\ mod{\isacharunderscore}hom{\isachardot}ker{\isachardoublequoteclose}\isanewline
\isanewline
\isacommand{lemmas}\isamarkupfalse%
\ T{\isadigit{0}}{\isacharunderscore}is{\isacharunderscore}{\isadigit{0}}{\isacharbrackleft}simp{\isacharbrackright}\ {\isacharequal}\ f{\isadigit{0}}{\isacharunderscore}is{\isacharunderscore}{\isadigit{0}}\isanewline
\isanewline
\isacommand{lemma}\isamarkupfalse%
\ kerT{\isacharunderscore}is{\isacharunderscore}subspace{\isacharcolon}\ {\isachardoublequoteopen}subspace\ K\ ker\ V{\isachardoublequoteclose}\isanewline
%
\isadelimproof
%
\endisadelimproof
%
\isatagproof
\isacommand{proof}\isamarkupfalse%
\ {\isacharminus}\ \isanewline
\ \ \isacommand{have}\isamarkupfalse%
\ vs{\isacharcolon}\ {\isachardoublequoteopen}vectorspace\ K\ V{\isachardoublequoteclose}\isacommand{{\isachardot}{\isachardot}}\isamarkupfalse%
\isanewline
\ \ \isacommand{from}\isamarkupfalse%
\ vs\ \isacommand{show}\isamarkupfalse%
\ {\isacharquery}thesis\isanewline
\ \ \ \ \isacommand{apply}\isamarkupfalse%
\ {\isacharparenleft}unfold\ subspace{\isacharunderscore}def{\isacharcomma}\ auto{\isacharparenright}\isanewline
\ \ \ \ \isacommand{by}\isamarkupfalse%
\ {\isacharparenleft}rule\ ker{\isacharunderscore}is{\isacharunderscore}submodule{\isacharparenright}\isanewline
\isacommand{qed}\isamarkupfalse%
%
\endisatagproof
{\isafoldproof}%
%
\isadelimproof
\isanewline
%
\endisadelimproof
\isanewline
\isacommand{lemma}\isamarkupfalse%
\ imT{\isacharunderscore}is{\isacharunderscore}subspace{\isacharcolon}\ {\isachardoublequoteopen}subspace\ K\ imT\ W{\isachardoublequoteclose}\isanewline
%
\isadelimproof
%
\endisadelimproof
%
\isatagproof
\isacommand{proof}\isamarkupfalse%
\ {\isacharminus}\ \isanewline
\ \ \isacommand{have}\isamarkupfalse%
\ vs{\isacharcolon}\ {\isachardoublequoteopen}vectorspace\ K\ W{\isachardoublequoteclose}\isacommand{{\isachardot}{\isachardot}}\isamarkupfalse%
\isanewline
\ \ \isacommand{from}\isamarkupfalse%
\ vs\ \isacommand{show}\isamarkupfalse%
\ {\isacharquery}thesis\isanewline
\ \ \ \ \isacommand{apply}\isamarkupfalse%
\ {\isacharparenleft}unfold\ subspace{\isacharunderscore}def{\isacharcomma}\ auto{\isacharparenright}\isanewline
\ \ \ \ \isacommand{by}\isamarkupfalse%
\ {\isacharparenleft}rule\ im{\isacharunderscore}is{\isacharunderscore}submodule{\isacharparenright}\isanewline
\isacommand{qed}\isamarkupfalse%
%
\endisatagproof
{\isafoldproof}%
%
\isadelimproof
\isanewline
%
\endisadelimproof
\isacommand{end}\isamarkupfalse%
\isanewline
\isanewline
\isacommand{lemma}\isamarkupfalse%
\ vs{\isacharunderscore}criteria{\isacharcolon}\isanewline
\ \ \isakeyword{fixes}\ K\ \isakeyword{and}\ V\ \isanewline
\ \ \isakeyword{assumes}\ field{\isacharcolon}\ {\isachardoublequoteopen}field\ K{\isachardoublequoteclose}\isanewline
\ \ \ \ \ \ \isakeyword{and}\ zero{\isacharcolon}\ {\isachardoublequoteopen}{\isasymzero}\isactrlbsub V\isactrlesub {\isasymin}\ carrier\ V{\isachardoublequoteclose}\ \isanewline
\ \ \ \ \ \ \isakeyword{and}\ add{\isacharcolon}\ {\isachardoublequoteopen}{\isasymforall}v\ w{\isachardot}\ v{\isasymin}carrier\ V\ {\isasymand}\ w{\isasymin}carrier\ V{\isasymlongrightarrow}\ v{\isasymoplus}\isactrlbsub V\isactrlesub \ w{\isasymin}\ carrier\ V{\isachardoublequoteclose}\isanewline
\ \ \ \ \ \ \isakeyword{and}\ neg{\isacharcolon}\ {\isachardoublequoteopen}{\isasymforall}v{\isasymin}carrier\ V{\isachardot}\ {\isacharparenleft}{\isasymexists}neg{\isacharunderscore}v{\isasymin}carrier\ V{\isachardot}\ v{\isasymoplus}\isactrlbsub V\isactrlesub neg{\isacharunderscore}v{\isacharequal}{\isasymzero}\isactrlbsub V\isactrlesub {\isacharparenright}{\isachardoublequoteclose}\isanewline
\ \ \ \ \ \ \isakeyword{and}\ smult{\isacharcolon}\ {\isachardoublequoteopen}{\isasymforall}c\ v{\isachardot}\ c{\isasymin}\ carrier\ K\ {\isasymand}\ v{\isasymin}carrier\ V{\isasymlongrightarrow}\ c{\isasymodot}\isactrlbsub V\isactrlesub \ v\ {\isasymin}\ carrier\ V{\isachardoublequoteclose}\isanewline
\ \ \ \ \ \ \isakeyword{and}\ comm{\isacharcolon}\ {\isachardoublequoteopen}{\isasymforall}v\ w{\isachardot}\ v{\isasymin}carrier\ V\ {\isasymand}\ w{\isasymin}carrier\ V{\isasymlongrightarrow}\ v{\isasymoplus}\isactrlbsub V\isactrlesub \ w{\isacharequal}w{\isasymoplus}\isactrlbsub V\isactrlesub \ v{\isachardoublequoteclose}\isanewline
\ \ \ \ \ \ \isakeyword{and}\ assoc{\isacharcolon}\ {\isachardoublequoteopen}{\isasymforall}v\ w\ x{\isachardot}\ v{\isasymin}carrier\ V\ {\isasymand}\ w{\isasymin}carrier\ V\ {\isasymand}\ x{\isasymin}carrier\ V{\isasymlongrightarrow}\ {\isacharparenleft}v{\isasymoplus}\isactrlbsub V\isactrlesub \ w{\isacharparenright}{\isasymoplus}\isactrlbsub V\isactrlesub \ x{\isacharequal}\ v{\isasymoplus}\isactrlbsub V\isactrlesub \ {\isacharparenleft}w{\isasymoplus}\isactrlbsub V\isactrlesub \ x{\isacharparenright}{\isachardoublequoteclose}\isanewline
\ \ \ \ \ \ \isakeyword{and}\ add{\isacharunderscore}id{\isacharcolon}\ {\isachardoublequoteopen}{\isasymforall}v{\isasymin}carrier\ V{\isachardot}\ {\isacharparenleft}v{\isasymoplus}\isactrlbsub V\isactrlesub \ {\isasymzero}\isactrlbsub V\isactrlesub \ {\isacharequal}v{\isacharparenright}{\isachardoublequoteclose}\isanewline
\ \ \ \ \ \ \isakeyword{and}\ compat{\isacharcolon}\ {\isachardoublequoteopen}{\isasymforall}a\ b\ v{\isachardot}\ a{\isasymin}\ carrier\ K\ {\isasymand}\ b{\isasymin}\ carrier\ K\ {\isasymand}\ v{\isasymin}carrier\ V{\isasymlongrightarrow}\ {\isacharparenleft}a{\isasymotimes}\isactrlbsub K\isactrlesub \ b{\isacharparenright}{\isasymodot}\isactrlbsub V\isactrlesub \ v\ {\isacharequal}a{\isasymodot}\isactrlbsub V\isactrlesub \ {\isacharparenleft}b{\isasymodot}\isactrlbsub V\isactrlesub \ v{\isacharparenright}{\isachardoublequoteclose}\isanewline
\ \ \ \ \ \ \isakeyword{and}\ smult{\isacharunderscore}id{\isacharcolon}\ {\isachardoublequoteopen}{\isasymforall}v{\isasymin}carrier\ V{\isachardot}\ {\isacharparenleft}{\isasymone}\isactrlbsub K\isactrlesub \ {\isasymodot}\isactrlbsub V\isactrlesub \ v\ {\isacharequal}v{\isacharparenright}{\isachardoublequoteclose}\isanewline
\ \ \ \ \ \ \isakeyword{and}\ dist{\isacharunderscore}f{\isacharcolon}\ {\isachardoublequoteopen}{\isasymforall}a\ b\ v{\isachardot}\ a{\isasymin}\ carrier\ K\ {\isasymand}\ b{\isasymin}\ carrier\ K\ {\isasymand}\ v{\isasymin}carrier\ V{\isasymlongrightarrow}\ {\isacharparenleft}a{\isasymoplus}\isactrlbsub K\isactrlesub \ b{\isacharparenright}{\isasymodot}\isactrlbsub V\isactrlesub \ v\ {\isacharequal}{\isacharparenleft}a{\isasymodot}\isactrlbsub V\isactrlesub \ v{\isacharparenright}\ {\isasymoplus}\isactrlbsub V\isactrlesub \ {\isacharparenleft}b{\isasymodot}\isactrlbsub V\isactrlesub \ v{\isacharparenright}{\isachardoublequoteclose}\isanewline
\ \ \ \ \ \ \isakeyword{and}\ dist{\isacharunderscore}add{\isacharcolon}\ {\isachardoublequoteopen}{\isasymforall}a\ v\ w{\isachardot}\ a{\isasymin}\ carrier\ K\ {\isasymand}\ v{\isasymin}carrier\ V\ {\isasymand}\ w{\isasymin}carrier\ V{\isasymlongrightarrow}\ a{\isasymodot}\isactrlbsub V\isactrlesub \ {\isacharparenleft}v{\isasymoplus}\isactrlbsub V\isactrlesub \ w{\isacharparenright}\ {\isacharequal}{\isacharparenleft}a{\isasymodot}\isactrlbsub V\isactrlesub \ v{\isacharparenright}\ {\isasymoplus}\isactrlbsub V\isactrlesub \ {\isacharparenleft}a{\isasymodot}\isactrlbsub V\isactrlesub \ w{\isacharparenright}{\isachardoublequoteclose}\isanewline
\ \ \isakeyword{shows}\ {\isachardoublequoteopen}vectorspace\ K\ V{\isachardoublequoteclose}\isanewline
%
\isadelimproof
%
\endisadelimproof
%
\isatagproof
\isacommand{proof}\isamarkupfalse%
\ {\isacharminus}\ \isanewline
\ \ \isacommand{from}\isamarkupfalse%
\ field\ \isacommand{have}\isamarkupfalse%
\ {\isadigit{1}}{\isacharcolon}\ {\isachardoublequoteopen}cring\ K{\isachardoublequoteclose}\ \isacommand{by}\isamarkupfalse%
\ {\isacharparenleft}unfold\ field{\isacharunderscore}def\ domain{\isacharunderscore}def{\isacharcomma}\ auto{\isacharparenright}\isanewline
\ \ \isacommand{from}\isamarkupfalse%
\ assms\ {\isadigit{1}}\ \isacommand{have}\isamarkupfalse%
\ {\isadigit{2}}{\isacharcolon}\ {\isachardoublequoteopen}module\ K\ V{\isachardoublequoteclose}\ \isacommand{by}\isamarkupfalse%
\ {\isacharparenleft}intro\ module{\isacharunderscore}criteria{\isacharcomma}\ auto{\isacharparenright}\isanewline
\ \ \isacommand{from}\isamarkupfalse%
\ field\ {\isadigit{2}}\ \isacommand{show}\isamarkupfalse%
\ {\isacharquery}thesis\ \isacommand{by}\isamarkupfalse%
\ {\isacharparenleft}unfold\ vectorspace{\isacharunderscore}def\ module{\isacharunderscore}def{\isacharcomma}\ auto{\isacharparenright}\isanewline
\isacommand{qed}\isamarkupfalse%
%
\endisatagproof
{\isafoldproof}%
%
\isadelimproof
%
\endisadelimproof
%
\begin{isamarkuptext}%
For any set $S$, the space of functions $S\to K$ forms a vector space.%
\end{isamarkuptext}%
\isamarkuptrue%
\isacommand{lemma}\isamarkupfalse%
\ {\isacharparenleft}\isakeyword{in}\ vectorspace{\isacharparenright}\ func{\isacharunderscore}space{\isacharunderscore}is{\isacharunderscore}vs{\isacharcolon}\isanewline
\ \ \isakeyword{fixes}\ S\isanewline
\ \ \isakeyword{shows}\ {\isachardoublequoteopen}vectorspace\ K\ {\isacharparenleft}func{\isacharunderscore}space\ S{\isacharparenright}{\isachardoublequoteclose}\ \isanewline
%
\isadelimproof
%
\endisadelimproof
%
\isatagproof
\isacommand{proof}\isamarkupfalse%
\ {\isacharminus}\isanewline
\ \ \isacommand{have}\isamarkupfalse%
\ {\isadigit{0}}{\isacharcolon}\ {\isachardoublequoteopen}field\ K{\isachardoublequoteclose}\isacommand{{\isachardot}{\isachardot}}\isamarkupfalse%
\isanewline
\ \ \isacommand{have}\isamarkupfalse%
\ {\isadigit{1}}{\isacharcolon}\ {\isachardoublequoteopen}module\ K\ {\isacharparenleft}func{\isacharunderscore}space\ S{\isacharparenright}{\isachardoublequoteclose}\ \isacommand{by}\isamarkupfalse%
\ {\isacharparenleft}rule\ func{\isacharunderscore}space{\isacharunderscore}is{\isacharunderscore}module{\isacharparenright}\isanewline
\ \ \isacommand{from}\isamarkupfalse%
\ {\isadigit{0}}\ {\isadigit{1}}\ \isacommand{show}\isamarkupfalse%
\ {\isacharquery}thesis\ \isacommand{by}\isamarkupfalse%
\ {\isacharparenleft}unfold\ vectorspace{\isacharunderscore}def\ module{\isacharunderscore}def{\isacharcomma}\ auto{\isacharparenright}\isanewline
\isacommand{qed}\isamarkupfalse%
%
\endisatagproof
{\isafoldproof}%
%
\isadelimproof
\isanewline
%
\endisadelimproof
\isanewline
\isanewline
\isacommand{lemma}\isamarkupfalse%
\ direct{\isacharunderscore}sum{\isacharunderscore}is{\isacharunderscore}vs{\isacharcolon}\ \isanewline
\ \ \isakeyword{fixes}\ K\ V{\isadigit{1}}\ V{\isadigit{2}}\isanewline
\ \ \isakeyword{assumes}\ h{\isadigit{1}}{\isacharcolon}\ {\isachardoublequoteopen}vectorspace\ K\ V{\isadigit{1}}{\isachardoublequoteclose}\ \isakeyword{and}\ h{\isadigit{2}}{\isacharcolon}\ {\isachardoublequoteopen}vectorspace\ K\ V{\isadigit{2}}{\isachardoublequoteclose}\isanewline
\ \ \isakeyword{shows}\ {\isachardoublequoteopen}vectorspace\ K\ {\isacharparenleft}direct{\isacharunderscore}sum\ V{\isadigit{1}}\ V{\isadigit{2}}{\isacharparenright}{\isachardoublequoteclose}\isanewline
%
\isadelimproof
%
\endisadelimproof
%
\isatagproof
\isacommand{proof}\isamarkupfalse%
\ {\isacharminus}\isanewline
\ \ \isacommand{from}\isamarkupfalse%
\ h{\isadigit{1}}\ h{\isadigit{2}}\ \isacommand{have}\isamarkupfalse%
\ mod{\isacharcolon}\ {\isachardoublequoteopen}module\ K\ {\isacharparenleft}direct{\isacharunderscore}sum\ V{\isadigit{1}}\ V{\isadigit{2}}{\isacharparenright}{\isachardoublequoteclose}\ \isacommand{by}\isamarkupfalse%
\ {\isacharparenleft}unfold\ vectorspace{\isacharunderscore}def{\isacharcomma}\ intro\ direct{\isacharunderscore}sum{\isacharunderscore}is{\isacharunderscore}module{\isacharcomma}\ auto{\isacharparenright}\isanewline
\ \ \isacommand{from}\isamarkupfalse%
\ mod\ h{\isadigit{1}}\ \isacommand{show}\isamarkupfalse%
\ {\isacharquery}thesis\ \isacommand{by}\isamarkupfalse%
\ {\isacharparenleft}unfold\ vectorspace{\isacharunderscore}def{\isacharcomma}\ auto{\isacharparenright}\isanewline
\isacommand{qed}\isamarkupfalse%
%
\endisatagproof
{\isafoldproof}%
%
\isadelimproof
\isanewline
%
\endisadelimproof
\isanewline
\isacommand{lemma}\isamarkupfalse%
\ inj{\isadigit{1}}{\isacharunderscore}linear{\isacharcolon}\isanewline
\ \ \isakeyword{fixes}\ K\ V{\isadigit{1}}\ V{\isadigit{2}}\isanewline
\ \ \isakeyword{assumes}\ h{\isadigit{1}}{\isacharcolon}\ {\isachardoublequoteopen}vectorspace\ K\ V{\isadigit{1}}{\isachardoublequoteclose}\ \isakeyword{and}\ h{\isadigit{2}}{\isacharcolon}\ {\isachardoublequoteopen}vectorspace\ K\ V{\isadigit{2}}{\isachardoublequoteclose}\isanewline
\ \ \isakeyword{shows}\ {\isachardoublequoteopen}linear{\isacharunderscore}map\ K\ V{\isadigit{1}}\ {\isacharparenleft}direct{\isacharunderscore}sum\ V{\isadigit{1}}\ V{\isadigit{2}}{\isacharparenright}\ {\isacharparenleft}inj{\isadigit{1}}\ V{\isadigit{1}}\ V{\isadigit{2}}{\isacharparenright}{\isachardoublequoteclose}\isanewline
%
\isadelimproof
%
\endisadelimproof
%
\isatagproof
\isacommand{proof}\isamarkupfalse%
\ {\isacharminus}\ \isanewline
\ \ \isacommand{from}\isamarkupfalse%
\ h{\isadigit{1}}\ h{\isadigit{2}}\ \isacommand{have}\isamarkupfalse%
\ mod{\isacharcolon}\ {\isachardoublequoteopen}mod{\isacharunderscore}hom\ K\ V{\isadigit{1}}\ {\isacharparenleft}direct{\isacharunderscore}sum\ V{\isadigit{1}}\ V{\isadigit{2}}{\isacharparenright}\ {\isacharparenleft}inj{\isadigit{1}}\ V{\isadigit{1}}\ V{\isadigit{2}}{\isacharparenright}{\isachardoublequoteclose}\ \isacommand{by}\isamarkupfalse%
\ {\isacharparenleft}unfold\ vectorspace{\isacharunderscore}def{\isacharcomma}\ intro\ inj{\isadigit{1}}{\isacharunderscore}hom{\isacharcomma}\ auto{\isacharparenright}\isanewline
\ \ \isacommand{from}\isamarkupfalse%
\ mod\ h{\isadigit{1}}\ h{\isadigit{2}}\ \isacommand{show}\isamarkupfalse%
\ {\isacharquery}thesis\ \isanewline
\ \ \ \ \isacommand{by}\isamarkupfalse%
\ {\isacharparenleft}unfold\ linear{\isacharunderscore}map{\isacharunderscore}def\ vectorspace{\isacharunderscore}def\ {\isacharcomma}\ auto{\isacharcomma}\ intro\ direct{\isacharunderscore}sum{\isacharunderscore}is{\isacharunderscore}module{\isacharcomma}\ auto{\isacharparenright}\isanewline
\isacommand{qed}\isamarkupfalse%
%
\endisatagproof
{\isafoldproof}%
%
\isadelimproof
\isanewline
%
\endisadelimproof
\isanewline
\isacommand{lemma}\isamarkupfalse%
\ inj{\isadigit{2}}{\isacharunderscore}linear{\isacharcolon}\isanewline
\ \ \isakeyword{fixes}\ K\ V{\isadigit{1}}\ V{\isadigit{2}}\isanewline
\ \ \isakeyword{assumes}\ h{\isadigit{1}}{\isacharcolon}\ {\isachardoublequoteopen}vectorspace\ K\ V{\isadigit{1}}{\isachardoublequoteclose}\ \isakeyword{and}\ h{\isadigit{2}}{\isacharcolon}\ {\isachardoublequoteopen}vectorspace\ K\ V{\isadigit{2}}{\isachardoublequoteclose}\isanewline
\ \ \isakeyword{shows}\ {\isachardoublequoteopen}linear{\isacharunderscore}map\ K\ V{\isadigit{2}}\ {\isacharparenleft}direct{\isacharunderscore}sum\ V{\isadigit{1}}\ V{\isadigit{2}}{\isacharparenright}\ {\isacharparenleft}inj{\isadigit{2}}\ V{\isadigit{1}}\ V{\isadigit{2}}{\isacharparenright}{\isachardoublequoteclose}\isanewline
%
\isadelimproof
%
\endisadelimproof
%
\isatagproof
\isacommand{proof}\isamarkupfalse%
\ {\isacharminus}\ \isanewline
\ \ \isacommand{from}\isamarkupfalse%
\ h{\isadigit{1}}\ h{\isadigit{2}}\ \isacommand{have}\isamarkupfalse%
\ mod{\isacharcolon}\ {\isachardoublequoteopen}mod{\isacharunderscore}hom\ K\ V{\isadigit{2}}\ {\isacharparenleft}direct{\isacharunderscore}sum\ V{\isadigit{1}}\ V{\isadigit{2}}{\isacharparenright}\ {\isacharparenleft}inj{\isadigit{2}}\ V{\isadigit{1}}\ V{\isadigit{2}}{\isacharparenright}{\isachardoublequoteclose}\ \isacommand{by}\isamarkupfalse%
\ {\isacharparenleft}unfold\ vectorspace{\isacharunderscore}def{\isacharcomma}\ intro\ inj{\isadigit{2}}{\isacharunderscore}hom{\isacharcomma}\ auto{\isacharparenright}\isanewline
\ \ \isacommand{from}\isamarkupfalse%
\ mod\ h{\isadigit{1}}\ h{\isadigit{2}}\ \isacommand{show}\isamarkupfalse%
\ {\isacharquery}thesis\ \isanewline
\ \ \ \ \isacommand{by}\isamarkupfalse%
\ {\isacharparenleft}unfold\ linear{\isacharunderscore}map{\isacharunderscore}def\ vectorspace{\isacharunderscore}def\ {\isacharcomma}\ auto{\isacharcomma}\ intro\ direct{\isacharunderscore}sum{\isacharunderscore}is{\isacharunderscore}module{\isacharcomma}\ auto{\isacharparenright}\isanewline
\isacommand{qed}\isamarkupfalse%
%
\endisatagproof
{\isafoldproof}%
%
\isadelimproof
%
\endisadelimproof
%
\begin{isamarkuptext}%
For subspaces $V_1,V_2\subseteq V$, the map $V_1\oplus V_2\to V$ given by $(v_1,v_2)\mapsto 
v_1+v_2$ is linear.%
\end{isamarkuptext}%
\isamarkuptrue%
\isacommand{lemma}\isamarkupfalse%
\ {\isacharparenleft}\isakeyword{in}\ vectorspace{\isacharparenright}\ sum{\isacharunderscore}map{\isacharunderscore}linear{\isacharcolon}\ \isanewline
\ \ \isakeyword{fixes}\ V{\isadigit{1}}\ V{\isadigit{2}}\isanewline
\ \ \isakeyword{assumes}\ h{\isadigit{1}}{\isacharcolon}\ {\isachardoublequoteopen}subspace\ K\ V{\isadigit{1}}\ V{\isachardoublequoteclose}\ \isakeyword{and}\ h{\isadigit{2}}{\isacharcolon}\ {\isachardoublequoteopen}subspace\ K\ V{\isadigit{2}}\ V{\isachardoublequoteclose}\isanewline
\ \ \isakeyword{shows}\ {\isachardoublequoteopen}linear{\isacharunderscore}map\ K\ {\isacharparenleft}direct{\isacharunderscore}sum\ {\isacharparenleft}vs\ V{\isadigit{1}}{\isacharparenright}\ {\isacharparenleft}vs\ V{\isadigit{2}}{\isacharparenright}{\isacharparenright}\ V\ {\isacharparenleft}{\isasymlambda}\ v{\isachardot}\ {\isacharparenleft}fst\ v{\isacharparenright}\ {\isasymoplus}\isactrlbsub V\isactrlesub \ {\isacharparenleft}snd\ v{\isacharparenright}{\isacharparenright}{\isachardoublequoteclose}\isanewline
%
\isadelimproof
%
\endisadelimproof
%
\isatagproof
\isacommand{proof}\isamarkupfalse%
\ {\isacharminus}\ \isanewline
\ \ \isacommand{from}\isamarkupfalse%
\ h{\isadigit{1}}\ h{\isadigit{2}}\ \isacommand{have}\isamarkupfalse%
\ mod{\isacharcolon}\ {\isachardoublequoteopen}mod{\isacharunderscore}hom\ K\ {\isacharparenleft}direct{\isacharunderscore}sum\ {\isacharparenleft}vs\ V{\isadigit{1}}{\isacharparenright}\ {\isacharparenleft}vs\ V{\isadigit{2}}{\isacharparenright}{\isacharparenright}\ V\ {\isacharparenleft}{\isasymlambda}\ v{\isachardot}\ {\isacharparenleft}fst\ v{\isacharparenright}\ {\isasymoplus}\isactrlbsub V\isactrlesub \ {\isacharparenleft}snd\ v{\isacharparenright}{\isacharparenright}{\isachardoublequoteclose}\ \isanewline
\ \ \ \ \isacommand{by}\isamarkupfalse%
\ {\isacharparenleft}\ intro\ sum{\isacharunderscore}map{\isacharunderscore}hom{\isacharcomma}\ unfold\ subspace{\isacharunderscore}def{\isacharcomma}\ auto{\isacharparenright}\isanewline
\ \ \isacommand{from}\isamarkupfalse%
\ mod\ h{\isadigit{1}}\ h{\isadigit{2}}\ \isacommand{show}\isamarkupfalse%
\ {\isacharquery}thesis\ \isanewline
\ \ \ \ \isacommand{apply}\isamarkupfalse%
\ {\isacharparenleft}unfold\ linear{\isacharunderscore}map{\isacharunderscore}def{\isacharcomma}\ auto{\isacharparenright}\ \isacommand{apply}\isamarkupfalse%
\ {\isacharparenleft}intro\ direct{\isacharunderscore}sum{\isacharunderscore}is{\isacharunderscore}vs\ subspace{\isacharunderscore}is{\isacharunderscore}vs{\isacharcomma}\ auto{\isacharparenright}\isacommand{{\isachardot}{\isachardot}}\isamarkupfalse%
\ \isanewline
\isacommand{qed}\isamarkupfalse%
%
\endisatagproof
{\isafoldproof}%
%
\isadelimproof
\isanewline
%
\endisadelimproof
\isanewline
\isacommand{lemma}\isamarkupfalse%
\ {\isacharparenleft}\isakeyword{in}\ vectorspace{\isacharparenright}\ sum{\isacharunderscore}is{\isacharunderscore}subspace{\isacharcolon}\isanewline
\ \ \isakeyword{fixes}\ W{\isadigit{1}}\ W{\isadigit{2}}\isanewline
\ \ \isakeyword{assumes}\ h{\isadigit{1}}{\isacharcolon}\ {\isachardoublequoteopen}subspace\ K\ W{\isadigit{1}}\ V{\isachardoublequoteclose}\ \isakeyword{and}\ h{\isadigit{2}}{\isacharcolon}\ {\isachardoublequoteopen}subspace\ K\ W{\isadigit{2}}\ V{\isachardoublequoteclose}\isanewline
\ \ \isakeyword{shows}\ {\isachardoublequoteopen}subspace\ K\ {\isacharparenleft}subspace{\isacharunderscore}sum\ W{\isadigit{1}}\ W{\isadigit{2}}{\isacharparenright}\ V{\isachardoublequoteclose}\isanewline
%
\isadelimproof
%
\endisadelimproof
%
\isatagproof
\isacommand{proof}\isamarkupfalse%
\ {\isacharminus}\isanewline
\ \ \isacommand{from}\isamarkupfalse%
\ h{\isadigit{1}}\ h{\isadigit{2}}\ \isacommand{have}\isamarkupfalse%
\ mod{\isacharcolon}\ {\isachardoublequoteopen}submodule\ K\ {\isacharparenleft}submodule{\isacharunderscore}sum\ W{\isadigit{1}}\ W{\isadigit{2}}{\isacharparenright}\ V{\isachardoublequoteclose}\ \isanewline
\ \ \ \ \isacommand{by}\isamarkupfalse%
\ {\isacharparenleft}\ intro\ sum{\isacharunderscore}is{\isacharunderscore}submodule{\isacharcomma}\ unfold\ subspace{\isacharunderscore}def{\isacharcomma}\ auto{\isacharparenright}\isanewline
\ \ \isacommand{from}\isamarkupfalse%
\ mod\ h{\isadigit{1}}\ h{\isadigit{2}}\ \isacommand{show}\isamarkupfalse%
\ {\isacharquery}thesis\ \isanewline
\ \ \ \ \isacommand{by}\isamarkupfalse%
\ {\isacharparenleft}unfold\ subspace{\isacharunderscore}def{\isacharcomma}\ auto{\isacharparenright}\isanewline
\isacommand{qed}\isamarkupfalse%
%
\endisatagproof
{\isafoldproof}%
%
\isadelimproof
%
\endisadelimproof
%
\begin{isamarkuptext}%
If $W_1,W_2\subseteq V$ are subspaces, $W_1\subseteq W_1 + W_2$%
\end{isamarkuptext}%
\isamarkuptrue%
\isacommand{lemma}\isamarkupfalse%
\ {\isacharparenleft}\isakeyword{in}\ vectorspace{\isacharparenright}\ in{\isacharunderscore}sum{\isacharunderscore}vs{\isacharcolon}\isanewline
\ \ \isakeyword{fixes}\ W{\isadigit{1}}\ W{\isadigit{2}}\isanewline
\ \ \isakeyword{assumes}\ h{\isadigit{1}}{\isacharcolon}\ {\isachardoublequoteopen}subspace\ K\ W{\isadigit{1}}\ V{\isachardoublequoteclose}\ \isakeyword{and}\ h{\isadigit{2}}{\isacharcolon}\ {\isachardoublequoteopen}subspace\ K\ W{\isadigit{2}}\ V{\isachardoublequoteclose}\isanewline
\ \ \isakeyword{shows}\ {\isachardoublequoteopen}W{\isadigit{1}}\ {\isasymsubseteq}\ subspace{\isacharunderscore}sum\ W{\isadigit{1}}\ W{\isadigit{2}}{\isachardoublequoteclose}\isanewline
%
\isadelimproof
%
\endisadelimproof
%
\isatagproof
\isacommand{proof}\isamarkupfalse%
\ {\isacharminus}\isanewline
\ \ \isacommand{from}\isamarkupfalse%
\ h{\isadigit{1}}\ h{\isadigit{2}}\ \isacommand{show}\isamarkupfalse%
\ {\isacharquery}thesis\ \isacommand{by}\isamarkupfalse%
\ {\isacharparenleft}intro\ in{\isacharunderscore}sum{\isacharcomma}\ unfold\ subspace{\isacharunderscore}def{\isacharcomma}\ auto{\isacharparenright}\isanewline
\isacommand{qed}\isamarkupfalse%
%
\endisatagproof
{\isafoldproof}%
%
\isadelimproof
\isanewline
%
\endisadelimproof
\isanewline
\isacommand{lemma}\isamarkupfalse%
\ {\isacharparenleft}\isakeyword{in}\ vectorspace{\isacharparenright}\ vsum{\isacharunderscore}comm{\isacharcolon}\isanewline
\ \ \isakeyword{fixes}\ W{\isadigit{1}}\ W{\isadigit{2}}\isanewline
\ \ \isakeyword{assumes}\ h{\isadigit{1}}{\isacharcolon}\ {\isachardoublequoteopen}subspace\ K\ W{\isadigit{1}}\ V{\isachardoublequoteclose}\ \isakeyword{and}\ h{\isadigit{2}}{\isacharcolon}\ {\isachardoublequoteopen}subspace\ K\ W{\isadigit{2}}\ V{\isachardoublequoteclose}\isanewline
\ \ \isakeyword{shows}\ {\isachardoublequoteopen}{\isacharparenleft}subspace{\isacharunderscore}sum\ W{\isadigit{1}}\ W{\isadigit{2}}{\isacharparenright}\ {\isacharequal}\ {\isacharparenleft}subspace{\isacharunderscore}sum\ W{\isadigit{2}}\ W{\isadigit{1}}{\isacharparenright}{\isachardoublequoteclose}\isanewline
%
\isadelimproof
%
\endisadelimproof
%
\isatagproof
\isacommand{proof}\isamarkupfalse%
\ {\isacharminus}\ \isanewline
\ \ \isacommand{from}\isamarkupfalse%
\ h{\isadigit{1}}\ h{\isadigit{2}}\ \isacommand{show}\isamarkupfalse%
\ {\isacharquery}thesis\ \isacommand{by}\isamarkupfalse%
\ {\isacharparenleft}intro\ msum{\isacharunderscore}comm{\isacharcomma}\ unfold\ subspace{\isacharunderscore}def{\isacharcomma}\ auto{\isacharparenright}\isanewline
\isacommand{qed}\isamarkupfalse%
%
\endisatagproof
{\isafoldproof}%
%
\isadelimproof
%
\endisadelimproof
%
\begin{isamarkuptext}%
If $W_1,W_2\subseteq V$ are subspaces, then $W_1+W_2$ is the minimal subspace such that 
both $W_1\subseteq W$ and $W_2\subseteq W$.%
\end{isamarkuptext}%
\isamarkuptrue%
\isacommand{lemma}\isamarkupfalse%
\ {\isacharparenleft}\isakeyword{in}\ vectorspace{\isacharparenright}\ vsum{\isacharunderscore}is{\isacharunderscore}minimal{\isacharcolon}\isanewline
\ \ \isakeyword{fixes}\ W\ W{\isadigit{1}}\ W{\isadigit{2}}\isanewline
\ \ \isakeyword{assumes}\ h{\isadigit{1}}{\isacharcolon}\ {\isachardoublequoteopen}subspace\ K\ W{\isadigit{1}}\ V{\isachardoublequoteclose}\ \isakeyword{and}\ h{\isadigit{2}}{\isacharcolon}\ {\isachardoublequoteopen}subspace\ K\ W{\isadigit{2}}\ V{\isachardoublequoteclose}\ \isakeyword{and}\ h{\isadigit{3}}{\isacharcolon}\ {\isachardoublequoteopen}subspace\ K\ W\ V{\isachardoublequoteclose}\isanewline
\ \ \isakeyword{shows}\ {\isachardoublequoteopen}{\isacharparenleft}subspace{\isacharunderscore}sum\ W{\isadigit{1}}\ W{\isadigit{2}}{\isacharparenright}\ {\isasymsubseteq}\ W\ {\isasymlongleftrightarrow}\ W{\isadigit{1}}\ {\isasymsubseteq}\ W\ {\isasymand}\ W{\isadigit{2}}\ {\isasymsubseteq}\ W{\isachardoublequoteclose}\isanewline
%
\isadelimproof
%
\endisadelimproof
%
\isatagproof
\isacommand{proof}\isamarkupfalse%
\ {\isacharminus}\ \isanewline
\ \ \isacommand{from}\isamarkupfalse%
\ h{\isadigit{1}}\ h{\isadigit{2}}\ h{\isadigit{3}}\ \isacommand{show}\isamarkupfalse%
\ {\isacharquery}thesis\ \isacommand{by}\isamarkupfalse%
\ {\isacharparenleft}intro\ sum{\isacharunderscore}is{\isacharunderscore}minimal{\isacharcomma}\ unfold\ subspace{\isacharunderscore}def{\isacharcomma}\ auto{\isacharparenright}\isanewline
\isacommand{qed}\isamarkupfalse%
%
\endisatagproof
{\isafoldproof}%
%
\isadelimproof
\isanewline
%
\endisadelimproof
\isanewline
\isanewline
\isacommand{lemma}\isamarkupfalse%
\ {\isacharparenleft}\isakeyword{in}\ vectorspace{\isacharparenright}\ span{\isacharunderscore}is{\isacharunderscore}subspace{\isacharcolon}\isanewline
\ \ \isakeyword{fixes}\ S\isanewline
\ \ \isakeyword{assumes}\ h{\isadigit{2}}{\isacharcolon}\ {\isachardoublequoteopen}S{\isasymsubseteq}carrier\ V{\isachardoublequoteclose}\isanewline
\ \ \isakeyword{shows}\ {\isachardoublequoteopen}subspace\ K\ {\isacharparenleft}span\ S{\isacharparenright}\ V{\isachardoublequoteclose}\isanewline
%
\isadelimproof
%
\endisadelimproof
%
\isatagproof
\isacommand{proof}\isamarkupfalse%
\ {\isacharminus}\isanewline
\ \ \isacommand{have}\isamarkupfalse%
\ {\isadigit{0}}{\isacharcolon}\ {\isachardoublequoteopen}vectorspace\ K\ V{\isachardoublequoteclose}\isacommand{{\isachardot}{\isachardot}}\isamarkupfalse%
\isanewline
\ \ \isacommand{from}\isamarkupfalse%
\ h{\isadigit{2}}\ \isacommand{have}\isamarkupfalse%
\ {\isadigit{1}}{\isacharcolon}\ {\isachardoublequoteopen}submodule\ K\ {\isacharparenleft}span\ S{\isacharparenright}\ V{\isachardoublequoteclose}\ \isacommand{by}\isamarkupfalse%
\ {\isacharparenleft}rule\ span{\isacharunderscore}is{\isacharunderscore}submodule{\isacharparenright}\isanewline
\ \ \isacommand{from}\isamarkupfalse%
\ {\isadigit{0}}\ {\isadigit{1}}\ \isacommand{show}\isamarkupfalse%
\ {\isacharquery}thesis\ \isacommand{by}\isamarkupfalse%
\ {\isacharparenleft}unfold\ subspace{\isacharunderscore}def\ mod{\isacharunderscore}hom{\isacharunderscore}def\ linear{\isacharunderscore}map{\isacharunderscore}def{\isacharcomma}\ auto{\isacharparenright}\isanewline
\isacommand{qed}\isamarkupfalse%
%
\endisatagproof
{\isafoldproof}%
%
\isadelimproof
%
\endisadelimproof
%
\isamarkupsubsubsection{Facts specific to vector spaces%
}
\isamarkuptrue%
%
\begin{isamarkuptext}%
If $av = w$ and $a\neq 0$, $v=a^{-1}w$.%
\end{isamarkuptext}%
\isamarkuptrue%
\isacommand{lemma}\isamarkupfalse%
\ {\isacharparenleft}\isakeyword{in}\ vectorspace{\isacharparenright}\ mult{\isacharunderscore}inverse{\isacharcolon}\isanewline
\ \ \isakeyword{assumes}\ h{\isadigit{1}}{\isacharcolon}\ {\isachardoublequoteopen}a{\isasymin}carrier\ K{\isachardoublequoteclose}\ \isakeyword{and}\ h{\isadigit{2}}{\isacharcolon}\ {\isachardoublequoteopen}v{\isasymin}carrier\ V{\isachardoublequoteclose}\ \isakeyword{and}\ h{\isadigit{3}}{\isacharcolon}\ {\isachardoublequoteopen}a\ {\isasymodot}\isactrlbsub V\isactrlesub \ v\ {\isacharequal}\ w{\isachardoublequoteclose}\ \isakeyword{and}\ h{\isadigit{4}}{\isacharcolon}\ {\isachardoublequoteopen}a{\isasymnoteq}{\isasymzero}\isactrlbsub K\isactrlesub {\isachardoublequoteclose}\isanewline
\ \ \isakeyword{shows}\ {\isachardoublequoteopen}v\ {\isacharequal}\ {\isacharparenleft}inv\isactrlbsub K\isactrlesub \ a\ {\isacharparenright}{\isasymodot}\isactrlbsub V\isactrlesub \ w{\isachardoublequoteclose}\isanewline
%
\isadelimproof
%
\endisadelimproof
%
\isatagproof
\isacommand{proof}\isamarkupfalse%
\ {\isacharminus}\isanewline
\ \ \isacommand{from}\isamarkupfalse%
\ h{\isadigit{1}}\ h{\isadigit{2}}\ h{\isadigit{3}}\ \isacommand{have}\isamarkupfalse%
\ {\isadigit{1}}{\isacharcolon}\ {\isachardoublequoteopen}w{\isasymin}carrier\ V{\isachardoublequoteclose}\ \isacommand{by}\isamarkupfalse%
\ auto\isanewline
\ \ \isacommand{from}\isamarkupfalse%
\ h{\isadigit{3}}\ {\isadigit{1}}\ \isacommand{have}\isamarkupfalse%
\ {\isadigit{2}}{\isacharcolon}\ {\isachardoublequoteopen}{\isacharparenleft}inv\isactrlbsub K\isactrlesub \ a\ {\isacharparenright}{\isasymodot}\isactrlbsub V\isactrlesub {\isacharparenleft}a\ {\isasymodot}\isactrlbsub V\isactrlesub \ v{\isacharparenright}\ {\isacharequal}{\isacharparenleft}inv\isactrlbsub K\isactrlesub \ a\ {\isacharparenright}{\isasymodot}\isactrlbsub V\isactrlesub w{\isachardoublequoteclose}\ \isacommand{by}\isamarkupfalse%
\ auto\isanewline
\ \ \isacommand{from}\isamarkupfalse%
\ h{\isadigit{1}}\ h{\isadigit{4}}\ \isacommand{have}\isamarkupfalse%
\ {\isadigit{3}}{\isacharcolon}\ {\isachardoublequoteopen}inv\isactrlbsub K\isactrlesub \ a{\isasymin}carrier\ K{\isachardoublequoteclose}\ \isacommand{by}\isamarkupfalse%
\ auto\isanewline
\ \ \isacommand{interpret}\isamarkupfalse%
\ g{\isacharcolon}\ group\ {\isachardoublequoteopen}{\isacharparenleft}units{\isacharunderscore}group\ K{\isacharparenright}{\isachardoublequoteclose}\ \isacommand{by}\isamarkupfalse%
\ {\isacharparenleft}rule\ units{\isacharunderscore}form{\isacharunderscore}group{\isacharparenright}\isanewline
\ \ \isacommand{have}\isamarkupfalse%
\ f{\isacharcolon}\ {\isachardoublequoteopen}field\ K{\isachardoublequoteclose}\isacommand{{\isachardot}{\isachardot}}\isamarkupfalse%
\isanewline
\ \ \isacommand{from}\isamarkupfalse%
\ f\ h{\isadigit{1}}\ h{\isadigit{4}}\ \isacommand{have}\isamarkupfalse%
\ {\isadigit{4}}{\isacharcolon}\ {\isachardoublequoteopen}a{\isasymin}Units\ K{\isachardoublequoteclose}\ \isanewline
\ \ \ \ \isacommand{by}\isamarkupfalse%
\ {\isacharparenleft}unfold\ field{\isacharunderscore}def\ field{\isacharunderscore}axioms{\isacharunderscore}def{\isacharcomma}\ simp{\isacharparenright}\isanewline
\ \ \isacommand{from}\isamarkupfalse%
\ {\isadigit{4}}\ h{\isadigit{1}}\ h{\isadigit{4}}\ \isacommand{have}\isamarkupfalse%
\ {\isadigit{5}}{\isacharcolon}\ {\isachardoublequoteopen}{\isacharparenleft}{\isacharparenleft}inv\isactrlbsub K\isactrlesub \ a{\isacharparenright}\ {\isasymotimes}\isactrlbsub K\isactrlesub a{\isacharparenright}\ {\isacharequal}\ {\isasymone}\isactrlbsub K\isactrlesub {\isachardoublequoteclose}\ \isanewline
\ \ \ \ \isacommand{by}\isamarkupfalse%
\ {\isacharparenleft}intro\ Units{\isacharunderscore}l{\isacharunderscore}inv{\isacharcomma}\ auto{\isacharparenright}\isanewline
\ \ \isacommand{from}\isamarkupfalse%
\ {\isadigit{5}}\ \isacommand{have}\isamarkupfalse%
\ {\isadigit{6}}{\isacharcolon}\ {\isachardoublequoteopen}{\isacharparenleft}inv\isactrlbsub K\isactrlesub \ a\ {\isacharparenright}{\isasymodot}\isactrlbsub V\isactrlesub {\isacharparenleft}a\ {\isasymodot}\isactrlbsub V\isactrlesub \ v{\isacharparenright}\ {\isacharequal}\ v{\isachardoublequoteclose}\ \isanewline
\ \ \isacommand{proof}\isamarkupfalse%
\ {\isacharminus}\ \isanewline
\ \ \ \ \isacommand{from}\isamarkupfalse%
\ h{\isadigit{1}}\ h{\isadigit{2}}\ h{\isadigit{4}}\ \isacommand{have}\isamarkupfalse%
\ {\isadigit{7}}{\isacharcolon}\ {\isachardoublequoteopen}{\isacharparenleft}inv\isactrlbsub K\isactrlesub \ a\ {\isacharparenright}{\isasymodot}\isactrlbsub V\isactrlesub {\isacharparenleft}a\ {\isasymodot}\isactrlbsub V\isactrlesub \ v{\isacharparenright}\ {\isacharequal}{\isacharparenleft}inv\isactrlbsub K\isactrlesub \ a\ {\isasymotimes}\isactrlbsub K\isactrlesub a{\isacharparenright}\ {\isasymodot}\isactrlbsub V\isactrlesub \ v{\isachardoublequoteclose}\ \isacommand{by}\isamarkupfalse%
\ {\isacharparenleft}auto\ simp\ add{\isacharcolon}\ smult{\isacharunderscore}assoc{\isadigit{1}}{\isacharparenright}\isanewline
\ \ \ \ \isacommand{from}\isamarkupfalse%
\ {\isadigit{5}}\ h{\isadigit{2}}\ \isacommand{have}\isamarkupfalse%
\ {\isadigit{8}}{\isacharcolon}\ {\isachardoublequoteopen}{\isacharparenleft}inv\isactrlbsub K\isactrlesub \ a\ {\isasymotimes}\isactrlbsub K\isactrlesub a{\isacharparenright}\ {\isasymodot}\isactrlbsub V\isactrlesub \ v\ {\isacharequal}\ v{\isachardoublequoteclose}\ \isacommand{by}\isamarkupfalse%
\ auto\isanewline
\ \ \ \ \isacommand{from}\isamarkupfalse%
\ {\isadigit{7}}\ {\isadigit{8}}\ \isacommand{show}\isamarkupfalse%
\ {\isacharquery}thesis\ \isacommand{by}\isamarkupfalse%
\ auto\isanewline
\ \ \isacommand{qed}\isamarkupfalse%
\isanewline
\ \ \isacommand{from}\isamarkupfalse%
\ {\isadigit{2}}\ {\isadigit{6}}\ \isacommand{show}\isamarkupfalse%
\ {\isacharquery}thesis\ \isacommand{by}\isamarkupfalse%
\ auto\isanewline
\isacommand{qed}\isamarkupfalse%
%
\endisatagproof
{\isafoldproof}%
%
\isadelimproof
%
\endisadelimproof
%
\begin{isamarkuptext}%
If $w\in S$ and $\sum_{w\in S} a_ww=0$, we have $v=\sum_{w\not\in S}a_v^{-1}a_ww$%
\end{isamarkuptext}%
\isamarkuptrue%
\isacommand{lemma}\isamarkupfalse%
\ {\isacharparenleft}\isakeyword{in}\ vectorspace{\isacharparenright}\ lincomb{\isacharunderscore}isolate{\isacharcolon}\ \isanewline
\ \ \isakeyword{fixes}\ A\ v\isanewline
\ \ \isakeyword{assumes}\ h{\isadigit{1}}{\isacharcolon}\ {\isachardoublequoteopen}finite\ A{\isachardoublequoteclose}\ \isakeyword{and}\ h{\isadigit{2}}{\isacharcolon}\ {\isachardoublequoteopen}A{\isasymsubseteq}carrier\ V{\isachardoublequoteclose}\ \ \isakeyword{and}\ h{\isadigit{3}}{\isacharcolon}\ {\isachardoublequoteopen}a{\isasymin}A{\isasymrightarrow}carrier\ K{\isachardoublequoteclose}\ \isakeyword{and}\ h{\isadigit{4}}{\isacharcolon}\ {\isachardoublequoteopen}v{\isasymin}A{\isachardoublequoteclose}\isanewline
\ \ \ \ \isakeyword{and}\ h{\isadigit{5}}{\isacharcolon}\ {\isachardoublequoteopen}a\ v\ {\isasymnoteq}\ {\isasymzero}\isactrlbsub K\isactrlesub {\isachardoublequoteclose}\ \isakeyword{and}\ h{\isadigit{6}}{\isacharcolon}\ {\isachardoublequoteopen}lincomb\ a\ A{\isacharequal}{\isasymzero}\isactrlbsub V\isactrlesub {\isachardoublequoteclose}\isanewline
\ \ \isakeyword{shows}\ {\isachardoublequoteopen}v{\isacharequal}lincomb\ {\isacharparenleft}{\isasymlambda}w{\isachardot}\ {\isasymominus}\isactrlbsub K\isactrlesub {\isacharparenleft}inv\isactrlbsub K\isactrlesub \ {\isacharparenleft}a\ v{\isacharparenright}{\isacharparenright}\ {\isasymotimes}\isactrlbsub K\isactrlesub \ a\ w{\isacharparenright}\ {\isacharparenleft}A{\isacharminus}{\isacharbraceleft}v{\isacharbraceright}{\isacharparenright}{\isachardoublequoteclose}\ \isakeyword{and}\ {\isachardoublequoteopen}v{\isasymin}\ span\ {\isacharparenleft}A{\isacharminus}{\isacharbraceleft}v{\isacharbraceright}{\isacharparenright}{\isachardoublequoteclose}\isanewline
%
\isadelimproof
%
\endisadelimproof
%
\isatagproof
\isacommand{proof}\isamarkupfalse%
\ {\isacharminus}\isanewline
\ \ \isacommand{from}\isamarkupfalse%
\ h{\isadigit{1}}\ h{\isadigit{2}}\ h{\isadigit{3}}\ h{\isadigit{4}}\ \isacommand{have}\isamarkupfalse%
\ {\isadigit{1}}{\isacharcolon}\ {\isachardoublequoteopen}lincomb\ a\ A\ {\isacharequal}\ {\isacharparenleft}{\isacharparenleft}a\ v{\isacharparenright}\ {\isasymodot}\isactrlbsub V\isactrlesub \ v{\isacharparenright}\ {\isasymoplus}\isactrlbsub V\isactrlesub \ lincomb\ a\ {\isacharparenleft}A{\isacharminus}{\isacharbraceleft}v{\isacharbraceright}{\isacharparenright}{\isachardoublequoteclose}\ \isanewline
\ \ \ \ \isacommand{by}\isamarkupfalse%
\ {\isacharparenleft}rule\ lincomb{\isacharunderscore}del{\isadigit{2}}{\isacharparenright}\isanewline
\ \ \isacommand{from}\isamarkupfalse%
\ {\isadigit{1}}\ \isacommand{have}\isamarkupfalse%
\ {\isadigit{2}}{\isacharcolon}\ {\isachardoublequoteopen}{\isasymzero}\isactrlbsub V\isactrlesub {\isacharequal}\ {\isacharparenleft}{\isacharparenleft}a\ v{\isacharparenright}\ {\isasymodot}\isactrlbsub V\isactrlesub \ v{\isacharparenright}\ {\isasymoplus}\isactrlbsub V\isactrlesub \ lincomb\ a\ {\isacharparenleft}A{\isacharminus}{\isacharbraceleft}v{\isacharbraceright}{\isacharparenright}{\isachardoublequoteclose}\ \isacommand{by}\isamarkupfalse%
\ {\isacharparenleft}simp\ add{\isacharcolon}\ h{\isadigit{6}}{\isacharparenright}\isanewline
\ \ \isacommand{from}\isamarkupfalse%
\ h{\isadigit{1}}\ h{\isadigit{2}}\ h{\isadigit{3}}\ \isacommand{have}\isamarkupfalse%
\ {\isadigit{5}}{\isacharcolon}\ {\isachardoublequoteopen}lincomb\ a\ {\isacharparenleft}A{\isacharminus}{\isacharbraceleft}v{\isacharbraceright}{\isacharparenright}\ {\isasymin}carrier\ V{\isachardoublequoteclose}\ \isacommand{by}\isamarkupfalse%
\ auto\ \isanewline
\ \ \isacommand{from}\isamarkupfalse%
\ {\isadigit{2}}\ h{\isadigit{1}}\ h{\isadigit{2}}\ h{\isadigit{3}}\ h{\isadigit{4}}\ \isacommand{have}\isamarkupfalse%
\ {\isadigit{3}}{\isacharcolon}\ {\isachardoublequoteopen}\ {\isasymominus}\isactrlbsub V\isactrlesub \ lincomb\ a\ {\isacharparenleft}A{\isacharminus}{\isacharbraceleft}v{\isacharbraceright}{\isacharparenright}\ {\isacharequal}\ {\isacharparenleft}{\isacharparenleft}a\ v{\isacharparenright}\ {\isasymodot}\isactrlbsub V\isactrlesub \ v{\isacharparenright}{\isachardoublequoteclose}\ \isanewline
\ \ \ \ \isacommand{by}\isamarkupfalse%
\ {\isacharparenleft}auto\ intro{\isacharbang}{\isacharcolon}\ M{\isachardot}minus{\isacharunderscore}equality{\isacharparenright}\isanewline
\ \ \isacommand{have}\isamarkupfalse%
\ {\isadigit{6}}{\isacharcolon}\ {\isachardoublequoteopen}v\ {\isacharequal}\ {\isacharparenleft}{\isasymominus}\isactrlbsub K\isactrlesub \ {\isacharparenleft}inv\isactrlbsub K\isactrlesub \ {\isacharparenleft}a\ v{\isacharparenright}{\isacharparenright}{\isacharparenright}\ {\isasymodot}\isactrlbsub V\isactrlesub \ lincomb\ a\ {\isacharparenleft}A{\isacharminus}{\isacharbraceleft}v{\isacharbraceright}{\isacharparenright}{\isachardoublequoteclose}\isanewline
\ \ \isacommand{proof}\isamarkupfalse%
\ {\isacharminus}\ \isanewline
\ \ \ \ \isacommand{from}\isamarkupfalse%
\ h{\isadigit{2}}\ h{\isadigit{3}}\ h{\isadigit{4}}\ h{\isadigit{5}}\ {\isadigit{3}}\ \isacommand{have}\isamarkupfalse%
\ {\isadigit{7}}{\isacharcolon}\ {\isachardoublequoteopen}v\ {\isacharequal}\ inv\isactrlbsub K\isactrlesub \ {\isacharparenleft}a\ v{\isacharparenright}\ {\isasymodot}\isactrlbsub V\isactrlesub \ {\isacharparenleft}{\isasymominus}\isactrlbsub V\isactrlesub \ lincomb\ a\ {\isacharparenleft}A{\isacharminus}{\isacharbraceleft}v{\isacharbraceright}{\isacharparenright}{\isacharparenright}{\isachardoublequoteclose}\ \isanewline
\ \ \ \ \ \ \isacommand{by}\isamarkupfalse%
\ {\isacharparenleft}intro\ mult{\isacharunderscore}inverse{\isacharcomma}\ auto{\isacharparenright}\isanewline
\ \ \ \ \isacommand{from}\isamarkupfalse%
\ assms\ \isacommand{have}\isamarkupfalse%
\ {\isadigit{8}}{\isacharcolon}\ {\isachardoublequoteopen}inv\isactrlbsub K\isactrlesub \ {\isacharparenleft}a\ v{\isacharparenright}{\isasymin}carrier\ K{\isachardoublequoteclose}\ \isacommand{by}\isamarkupfalse%
\ auto\isanewline
\ \ \ \ \isacommand{from}\isamarkupfalse%
\ assms\ {\isadigit{5}}\ {\isadigit{8}}\ \isacommand{have}\isamarkupfalse%
\ {\isadigit{9}}{\isacharcolon}\ {\isachardoublequoteopen}inv\isactrlbsub K\isactrlesub \ {\isacharparenleft}a\ v{\isacharparenright}\ {\isasymodot}\isactrlbsub V\isactrlesub \ {\isacharparenleft}{\isasymominus}\isactrlbsub V\isactrlesub \ lincomb\ a\ {\isacharparenleft}A{\isacharminus}{\isacharbraceleft}v{\isacharbraceright}{\isacharparenright}{\isacharparenright}\ \isanewline
\ \ \ \ \ \ {\isacharequal}\ {\isacharparenleft}{\isasymominus}\isactrlbsub K\isactrlesub \ {\isacharparenleft}inv\isactrlbsub K\isactrlesub \ {\isacharparenleft}a\ v{\isacharparenright}{\isacharparenright}{\isacharparenright}\ {\isasymodot}\isactrlbsub V\isactrlesub \ lincomb\ a\ {\isacharparenleft}A{\isacharminus}{\isacharbraceleft}v{\isacharbraceright}{\isacharparenright}{\isachardoublequoteclose}\isanewline
\ \ \ \ \ \ \ \ \isacommand{by}\isamarkupfalse%
\ {\isacharparenleft}simp\ add{\isacharcolon}\ smult{\isacharunderscore}assoc{\isacharunderscore}simp\ smult{\isacharunderscore}minus{\isacharunderscore}{\isadigit{1}}{\isacharunderscore}back\ r{\isacharunderscore}minus{\isacharparenright}\isanewline
\ \ \ \ \isacommand{from}\isamarkupfalse%
\ {\isadigit{7}}\ {\isadigit{9}}\ \isacommand{show}\isamarkupfalse%
\ {\isacharquery}thesis\ \isacommand{by}\isamarkupfalse%
\ auto\isanewline
\ \ \isacommand{qed}\isamarkupfalse%
\isanewline
\ \ \isacommand{from}\isamarkupfalse%
\ h{\isadigit{1}}\ \isacommand{have}\isamarkupfalse%
\ {\isadigit{1}}{\isadigit{0}}{\isacharcolon}\ {\isachardoublequoteopen}finite\ {\isacharparenleft}A{\isacharminus}{\isacharbraceleft}v{\isacharbraceright}{\isacharparenright}{\isachardoublequoteclose}\ \isacommand{by}\isamarkupfalse%
\ auto\isanewline
\ \ \isacommand{from}\isamarkupfalse%
\ assms\ \isacommand{have}\isamarkupfalse%
\ {\isadigit{1}}{\isadigit{1}}\ {\isacharcolon}\ {\isachardoublequoteopen}{\isacharparenleft}{\isasymominus}\isactrlbsub K\isactrlesub \ {\isacharparenleft}inv\isactrlbsub K\isactrlesub \ {\isacharparenleft}a\ v{\isacharparenright}{\isacharparenright}{\isacharparenright}{\isasymin}\ carrier\ K{\isachardoublequoteclose}\ \isacommand{by}\isamarkupfalse%
\ auto\isanewline
\ \ \isacommand{from}\isamarkupfalse%
\ assms\ \isacommand{have}\isamarkupfalse%
\ {\isadigit{1}}{\isadigit{2}}{\isacharcolon}\ {\isachardoublequoteopen}lincomb\ {\isacharparenleft}{\isasymlambda}w{\isachardot}\ {\isasymominus}\isactrlbsub K\isactrlesub {\isacharparenleft}inv\isactrlbsub K\isactrlesub \ {\isacharparenleft}a\ v{\isacharparenright}{\isacharparenright}\ {\isasymotimes}\isactrlbsub K\isactrlesub \ a\ w{\isacharparenright}\ {\isacharparenleft}A{\isacharminus}{\isacharbraceleft}v{\isacharbraceright}{\isacharparenright}\ {\isacharequal}\ \isanewline
\ \ \ \ {\isacharparenleft}{\isasymominus}\isactrlbsub K\isactrlesub \ {\isacharparenleft}inv\isactrlbsub K\isactrlesub \ {\isacharparenleft}a\ v{\isacharparenright}{\isacharparenright}{\isacharparenright}\ {\isasymodot}\isactrlbsub V\isactrlesub \ lincomb\ a\ {\isacharparenleft}A{\isacharminus}{\isacharbraceleft}v{\isacharbraceright}{\isacharparenright}{\isachardoublequoteclose}\ \isanewline
\ \ \ \ \isacommand{by}\isamarkupfalse%
\ {\isacharparenleft}intro\ lincomb{\isacharunderscore}smult{\isacharcomma}\ auto{\isacharparenright}\isanewline
\ \ \isacommand{from}\isamarkupfalse%
\ {\isadigit{6}}\ {\isadigit{1}}{\isadigit{2}}\ \isacommand{show}\isamarkupfalse%
\ {\isadigit{1}}{\isadigit{3}}{\isacharcolon}\ {\isachardoublequoteopen}v{\isacharequal}lincomb\ {\isacharparenleft}{\isasymlambda}w{\isachardot}\ {\isasymominus}\isactrlbsub K\isactrlesub {\isacharparenleft}inv\isactrlbsub K\isactrlesub \ {\isacharparenleft}a\ v{\isacharparenright}{\isacharparenright}\ {\isasymotimes}\isactrlbsub K\isactrlesub \ a\ w{\isacharparenright}\ {\isacharparenleft}A{\isacharminus}{\isacharbraceleft}v{\isacharbraceright}{\isacharparenright}{\isachardoublequoteclose}\ \isacommand{by}\isamarkupfalse%
\ auto\isanewline
\ \ \isacommand{from}\isamarkupfalse%
\ {\isadigit{1}}{\isadigit{3}}\ assms\ \isacommand{show}\isamarkupfalse%
\ {\isadigit{1}}{\isadigit{4}}\ {\isacharcolon}\ {\isachardoublequoteopen}v{\isasymin}\ span\ {\isacharparenleft}A{\isacharminus}{\isacharbraceleft}v{\isacharbraceright}{\isacharparenright}{\isachardoublequoteclose}\ \isanewline
\ \ \ \ \isacommand{apply}\isamarkupfalse%
\ {\isacharparenleft}unfold\ span{\isacharunderscore}def{\isacharcomma}\ auto{\isacharparenright}\ \isanewline
\ \ \ \ \isacommand{apply}\isamarkupfalse%
\ {\isacharparenleft}rule{\isacharunderscore}tac\ x{\isacharequal}{\isachardoublequoteopen}{\isacharparenleft}{\isasymlambda}w{\isachardot}\ {\isasymominus}\isactrlbsub K\isactrlesub {\isacharparenleft}inv\isactrlbsub K\isactrlesub \ {\isacharparenleft}a\ v{\isacharparenright}{\isacharparenright}\ {\isasymotimes}\isactrlbsub K\isactrlesub \ a\ w{\isacharparenright}{\isachardoublequoteclose}\ \isakeyword{in}\ exI{\isacharparenright}\ \isanewline
\ \ \ \ \isacommand{apply}\isamarkupfalse%
\ {\isacharparenleft}drule\ Pi{\isacharunderscore}implies{\isacharunderscore}Pi{\isadigit{2}}{\isacharparenright}\isanewline
\ \ \ \ \isacommand{by}\isamarkupfalse%
\ {\isacharparenleft}auto\ simp\ add{\isacharcolon}\ Pi{\isacharunderscore}simp\ ring{\isacharunderscore}subset{\isacharunderscore}carrier{\isacharparenright}\isanewline
\isacommand{qed}\isamarkupfalse%
%
\endisatagproof
{\isafoldproof}%
%
\isadelimproof
%
\endisadelimproof
%
\begin{isamarkuptext}%
The map $(S\to K)\mapsto V$ given by $(a_v)_{v\in S}\mapsto \sum_{v\in S} a_v v$ is linear.%
\end{isamarkuptext}%
\isamarkuptrue%
\isacommand{lemma}\isamarkupfalse%
\ {\isacharparenleft}\isakeyword{in}\ vectorspace{\isacharparenright}\ lincomb{\isacharunderscore}is{\isacharunderscore}linear{\isacharcolon}\isanewline
\ \ \isakeyword{fixes}\ S\isanewline
\ \ \isakeyword{assumes}\ h{\isacharcolon}\ {\isachardoublequoteopen}finite\ S{\isachardoublequoteclose}\ \isakeyword{and}\ h{\isadigit{2}}{\isacharcolon}\ {\isachardoublequoteopen}S{\isasymsubseteq}carrier\ V{\isachardoublequoteclose}\isanewline
\ \ \isakeyword{shows}\ {\isachardoublequoteopen}linear{\isacharunderscore}map\ K\ {\isacharparenleft}func{\isacharunderscore}space\ S{\isacharparenright}\ V\ {\isacharparenleft}{\isasymlambda}a{\isachardot}\ lincomb\ a\ S{\isacharparenright}{\isachardoublequoteclose}\ \isanewline
%
\isadelimproof
%
\endisadelimproof
%
\isatagproof
\isacommand{proof}\isamarkupfalse%
\ {\isacharminus}\isanewline
\ \ \isacommand{have}\isamarkupfalse%
\ {\isadigit{0}}{\isacharcolon}\ {\isachardoublequoteopen}vectorspace\ K\ V{\isachardoublequoteclose}\isacommand{{\isachardot}{\isachardot}}\isamarkupfalse%
\isanewline
\ \ \isacommand{from}\isamarkupfalse%
\ h\ h{\isadigit{2}}\ \isacommand{have}\isamarkupfalse%
\ {\isadigit{1}}{\isacharcolon}\ {\isachardoublequoteopen}mod{\isacharunderscore}hom\ K\ {\isacharparenleft}func{\isacharunderscore}space\ S{\isacharparenright}\ V\ {\isacharparenleft}{\isasymlambda}a{\isachardot}\ lincomb\ a\ S{\isacharparenright}{\isachardoublequoteclose}\ \isacommand{by}\isamarkupfalse%
\ {\isacharparenleft}rule\ lincomb{\isacharunderscore}is{\isacharunderscore}mod{\isacharunderscore}hom{\isacharparenright}\isanewline
\ \ \isacommand{from}\isamarkupfalse%
\ {\isadigit{0}}\ {\isadigit{1}}\ \isacommand{show}\isamarkupfalse%
\ {\isacharquery}thesis\ \isacommand{by}\isamarkupfalse%
\ {\isacharparenleft}unfold\ vectorspace{\isacharunderscore}def\ mod{\isacharunderscore}hom{\isacharunderscore}def\ linear{\isacharunderscore}map{\isacharunderscore}def{\isacharcomma}\ auto{\isacharparenright}\isanewline
\isacommand{qed}\isamarkupfalse%
%
\endisatagproof
{\isafoldproof}%
%
\isadelimproof
%
\endisadelimproof
%
\isamarkupsubsection{Basic facts about span and linear independence%
}
\isamarkuptrue%
%
\begin{isamarkuptext}%
If $S$ is linearly independent, then $v\in \text{span}S$ iff $S\cup \{v\}$ is linearly 
dependent.%
\end{isamarkuptext}%
\isamarkuptrue%
\isacommand{theorem}\isamarkupfalse%
\ {\isacharparenleft}\isakeyword{in}\ vectorspace{\isacharparenright}\ lin{\isacharunderscore}dep{\isacharunderscore}iff{\isacharunderscore}in{\isacharunderscore}span{\isacharcolon}\isanewline
\ \ \isakeyword{fixes}\ A\ v\ S\isanewline
\ \ \isakeyword{assumes}\ \ h{\isadigit{1}}{\isacharcolon}\ {\isachardoublequoteopen}S\ {\isasymsubseteq}\ carrier\ V{\isachardoublequoteclose}\ \isakeyword{and}\ h{\isadigit{2}}{\isacharcolon}\ {\isachardoublequoteopen}lin{\isacharunderscore}indpt\ S{\isachardoublequoteclose}\ \isakeyword{and}\ h{\isadigit{3}}{\isacharcolon}\ {\isachardoublequoteopen}v{\isasymin}\ carrier\ V{\isachardoublequoteclose}\ \isakeyword{and}\ h{\isadigit{4}}{\isacharcolon}\ {\isachardoublequoteopen}v{\isasymnotin}S{\isachardoublequoteclose}\isanewline
\ \ \isakeyword{shows}\ {\isachardoublequoteopen}v{\isasymin}\ span\ S\ {\isasymlongleftrightarrow}\ lin{\isacharunderscore}dep\ {\isacharparenleft}S\ {\isasymunion}\ {\isacharbraceleft}v{\isacharbraceright}{\isacharparenright}{\isachardoublequoteclose}\isanewline
%
\isadelimproof
%
\endisadelimproof
%
\isatagproof
\isacommand{proof}\isamarkupfalse%
\ {\isacharminus}\isanewline
\ \ \isacommand{let}\isamarkupfalse%
\ {\isacharquery}T\ {\isacharequal}\ {\isachardoublequoteopen}S\ {\isasymunion}\ {\isacharbraceleft}v{\isacharbraceright}{\isachardoublequoteclose}\ \isanewline
\ \ \isacommand{have}\isamarkupfalse%
\ {\isadigit{0}}{\isacharcolon}\ {\isachardoublequoteopen}v{\isasymin}{\isacharquery}T\ {\isachardoublequoteclose}\ \isacommand{by}\isamarkupfalse%
\ auto\isanewline
\ \ \isacommand{from}\isamarkupfalse%
\ h{\isadigit{1}}\ h{\isadigit{3}}\ \isacommand{have}\isamarkupfalse%
\ h{\isadigit{1}}{\isacharunderscore}{\isadigit{1}}{\isacharcolon}\ {\isachardoublequoteopen}{\isacharquery}T\ {\isasymsubseteq}\ carrier\ V{\isachardoublequoteclose}\ \isacommand{by}\isamarkupfalse%
\ auto\isanewline
\ \ \isacommand{have}\isamarkupfalse%
\ a{\isadigit{1}}{\isacharcolon}{\isachardoublequoteopen}lin{\isacharunderscore}dep\ {\isacharquery}T\ {\isasymLongrightarrow}\ v{\isasymin}\ span\ S{\isachardoublequoteclose}\isanewline
\ \ \isacommand{proof}\isamarkupfalse%
\ {\isacharminus}\ \isanewline
\ \ \ \ \isacommand{assume}\isamarkupfalse%
\ a{\isadigit{1}}{\isadigit{1}}{\isacharcolon}\ {\isachardoublequoteopen}lin{\isacharunderscore}dep\ {\isacharquery}T{\isachardoublequoteclose}\isanewline
\ \ \ \ \isacommand{from}\isamarkupfalse%
\ a{\isadigit{1}}{\isadigit{1}}\ \isacommand{obtain}\isamarkupfalse%
\ a\ w\ A\ \isakeyword{where}\ a{\isacharcolon}\ {\isachardoublequoteopen}{\isacharparenleft}finite\ A\ {\isasymand}\ A{\isasymsubseteq}{\isacharquery}T\ {\isasymand}\ {\isacharparenleft}a{\isasymin}\ {\isacharparenleft}A{\isasymrightarrow}carrier\ K{\isacharparenright}{\isacharparenright}\ {\isasymand}\ {\isacharparenleft}lincomb\ a\ A\ {\isacharequal}\ {\isasymzero}\isactrlbsub V\isactrlesub {\isacharparenright}\ {\isasymand}\ {\isacharparenleft}w{\isasymin}A{\isacharparenright}\ {\isasymand}\ {\isacharparenleft}a\ w{\isasymnoteq}\ {\isasymzero}\isactrlbsub K\isactrlesub {\isacharparenright}{\isacharparenright}{\isachardoublequoteclose}\isanewline
\ \ \ \ \ \ \isacommand{by}\isamarkupfalse%
\ {\isacharparenleft}metis\ lin{\isacharunderscore}dep{\isacharunderscore}def{\isacharparenright}\isanewline
\ \ \ \ \isacommand{from}\isamarkupfalse%
\ assms\ a\ \isacommand{have}\isamarkupfalse%
\ nz{\isadigit{2}}{\isacharcolon}\ {\isachardoublequoteopen}{\isasymexists}v{\isasymin}A{\isacharminus}S{\isachardot}\ a\ v{\isasymnoteq}{\isasymzero}\isactrlbsub K\isactrlesub {\isachardoublequoteclose}\ \isanewline
\ \ \ \ \ \ \isacommand{by}\isamarkupfalse%
\ {\isacharparenleft}intro\ lincomb{\isacharunderscore}must{\isacharunderscore}include{\isacharbrackleft}\isakeyword{where}\ {\isacharquery}v{\isacharequal}{\isachardoublequoteopen}w{\isachardoublequoteclose}\ \isakeyword{and}\ {\isacharquery}T{\isacharequal}{\isachardoublequoteopen}S{\isasymunion}{\isacharbraceleft}v{\isacharbraceright}{\isachardoublequoteclose}{\isacharbrackright}{\isacharcomma}\ auto{\isacharparenright}\isanewline
\ \ \ \ \isacommand{from}\isamarkupfalse%
\ a\ nz{\isadigit{2}}\ \isacommand{have}\isamarkupfalse%
\ singleton{\isacharcolon}\ {\isachardoublequoteopen}{\isacharbraceleft}v{\isacharbraceright}{\isacharequal}A{\isacharminus}S{\isachardoublequoteclose}\ \isacommand{by}\isamarkupfalse%
\ auto\isanewline
\ \ \ \ \isacommand{from}\isamarkupfalse%
\ singleton\ nz{\isadigit{2}}\ \isacommand{have}\isamarkupfalse%
\ nz{\isadigit{3}}{\isacharcolon}\ {\isachardoublequoteopen}a\ v{\isasymnoteq}{\isasymzero}\isactrlbsub K\isactrlesub {\isachardoublequoteclose}\ \isacommand{by}\isamarkupfalse%
\ auto\isanewline
\isanewline
\ \ \ \ \isacommand{let}\isamarkupfalse%
\ {\isacharquery}b{\isacharequal}{\isachardoublequoteopen}{\isacharparenleft}{\isasymlambda}w{\isachardot}\ {\isasymominus}\isactrlbsub K\isactrlesub \ {\isacharparenleft}inv\isactrlbsub K\isactrlesub \ {\isacharparenleft}a\ v{\isacharparenright}{\isacharparenright}\ {\isasymotimes}\isactrlbsub K\isactrlesub \ {\isacharparenleft}a\ w{\isacharparenright}{\isacharparenright}{\isachardoublequoteclose}\isanewline
\ \ \ \ \isacommand{from}\isamarkupfalse%
\ singleton\ \isacommand{have}\isamarkupfalse%
\ Ains{\isacharcolon}\ {\isachardoublequoteopen}{\isacharparenleft}A{\isasyminter}S{\isacharparenright}\ {\isacharequal}\ A{\isacharminus}{\isacharbraceleft}v{\isacharbraceright}{\isachardoublequoteclose}\ \isacommand{by}\isamarkupfalse%
\ auto\isanewline
\ \ \ \ \isacommand{from}\isamarkupfalse%
\ assms\ a\ singleton\ nz{\isadigit{3}}\ \isacommand{have}\isamarkupfalse%
\ a{\isadigit{3}}{\isadigit{1}}{\isacharcolon}\ {\isachardoublequoteopen}v{\isacharequal}\ lincomb\ {\isacharquery}b\ {\isacharparenleft}A{\isasyminter}S{\isacharparenright}{\isachardoublequoteclose}\ \isanewline
\ \ \ \ \ \ \isacommand{apply}\isamarkupfalse%
\ {\isacharparenleft}subst\ Ains{\isacharparenright}\isanewline
\ \ \ \ \ \ \isacommand{by}\isamarkupfalse%
\ {\isacharparenleft}intro\ lincomb{\isacharunderscore}isolate{\isacharparenleft}{\isadigit{1}}{\isacharparenright}{\isacharcomma}\ auto{\isacharparenright}\isanewline
\ \ \ \ \isacommand{from}\isamarkupfalse%
\ a\ a{\isadigit{3}}{\isadigit{1}}\ nz{\isadigit{3}}\ singleton\ \isacommand{show}\isamarkupfalse%
\ {\isacharquery}thesis\ \isanewline
\ \ \ \ \ \ \isacommand{apply}\isamarkupfalse%
\ {\isacharparenleft}unfold\ span{\isacharunderscore}def{\isacharcomma}\ auto{\isacharparenright}\ \isanewline
\ \ \ \ \ \ \isacommand{apply}\isamarkupfalse%
\ {\isacharparenleft}rule{\isacharunderscore}tac\ x{\isacharequal}{\isachardoublequoteopen}{\isacharquery}b{\isachardoublequoteclose}\ \isakeyword{in}\ exI{\isacharparenright}\isanewline
\ \ \ \ \ \ \isacommand{apply}\isamarkupfalse%
\ {\isacharparenleft}rule{\isacharunderscore}tac\ x{\isacharequal}{\isachardoublequoteopen}A{\isasyminter}S{\isachardoublequoteclose}\ \isakeyword{in}\ exI{\isacharparenright}\ \isanewline
\ \ \ \ \ \ \isacommand{by}\isamarkupfalse%
\ {\isacharparenleft}auto\ intro{\isacharbang}{\isacharcolon}\ m{\isacharunderscore}closed{\isacharparenright}\isanewline
\ \ \isacommand{qed}\isamarkupfalse%
\isanewline
\ \ \isacommand{have}\isamarkupfalse%
\ a{\isadigit{2}}{\isacharcolon}\ {\isachardoublequoteopen}v{\isasymin}\ {\isacharparenleft}span\ S{\isacharparenright}\ {\isasymLongrightarrow}\ lin{\isacharunderscore}dep\ {\isacharquery}T{\isachardoublequoteclose}\isanewline
\ \ \isacommand{proof}\isamarkupfalse%
\ {\isacharminus}\ \isanewline
\ \ \ \ \isacommand{assume}\isamarkupfalse%
\ inspan{\isacharcolon}\ {\isachardoublequoteopen}v{\isasymin}\ {\isacharparenleft}span\ S{\isacharparenright}{\isachardoublequoteclose}\isanewline
\ \ \ \ \isacommand{from}\isamarkupfalse%
\ inspan\ \isacommand{obtain}\isamarkupfalse%
\ a\ A\ \isakeyword{where}\ a{\isacharcolon}\ {\isachardoublequoteopen}A{\isasymsubseteq}S\ {\isasymand}\ finite\ A\ {\isasymand}\ {\isacharparenleft}v\ {\isacharequal}\ lincomb\ a\ A{\isacharparenright}{\isasymand}\ a{\isasymin}A{\isasymrightarrow}carrier\ K{\isachardoublequoteclose}\ \isacommand{by}\isamarkupfalse%
\ {\isacharparenleft}simp\ add{\isacharcolon}\ span{\isacharunderscore}def{\isacharcomma}\ auto{\isacharparenright}\isanewline
\ \ \ \ \isacommand{let}\isamarkupfalse%
\ {\isacharquery}b\ {\isacharequal}\ {\isachardoublequoteopen}{\isasymlambda}\ w{\isachardot}\ if\ {\isacharparenleft}w{\isacharequal}v{\isacharparenright}\ then\ {\isacharparenleft}{\isasymominus}\isactrlbsub K\isactrlesub \ {\isasymone}\isactrlbsub K\isactrlesub {\isacharparenright}\ else\ a\ w{\isachardoublequoteclose}\ \isanewline
\ \ \ \ \isacommand{have}\isamarkupfalse%
\ lc{\isadigit{0}}{\isacharcolon}\ {\isachardoublequoteopen}\ lincomb\ {\isacharquery}b\ {\isacharparenleft}A{\isasymunion}{\isacharbraceleft}v{\isacharbraceright}{\isacharparenright}{\isacharequal}{\isasymzero}\isactrlbsub V\isactrlesub {\isachardoublequoteclose}\ \isanewline
\ \ \ \ \isacommand{proof}\isamarkupfalse%
\ {\isacharminus}\ \isanewline
\ \ \ \ \ \ \isacommand{from}\isamarkupfalse%
\ assms\ a\ \isacommand{have}\isamarkupfalse%
\ lc{\isacharunderscore}ins{\isacharcolon}\ {\isachardoublequoteopen}lincomb\ {\isacharquery}b\ {\isacharparenleft}A{\isasymunion}{\isacharbraceleft}v{\isacharbraceright}{\isacharparenright}\ {\isacharequal}\ {\isacharparenleft}{\isacharparenleft}{\isacharquery}b\ v{\isacharparenright}\ {\isasymodot}\isactrlbsub V\isactrlesub \ v{\isacharparenright}\ {\isasymoplus}\isactrlbsub V\isactrlesub \ lincomb\ {\isacharquery}b\ A{\isachardoublequoteclose}\ \isanewline
\ \ \ \ \ \ \ \ \isacommand{by}\isamarkupfalse%
\ {\isacharparenleft}intro\ lincomb{\isacharunderscore}insert{\isacharcomma}\ auto{\isacharparenright}\isanewline
\ \ \ \ \ \ \isacommand{from}\isamarkupfalse%
\ assms\ a\ \isacommand{have}\isamarkupfalse%
\ lc{\isacharunderscore}elim{\isacharcolon}\ {\isachardoublequoteopen}lincomb\ {\isacharquery}b\ A{\isacharequal}lincomb\ a\ A{\isachardoublequoteclose}\ \isacommand{by}\isamarkupfalse%
\ {\isacharparenleft}intro\ lincomb{\isacharunderscore}elim{\isacharunderscore}if{\isacharcomma}\ auto{\isacharparenright}\isanewline
\ \ \ \ \ \ \isacommand{from}\isamarkupfalse%
\ assms\ lc{\isacharunderscore}ins\ lc{\isacharunderscore}elim\ a\ \isacommand{show}\isamarkupfalse%
\ {\isacharquery}thesis\ \isacommand{by}\isamarkupfalse%
\ {\isacharparenleft}simp\ add{\isacharcolon}\ M{\isachardot}l{\isacharunderscore}neg\ smult{\isacharunderscore}minus{\isacharunderscore}{\isadigit{1}}{\isacharparenright}\isanewline
\ \ \ \ \isacommand{qed}\isamarkupfalse%
\isanewline
\ \ \ \ \isacommand{from}\isamarkupfalse%
\ \ a\ lc{\isadigit{0}}\ \isacommand{show}\isamarkupfalse%
\ {\isacharquery}thesis\ \isanewline
\ \ \ \ \ \ \isacommand{apply}\isamarkupfalse%
\ {\isacharparenleft}unfold\ lin{\isacharunderscore}dep{\isacharunderscore}def{\isacharparenright}\isanewline
\ \ \ \ \ \ \isacommand{apply}\isamarkupfalse%
\ {\isacharparenleft}rule{\isacharunderscore}tac\ x{\isacharequal}{\isachardoublequoteopen}A{\isasymunion}{\isacharbraceleft}v{\isacharbraceright}{\isachardoublequoteclose}\ \isakeyword{in}\ exI{\isacharparenright}\isanewline
\ \ \ \ \ \ \isacommand{apply}\isamarkupfalse%
\ {\isacharparenleft}rule{\isacharunderscore}tac\ x{\isacharequal}{\isachardoublequoteopen}{\isacharquery}b{\isachardoublequoteclose}\ \isakeyword{in}\ exI{\isacharparenright}\isanewline
\ \ \ \ \ \ \isacommand{apply}\isamarkupfalse%
\ {\isacharparenleft}rule{\isacharunderscore}tac\ x{\isacharequal}{\isachardoublequoteopen}v{\isachardoublequoteclose}\ \isakeyword{in}\ exI{\isacharparenright}\isanewline
\ \ \ \ \ \ \isacommand{by}\isamarkupfalse%
\ auto\isanewline
\ \ \isacommand{qed}\isamarkupfalse%
\isanewline
\ \ \isacommand{from}\isamarkupfalse%
\ a{\isadigit{1}}\ a{\isadigit{2}}\ \isacommand{show}\isamarkupfalse%
\ {\isacharquery}thesis\ \isacommand{by}\isamarkupfalse%
\ auto\isanewline
\isacommand{qed}\isamarkupfalse%
%
\endisatagproof
{\isafoldproof}%
%
\isadelimproof
%
\endisadelimproof
%
\begin{isamarkuptext}%
If $v\in \text{span} A$ then $\text{span}A =\text{span}(A\cup \{v\})$%
\end{isamarkuptext}%
\isamarkuptrue%
\isacommand{lemma}\isamarkupfalse%
\ {\isacharparenleft}\isakeyword{in}\ vectorspace{\isacharparenright}\ already{\isacharunderscore}in{\isacharunderscore}span{\isacharcolon}\isanewline
\ \ \isakeyword{fixes}\ v\ A\isanewline
\ \ \isakeyword{assumes}\ \ inC{\isacharcolon}\ {\isachardoublequoteopen}A{\isasymsubseteq}carrier\ V{\isachardoublequoteclose}\ \isakeyword{and}\ inspan{\isacharcolon}\ {\isachardoublequoteopen}v{\isasymin}span\ A{\isachardoublequoteclose}\isanewline
\ \ \isakeyword{shows}\ {\isachardoublequoteopen}span\ A{\isacharequal}\ span\ {\isacharparenleft}A{\isasymunion}{\isacharbraceleft}v{\isacharbraceright}{\isacharparenright}{\isachardoublequoteclose}\isanewline
%
\isadelimproof
%
\endisadelimproof
%
\isatagproof
\isacommand{proof}\isamarkupfalse%
\ {\isacharminus}\ \isanewline
\ \ \isacommand{from}\isamarkupfalse%
\ inC\ inspan\ \isacommand{have}\isamarkupfalse%
\ dir{\isadigit{1}}{\isacharcolon}\ {\isachardoublequoteopen}span\ A\ {\isasymsubseteq}\ span\ {\isacharparenleft}A{\isasymunion}{\isacharbraceleft}v{\isacharbraceright}{\isacharparenright}{\isachardoublequoteclose}\ \isacommand{by}\isamarkupfalse%
\ {\isacharparenleft}intro\ span{\isacharunderscore}is{\isacharunderscore}monotone{\isacharcomma}\ auto{\isacharparenright}\isanewline
\isanewline
\ \ \isacommand{from}\isamarkupfalse%
\ inC\ \isacommand{have}\isamarkupfalse%
\ inown{\isacharcolon}\ {\isachardoublequoteopen}A{\isasymsubseteq}span\ A{\isachardoublequoteclose}\ \isacommand{by}\isamarkupfalse%
\ {\isacharparenleft}rule\ in{\isacharunderscore}own{\isacharunderscore}span{\isacharparenright}\isanewline
\ \ \isacommand{from}\isamarkupfalse%
\ inC\ \isacommand{have}\isamarkupfalse%
\ subm{\isacharcolon}\ {\isachardoublequoteopen}submodule\ K\ {\isacharparenleft}span\ A{\isacharparenright}\ V{\isachardoublequoteclose}\ \isacommand{by}\isamarkupfalse%
\ {\isacharparenleft}rule\ span{\isacharunderscore}is{\isacharunderscore}submodule{\isacharparenright}\isanewline
\ \ \isacommand{from}\isamarkupfalse%
\ inown\ inspan\ subm\ \isacommand{have}\isamarkupfalse%
\ dir{\isadigit{2}}{\isacharcolon}\ {\isachardoublequoteopen}span\ {\isacharparenleft}A\ {\isasymunion}\ {\isacharbraceleft}v{\isacharbraceright}{\isacharparenright}\ {\isasymsubseteq}\ span\ A{\isachardoublequoteclose}\ \isacommand{by}\isamarkupfalse%
\ {\isacharparenleft}intro\ span{\isacharunderscore}is{\isacharunderscore}subset{\isacharcomma}\ auto{\isacharparenright}\ \isanewline
\isanewline
\ \ \isacommand{from}\isamarkupfalse%
\ dir{\isadigit{1}}\ dir{\isadigit{2}}\ \isacommand{show}\isamarkupfalse%
\ {\isacharquery}thesis\ \isacommand{by}\isamarkupfalse%
\ auto\isanewline
\isacommand{qed}\isamarkupfalse%
%
\endisatagproof
{\isafoldproof}%
%
\isadelimproof
%
\endisadelimproof
%
\isamarkupsubsection{The Replacement Theorem%
}
\isamarkuptrue%
%
\begin{isamarkuptext}%
If $A,B\subseteq V$ are finite, $A$ is linearly independent, $B$ generates $W$, 
and $A\subseteq W$, then there exists $C\subseteq V$ disjoint from $A$ such that
$\text{span}(A\cup C) = W$ and $|C|\le |B|-|A|$. In other words, we can complete
any linearly independent set to a generating set of $W$ by adding at most $|B|-|A|$ more elements.%
\end{isamarkuptext}%
\isamarkuptrue%
\isacommand{theorem}\isamarkupfalse%
\ {\isacharparenleft}\isakeyword{in}\ vectorspace{\isacharparenright}\ replacement{\isacharcolon}\isanewline
\ \ \isakeyword{fixes}\ A\ B\ \ \isanewline
\ \ \isakeyword{assumes}\ h{\isadigit{1}}{\isacharcolon}\ {\isachardoublequoteopen}finite\ A{\isachardoublequoteclose}\isanewline
\ \ \ \ \ \ \isakeyword{and}\ h{\isadigit{2}}{\isacharcolon}\ {\isachardoublequoteopen}finite\ B{\isachardoublequoteclose}\isanewline
\ \ \ \ \ \ \isakeyword{and}\ h{\isadigit{3}}{\isacharcolon}\ {\isachardoublequoteopen}B{\isasymsubseteq}carrier\ V{\isachardoublequoteclose}\isanewline
\ \ \ \ \ \ \isakeyword{and}\ h{\isadigit{4}}{\isacharcolon}\ {\isachardoublequoteopen}lin{\isacharunderscore}indpt\ A{\isachardoublequoteclose}\ \isanewline
\ \ \ \ \ \ \isakeyword{and}\ h{\isadigit{5}}{\isacharcolon}\ {\isachardoublequoteopen}A{\isasymsubseteq}span\ B{\isachardoublequoteclose}\ \isanewline
\ \ \isakeyword{shows}\ {\isachardoublequoteopen}{\isasymexists}C{\isachardot}\ finite\ C\ {\isasymand}\ C{\isasymsubseteq}carrier\ V\ {\isasymand}\ C{\isasymsubseteq}span\ B\ {\isasymand}\ C{\isasyminter}A{\isacharequal}{\isacharbraceleft}{\isacharbraceright}\ {\isasymand}\ int\ {\isacharparenleft}card\ C{\isacharparenright}\ {\isasymle}\ {\isacharparenleft}int\ {\isacharparenleft}card\ B{\isacharparenright}{\isacharparenright}\ {\isacharminus}\ {\isacharparenleft}int\ {\isacharparenleft}card\ A{\isacharparenright}{\isacharparenright}\ {\isasymand}\ {\isacharparenleft}span\ {\isacharparenleft}A\ {\isasymunion}\ C{\isacharparenright}\ {\isacharequal}\ span\ B{\isacharparenright}{\isachardoublequoteclose}\ \isanewline
\ \ {\isacharparenleft}\isakeyword{is}\ {\isachardoublequoteopen}{\isasymexists}C{\isachardot}\ {\isacharquery}P\ A\ B\ C{\isachardoublequoteclose}{\isacharparenright}\isanewline
\ \ \isanewline
%
\isadelimproof
%
\endisadelimproof
%
\isatagproof
\isacommand{using}\isamarkupfalse%
\ h{\isadigit{1}}\ h{\isadigit{2}}\ h{\isadigit{3}}\ h{\isadigit{4}}\ h{\isadigit{5}}\ \isanewline
\isacommand{proof}\isamarkupfalse%
\ {\isacharparenleft}induct\ {\isachardoublequoteopen}card\ A{\isachardoublequoteclose}\ arbitrary{\isacharcolon}\ A\ B{\isacharparenright}\isanewline
\ \ \isacommand{case}\isamarkupfalse%
\ {\isadigit{0}}\isanewline
\ \ \isacommand{from}\isamarkupfalse%
\ {\isachardoublequoteopen}{\isadigit{0}}{\isachardot}prems{\isachardoublequoteclose}{\isacharparenleft}{\isadigit{1}}{\isacharparenright}\ {\isachardoublequoteopen}{\isadigit{0}}{\isachardot}hyps{\isachardoublequoteclose}\ \isacommand{have}\isamarkupfalse%
\ a{\isadigit{0}}{\isacharcolon}\ {\isachardoublequoteopen}A{\isacharequal}{\isacharbraceleft}{\isacharbraceright}{\isachardoublequoteclose}\ \isacommand{by}\isamarkupfalse%
\ auto\isanewline
\ \ \isacommand{from}\isamarkupfalse%
\ {\isachardoublequoteopen}{\isadigit{0}}{\isachardot}prems{\isachardoublequoteclose}{\isacharparenleft}{\isadigit{3}}{\isacharparenright}\ \isacommand{have}\isamarkupfalse%
\ a{\isadigit{3}}{\isacharcolon}\ {\isachardoublequoteopen}B{\isasymsubseteq}span\ B{\isachardoublequoteclose}\ \isacommand{by}\isamarkupfalse%
\ {\isacharparenleft}rule\ in{\isacharunderscore}own{\isacharunderscore}span{\isacharparenright}\isanewline
\ \ \isacommand{from}\isamarkupfalse%
\ a{\isadigit{0}}\ a{\isadigit{3}}\ {\isachardoublequoteopen}{\isadigit{0}}{\isachardot}prems{\isachardoublequoteclose}\ \isacommand{show}\isamarkupfalse%
\ {\isacharquery}case\ \isacommand{by}\isamarkupfalse%
\ {\isacharparenleft}rule{\isacharunderscore}tac\ x{\isacharequal}{\isachardoublequoteopen}B{\isachardoublequoteclose}\ \isakeyword{in}\ exI{\isacharcomma}\ auto{\isacharparenright}\isanewline
\isacommand{next}\isamarkupfalse%
\isanewline
\ \ \isacommand{case}\isamarkupfalse%
\ {\isacharparenleft}Suc\ m{\isacharparenright}\isanewline
\ \ \isacommand{let}\isamarkupfalse%
\ {\isacharquery}W{\isacharequal}{\isachardoublequoteopen}span\ B{\isachardoublequoteclose}\isanewline
\ \ \isacommand{from}\isamarkupfalse%
\ Suc{\isachardot}prems{\isacharparenleft}{\isadigit{3}}{\isacharparenright}\ \isacommand{have}\isamarkupfalse%
\ BinC{\isacharcolon}\ {\isachardoublequoteopen}span\ B{\isasymsubseteq}carrier\ V{\isachardoublequoteclose}\ \isacommand{by}\isamarkupfalse%
\ {\isacharparenleft}rule\ span{\isacharunderscore}is{\isacharunderscore}subset{\isadigit{2}}{\isacharparenright}\isanewline
\ \ \isanewline
\ \ \isacommand{from}\isamarkupfalse%
\ Suc{\isachardot}prems\ Suc{\isachardot}hyps\ BinC\ \isacommand{have}\isamarkupfalse%
\ A{\isacharcolon}\ {\isachardoublequoteopen}finite\ A{\isachardoublequoteclose}\ {\isachardoublequoteopen}lin{\isacharunderscore}indpt\ A{\isachardoublequoteclose}\ {\isachardoublequoteopen}A{\isasymsubseteq}span\ B{\isachardoublequoteclose}\ {\isachardoublequoteopen}Suc\ m\ {\isacharequal}\ card\ A{\isachardoublequoteclose}\ {\isachardoublequoteopen}A{\isasymsubseteq}carrier\ V{\isachardoublequoteclose}\ \isanewline
\ \ \ \ \isacommand{by}\isamarkupfalse%
\ auto\isanewline
\ \ \isanewline
\ \ \isacommand{from}\isamarkupfalse%
\ Suc{\isachardot}prems\ \isacommand{have}\isamarkupfalse%
\ B{\isacharcolon}\ {\isachardoublequoteopen}finite\ B{\isachardoublequoteclose}\ {\isachardoublequoteopen}B{\isasymsubseteq}carrier\ V{\isachardoublequoteclose}\ \isacommand{by}\isamarkupfalse%
\ auto\isanewline
\ \ \ \isanewline
\ \ \isacommand{from}\isamarkupfalse%
\ Suc{\isachardot}hyps{\isacharparenleft}{\isadigit{2}}{\isacharparenright}\ \isacommand{obtain}\isamarkupfalse%
\ v\ \isakeyword{where}\ v{\isacharcolon}\ {\isachardoublequoteopen}v{\isasymin}A{\isachardoublequoteclose}\ \isacommand{by}\isamarkupfalse%
\ fastforce\isanewline
\ \ \isacommand{let}\isamarkupfalse%
\ {\isacharquery}A{\isacharprime}{\isacharequal}{\isachardoublequoteopen}A{\isacharminus}{\isacharbraceleft}v{\isacharbraceright}{\isachardoublequoteclose}\isanewline
\ \ \isanewline
\ \ \isacommand{from}\isamarkupfalse%
\ A{\isacharparenleft}{\isadigit{2}}{\isacharparenright}\ \isacommand{have}\isamarkupfalse%
\ liA{\isacharprime}{\isacharcolon}\ {\isachardoublequoteopen}lin{\isacharunderscore}indpt\ {\isacharquery}A{\isacharprime}{\isachardoublequoteclose}\ \isanewline
\ \ \ \ \isacommand{apply}\isamarkupfalse%
\ {\isacharparenleft}intro\ subset{\isacharunderscore}li{\isacharunderscore}is{\isacharunderscore}li{\isacharbrackleft}of\ {\isachardoublequoteopen}A{\isachardoublequoteclose}\ {\isachardoublequoteopen}{\isacharquery}A{\isacharprime}{\isachardoublequoteclose}{\isacharbrackright}{\isacharparenright}\ \isanewline
\ \ \ \ \ \isacommand{by}\isamarkupfalse%
\ auto\isanewline
\ \ \isacommand{from}\isamarkupfalse%
\ v\ liA{\isacharprime}\ Suc{\isachardot}prems\ Suc{\isachardot}hyps{\isacharparenleft}{\isadigit{2}}{\isacharparenright}\ \isacommand{have}\isamarkupfalse%
\ {\isachardoublequoteopen}{\isasymexists}C{\isacharprime}{\isachardot}\ {\isacharquery}P\ {\isacharquery}A{\isacharprime}\ B\ C{\isacharprime}{\isachardoublequoteclose}\ \isanewline
\ \ \ \ \isacommand{apply}\isamarkupfalse%
\ {\isacharparenleft}intro\ Suc{\isachardot}hyps{\isacharparenleft}{\isadigit{1}}{\isacharparenright}{\isacharparenright}\isanewline
\ \ \ \ \ \ \ \ \ \isacommand{by}\isamarkupfalse%
\ auto\isanewline
\ \ \isacommand{from}\isamarkupfalse%
\ this\ \isacommand{obtain}\isamarkupfalse%
\ C{\isacharprime}\ \isakeyword{where}\ C{\isacharprime}{\isacharcolon}\ {\isachardoublequoteopen}{\isacharquery}P\ {\isacharquery}A{\isacharprime}\ B\ C{\isacharprime}{\isachardoublequoteclose}\ \isacommand{by}\isamarkupfalse%
\ auto\isanewline
\isanewline
\ \ \isacommand{show}\isamarkupfalse%
\ {\isacharquery}case\ \isanewline
\ \ \isacommand{proof}\isamarkupfalse%
\ {\isacharparenleft}cases\ {\isachardoublequoteopen}v{\isasymin}\ C{\isacharprime}{\isachardoublequoteclose}{\isacharparenright}\isanewline
\ \ \ \ \isacommand{case}\isamarkupfalse%
\ True\isanewline
\ \ \ \ \isacommand{have}\isamarkupfalse%
\ vinC{\isacharprime}{\isacharcolon}\ {\isachardoublequoteopen}v{\isasymin}C{\isacharprime}{\isachardoublequoteclose}\ \isacommand{by}\isamarkupfalse%
\ fact\isanewline
\ \ \ \ \isacommand{from}\isamarkupfalse%
\ vinC{\isacharprime}\ v\ \isacommand{have}\isamarkupfalse%
\ seteq{\isacharcolon}\ {\isachardoublequoteopen}A\ {\isacharminus}\ {\isacharbraceleft}v{\isacharbraceright}\ {\isasymunion}\ C{\isacharprime}\ {\isacharequal}\ A\ {\isasymunion}\ {\isacharparenleft}C{\isacharprime}\ {\isacharminus}\ {\isacharbraceleft}v{\isacharbraceright}{\isacharparenright}{\isachardoublequoteclose}\ \isacommand{by}\isamarkupfalse%
\ auto\isanewline
\ \ \ \ \isacommand{from}\isamarkupfalse%
\ C{\isacharprime}\ seteq\ \isacommand{have}\isamarkupfalse%
\ spaneq{\isacharcolon}\ {\isachardoublequoteopen}span\ {\isacharparenleft}A\ {\isasymunion}\ {\isacharparenleft}C{\isacharprime}\ {\isacharminus}\ {\isacharbraceleft}v{\isacharbraceright}{\isacharparenright}{\isacharparenright}\ {\isacharequal}\ span\ {\isacharparenleft}B{\isacharparenright}{\isachardoublequoteclose}\ \isacommand{by}\isamarkupfalse%
\ algebra\isanewline
\ \ \ \ \isacommand{from}\isamarkupfalse%
\ Suc{\isachardot}prems\ Suc{\isachardot}hyps\ C{\isacharprime}\ vinC{\isacharprime}\ v\ spaneq\ \isacommand{show}\isamarkupfalse%
\ {\isacharquery}thesis\isanewline
\ \ \ \ \ \ \isacommand{apply}\isamarkupfalse%
\ {\isacharparenleft}rule{\isacharunderscore}tac\ x{\isacharequal}{\isachardoublequoteopen}C{\isacharprime}{\isacharminus}{\isacharbraceleft}v{\isacharbraceright}{\isachardoublequoteclose}\ \isakeyword{in}\ exI{\isacharparenright}\isanewline
\ \ \ \ \ \ \isacommand{apply}\isamarkupfalse%
\ {\isacharparenleft}subgoal{\isacharunderscore}tac\ {\isachardoublequoteopen}card\ C{\isacharprime}\ {\isachargreater}{\isadigit{0}}{\isachardoublequoteclose}{\isacharparenright}\isanewline
\ \ \ \ \ \ \ \isacommand{by}\isamarkupfalse%
\ auto\isanewline
\ \ \isacommand{next}\isamarkupfalse%
\ \isanewline
\ \ \ \ \isacommand{case}\isamarkupfalse%
\ False\isanewline
\ \ \ \ \isacommand{have}\isamarkupfalse%
\ f{\isacharcolon}\ {\isachardoublequoteopen}v{\isasymnotin}C{\isacharprime}{\isachardoublequoteclose}\ \isacommand{by}\isamarkupfalse%
\ fact\isanewline
\ \ \ \ \isacommand{from}\isamarkupfalse%
\ A\ v\ C{\isacharprime}\ \isacommand{have}\isamarkupfalse%
\ {\isachardoublequoteopen}{\isasymexists}a{\isachardot}\ a{\isasymin}{\isacharparenleft}{\isacharquery}A{\isacharprime}{\isasymunion}C{\isacharprime}{\isacharparenright}{\isasymrightarrow}carrier\ K\ {\isasymand}\ \ lincomb\ a\ {\isacharparenleft}{\isacharquery}A{\isacharprime}\ {\isasymunion}\ C{\isacharprime}{\isacharparenright}\ {\isacharequal}v{\isachardoublequoteclose}\ \isanewline
\ \ \ \ \ \ \isacommand{by}\isamarkupfalse%
\ {\isacharparenleft}intro\ finite{\isacharunderscore}in{\isacharunderscore}span{\isacharcomma}\ auto{\isacharparenright}\isanewline
\ \ \ \ \isacommand{from}\isamarkupfalse%
\ this\ \isacommand{obtain}\isamarkupfalse%
\ a\ \isakeyword{where}\ a{\isacharcolon}\ {\isachardoublequoteopen}a{\isasymin}{\isacharparenleft}{\isacharquery}A{\isacharprime}{\isasymunion}C{\isacharprime}{\isacharparenright}{\isasymrightarrow}carrier\ K\ {\isasymand}\ v{\isacharequal}\ lincomb\ a\ {\isacharparenleft}{\isacharquery}A{\isacharprime}\ {\isasymunion}\ C{\isacharprime}{\isacharparenright}{\isachardoublequoteclose}\ \isacommand{by}\isamarkupfalse%
\ metis\isanewline
\ \ \ \ \isacommand{let}\isamarkupfalse%
\ {\isacharquery}b{\isacharequal}{\isachardoublequoteopen}{\isacharparenleft}{\isasymlambda}\ w{\isachardot}\ if\ {\isacharparenleft}w{\isacharequal}v{\isacharparenright}\ then\ {\isasymominus}\isactrlbsub K\isactrlesub {\isasymone}\isactrlbsub K\isactrlesub \ else\ a\ w{\isacharparenright}{\isachardoublequoteclose}\isanewline
\ \ \ \ \isacommand{from}\isamarkupfalse%
\ a\ \isacommand{have}\isamarkupfalse%
\ b{\isacharcolon}\ {\isachardoublequoteopen}{\isacharquery}b{\isasymin}A{\isasymunion}C{\isacharprime}{\isasymrightarrow}carrier\ K{\isachardoublequoteclose}\ \isacommand{by}\isamarkupfalse%
\ auto\isanewline
\ \ \ \ \isacommand{from}\isamarkupfalse%
\ v\ \isacommand{have}\isamarkupfalse%
\ rewrite{\isacharunderscore}ins{\isacharcolon}\ {\isachardoublequoteopen}A{\isasymunion}C{\isacharprime}{\isacharequal}{\isacharparenleft}{\isacharquery}A{\isacharprime}{\isasymunion}C{\isacharprime}{\isacharparenright}{\isasymunion}{\isacharbraceleft}v{\isacharbraceright}{\isachardoublequoteclose}\ \isacommand{by}\isamarkupfalse%
\ auto\isanewline
\ \ \ \ \isacommand{from}\isamarkupfalse%
\ f\ \isacommand{have}\isamarkupfalse%
\ {\isachardoublequoteopen}v{\isasymnotin}{\isacharquery}A{\isacharprime}{\isasymunion}C{\isacharprime}{\isachardoublequoteclose}\ \isacommand{by}\isamarkupfalse%
\ auto\isanewline
\ \ \ \ \isacommand{from}\isamarkupfalse%
\ this\ A\ C{\isacharprime}\ v\ a\ f\ \isacommand{have}\isamarkupfalse%
\ lcb{\isacharcolon}\ {\isachardoublequoteopen}lincomb\ {\isacharquery}b\ {\isacharparenleft}A\ {\isasymunion}\ C{\isacharprime}{\isacharparenright}\ {\isacharequal}\ {\isasymzero}\isactrlbsub V\isactrlesub {\isachardoublequoteclose}\isanewline
\ \ \ \ \ \ \isacommand{apply}\isamarkupfalse%
\ {\isacharparenleft}subst\ rewrite{\isacharunderscore}ins{\isacharparenright}\isanewline
\ \ \ \ \ \ \isacommand{apply}\isamarkupfalse%
\ {\isacharparenleft}subst\ lincomb{\isacharunderscore}insert{\isacharparenright}\isanewline
\ \ \ \ \ \ \ \ \ \ \ \isacommand{apply}\isamarkupfalse%
\ {\isacharparenleft}simp{\isacharunderscore}all\ add{\isacharcolon}\ ring{\isacharunderscore}subset{\isacharunderscore}carrier\ coeff{\isacharunderscore}in{\isacharunderscore}ring{\isacharparenright}\isanewline
\ \ \ \ \ \ \ \ \isacommand{apply}\isamarkupfalse%
\ {\isacharparenleft}auto\ split{\isacharcolon}\ split{\isacharunderscore}if{\isacharunderscore}asm{\isacharparenright}\isanewline
\ \ \ \ \ \ \isacommand{apply}\isamarkupfalse%
\ {\isacharparenleft}subst\ lincomb{\isacharunderscore}elim{\isacharunderscore}if{\isacharparenright}\isanewline
\ \ \ \ \ \ \ \ \ \ \isacommand{by}\isamarkupfalse%
\ {\isacharparenleft}auto\ simp\ add{\isacharcolon}\ smult{\isacharunderscore}minus{\isacharunderscore}{\isadigit{1}}\ l{\isacharunderscore}neg\ ring{\isacharunderscore}subset{\isacharunderscore}carrier{\isacharparenright}\isanewline
\isanewline
\ \ \ \ \isacommand{from}\isamarkupfalse%
\ C{\isacharprime}\ f\ \isacommand{have}\isamarkupfalse%
\ rewrite{\isacharunderscore}minus{\isacharcolon}\ {\isachardoublequoteopen}C{\isacharprime}{\isacharequal}{\isacharparenleft}A{\isasymunion}C{\isacharprime}{\isacharparenright}{\isacharminus}A{\isachardoublequoteclose}\ \isacommand{by}\isamarkupfalse%
\ auto\isanewline
\ \ \ \ \isacommand{from}\isamarkupfalse%
\ A\ C{\isacharprime}\ b\ lcb\ v\ \isacommand{have}\isamarkupfalse%
\ exw{\isacharcolon}\ {\isachardoublequoteopen}{\isasymexists}w{\isasymin}\ C{\isacharprime}{\isachardot}\ {\isacharquery}b\ w{\isasymnoteq}{\isasymzero}\isactrlbsub K\isactrlesub {\isachardoublequoteclose}\ \isanewline
\ \ \ \ \ \ \isacommand{apply}\isamarkupfalse%
\ {\isacharparenleft}subst\ rewrite{\isacharunderscore}minus{\isacharparenright}\isanewline
\ \ \ \ \ \ \isacommand{apply}\isamarkupfalse%
\ {\isacharparenleft}intro\ lincomb{\isacharunderscore}must{\isacharunderscore}include{\isacharbrackleft}\isakeyword{where}\ {\isacharquery}T{\isacharequal}{\isachardoublequoteopen}A{\isasymunion}C{\isacharprime}{\isachardoublequoteclose}\ \isakeyword{and}\ {\isacharquery}v{\isacharequal}{\isachardoublequoteopen}v{\isachardoublequoteclose}{\isacharbrackright}{\isacharparenright}\isanewline
\ \ \ \ \ \ \ \ \ \ \ \ \ \ \isacommand{by}\isamarkupfalse%
\ auto\isanewline
\ \ \ \ \isacommand{from}\isamarkupfalse%
\ exw\ \isacommand{obtain}\isamarkupfalse%
\ w\ \isakeyword{where}\ w{\isacharcolon}\ {\isachardoublequoteopen}w{\isasymin}\ C{\isacharprime}{\isachardoublequoteclose}\ {\isachardoublequoteopen}{\isacharquery}b\ w{\isasymnoteq}{\isasymzero}\isactrlbsub K\isactrlesub {\isachardoublequoteclose}\ \isacommand{by}\isamarkupfalse%
\ auto\isanewline
\ \ \ \ \isacommand{from}\isamarkupfalse%
\ A\ C{\isacharprime}\ w\ f\ b\ lcb\ \isacommand{have}\isamarkupfalse%
\ w{\isacharunderscore}in{\isacharcolon}\ {\isachardoublequoteopen}w{\isasymin}span\ {\isacharparenleft}{\isacharparenleft}A{\isasymunion}\ C{\isacharprime}{\isacharparenright}\ {\isacharminus}{\isacharbraceleft}w{\isacharbraceright}{\isacharparenright}{\isachardoublequoteclose}\ \isanewline
\ \ \ \ \ \ \isacommand{apply}\isamarkupfalse%
\ {\isacharparenleft}intro\ lincomb{\isacharunderscore}isolate{\isacharbrackleft}\isakeyword{where}\ a{\isacharequal}{\isachardoublequoteopen}{\isacharquery}b{\isachardoublequoteclose}{\isacharbrackright}{\isacharparenright}\ \isanewline
\ \ \ \ \ \ \ \ \ \ \ \isacommand{by}\isamarkupfalse%
\ auto\isanewline
\ \ \ \ \isacommand{have}\isamarkupfalse%
\ spaneq{\isadigit{2}}{\isacharcolon}\ {\isachardoublequoteopen}span\ {\isacharparenleft}A{\isasymunion}{\isacharparenleft}C{\isacharprime}{\isacharminus}{\isacharbraceleft}w{\isacharbraceright}{\isacharparenright}{\isacharparenright}\ {\isacharequal}\ span\ B{\isachardoublequoteclose}\isanewline
\ \ \ \ \isacommand{proof}\isamarkupfalse%
\ {\isacharminus}\ \isanewline
\ \ \ \ \ \ \isacommand{have}\isamarkupfalse%
\ {\isadigit{1}}{\isacharcolon}\ {\isachardoublequoteopen}span\ {\isacharparenleft}{\isacharquery}A{\isacharprime}{\isasymunion}C{\isacharprime}{\isacharparenright}\ {\isacharequal}\ span\ {\isacharparenleft}A{\isasymunion}\ C{\isacharprime}{\isacharparenright}{\isachardoublequoteclose}\ \isanewline
\ \ \ \ \ \ \ \ \isacommand{proof}\isamarkupfalse%
\ {\isacharminus}\ \isanewline
\ \ \ \ \ \ \ \ \ \ \isacommand{from}\isamarkupfalse%
\ A\ C{\isacharprime}\ v\ \isacommand{have}\isamarkupfalse%
\ m{\isadigit{1}}{\isacharcolon}\ {\isachardoublequoteopen}span\ {\isacharparenleft}{\isacharquery}A{\isacharprime}{\isasymunion}C{\isacharprime}{\isacharparenright}\ {\isacharequal}\ span\ {\isacharparenleft}{\isacharparenleft}{\isacharquery}A{\isacharprime}{\isasymunion}\ C{\isacharprime}{\isacharparenright}{\isasymunion}{\isacharbraceleft}v{\isacharbraceright}{\isacharparenright}{\isachardoublequoteclose}\isanewline
\ \ \ \ \ \ \ \ \ \ \ \ \isacommand{apply}\isamarkupfalse%
\ {\isacharparenleft}intro\ already{\isacharunderscore}in{\isacharunderscore}span{\isacharparenright}\ \isanewline
\ \ \ \ \ \ \ \ \ \ \ \ \ \isacommand{by}\isamarkupfalse%
\ auto\isanewline
\ \ \ \ \ \ \ \ \ \ \isacommand{from}\isamarkupfalse%
\ f\ m{\isadigit{1}}\ \isacommand{show}\isamarkupfalse%
\ {\isacharquery}thesis\ \isacommand{by}\isamarkupfalse%
\ {\isacharparenleft}metis\ rewrite{\isacharunderscore}ins{\isacharparenright}\isanewline
\ \ \ \ \ \ \ \ \isacommand{qed}\isamarkupfalse%
\isanewline
\ \ \ \ \ \ \isacommand{have}\isamarkupfalse%
\ {\isadigit{2}}{\isacharcolon}\ {\isachardoublequoteopen}span\ {\isacharparenleft}A{\isasymunion}\ {\isacharparenleft}C{\isacharprime}{\isacharminus}{\isacharbraceleft}w{\isacharbraceright}{\isacharparenright}{\isacharparenright}\ {\isacharequal}\ span\ {\isacharparenleft}A{\isasymunion}\ C{\isacharprime}{\isacharparenright}{\isachardoublequoteclose}\ \isanewline
\ \ \ \ \ \ \isacommand{proof}\isamarkupfalse%
\ {\isacharminus}\ \isanewline
\ \ \ \ \ \ \ \ \isacommand{from}\isamarkupfalse%
\ C{\isacharprime}\ w{\isacharparenleft}{\isadigit{1}}{\isacharparenright}\ f\ \isacommand{have}\isamarkupfalse%
\ b{\isadigit{6}}{\isadigit{0}}{\isacharcolon}\ {\isachardoublequoteopen}A{\isasymunion}\ {\isacharparenleft}C{\isacharprime}{\isacharminus}{\isacharbraceleft}w{\isacharbraceright}{\isacharparenright}\ {\isacharequal}\ {\isacharparenleft}A{\isasymunion}\ C{\isacharprime}{\isacharparenright}\ {\isacharminus}{\isacharbraceleft}w{\isacharbraceright}{\isachardoublequoteclose}\ \isacommand{by}\isamarkupfalse%
\ auto\isanewline
\ \ \ \ \ \ \ \ \isacommand{from}\isamarkupfalse%
\ \ w{\isacharparenleft}{\isadigit{1}}{\isacharparenright}\ \isacommand{have}\isamarkupfalse%
\ b{\isadigit{6}}{\isadigit{1}}{\isacharcolon}\ {\isachardoublequoteopen}A{\isasymunion}\ C{\isacharprime}{\isacharequal}\ {\isacharparenleft}A{\isasymunion}\ C{\isacharprime}\ {\isacharminus}{\isacharbraceleft}w{\isacharbraceright}{\isacharparenright}{\isasymunion}{\isacharbraceleft}w{\isacharbraceright}{\isachardoublequoteclose}\ \isacommand{by}\isamarkupfalse%
\ auto\isanewline
\ \ \ \ \ \ \ \ \isacommand{from}\isamarkupfalse%
\ A\ C{\isacharprime}\ \ w{\isacharunderscore}in\ \isacommand{show}\isamarkupfalse%
\ {\isacharquery}thesis\ \isanewline
\ \ \ \ \ \ \ \ \ \ \isacommand{apply}\isamarkupfalse%
\ {\isacharparenleft}subst\ b{\isadigit{6}}{\isadigit{1}}{\isacharparenright}\ \isanewline
\ \ \ \ \ \ \ \ \ \ \isacommand{apply}\isamarkupfalse%
\ {\isacharparenleft}subst\ b{\isadigit{6}}{\isadigit{0}}{\isacharparenright}\ \isanewline
\ \ \ \ \ \ \ \ \ \ \isacommand{apply}\isamarkupfalse%
\ {\isacharparenleft}intro\ already{\isacharunderscore}in{\isacharunderscore}span{\isacharparenright}\ \isanewline
\ \ \ \ \ \ \ \ \ \ \ \isacommand{by}\isamarkupfalse%
\ auto\isanewline
\ \ \ \ \ \ \ \ \isacommand{qed}\isamarkupfalse%
\isanewline
\ \ \ \ \isacommand{from}\isamarkupfalse%
\ C{\isacharprime}\ {\isadigit{1}}\ {\isadigit{2}}\ \isacommand{show}\isamarkupfalse%
\ {\isacharquery}thesis\ \isacommand{by}\isamarkupfalse%
\ auto\isanewline
\ \ \isacommand{qed}\isamarkupfalse%
\isanewline
\ \ \ \ \isacommand{from}\isamarkupfalse%
\ A\ C{\isacharprime}\ w\ f\ v\ spaneq{\isadigit{2}}\ \isacommand{show}\isamarkupfalse%
\ {\isacharquery}thesis\isanewline
\ \ \ \ \ \ \isacommand{apply}\isamarkupfalse%
\ {\isacharparenleft}rule{\isacharunderscore}tac\ x{\isacharequal}{\isachardoublequoteopen}C{\isacharprime}{\isacharminus}{\isacharbraceleft}w{\isacharbraceright}{\isachardoublequoteclose}\ \isakeyword{in}\ exI{\isacharparenright}\isanewline
\ \ \ \ \ \ \isacommand{apply}\isamarkupfalse%
\ {\isacharparenleft}subgoal{\isacharunderscore}tac\ {\isachardoublequoteopen}card\ C{\isacharprime}\ {\isachargreater}{\isadigit{0}}{\isachardoublequoteclose}{\isacharparenright}\isanewline
\ \ \ \ \ \ \ \isacommand{by}\isamarkupfalse%
\ auto\isanewline
\ \ \isacommand{qed}\isamarkupfalse%
\isanewline
\isacommand{qed}\isamarkupfalse%
%
\endisatagproof
{\isafoldproof}%
%
\isadelimproof
%
\endisadelimproof
%
\isamarkupsubsection{Defining dimension and bases.%
}
\isamarkuptrue%
%
\begin{isamarkuptext}%
Finite dimensional is defined as having a finite generating set.%
\end{isamarkuptext}%
\isamarkuptrue%
\isacommand{definition}\isamarkupfalse%
\ {\isacharparenleft}\isakeyword{in}\ vectorspace{\isacharparenright}\ fin{\isacharunderscore}dim{\isacharcolon}{\isacharcolon}\ {\isachardoublequoteopen}bool{\isachardoublequoteclose}\isanewline
\ \ \isakeyword{where}\ {\isachardoublequoteopen}fin{\isacharunderscore}dim\ {\isacharequal}\ {\isacharparenleft}{\isasymexists}\ A{\isachardot}\ {\isacharparenleft}{\isacharparenleft}finite\ A{\isacharparenright}\ {\isasymand}\ {\isacharparenleft}A\ {\isasymsubseteq}\ carrier\ V{\isacharparenright}\ {\isasymand}\ {\isacharparenleft}gen{\isacharunderscore}set\ A{\isacharparenright}{\isacharparenright}{\isacharparenright}{\isachardoublequoteclose}%
\begin{isamarkuptext}%
The dimension is the size of the smallest generating set. For equivalent
characterizations see below.%
\end{isamarkuptext}%
\isamarkuptrue%
\isacommand{definition}\isamarkupfalse%
\ {\isacharparenleft}\isakeyword{in}\ vectorspace{\isacharparenright}\ dim{\isacharcolon}{\isacharcolon}\ {\isachardoublequoteopen}nat{\isachardoublequoteclose}\isanewline
\ \ \isakeyword{where}\ {\isachardoublequoteopen}dim\ {\isacharequal}\ {\isacharparenleft}LEAST\ n{\isachardot}\ {\isacharparenleft}{\isasymexists}\ A{\isachardot}\ {\isacharparenleft}{\isacharparenleft}finite\ A{\isacharparenright}\ {\isasymand}\ {\isacharparenleft}card\ A\ {\isacharequal}\ n{\isacharparenright}\ {\isasymand}\ {\isacharparenleft}A\ {\isasymsubseteq}\ carrier\ V{\isacharparenright}\ {\isasymand}\ {\isacharparenleft}gen{\isacharunderscore}set\ A{\isacharparenright}{\isacharparenright}{\isacharparenright}{\isacharparenright}{\isachardoublequoteclose}%
\begin{isamarkuptext}%
A \isa{basis} is a linearly independent generating set.%
\end{isamarkuptext}%
\isamarkuptrue%
\isacommand{definition}\isamarkupfalse%
\ {\isacharparenleft}\isakeyword{in}\ vectorspace{\isacharparenright}\ basis{\isacharcolon}{\isacharcolon}\ {\isachardoublequoteopen}{\isacharprime}c\ set\ {\isasymRightarrow}\ bool{\isachardoublequoteclose}\isanewline
\ \ \isakeyword{where}\ {\isachardoublequoteopen}basis\ A\ {\isacharequal}\ {\isacharparenleft}{\isacharparenleft}lin{\isacharunderscore}indpt\ A{\isacharparenright}\ {\isasymand}\ {\isacharparenleft}gen{\isacharunderscore}set\ A{\isacharparenright}{\isasymand}\ {\isacharparenleft}A{\isasymsubseteq}carrier\ V{\isacharparenright}{\isacharparenright}{\isachardoublequoteclose}%
\begin{isamarkuptext}%
From the replacement theorem, any linearly independent set is smaller than any generating set.%
\end{isamarkuptext}%
\isamarkuptrue%
\isacommand{lemma}\isamarkupfalse%
\ {\isacharparenleft}\isakeyword{in}\ vectorspace{\isacharparenright}\ li{\isacharunderscore}smaller{\isacharunderscore}than{\isacharunderscore}gen{\isacharcolon}\isanewline
\ \ \isakeyword{fixes}\ A\ B\isanewline
\ \ \isakeyword{assumes}\ h{\isadigit{1}}{\isacharcolon}\ {\isachardoublequoteopen}finite\ A{\isachardoublequoteclose}\ \isakeyword{and}\ h{\isadigit{2}}{\isacharcolon}\ {\isachardoublequoteopen}finite\ B{\isachardoublequoteclose}\ \isakeyword{and}\ h{\isadigit{3}}{\isacharcolon}\ {\isachardoublequoteopen}A{\isasymsubseteq}carrier\ V{\isachardoublequoteclose}\ \isakeyword{and}\ h{\isadigit{4}}{\isacharcolon}\ {\isachardoublequoteopen}B{\isasymsubseteq}carrier\ V{\isachardoublequoteclose}\ \isanewline
\ \ \ \ \isakeyword{and}\ h{\isadigit{5}}{\isacharcolon}\ {\isachardoublequoteopen}lin{\isacharunderscore}indpt\ A{\isachardoublequoteclose}\ \isakeyword{and}\ h{\isadigit{6}}{\isacharcolon}\ {\isachardoublequoteopen}gen{\isacharunderscore}set\ B{\isachardoublequoteclose}\isanewline
\ \ \isakeyword{shows}\ {\isachardoublequoteopen}card\ A\ {\isasymle}\ card\ B{\isachardoublequoteclose}\isanewline
%
\isadelimproof
%
\endisadelimproof
%
\isatagproof
\isacommand{proof}\isamarkupfalse%
\ {\isacharminus}\ \isanewline
\ \ \isacommand{from}\isamarkupfalse%
\ h{\isadigit{3}}\ h{\isadigit{6}}\ \isacommand{have}\isamarkupfalse%
\ {\isadigit{1}}{\isacharcolon}\ {\isachardoublequoteopen}A{\isasymsubseteq}span\ B{\isachardoublequoteclose}\ \isacommand{by}\isamarkupfalse%
\ \ auto\isanewline
\ \ \isacommand{from}\isamarkupfalse%
\ h{\isadigit{1}}\ h{\isadigit{2}}\ h{\isadigit{4}}\ h{\isadigit{5}}\ {\isadigit{1}}\ \isacommand{obtain}\isamarkupfalse%
\ C\ \isakeyword{where}\ \isanewline
\ \ \ \ {\isadigit{2}}{\isacharcolon}\ {\isachardoublequoteopen}finite\ C\ {\isasymand}\ C{\isasymsubseteq}carrier\ V\ {\isasymand}\ C{\isasymsubseteq}span\ B\ {\isasymand}\ C{\isasyminter}A{\isacharequal}{\isacharbraceleft}{\isacharbraceright}\ {\isasymand}\ int\ {\isacharparenleft}card\ C{\isacharparenright}\ {\isasymle}\ int\ {\isacharparenleft}card\ B{\isacharparenright}\ {\isacharminus}\ int\ {\isacharparenleft}card\ A{\isacharparenright}\ {\isasymand}\ {\isacharparenleft}span\ {\isacharparenleft}A\ {\isasymunion}\ C{\isacharparenright}\ {\isacharequal}\ span\ B{\isacharparenright}{\isachardoublequoteclose}\ \isanewline
\ \ \ \ \isacommand{by}\isamarkupfalse%
\ {\isacharparenleft}metis\ replacement{\isacharparenright}\ \isanewline
\ \ \isacommand{from}\isamarkupfalse%
\ {\isadigit{2}}\ \isacommand{show}\isamarkupfalse%
\ {\isacharquery}thesis\ \isacommand{by}\isamarkupfalse%
\ arith\isanewline
\isacommand{qed}\isamarkupfalse%
%
\endisatagproof
{\isafoldproof}%
%
\isadelimproof
%
\endisadelimproof
%
\begin{isamarkuptext}%
The dimension is the cardinality of any basis. (In particular, all bases are the same size.)%
\end{isamarkuptext}%
\isamarkuptrue%
\isacommand{lemma}\isamarkupfalse%
\ {\isacharparenleft}\isakeyword{in}\ vectorspace{\isacharparenright}\ dim{\isacharunderscore}basis{\isacharcolon}\isanewline
\ \ \isakeyword{fixes}\ A\ \isanewline
\ \ \isakeyword{assumes}\ \ fin{\isacharcolon}\ {\isachardoublequoteopen}finite\ A{\isachardoublequoteclose}\ \isakeyword{and}\ h{\isadigit{2}}{\isacharcolon}\ {\isachardoublequoteopen}basis\ A{\isachardoublequoteclose}\isanewline
\ \ \isakeyword{shows}\ {\isachardoublequoteopen}dim\ {\isacharequal}\ card\ A{\isachardoublequoteclose}\isanewline
%
\isadelimproof
%
\endisadelimproof
%
\isatagproof
\isacommand{proof}\isamarkupfalse%
\ {\isacharminus}\ \isanewline
\ \ \isacommand{have}\isamarkupfalse%
\ {\isadigit{0}}{\isacharcolon}\ {\isachardoublequoteopen}{\isasymAnd}B\ m{\isachardot}\ {\isacharparenleft}{\isacharparenleft}finite\ B{\isacharparenright}\ {\isasymand}\ {\isacharparenleft}card\ B\ {\isacharequal}\ m{\isacharparenright}\ {\isasymand}\ {\isacharparenleft}B\ {\isasymsubseteq}\ carrier\ V{\isacharparenright}\ {\isasymand}\ {\isacharparenleft}gen{\isacharunderscore}set\ B{\isacharparenright}{\isacharparenright}\ {\isasymLongrightarrow}\ card\ A\ {\isasymle}\ m{\isachardoublequoteclose}\isanewline
\ \ \isacommand{proof}\isamarkupfalse%
\ {\isacharminus}\ \isanewline
\ \ \ \ \isacommand{fix}\isamarkupfalse%
\ B\ m\ \isanewline
\ \ \ \ \isacommand{assume}\isamarkupfalse%
\ {\isadigit{1}}{\isacharcolon}\ {\isachardoublequoteopen}{\isacharparenleft}{\isacharparenleft}finite\ B{\isacharparenright}\ {\isasymand}\ {\isacharparenleft}card\ B\ {\isacharequal}\ m{\isacharparenright}\ {\isasymand}\ {\isacharparenleft}B\ {\isasymsubseteq}\ carrier\ V{\isacharparenright}\ {\isasymand}\ {\isacharparenleft}gen{\isacharunderscore}set\ B{\isacharparenright}{\isacharparenright}{\isachardoublequoteclose}\isanewline
\ \ \ \ \isacommand{from}\isamarkupfalse%
\ {\isadigit{1}}\ fin\ h{\isadigit{2}}\ \isacommand{have}\isamarkupfalse%
\ {\isadigit{2}}{\isacharcolon}\ {\isachardoublequoteopen}card\ A\ {\isasymle}\ card\ B{\isachardoublequoteclose}\ \isanewline
\ \ \ \ \ \ \isacommand{apply}\isamarkupfalse%
\ {\isacharparenleft}unfold\ basis{\isacharunderscore}def{\isacharparenright}\ \isanewline
\ \ \ \ \ \ \isacommand{apply}\isamarkupfalse%
\ {\isacharparenleft}intro\ li{\isacharunderscore}smaller{\isacharunderscore}than{\isacharunderscore}gen{\isacharparenright}\ \isanewline
\ \ \ \ \ \ \ \ \ \ \ \isacommand{by}\isamarkupfalse%
\ auto\isanewline
\ \ \ \ \isacommand{from}\isamarkupfalse%
\ {\isadigit{1}}\ {\isadigit{2}}\ \isacommand{show}\isamarkupfalse%
\ {\isachardoublequoteopen}{\isacharquery}thesis\ B\ m{\isachardoublequoteclose}\ \isacommand{by}\isamarkupfalse%
\ auto\isanewline
\ \ \isacommand{qed}\isamarkupfalse%
\isanewline
\ \ \isacommand{from}\isamarkupfalse%
\ fin\ h{\isadigit{2}}\ {\isadigit{0}}\ \isacommand{show}\isamarkupfalse%
\ {\isacharquery}thesis\isanewline
\ \ \ \ \isacommand{apply}\isamarkupfalse%
\ {\isacharparenleft}unfold\ dim{\isacharunderscore}def\ basis{\isacharunderscore}def{\isacharparenright}\ \isanewline
\ \ \ \ \isacommand{apply}\isamarkupfalse%
\ {\isacharparenleft}intro\ Least{\isacharunderscore}equality{\isacharparenright}\isanewline
\ \ \ \ \ \isacommand{apply}\isamarkupfalse%
\ {\isacharparenleft}rule{\isacharunderscore}tac\ x{\isacharequal}{\isachardoublequoteopen}A{\isachardoublequoteclose}\ \isakeyword{in}\ exI{\isacharparenright}\isanewline
\ \ \ \ \ \isacommand{by}\isamarkupfalse%
\ auto\ \isanewline
\isacommand{qed}\isamarkupfalse%
%
\endisatagproof
{\isafoldproof}%
%
\isadelimproof
%
\endisadelimproof
%
\begin{isamarkuptext}%
A \isa{maximal} set with respect to $P$ is such that if $B\supseteq A$ and $P$ is also 
satisfied for $B$, then $B=A$.%
\end{isamarkuptext}%
\isamarkuptrue%
\isacommand{definition}\isamarkupfalse%
\ maximal{\isacharcolon}{\isacharcolon}{\isachardoublequoteopen}{\isacharprime}a\ set\ {\isasymRightarrow}\ {\isacharparenleft}{\isacharprime}a\ set\ {\isasymRightarrow}\ bool{\isacharparenright}\ {\isasymRightarrow}\ bool{\isachardoublequoteclose}\isanewline
\ \ \isakeyword{where}\ {\isachardoublequoteopen}maximal\ A\ P\ {\isacharequal}\ {\isacharparenleft}{\isacharparenleft}P\ A{\isacharparenright}\ {\isasymand}\ {\isacharparenleft}{\isasymforall}B{\isachardot}\ B{\isasymsupseteq}A\ {\isasymand}\ P\ B\ {\isasymlongrightarrow}\ B\ {\isacharequal}\ A{\isacharparenright}{\isacharparenright}{\isachardoublequoteclose}%
\begin{isamarkuptext}%
A \isa{minimal} set with respect to $P$ is such that if $B\subseteq A$ and $P$ is also 
satisfied for $B$, then $B=A$.%
\end{isamarkuptext}%
\isamarkuptrue%
\isacommand{definition}\isamarkupfalse%
\ minimal{\isacharcolon}{\isacharcolon}{\isachardoublequoteopen}{\isacharprime}a\ set\ {\isasymRightarrow}\ {\isacharparenleft}{\isacharprime}a\ set\ {\isasymRightarrow}\ bool{\isacharparenright}\ {\isasymRightarrow}\ bool{\isachardoublequoteclose}\isanewline
\ \ \isakeyword{where}\ {\isachardoublequoteopen}minimal\ A\ P\ {\isacharequal}\ {\isacharparenleft}{\isacharparenleft}P\ A{\isacharparenright}\ {\isasymand}\ {\isacharparenleft}{\isasymforall}B{\isachardot}\ B{\isasymsubseteq}A\ {\isasymand}\ P\ B\ {\isasymlongrightarrow}\ B\ {\isacharequal}\ A{\isacharparenright}{\isacharparenright}{\isachardoublequoteclose}%
\begin{isamarkuptext}%
A maximal linearly independent set is a generating set.%
\end{isamarkuptext}%
\isamarkuptrue%
\isacommand{lemma}\isamarkupfalse%
\ {\isacharparenleft}\isakeyword{in}\ vectorspace{\isacharparenright}\ max{\isacharunderscore}li{\isacharunderscore}is{\isacharunderscore}gen{\isacharcolon}\isanewline
\ \ \isakeyword{fixes}\ A\isanewline
\ \ \isakeyword{assumes}\ h{\isadigit{1}}{\isacharcolon}\ {\isachardoublequoteopen}maximal\ A\ {\isacharparenleft}{\isasymlambda}S{\isachardot}\ S{\isasymsubseteq}carrier\ V\ {\isasymand}\ lin{\isacharunderscore}indpt\ S{\isacharparenright}{\isachardoublequoteclose}\isanewline
\ \ \isakeyword{shows}\ {\isachardoublequoteopen}gen{\isacharunderscore}set\ A{\isachardoublequoteclose}\isanewline
%
\isadelimproof
%
\endisadelimproof
%
\isatagproof
\isacommand{proof}\isamarkupfalse%
\ {\isacharparenleft}rule\ ccontr{\isacharparenright}\isanewline
\ \ \isacommand{assume}\isamarkupfalse%
\ {\isadigit{0}}{\isacharcolon}\ {\isachardoublequoteopen}{\isasymnot}{\isacharparenleft}gen{\isacharunderscore}set\ A{\isacharparenright}{\isachardoublequoteclose}\isanewline
\ \ \isacommand{from}\isamarkupfalse%
\ h{\isadigit{1}}\ \isacommand{have}\isamarkupfalse%
\ {\isadigit{1}}{\isacharcolon}\ {\isachardoublequoteopen}\ A{\isasymsubseteq}\ carrier\ V\ {\isasymand}\ lin{\isacharunderscore}indpt\ A{\isachardoublequoteclose}\ \isacommand{by}\isamarkupfalse%
\ {\isacharparenleft}unfold\ maximal{\isacharunderscore}def{\isacharcomma}\ auto{\isacharparenright}\isanewline
\ \ \isacommand{from}\isamarkupfalse%
\ {\isadigit{1}}\ \isacommand{have}\isamarkupfalse%
\ {\isadigit{2}}{\isacharcolon}\ {\isachardoublequoteopen}span\ A\ {\isasymsubseteq}\ carrier\ V{\isachardoublequoteclose}\ \isacommand{by}\isamarkupfalse%
\ {\isacharparenleft}intro\ span{\isacharunderscore}is{\isacharunderscore}subset{\isadigit{2}}{\isacharcomma}\ auto{\isacharparenright}\isanewline
\ \ \isacommand{from}\isamarkupfalse%
\ {\isadigit{0}}\ {\isadigit{1}}\ {\isadigit{2}}\ \isacommand{have}\isamarkupfalse%
\ {\isadigit{3}}{\isacharcolon}\ {\isachardoublequoteopen}{\isasymexists}v{\isachardot}\ v{\isasymin}carrier\ V\ {\isasymand}\ v\ {\isasymnotin}\ {\isacharparenleft}span\ A{\isacharparenright}{\isachardoublequoteclose}\isanewline
\ \ \ \ \isacommand{by}\isamarkupfalse%
\ auto\isanewline
\ \ \isacommand{from}\isamarkupfalse%
\ {\isadigit{3}}\ \isacommand{obtain}\isamarkupfalse%
\ v\ \isakeyword{where}\ {\isadigit{4}}{\isacharcolon}\ {\isachardoublequoteopen}v{\isasymin}carrier\ V\ {\isasymand}\ v\ {\isasymnotin}\ {\isacharparenleft}span\ A{\isacharparenright}{\isachardoublequoteclose}\ \isacommand{by}\isamarkupfalse%
\ auto\isanewline
\ \ \isacommand{have}\isamarkupfalse%
\ {\isadigit{5}}{\isacharcolon}\ {\isachardoublequoteopen}v{\isasymnotin}A{\isachardoublequoteclose}\ \isanewline
\ \ \isacommand{proof}\isamarkupfalse%
\ {\isacharminus}\ \isanewline
\ \ \ \ \isacommand{from}\isamarkupfalse%
\ h{\isadigit{1}}\ {\isadigit{1}}\ \isacommand{have}\isamarkupfalse%
\ {\isadigit{5}}{\isadigit{1}}{\isacharcolon}\ {\isachardoublequoteopen}A{\isasymsubseteq}span\ A{\isachardoublequoteclose}\ \isacommand{apply}\isamarkupfalse%
\ {\isacharparenleft}intro\ in{\isacharunderscore}own{\isacharunderscore}span{\isacharparenright}\ \isacommand{by}\isamarkupfalse%
\ auto\isanewline
\ \ \ \ \isacommand{from}\isamarkupfalse%
\ {\isadigit{4}}\ {\isadigit{5}}{\isadigit{1}}\ \isacommand{show}\isamarkupfalse%
\ {\isacharquery}thesis\ \isacommand{by}\isamarkupfalse%
\ auto\isanewline
\ \ \isacommand{qed}\isamarkupfalse%
\isanewline
\ \ \isacommand{from}\isamarkupfalse%
\ lin{\isacharunderscore}dep{\isacharunderscore}iff{\isacharunderscore}in{\isacharunderscore}span\ \isacommand{have}\isamarkupfalse%
\ {\isadigit{6}}{\isacharcolon}\ {\isachardoublequoteopen}{\isasymAnd}S\ v{\isachardot}\ S\ {\isasymsubseteq}\ carrier\ V{\isasymand}\ lin{\isacharunderscore}indpt\ S\ {\isasymand}\ v{\isasymin}\ carrier\ V\ {\isasymand}\ v{\isasymnotin}S\isanewline
\ \ \ \ {\isasymand}\ v{\isasymnotin}\ span\ S\ {\isasymLongrightarrow}\ \ {\isacharparenleft}lin{\isacharunderscore}indpt\ {\isacharparenleft}S\ {\isasymunion}\ {\isacharbraceleft}v{\isacharbraceright}{\isacharparenright}{\isacharparenright}{\isachardoublequoteclose}\ \isacommand{by}\isamarkupfalse%
\ auto\isanewline
\ \ \isacommand{from}\isamarkupfalse%
\ {\isadigit{1}}\ {\isadigit{4}}\ {\isadigit{5}}\ \isacommand{have}\isamarkupfalse%
\ {\isadigit{7}}{\isacharcolon}\ {\isachardoublequoteopen}lin{\isacharunderscore}indpt\ {\isacharparenleft}A\ {\isasymunion}\ {\isacharbraceleft}v{\isacharbraceright}{\isacharparenright}{\isachardoublequoteclose}\ \isacommand{apply}\isamarkupfalse%
\ {\isacharparenleft}intro\ {\isadigit{6}}{\isacharparenright}\ \isacommand{by}\isamarkupfalse%
\ auto\isanewline
\isanewline
\ \ \isacommand{have}\isamarkupfalse%
\ {\isadigit{9}}{\isacharcolon}\ {\isachardoublequoteopen}{\isasymnot}{\isacharparenleft}maximal\ A\ {\isacharparenleft}{\isasymlambda}S{\isachardot}\ S{\isasymsubseteq}carrier\ V\ {\isasymand}\ lin{\isacharunderscore}indpt\ S{\isacharparenright}{\isacharparenright}{\isachardoublequoteclose}\isanewline
\ \ \isacommand{proof}\isamarkupfalse%
\ {\isacharminus}\ \isanewline
\ \ \ \ \isacommand{from}\isamarkupfalse%
\ {\isadigit{1}}\ {\isadigit{4}}\ {\isadigit{5}}\ {\isadigit{7}}\ \isacommand{have}\isamarkupfalse%
\ {\isadigit{8}}{\isacharcolon}\ {\isachardoublequoteopen}{\isacharparenleft}{\isasymexists}B{\isachardot}\ A\ {\isasymsubseteq}\ B\ \ {\isasymand}\ B\ {\isasymsubseteq}\ carrier\ V\ {\isasymand}\ lin{\isacharunderscore}indpt\ B\ {\isasymand}\ B\ {\isasymnoteq}\ A{\isacharparenright}{\isachardoublequoteclose}\isanewline
\ \ \ \ \ \ \isacommand{apply}\isamarkupfalse%
\ {\isacharparenleft}rule{\isacharunderscore}tac\ x{\isacharequal}{\isachardoublequoteopen}A{\isasymunion}{\isacharbraceleft}v{\isacharbraceright}{\isachardoublequoteclose}\ \isakeyword{in}\ exI{\isacharparenright}\ \isanewline
\ \ \ \ \ \ \isacommand{by}\isamarkupfalse%
\ auto\ \ \ \ \isanewline
\ \ \ \ \isacommand{from}\isamarkupfalse%
\ {\isadigit{8}}\ \isacommand{show}\isamarkupfalse%
\ {\isacharquery}thesis\ \isanewline
\ \ \ \ \ \ \isacommand{apply}\isamarkupfalse%
\ {\isacharparenleft}unfold\ maximal{\isacharunderscore}def{\isacharparenright}\ \isanewline
\ \ \ \ \ \ \isacommand{by}\isamarkupfalse%
\ simp\isanewline
\ \ \isacommand{qed}\isamarkupfalse%
\isanewline
\ \ \isacommand{from}\isamarkupfalse%
\ h{\isadigit{1}}\ {\isadigit{9}}\ \isacommand{show}\isamarkupfalse%
\ False\ \isacommand{by}\isamarkupfalse%
\ auto\isanewline
\isacommand{qed}\isamarkupfalse%
%
\endisatagproof
{\isafoldproof}%
%
\isadelimproof
%
\endisadelimproof
%
\begin{isamarkuptext}%
A minimal generating set is linearly independent.%
\end{isamarkuptext}%
\isamarkuptrue%
\isacommand{lemma}\isamarkupfalse%
\ {\isacharparenleft}\isakeyword{in}\ vectorspace{\isacharparenright}\ min{\isacharunderscore}gen{\isacharunderscore}is{\isacharunderscore}li{\isacharcolon}\isanewline
\ \ \isakeyword{fixes}\ A\isanewline
\ \ \isakeyword{assumes}\ h{\isadigit{1}}{\isacharcolon}\ {\isachardoublequoteopen}minimal\ A\ {\isacharparenleft}{\isasymlambda}S{\isachardot}\ S{\isasymsubseteq}carrier\ V\ {\isasymand}\ gen{\isacharunderscore}set\ S{\isacharparenright}{\isachardoublequoteclose}\isanewline
\ \ \isakeyword{shows}\ {\isachardoublequoteopen}lin{\isacharunderscore}indpt\ A{\isachardoublequoteclose}\isanewline
%
\isadelimproof
%
\endisadelimproof
%
\isatagproof
\isacommand{proof}\isamarkupfalse%
\ {\isacharparenleft}rule\ ccontr{\isacharparenright}\isanewline
\ \ \isacommand{assume}\isamarkupfalse%
\ {\isadigit{0}}{\isacharcolon}\ {\isachardoublequoteopen}{\isasymnot}lin{\isacharunderscore}indpt\ A{\isachardoublequoteclose}\isanewline
\ \ \isacommand{from}\isamarkupfalse%
\ h{\isadigit{1}}\ \isacommand{have}\isamarkupfalse%
\ {\isadigit{1}}{\isacharcolon}\ {\isachardoublequoteopen}\ A{\isasymsubseteq}\ carrier\ V\ {\isasymand}\ gen{\isacharunderscore}set\ A{\isachardoublequoteclose}\ \isacommand{by}\isamarkupfalse%
\ {\isacharparenleft}unfold\ minimal{\isacharunderscore}def{\isacharcomma}\ auto{\isacharparenright}\isanewline
\ \ \isacommand{from}\isamarkupfalse%
\ {\isadigit{1}}\ \isacommand{have}\isamarkupfalse%
\ {\isadigit{2}}{\isacharcolon}\ {\isachardoublequoteopen}span\ A\ {\isacharequal}\ carrier\ V{\isachardoublequoteclose}\ \isacommand{by}\isamarkupfalse%
\ auto\isanewline
\ \ \isacommand{from}\isamarkupfalse%
\ {\isadigit{0}}\ {\isadigit{1}}\ \isacommand{obtain}\isamarkupfalse%
\ a\ v\ A{\isacharprime}\ \isakeyword{where}\ \isanewline
\ \ \ \ {\isadigit{3}}{\isacharcolon}\ {\isachardoublequoteopen}finite\ A{\isacharprime}\ {\isasymand}\ A{\isacharprime}{\isasymsubseteq}A\ {\isasymand}\ a\ {\isasymin}\ A{\isacharprime}\ {\isasymrightarrow}\ carrier\ K\ {\isasymand}\ LinearCombinations{\isachardot}module{\isachardot}lincomb\ V\ a\ A{\isacharprime}\ {\isacharequal}\ {\isasymzero}\isactrlbsub V\isactrlesub \ {\isasymand}\ v\ {\isasymin}\ A{\isacharprime}\ {\isasymand}\ a\ v\ {\isasymnoteq}\ {\isasymzero}\isactrlbsub K\isactrlesub {\isachardoublequoteclose}\ \isanewline
\ \ \ \ \isacommand{by}\isamarkupfalse%
\ {\isacharparenleft}unfold\ lin{\isacharunderscore}dep{\isacharunderscore}def{\isacharcomma}\ auto{\isacharparenright}\isanewline
\ \ \isacommand{have}\isamarkupfalse%
\ {\isadigit{4}}{\isacharcolon}\ {\isachardoublequoteopen}gen{\isacharunderscore}set\ {\isacharparenleft}A{\isacharminus}{\isacharbraceleft}v{\isacharbraceright}{\isacharparenright}{\isachardoublequoteclose}\isanewline
\ \ \isacommand{proof}\isamarkupfalse%
\ {\isacharminus}\ \isanewline
\ \ \ \ \isacommand{from}\isamarkupfalse%
\ {\isadigit{1}}\ {\isadigit{3}}\ \isacommand{have}\isamarkupfalse%
\ {\isadigit{5}}{\isacharcolon}\ {\isachardoublequoteopen}v{\isasymin}span\ {\isacharparenleft}A{\isacharprime}{\isacharminus}{\isacharbraceleft}v{\isacharbraceright}{\isacharparenright}{\isachardoublequoteclose}\ \isanewline
\ \ \ \ \ \ \isacommand{apply}\isamarkupfalse%
\ {\isacharparenleft}intro\ lincomb{\isacharunderscore}isolate{\isacharbrackleft}\isakeyword{where}\ a{\isacharequal}{\isachardoublequoteopen}a{\isachardoublequoteclose}\ \isakeyword{and}\ v{\isacharequal}{\isachardoublequoteopen}v{\isachardoublequoteclose}{\isacharbrackright}{\isacharparenright}\ \isanewline
\ \ \ \ \ \ \ \ \ \ \ \isacommand{by}\isamarkupfalse%
\ auto\isanewline
\ \ \ \ \isacommand{from}\isamarkupfalse%
\ {\isadigit{3}}\ {\isadigit{5}}\ \isacommand{have}\isamarkupfalse%
\ {\isadigit{5}}{\isadigit{1}}{\isacharcolon}\ {\isachardoublequoteopen}v{\isasymin}span\ {\isacharparenleft}A{\isacharminus}{\isacharbraceleft}v{\isacharbraceright}{\isacharparenright}{\isachardoublequoteclose}\isanewline
\ \ \ \ \ \ \isacommand{apply}\isamarkupfalse%
\ {\isacharparenleft}intro\ subsetD{\isacharbrackleft}\isakeyword{where}\ {\isacharquery}A{\isacharequal}{\isachardoublequoteopen}span\ {\isacharparenleft}A{\isacharprime}{\isacharminus}{\isacharbraceleft}v{\isacharbraceright}{\isacharparenright}{\isachardoublequoteclose}\ \isakeyword{and}\ {\isacharquery}B{\isacharequal}{\isachardoublequoteopen}span\ {\isacharparenleft}A{\isacharminus}{\isacharbraceleft}v{\isacharbraceright}{\isacharparenright}{\isachardoublequoteclose}\ \isakeyword{and}\ {\isacharquery}c{\isacharequal}{\isachardoublequoteopen}v{\isachardoublequoteclose}{\isacharbrackright}{\isacharparenright}\isanewline
\ \ \ \ \ \ \ \isacommand{by}\isamarkupfalse%
\ {\isacharparenleft}intro\ span{\isacharunderscore}is{\isacharunderscore}monotone{\isacharcomma}\ auto{\isacharparenright}\isanewline
\ \ \ \ \isacommand{from}\isamarkupfalse%
\ {\isadigit{1}}\ \isacommand{have}\isamarkupfalse%
\ {\isadigit{6}}{\isacharcolon}\ {\isachardoublequoteopen}A{\isasymsubseteq}span\ A{\isachardoublequoteclose}\ \isacommand{apply}\isamarkupfalse%
\ {\isacharparenleft}intro\ in{\isacharunderscore}own{\isacharunderscore}span{\isacharparenright}\ \isacommand{by}\isamarkupfalse%
\ auto\isanewline
\ \ \ \ \isacommand{from}\isamarkupfalse%
\ {\isadigit{1}}\ {\isadigit{5}}{\isadigit{1}}\ \isacommand{have}\isamarkupfalse%
\ {\isadigit{7}}{\isacharcolon}\ {\isachardoublequoteopen}span\ {\isacharparenleft}A{\isacharminus}{\isacharbraceleft}v{\isacharbraceright}{\isacharparenright}\ {\isacharequal}\ span\ {\isacharparenleft}{\isacharparenleft}A{\isacharminus}{\isacharbraceleft}v{\isacharbraceright}{\isacharparenright}{\isasymunion}{\isacharbraceleft}v{\isacharbraceright}{\isacharparenright}{\isachardoublequoteclose}\ \isacommand{apply}\isamarkupfalse%
\ {\isacharparenleft}intro\ already{\isacharunderscore}in{\isacharunderscore}span{\isacharparenright}\ \isacommand{by}\isamarkupfalse%
\ auto\isanewline
\ \ \ \ \isacommand{from}\isamarkupfalse%
\ {\isadigit{3}}\ \isacommand{have}\isamarkupfalse%
\ {\isadigit{8}}{\isacharcolon}\ {\isachardoublequoteopen}A\ {\isacharequal}\ \ {\isacharparenleft}{\isacharparenleft}A{\isacharminus}{\isacharbraceleft}v{\isacharbraceright}{\isacharparenright}{\isasymunion}{\isacharbraceleft}v{\isacharbraceright}{\isacharparenright}{\isachardoublequoteclose}\ \isacommand{by}\isamarkupfalse%
\ auto\isanewline
\ \ \ \ \isacommand{from}\isamarkupfalse%
\ {\isadigit{2}}\ {\isadigit{7}}\ {\isadigit{8}}\ \isacommand{have}\isamarkupfalse%
\ {\isadigit{9}}{\isacharcolon}{\isachardoublequoteopen}span\ {\isacharparenleft}A{\isacharminus}{\isacharbraceleft}v{\isacharbraceright}{\isacharparenright}\ {\isacharequal}\ carrier\ V{\isachardoublequoteclose}\ \isacommand{by}\isamarkupfalse%
\ auto\ \isanewline
\ \ \ \ \isacommand{from}\isamarkupfalse%
\ {\isadigit{9}}\ \isacommand{show}\isamarkupfalse%
\ {\isacharquery}thesis\ \ \isacommand{by}\isamarkupfalse%
\ auto\isanewline
\ \ \isacommand{qed}\isamarkupfalse%
\isanewline
\ \ \isacommand{have}\isamarkupfalse%
\ {\isadigit{1}}{\isadigit{0}}{\isacharcolon}\ {\isachardoublequoteopen}{\isasymnot}{\isacharparenleft}minimal\ A\ {\isacharparenleft}{\isasymlambda}S{\isachardot}\ S{\isasymsubseteq}carrier\ V\ {\isasymand}\ gen{\isacharunderscore}set\ S{\isacharparenright}{\isacharparenright}{\isachardoublequoteclose}\isanewline
\ \ \isacommand{proof}\isamarkupfalse%
\ {\isacharminus}\ \isanewline
\ \ \ \ \isacommand{from}\isamarkupfalse%
\ {\isadigit{1}}\ {\isadigit{3}}\ {\isadigit{4}}\ \isacommand{have}\isamarkupfalse%
\ {\isadigit{1}}{\isadigit{1}}{\isacharcolon}\ {\isachardoublequoteopen}{\isacharparenleft}{\isasymexists}B{\isachardot}\ A\ {\isasymsupseteq}\ B\ {\isasymand}\ B\ {\isasymsubseteq}\ carrier\ V\ {\isasymand}\ gen{\isacharunderscore}set\ B\ {\isasymand}\ B\ {\isasymnoteq}\ A{\isacharparenright}{\isachardoublequoteclose}\isanewline
\ \ \ \ \ \ \isacommand{apply}\isamarkupfalse%
\ {\isacharparenleft}rule{\isacharunderscore}tac\ x{\isacharequal}{\isachardoublequoteopen}A{\isacharminus}{\isacharbraceleft}v{\isacharbraceright}{\isachardoublequoteclose}\ \isakeyword{in}\ exI{\isacharparenright}\ \isanewline
\ \ \ \ \ \ \isacommand{by}\isamarkupfalse%
\ auto\isanewline
\ \ \ \ \isacommand{from}\isamarkupfalse%
\ {\isadigit{1}}{\isadigit{1}}\ \isacommand{show}\isamarkupfalse%
\ {\isacharquery}thesis\ \isanewline
\ \ \ \ \ \ \isacommand{apply}\isamarkupfalse%
\ {\isacharparenleft}unfold\ minimal{\isacharunderscore}def{\isacharparenright}\ \isanewline
\ \ \ \ \ \ \isacommand{by}\isamarkupfalse%
\ auto\isanewline
\ \ \isacommand{qed}\isamarkupfalse%
\isanewline
\ \ \isacommand{from}\isamarkupfalse%
\ h{\isadigit{1}}\ {\isadigit{1}}{\isadigit{0}}\ \isacommand{show}\isamarkupfalse%
\ False\ \isacommand{by}\isamarkupfalse%
\ auto\isanewline
\isacommand{qed}\isamarkupfalse%
%
\endisatagproof
{\isafoldproof}%
%
\isadelimproof
%
\endisadelimproof
%
\begin{isamarkuptext}%
Given that some finite set satisfies $P$, there is a minimal set that satisfies $P$.%
\end{isamarkuptext}%
\isamarkuptrue%
\isacommand{lemma}\isamarkupfalse%
\ minimal{\isacharunderscore}exists{\isacharcolon}\isanewline
\ \ \isakeyword{fixes}\ A\ P\isanewline
\ \ \isakeyword{assumes}\ h{\isadigit{1}}{\isacharcolon}\ {\isachardoublequoteopen}finite\ A{\isachardoublequoteclose}\ \isakeyword{and}\ h{\isadigit{2}}{\isacharcolon}\ {\isachardoublequoteopen}P\ A{\isachardoublequoteclose}\isanewline
\ \ \isakeyword{shows}\ {\isachardoublequoteopen}{\isasymexists}B{\isachardot}\ B{\isasymsubseteq}A\ {\isasymand}\ minimal\ B\ P{\isachardoublequoteclose}\isanewline
%
\isadelimproof
%
\endisadelimproof
%
\isatagproof
\isacommand{using}\isamarkupfalse%
\ h{\isadigit{1}}\ h{\isadigit{2}}\ \isanewline
\isacommand{proof}\isamarkupfalse%
\ {\isacharparenleft}induct\ {\isachardoublequoteopen}card\ A{\isachardoublequoteclose}\ \ arbitrary{\isacharcolon}\ A\ rule{\isacharcolon}\ less{\isacharunderscore}induct{\isacharparenright}\isanewline
\isacommand{case}\isamarkupfalse%
\ {\isacharparenleft}less\ A{\isacharparenright}\isanewline
\ \ \isacommand{show}\isamarkupfalse%
\ {\isacharquery}case\isanewline
\ \ \isacommand{proof}\isamarkupfalse%
\ {\isacharparenleft}cases\ {\isachardoublequoteopen}card\ A\ {\isacharequal}\ {\isadigit{0}}{\isachardoublequoteclose}{\isacharparenright}\ \ \isanewline
\ \ \isacommand{case}\isamarkupfalse%
\ True\isanewline
\ \ \ \ \isacommand{from}\isamarkupfalse%
\ True\ less{\isachardot}hyps\ less{\isachardot}prems\ \isacommand{show}\isamarkupfalse%
\ {\isacharquery}thesis\isanewline
\ \ \ \ \ \ \isacommand{apply}\isamarkupfalse%
\ {\isacharparenleft}rule{\isacharunderscore}tac\ x{\isacharequal}{\isachardoublequoteopen}{\isacharbraceleft}{\isacharbraceright}{\isachardoublequoteclose}\ \isakeyword{in}\ exI{\isacharparenright}\isanewline
\ \ \ \ \ \ \isacommand{apply}\isamarkupfalse%
\ {\isacharparenleft}unfold\ minimal{\isacharunderscore}def{\isacharparenright}\isanewline
\ \ \ \ \ \ \isacommand{by}\isamarkupfalse%
\ \ auto\isanewline
\ \ \isacommand{next}\isamarkupfalse%
\isanewline
\ \ \isacommand{case}\isamarkupfalse%
\ False\isanewline
\ \ \ \ \isacommand{show}\isamarkupfalse%
\ {\isacharquery}thesis\isanewline
\ \ \ \ \isacommand{proof}\isamarkupfalse%
\ {\isacharparenleft}cases\ {\isachardoublequoteopen}minimal\ A\ P{\isachardoublequoteclose}{\isacharparenright}\isanewline
\ \ \ \ \ \ \isacommand{case}\isamarkupfalse%
\ True\isanewline
\ \ \ \ \ \ \ \ \isacommand{then}\isamarkupfalse%
\ \isacommand{show}\isamarkupfalse%
\ {\isacharquery}thesis\ \isanewline
\ \ \ \ \ \ \ \ \ \ \isacommand{apply}\isamarkupfalse%
\ {\isacharparenleft}rule{\isacharunderscore}tac\ x{\isacharequal}{\isachardoublequoteopen}A{\isachardoublequoteclose}\ \isakeyword{in}\ exI{\isacharparenright}\ \isanewline
\ \ \ \ \ \ \ \ \ \ \isacommand{by}\isamarkupfalse%
\ auto\isanewline
\ \ \ \ \ \ \isacommand{next}\isamarkupfalse%
\isanewline
\ \ \ \ \ \ \isacommand{case}\isamarkupfalse%
\ False\isanewline
\ \ \ \ \ \ \ \ \isacommand{have}\isamarkupfalse%
\ {\isadigit{2}}{\isacharcolon}\ {\isachardoublequoteopen}{\isasymnot}minimal\ A\ P{\isachardoublequoteclose}\ \isacommand{by}\isamarkupfalse%
\ fact\isanewline
\ \ \ \ \ \ \ \ \isacommand{from}\isamarkupfalse%
\ less{\isachardot}prems\ {\isadigit{2}}\ \isacommand{have}\isamarkupfalse%
\ {\isadigit{3}}{\isacharcolon}\ {\isachardoublequoteopen}{\isasymexists}B{\isachardot}\ P\ B\ {\isasymand}\ B\ {\isasymsubseteq}\ A\ {\isasymand}\ B{\isasymnoteq}A{\isachardoublequoteclose}\isanewline
\ \ \ \ \ \ \ \ \ \ \isacommand{apply}\isamarkupfalse%
\ {\isacharparenleft}unfold\ minimal{\isacharunderscore}def{\isacharparenright}\ \isanewline
\ \ \ \ \ \ \ \ \ \ \isacommand{by}\isamarkupfalse%
\ auto\isanewline
\ \ \ \ \ \ \ \ \isacommand{from}\isamarkupfalse%
\ {\isadigit{3}}\ \isacommand{obtain}\isamarkupfalse%
\ B\ \isakeyword{where}\ {\isadigit{4}}{\isacharcolon}\ {\isachardoublequoteopen}P\ B\ {\isasymand}\ B\ {\isasymsubset}\ A\ {\isasymand}\ B{\isasymnoteq}A{\isachardoublequoteclose}\ \isacommand{by}\isamarkupfalse%
\ auto\isanewline
\ \ \ \ \ \ \ \ \isacommand{from}\isamarkupfalse%
\ {\isadigit{4}}\ \isacommand{have}\isamarkupfalse%
\ {\isadigit{5}}{\isacharcolon}\ {\isachardoublequoteopen}card\ B\ {\isacharless}\ card\ A{\isachardoublequoteclose}\ \isacommand{by}\isamarkupfalse%
\ {\isacharparenleft}metis\ less{\isachardot}prems{\isacharparenleft}{\isadigit{1}}{\isacharparenright}\ psubset{\isacharunderscore}card{\isacharunderscore}mono{\isacharparenright}\isanewline
\ \ \ \ \ \ \ \ \isacommand{from}\isamarkupfalse%
\ less{\isachardot}hyps\ less{\isachardot}prems\ {\isadigit{3}}\ {\isadigit{4}}\ {\isadigit{5}}\ \isacommand{have}\isamarkupfalse%
\ {\isadigit{6}}{\isacharcolon}\ {\isachardoublequoteopen}{\isasymexists}C{\isasymsubseteq}B{\isachardot}\ minimal\ C\ P{\isachardoublequoteclose}\ \isanewline
\ \ \ \ \ \ \ \ \ \ \isacommand{apply}\isamarkupfalse%
\ {\isacharparenleft}intro\ less{\isachardot}hyps{\isacharparenright}\ \isanewline
\ \ \ \ \ \ \ \ \ \ \ \ \isacommand{apply}\isamarkupfalse%
\ auto\isanewline
\ \ \ \ \ \ \ \ \ \ \isacommand{by}\isamarkupfalse%
\ {\isacharparenleft}metis\ rev{\isacharunderscore}finite{\isacharunderscore}subset{\isacharparenright}\isanewline
\ \ \ \ \ \ \ \ \isacommand{from}\isamarkupfalse%
\ {\isadigit{6}}\ \isacommand{obtain}\isamarkupfalse%
\ C\ \isakeyword{where}\ {\isadigit{7}}{\isacharcolon}\ {\isachardoublequoteopen}C{\isasymsubseteq}B\ {\isasymand}\ minimal\ C\ P{\isachardoublequoteclose}\ \isacommand{by}\isamarkupfalse%
\ auto\isanewline
\ \ \ \ \ \ \ \ \isacommand{from}\isamarkupfalse%
\ {\isadigit{4}}\ {\isadigit{7}}\ \isacommand{show}\isamarkupfalse%
\ {\isacharquery}thesis\ \isanewline
\ \ \ \ \ \ \ \ \ \ \isacommand{apply}\isamarkupfalse%
\ {\isacharparenleft}rule{\isacharunderscore}tac\ x{\isacharequal}{\isachardoublequoteopen}C{\isachardoublequoteclose}\ \isakeyword{in}\ exI{\isacharparenright}\ \isanewline
\ \ \ \ \ \ \ \ \ \ \isacommand{apply}\isamarkupfalse%
\ {\isacharparenleft}unfold\ minimal{\isacharunderscore}def{\isacharparenright}\ \isanewline
\ \ \ \ \ \ \ \ \ \ \isacommand{by}\isamarkupfalse%
\ auto\isanewline
\ \ \ \ \ \isacommand{qed}\isamarkupfalse%
\isanewline
\ \ \ \isacommand{qed}\isamarkupfalse%
\isanewline
\isacommand{qed}\isamarkupfalse%
%
\endisatagproof
{\isafoldproof}%
%
\isadelimproof
%
\endisadelimproof
%
\begin{isamarkuptext}%
If $V$ is finite-dimensional, then any linearly independent set is finite.%
\end{isamarkuptext}%
\isamarkuptrue%
\isacommand{lemma}\isamarkupfalse%
\ {\isacharparenleft}\isakeyword{in}\ vectorspace{\isacharparenright}\ fin{\isacharunderscore}dim{\isacharunderscore}li{\isacharunderscore}fin{\isacharcolon}\isanewline
\ \ \isakeyword{assumes}\ fd{\isacharcolon}\ {\isachardoublequoteopen}fin{\isacharunderscore}dim{\isachardoublequoteclose}\ \isakeyword{and}\ li{\isacharcolon}\ {\isachardoublequoteopen}lin{\isacharunderscore}indpt\ A{\isachardoublequoteclose}\ \isakeyword{and}\ inC{\isacharcolon}\ {\isachardoublequoteopen}A{\isasymsubseteq}carrier\ V{\isachardoublequoteclose}\isanewline
\ \ \isakeyword{shows}\ fin{\isacharcolon}\ {\isachardoublequoteopen}finite\ A{\isachardoublequoteclose}\isanewline
%
\isadelimproof
%
\endisadelimproof
%
\isatagproof
\isacommand{proof}\isamarkupfalse%
\ {\isacharparenleft}rule\ ccontr{\isacharparenright}\isanewline
\ \ \isacommand{assume}\isamarkupfalse%
\ A{\isacharcolon}\ {\isachardoublequoteopen}{\isasymnot}finite\ A{\isachardoublequoteclose}\isanewline
\ \ \isacommand{from}\isamarkupfalse%
\ fd\ \isacommand{obtain}\isamarkupfalse%
\ C\ \isakeyword{where}\ C{\isacharcolon}\ {\isachardoublequoteopen}finite\ C\ {\isasymand}\ C{\isasymsubseteq}carrier\ V\ {\isasymand}\ gen{\isacharunderscore}set\ C{\isachardoublequoteclose}\ \isacommand{by}\isamarkupfalse%
\ {\isacharparenleft}unfold\ fin{\isacharunderscore}dim{\isacharunderscore}def{\isacharcomma}\ auto{\isacharparenright}\isanewline
\ \ \isacommand{from}\isamarkupfalse%
\ A\ \isacommand{obtain}\isamarkupfalse%
\ B\ \isakeyword{where}\ B{\isacharcolon}\ {\isachardoublequoteopen}B{\isasymsubseteq}A{\isasymand}\ finite\ B\ {\isasymand}\ card\ B\ {\isacharequal}\ card\ C\ {\isacharplus}\ {\isadigit{1}}{\isachardoublequoteclose}\isanewline
\ \ \ \ \isacommand{by}\isamarkupfalse%
\ {\isacharparenleft}metis\ infinite{\isacharunderscore}arbitrarily{\isacharunderscore}large{\isacharparenright}\ \isanewline
\ \ \isacommand{from}\isamarkupfalse%
\ B\ li\ \isacommand{have}\isamarkupfalse%
\ liB{\isacharcolon}\ {\isachardoublequoteopen}lin{\isacharunderscore}indpt\ B{\isachardoublequoteclose}\ \isanewline
\ \ \ \ \isacommand{by}\isamarkupfalse%
\ {\isacharparenleft}intro\ subset{\isacharunderscore}li{\isacharunderscore}is{\isacharunderscore}li{\isacharbrackleft}\isakeyword{where}\ {\isacharquery}A{\isacharequal}{\isachardoublequoteopen}A{\isachardoublequoteclose}\ \isakeyword{and}\ {\isacharquery}B{\isacharequal}{\isachardoublequoteopen}B{\isachardoublequoteclose}{\isacharbrackright}{\isacharcomma}\ auto{\isacharparenright}\isanewline
\ \ \isacommand{from}\isamarkupfalse%
\ B\ C\ liB\ inC\ \isacommand{have}\isamarkupfalse%
\ {\isachardoublequoteopen}card\ B\ {\isasymle}\ card\ C{\isachardoublequoteclose}\ \isacommand{by}\isamarkupfalse%
\ {\isacharparenleft}intro\ li{\isacharunderscore}smaller{\isacharunderscore}than{\isacharunderscore}gen{\isacharcomma}\ auto{\isacharparenright}\ \isanewline
\ \ \isacommand{from}\isamarkupfalse%
\ this\ B\ \isacommand{show}\isamarkupfalse%
\ False\ \isacommand{by}\isamarkupfalse%
\ auto\isanewline
\isacommand{qed}\isamarkupfalse%
%
\endisatagproof
{\isafoldproof}%
%
\isadelimproof
%
\endisadelimproof
%
\begin{isamarkuptext}%
If $V$ is finite-dimensional (has a finite generating set), then a finite basis exists.%
\end{isamarkuptext}%
\isamarkuptrue%
\isacommand{lemma}\isamarkupfalse%
\ {\isacharparenleft}\isakeyword{in}\ vectorspace{\isacharparenright}\ finite{\isacharunderscore}basis{\isacharunderscore}exists{\isacharcolon}\isanewline
\ \ \isakeyword{assumes}\ h{\isadigit{1}}{\isacharcolon}\ {\isachardoublequoteopen}fin{\isacharunderscore}dim{\isachardoublequoteclose}\isanewline
\ \ \isakeyword{shows}\ {\isachardoublequoteopen}{\isasymexists}{\isasymbeta}{\isachardot}\ finite\ {\isasymbeta}\ {\isasymand}\ basis\ {\isasymbeta}{\isachardoublequoteclose}\isanewline
%
\isadelimproof
%
\endisadelimproof
%
\isatagproof
\isacommand{proof}\isamarkupfalse%
\ {\isacharminus}\isanewline
\ \ \isacommand{from}\isamarkupfalse%
\ h{\isadigit{1}}\ \isacommand{obtain}\isamarkupfalse%
\ A\ \isakeyword{where}\ {\isadigit{1}}{\isacharcolon}\ {\isachardoublequoteopen}finite\ A\ {\isasymand}\ A{\isasymsubseteq}carrier\ V\ {\isasymand}\ gen{\isacharunderscore}set\ A{\isachardoublequoteclose}\ \isacommand{by}\isamarkupfalse%
\ {\isacharparenleft}metis\ fin{\isacharunderscore}dim{\isacharunderscore}def{\isacharparenright}\isanewline
\ \ \isacommand{hence}\isamarkupfalse%
\ {\isadigit{2}}{\isacharcolon}\ {\isachardoublequoteopen}{\isasymexists}{\isasymbeta}{\isachardot}\ {\isasymbeta}{\isasymsubseteq}A\ {\isasymand}\ minimal\ {\isasymbeta}\ {\isacharparenleft}{\isasymlambda}S{\isachardot}\ S{\isasymsubseteq}carrier\ V\ {\isasymand}\ gen{\isacharunderscore}set\ S{\isacharparenright}{\isachardoublequoteclose}\ \isanewline
\ \ \ \ \isacommand{apply}\isamarkupfalse%
\ {\isacharparenleft}intro\ minimal{\isacharunderscore}exists{\isacharparenright}\ \isanewline
\ \ \ \ \ \isacommand{by}\isamarkupfalse%
\ auto\isanewline
\ \ \isacommand{then}\isamarkupfalse%
\ \isacommand{obtain}\isamarkupfalse%
\ {\isasymbeta}\ \isakeyword{where}\ {\isadigit{3}}{\isacharcolon}\ {\isachardoublequoteopen}{\isasymbeta}{\isasymsubseteq}A\ {\isasymand}\ minimal\ {\isasymbeta}\ {\isacharparenleft}{\isasymlambda}S{\isachardot}\ S{\isasymsubseteq}carrier\ V\ {\isasymand}\ gen{\isacharunderscore}set\ S{\isacharparenright}{\isachardoublequoteclose}\ \isacommand{by}\isamarkupfalse%
\ auto\isanewline
\ \ \isacommand{hence}\isamarkupfalse%
\ {\isadigit{4}}{\isacharcolon}\ {\isachardoublequoteopen}lin{\isacharunderscore}indpt\ {\isasymbeta}{\isachardoublequoteclose}\ \isacommand{apply}\isamarkupfalse%
\ {\isacharparenleft}intro\ min{\isacharunderscore}gen{\isacharunderscore}is{\isacharunderscore}li{\isacharparenright}\ \isacommand{by}\isamarkupfalse%
\ auto\isanewline
\ \ \isacommand{moreover}\isamarkupfalse%
\ \isacommand{from}\isamarkupfalse%
\ {\isadigit{3}}\ \isacommand{have}\isamarkupfalse%
\ {\isadigit{5}}{\isacharcolon}\ {\isachardoublequoteopen}gen{\isacharunderscore}set\ {\isasymbeta}\ {\isasymand}\ {\isasymbeta}{\isasymsubseteq}carrier\ V{\isachardoublequoteclose}\ \isacommand{apply}\isamarkupfalse%
\ {\isacharparenleft}unfold\ minimal{\isacharunderscore}def{\isacharparenright}\ \isacommand{by}\isamarkupfalse%
\ auto\isanewline
\ \ \isacommand{moreover}\isamarkupfalse%
\ \isacommand{from}\isamarkupfalse%
\ {\isadigit{1}}\ {\isadigit{3}}\ \isacommand{have}\isamarkupfalse%
\ {\isadigit{6}}{\isacharcolon}\ {\isachardoublequoteopen}finite\ {\isasymbeta}{\isachardoublequoteclose}\ \isacommand{by}\isamarkupfalse%
\ {\isacharparenleft}auto\ simp\ add{\isacharcolon}\ finite{\isacharunderscore}subset{\isacharparenright}\isanewline
\ \ \isacommand{ultimately}\isamarkupfalse%
\ \isacommand{show}\isamarkupfalse%
\ {\isacharquery}thesis\ \isacommand{apply}\isamarkupfalse%
\ {\isacharparenleft}unfold\ basis{\isacharunderscore}def{\isacharparenright}\ \isacommand{by}\isamarkupfalse%
\ auto\isanewline
\isacommand{qed}\isamarkupfalse%
%
\endisatagproof
{\isafoldproof}%
%
\isadelimproof
%
\endisadelimproof
%
\begin{isamarkuptext}%
The proof is as follows.
\begin{enumerate}
\item Because $V$ is finite-dimensional, there is a finite generating set 
(we took this as our definition of finite-dimensional).
\item Hence, there is a minimal $\beta \subseteq A$ such that $\beta$ generates $V$.
\item $\beta$ is linearly independent because a minimal generating set is linearly independent.
\end{enumerate}
Finally, $\beta$ is a basis because it is both generating and linearly independent.%
\end{isamarkuptext}%
\isamarkuptrue%
%
\begin{isamarkuptext}%
Any linearly independent set has cardinality at most equal to the dimension.%
\end{isamarkuptext}%
\isamarkuptrue%
\isacommand{lemma}\isamarkupfalse%
\ {\isacharparenleft}\isakeyword{in}\ vectorspace{\isacharparenright}\ li{\isacharunderscore}le{\isacharunderscore}dim{\isacharcolon}\ \isanewline
\ \ \isakeyword{fixes}\ A\isanewline
\ \ \isakeyword{assumes}\ fd{\isacharcolon}\ {\isachardoublequoteopen}fin{\isacharunderscore}dim{\isachardoublequoteclose}\ \isakeyword{and}\ c{\isacharcolon}\ {\isachardoublequoteopen}A{\isasymsubseteq}carrier\ V{\isachardoublequoteclose}\ \isakeyword{and}\ l{\isacharcolon}\ {\isachardoublequoteopen}lin{\isacharunderscore}indpt\ A{\isachardoublequoteclose}\isanewline
\ \ \isakeyword{shows}\ {\isachardoublequoteopen}finite\ A{\isachardoublequoteclose}\ {\isachardoublequoteopen}card\ A\ {\isasymle}\ dim{\isachardoublequoteclose}\isanewline
%
\isadelimproof
%
\endisadelimproof
%
\isatagproof
\isacommand{proof}\isamarkupfalse%
\ \ {\isacharminus}\isanewline
\ \ \isacommand{from}\isamarkupfalse%
\ fd\ c\ l\ \isacommand{show}\isamarkupfalse%
\ fa{\isacharcolon}\ {\isachardoublequoteopen}finite\ A{\isachardoublequoteclose}\ \isacommand{by}\isamarkupfalse%
\ {\isacharparenleft}intro\ fin{\isacharunderscore}dim{\isacharunderscore}li{\isacharunderscore}fin{\isacharcomma}\ auto{\isacharparenright}\isanewline
\ \ \isacommand{from}\isamarkupfalse%
\ fd\ \isacommand{obtain}\isamarkupfalse%
\ {\isasymbeta}\ \isakeyword{where}\ {\isadigit{1}}{\isacharcolon}\ {\isachardoublequoteopen}finite\ {\isasymbeta}\ {\isasymand}\ basis\ {\isasymbeta}{\isachardoublequoteclose}\ \isanewline
\ \ \ \ \isacommand{by}\isamarkupfalse%
\ {\isacharparenleft}metis\ finite{\isacharunderscore}basis{\isacharunderscore}exists{\isacharparenright}\isanewline
\ \ \isacommand{from}\isamarkupfalse%
\ assms\ fa\ {\isadigit{1}}\ \isacommand{have}\isamarkupfalse%
\ {\isadigit{2}}{\isacharcolon}\ {\isachardoublequoteopen}card\ A\ {\isasymle}\ card\ {\isasymbeta}{\isachardoublequoteclose}\ \isanewline
\ \ \ \ \isacommand{apply}\isamarkupfalse%
\ {\isacharparenleft}intro\ li{\isacharunderscore}smaller{\isacharunderscore}than{\isacharunderscore}gen{\isacharcomma}\ auto{\isacharparenright}\ \isanewline
\ \ \ \ \ \ \isacommand{by}\isamarkupfalse%
\ {\isacharparenleft}unfold\ basis{\isacharunderscore}def{\isacharcomma}\ auto{\isacharparenright}\isanewline
\ \ \isacommand{from}\isamarkupfalse%
\ assms\ {\isadigit{1}}\ \isacommand{have}\isamarkupfalse%
\ {\isadigit{3}}{\isacharcolon}\ {\isachardoublequoteopen}dim\ {\isacharequal}\ card\ {\isasymbeta}{\isachardoublequoteclose}\ \isacommand{by}\isamarkupfalse%
\ {\isacharparenleft}intro\ dim{\isacharunderscore}basis{\isacharcomma}\ auto{\isacharparenright}\isanewline
\ \ \isacommand{from}\isamarkupfalse%
\ {\isadigit{2}}\ {\isadigit{3}}\ \isacommand{show}\isamarkupfalse%
\ {\isachardoublequoteopen}card\ A\ {\isasymle}\ dim{\isachardoublequoteclose}\ \isacommand{by}\isamarkupfalse%
\ auto\isanewline
\isacommand{qed}\isamarkupfalse%
%
\endisatagproof
{\isafoldproof}%
%
\isadelimproof
%
\endisadelimproof
%
\begin{isamarkuptext}%
Any generating set has cardinality at least equal to the dimension.%
\end{isamarkuptext}%
\isamarkuptrue%
\isacommand{lemma}\isamarkupfalse%
\ {\isacharparenleft}\isakeyword{in}\ vectorspace{\isacharparenright}\ gen{\isacharunderscore}ge{\isacharunderscore}dim{\isacharcolon}\ \isanewline
\ \ \isakeyword{fixes}\ A\isanewline
\ \ \isakeyword{assumes}\ fa{\isacharcolon}\ {\isachardoublequoteopen}finite\ A{\isachardoublequoteclose}\ \isakeyword{and}\ c{\isacharcolon}\ {\isachardoublequoteopen}A{\isasymsubseteq}carrier\ V{\isachardoublequoteclose}\ \isakeyword{and}\ l{\isacharcolon}\ {\isachardoublequoteopen}gen{\isacharunderscore}set\ A{\isachardoublequoteclose}\isanewline
\ \ \isakeyword{shows}\ {\isachardoublequoteopen}card\ A\ {\isasymge}\ dim{\isachardoublequoteclose}\isanewline
%
\isadelimproof
%
\endisadelimproof
%
\isatagproof
\isacommand{proof}\isamarkupfalse%
\ \ {\isacharminus}\isanewline
\ \ \isacommand{from}\isamarkupfalse%
\ assms\ \isacommand{have}\isamarkupfalse%
\ fd{\isacharcolon}\ {\isachardoublequoteopen}fin{\isacharunderscore}dim{\isachardoublequoteclose}\ \isacommand{by}\isamarkupfalse%
\ {\isacharparenleft}unfold\ fin{\isacharunderscore}dim{\isacharunderscore}def{\isacharcomma}\ auto{\isacharparenright}\isanewline
\ \ \isacommand{from}\isamarkupfalse%
\ fd\ \isacommand{obtain}\isamarkupfalse%
\ {\isasymbeta}\ \isakeyword{where}\ {\isadigit{1}}{\isacharcolon}\ {\isachardoublequoteopen}finite\ {\isasymbeta}\ {\isasymand}\ basis\ {\isasymbeta}{\isachardoublequoteclose}\ \isacommand{by}\isamarkupfalse%
\ {\isacharparenleft}metis\ finite{\isacharunderscore}basis{\isacharunderscore}exists{\isacharparenright}\isanewline
\ \ \isacommand{from}\isamarkupfalse%
\ assms\ {\isadigit{1}}\ \isacommand{have}\isamarkupfalse%
\ {\isadigit{2}}{\isacharcolon}\ {\isachardoublequoteopen}card\ A\ {\isasymge}\ card\ {\isasymbeta}{\isachardoublequoteclose}\ \isanewline
\ \ \ \ \isacommand{apply}\isamarkupfalse%
\ {\isacharparenleft}intro\ li{\isacharunderscore}smaller{\isacharunderscore}than{\isacharunderscore}gen{\isacharcomma}\ auto{\isacharparenright}\ \isanewline
\ \ \ \ \ \isacommand{by}\isamarkupfalse%
\ {\isacharparenleft}unfold\ basis{\isacharunderscore}def{\isacharcomma}\ auto{\isacharparenright}\isanewline
\ \ \isacommand{from}\isamarkupfalse%
\ assms\ {\isadigit{1}}\ \isacommand{have}\isamarkupfalse%
\ {\isadigit{3}}{\isacharcolon}\ {\isachardoublequoteopen}dim\ {\isacharequal}\ card\ {\isasymbeta}{\isachardoublequoteclose}\ \isacommand{by}\isamarkupfalse%
\ {\isacharparenleft}intro\ dim{\isacharunderscore}basis{\isacharcomma}\ auto{\isacharparenright}\isanewline
\ \ \isacommand{from}\isamarkupfalse%
\ {\isadigit{2}}\ {\isadigit{3}}\ \isacommand{show}\isamarkupfalse%
\ {\isacharquery}thesis\ \isacommand{by}\isamarkupfalse%
\ auto\isanewline
\isacommand{qed}\isamarkupfalse%
%
\endisatagproof
{\isafoldproof}%
%
\isadelimproof
%
\endisadelimproof
%
\begin{isamarkuptext}%
If there is an upper bound on the cardinality of sets satisfying $P$, then there is a maximal
set satisfying $P$.%
\end{isamarkuptext}%
\isamarkuptrue%
\isacommand{lemma}\isamarkupfalse%
\ maximal{\isacharunderscore}exists{\isacharcolon}\isanewline
\ \ \isakeyword{fixes}\ P\ B\ N\isanewline
\ \ \isakeyword{assumes}\ maxc{\isacharcolon}\ {\isachardoublequoteopen}{\isasymAnd}A{\isachardot}\ P\ A\ {\isasymLongrightarrow}\ finite\ A\ {\isasymand}\ {\isacharparenleft}card\ A\ {\isasymle}N{\isacharparenright}{\isachardoublequoteclose}\ \isakeyword{and}\ b{\isacharcolon}\ {\isachardoublequoteopen}P\ B{\isachardoublequoteclose}\isanewline
\ \ \isakeyword{shows}\ {\isachardoublequoteopen}{\isasymexists}A{\isachardot}\ finite\ A\ {\isasymand}\ maximal\ A\ P{\isachardoublequoteclose}\isanewline
%
\isadelimproof
%
\endisadelimproof
%
\isatagproof
\isacommand{proof}\isamarkupfalse%
\ {\isacharminus}\isanewline
\isanewline
\ \ \isacommand{let}\isamarkupfalse%
\ {\isacharquery}S{\isacharequal}{\isachardoublequoteopen}{\isacharbraceleft}card\ A{\isacharbar}\ A{\isachardot}\ P\ A{\isacharbraceright}{\isachardoublequoteclose}\isanewline
\ \ \isacommand{let}\isamarkupfalse%
\ {\isacharquery}n{\isacharequal}{\isachardoublequoteopen}Max\ {\isacharquery}S{\isachardoublequoteclose}\isanewline
\ \ \isacommand{from}\isamarkupfalse%
\ maxc\ \isacommand{have}\isamarkupfalse%
\ {\isadigit{1}}{\isacharcolon}{\isachardoublequoteopen}finite\ {\isacharquery}S{\isachardoublequoteclose}\isanewline
\ \ \ \ \isacommand{apply}\isamarkupfalse%
\ {\isacharparenleft}simp\ add{\isacharcolon}\ finite{\isacharunderscore}nat{\isacharunderscore}set{\isacharunderscore}iff{\isacharunderscore}bounded{\isacharunderscore}le{\isacharparenright}\ \isacommand{by}\isamarkupfalse%
\ auto\isanewline
\ \ \isacommand{from}\isamarkupfalse%
\ {\isadigit{1}}\ \isacommand{have}\isamarkupfalse%
\ {\isadigit{2}}{\isacharcolon}\ {\isachardoublequoteopen}{\isacharquery}n{\isasymin}{\isacharquery}S{\isachardoublequoteclose}\isanewline
\ \ \ \ \isacommand{by}\isamarkupfalse%
\ {\isacharparenleft}metis\ {\isacharparenleft}mono{\isacharunderscore}tags{\isacharcomma}\ lifting{\isacharparenright}\ Collect{\isacharunderscore}empty{\isacharunderscore}eq\ Max{\isacharunderscore}in\ b{\isacharparenright}\ \isanewline
\ \ \isacommand{from}\isamarkupfalse%
\ assms\ {\isadigit{2}}\ \isacommand{have}\isamarkupfalse%
\ {\isadigit{3}}{\isacharcolon}\ {\isachardoublequoteopen}{\isasymexists}A{\isachardot}\ P\ A\ {\isasymand}\ finite\ A\ {\isasymand}\ card\ A\ {\isacharequal}\ {\isacharquery}n{\isachardoublequoteclose}\ \isanewline
\ \ \ \ \isacommand{by}\isamarkupfalse%
\ auto\isanewline
\ \ \isacommand{from}\isamarkupfalse%
\ {\isadigit{3}}\ \isacommand{obtain}\isamarkupfalse%
\ A\ \isakeyword{where}\ {\isadigit{4}}{\isacharcolon}\ {\isachardoublequoteopen}P\ A\ {\isasymand}\ finite\ A\ {\isasymand}\ card\ A\ {\isacharequal}\ {\isacharquery}n{\isachardoublequoteclose}\ \isacommand{by}\isamarkupfalse%
\ auto\isanewline
\ \ \isacommand{from}\isamarkupfalse%
\ {\isadigit{1}}\ maxc\ \isacommand{have}\isamarkupfalse%
\ {\isadigit{5}}{\isacharcolon}\ {\isachardoublequoteopen}{\isasymAnd}A{\isachardot}\ P\ A\ {\isasymLongrightarrow}\ finite\ A\ {\isasymand}\ {\isacharparenleft}card\ A\ {\isasymle}{\isacharquery}n{\isacharparenright}{\isachardoublequoteclose}\isanewline
\ \ \ \ \isacommand{by}\isamarkupfalse%
\ {\isacharparenleft}metis\ {\isacharparenleft}mono{\isacharunderscore}tags{\isacharcomma}\ lifting{\isacharparenright}\ Max{\isachardot}coboundedI\ mem{\isacharunderscore}Collect{\isacharunderscore}eq{\isacharparenright}\ \isanewline
\ \ \isacommand{from}\isamarkupfalse%
\ {\isadigit{4}}\ {\isadigit{5}}\ \isacommand{have}\isamarkupfalse%
\ {\isadigit{6}}{\isacharcolon}\ {\isachardoublequoteopen}maximal\ A\ P{\isachardoublequoteclose}\isanewline
\ \ \ \ \isacommand{apply}\isamarkupfalse%
\ {\isacharparenleft}unfold\ maximal{\isacharunderscore}def{\isacharparenright}\isanewline
\ \ \ \ \isacommand{by}\isamarkupfalse%
\ {\isacharparenleft}metis\ card{\isacharunderscore}seteq{\isacharparenright}\isanewline
\ \ \isacommand{from}\isamarkupfalse%
\ {\isadigit{4}}\ {\isadigit{6}}\ \isacommand{show}\isamarkupfalse%
\ {\isacharquery}thesis\ \isacommand{by}\isamarkupfalse%
\ auto\isanewline
\isacommand{qed}\isamarkupfalse%
%
\endisatagproof
{\isafoldproof}%
%
\isadelimproof
%
\endisadelimproof
%
\begin{isamarkuptext}%
Any maximal linearly independent set is a basis.%
\end{isamarkuptext}%
\isamarkuptrue%
\isacommand{lemma}\isamarkupfalse%
\ {\isacharparenleft}\isakeyword{in}\ vectorspace{\isacharparenright}\ max{\isacharunderscore}li{\isacharunderscore}is{\isacharunderscore}basis{\isacharcolon}\isanewline
\ \ \isakeyword{fixes}\ A\isanewline
\ \ \isakeyword{assumes}\ h{\isadigit{1}}{\isacharcolon}\ {\isachardoublequoteopen}maximal\ A\ {\isacharparenleft}{\isasymlambda}S{\isachardot}\ S{\isasymsubseteq}carrier\ V\ {\isasymand}\ lin{\isacharunderscore}indpt\ S{\isacharparenright}{\isachardoublequoteclose}\isanewline
\ \ \isakeyword{shows}\ {\isachardoublequoteopen}basis\ A{\isachardoublequoteclose}\isanewline
%
\isadelimproof
%
\endisadelimproof
%
\isatagproof
\isacommand{proof}\isamarkupfalse%
\ {\isacharminus}\ \isanewline
\ \ \isacommand{from}\isamarkupfalse%
\ h{\isadigit{1}}\ \isacommand{have}\isamarkupfalse%
\ {\isadigit{1}}{\isacharcolon}\ {\isachardoublequoteopen}gen{\isacharunderscore}set\ A{\isachardoublequoteclose}\ \isacommand{by}\isamarkupfalse%
\ {\isacharparenleft}rule\ max{\isacharunderscore}li{\isacharunderscore}is{\isacharunderscore}gen{\isacharparenright}\isanewline
\ \ \isacommand{from}\isamarkupfalse%
\ assms\ {\isadigit{1}}\ \isacommand{show}\isamarkupfalse%
\ {\isacharquery}thesis\ \isacommand{by}\isamarkupfalse%
\ {\isacharparenleft}unfold\ basis{\isacharunderscore}def\ maximal{\isacharunderscore}def{\isacharcomma}\ auto{\isacharparenright}\isanewline
\isacommand{qed}\isamarkupfalse%
%
\endisatagproof
{\isafoldproof}%
%
\isadelimproof
%
\endisadelimproof
%
\begin{isamarkuptext}%
Any minimal linearly independent set is a generating set.%
\end{isamarkuptext}%
\isamarkuptrue%
\isacommand{lemma}\isamarkupfalse%
\ {\isacharparenleft}\isakeyword{in}\ vectorspace{\isacharparenright}\ min{\isacharunderscore}gen{\isacharunderscore}is{\isacharunderscore}basis{\isacharcolon}\isanewline
\ \ \isakeyword{fixes}\ A\isanewline
\ \ \isakeyword{assumes}\ h{\isadigit{1}}{\isacharcolon}\ {\isachardoublequoteopen}minimal\ A\ {\isacharparenleft}{\isasymlambda}S{\isachardot}\ S{\isasymsubseteq}carrier\ V\ {\isasymand}\ gen{\isacharunderscore}set\ S{\isacharparenright}{\isachardoublequoteclose}\isanewline
\ \ \isakeyword{shows}\ {\isachardoublequoteopen}basis\ A{\isachardoublequoteclose}\isanewline
%
\isadelimproof
%
\endisadelimproof
%
\isatagproof
\isacommand{proof}\isamarkupfalse%
\ {\isacharminus}\ \isanewline
\ \ \isacommand{from}\isamarkupfalse%
\ h{\isadigit{1}}\ \isacommand{have}\isamarkupfalse%
\ {\isadigit{1}}{\isacharcolon}\ {\isachardoublequoteopen}lin{\isacharunderscore}indpt\ A{\isachardoublequoteclose}\ \isacommand{by}\isamarkupfalse%
\ {\isacharparenleft}rule\ min{\isacharunderscore}gen{\isacharunderscore}is{\isacharunderscore}li{\isacharparenright}\isanewline
\ \ \isacommand{from}\isamarkupfalse%
\ assms\ {\isadigit{1}}\ \isacommand{show}\isamarkupfalse%
\ {\isacharquery}thesis\ \isacommand{by}\isamarkupfalse%
\ {\isacharparenleft}unfold\ basis{\isacharunderscore}def\ minimal{\isacharunderscore}def{\isacharcomma}\ auto{\isacharparenright}\isanewline
\isacommand{qed}\isamarkupfalse%
%
\endisatagproof
{\isafoldproof}%
%
\isadelimproof
%
\endisadelimproof
%
\begin{isamarkuptext}%
Any linearly independent set with cardinality at least the dimension is a basis.%
\end{isamarkuptext}%
\isamarkuptrue%
\isacommand{lemma}\isamarkupfalse%
\ {\isacharparenleft}\isakeyword{in}\ vectorspace{\isacharparenright}\ dim{\isacharunderscore}li{\isacharunderscore}is{\isacharunderscore}basis{\isacharcolon}\isanewline
\ \ \isakeyword{fixes}\ A\isanewline
\ \ \isakeyword{assumes}\ fd{\isacharcolon}\ {\isachardoublequoteopen}fin{\isacharunderscore}dim{\isachardoublequoteclose}\ \isakeyword{and}\ fa{\isacharcolon}\ {\isachardoublequoteopen}finite\ A{\isachardoublequoteclose}\ \isakeyword{and}\ ca{\isacharcolon}\ {\isachardoublequoteopen}A{\isasymsubseteq}carrier\ V{\isachardoublequoteclose}\ \isakeyword{and}\ li{\isacharcolon}\ {\isachardoublequoteopen}lin{\isacharunderscore}indpt\ A{\isachardoublequoteclose}\ \isanewline
\ \ \ \ \isakeyword{and}\ d{\isacharcolon}\ {\isachardoublequoteopen}card\ A\ {\isasymge}\ dim{\isachardoublequoteclose}\ \isanewline
\ \ \isakeyword{shows}\ {\isachardoublequoteopen}basis\ A{\isachardoublequoteclose}\isanewline
%
\isadelimproof
%
\endisadelimproof
%
\isatagproof
\isacommand{proof}\isamarkupfalse%
\ {\isacharminus}\ \isanewline
\ \ \isacommand{from}\isamarkupfalse%
\ fd\ \isacommand{have}\isamarkupfalse%
\ {\isadigit{0}}{\isacharcolon}\ {\isachardoublequoteopen}{\isasymAnd}S{\isachardot}\ S{\isasymsubseteq}carrier\ V\ {\isasymand}\ lin{\isacharunderscore}indpt\ S\ {\isasymLongrightarrow}\ finite\ S\ {\isasymand}\ card\ S\ {\isasymle}dim{\isachardoublequoteclose}\isanewline
\ \ \ \ \isacommand{by}\isamarkupfalse%
\ {\isacharparenleft}auto\ intro{\isacharcolon}\ li{\isacharunderscore}le{\isacharunderscore}dim{\isacharparenright}\isanewline
\isanewline
\ \ \isacommand{from}\isamarkupfalse%
\ {\isadigit{0}}\ assms\ \isacommand{have}\isamarkupfalse%
\ h{\isadigit{1}}{\isacharcolon}\ \ {\isachardoublequoteopen}finite\ A\ {\isasymand}\ maximal\ A\ {\isacharparenleft}{\isasymlambda}S{\isachardot}\ \ S{\isasymsubseteq}carrier\ V\ {\isasymand}\ lin{\isacharunderscore}indpt\ S{\isacharparenright}{\isachardoublequoteclose}\isanewline
\ \ \ \ \isacommand{apply}\isamarkupfalse%
\ {\isacharparenleft}unfold\ maximal{\isacharunderscore}def{\isacharparenright}\ \isanewline
\ \ \ \ \isacommand{apply}\isamarkupfalse%
\ auto\isanewline
\ \ \ \ \isacommand{by}\isamarkupfalse%
\ {\isacharparenleft}metis\ card{\isacharunderscore}seteq\ eq{\isacharunderscore}iff{\isacharparenright}\isanewline
\ \ \isacommand{from}\isamarkupfalse%
\ h{\isadigit{1}}\ \isacommand{show}\isamarkupfalse%
\ {\isacharquery}thesis\ \isacommand{by}\isamarkupfalse%
\ {\isacharparenleft}auto\ intro{\isacharcolon}\ max{\isacharunderscore}li{\isacharunderscore}is{\isacharunderscore}basis{\isacharparenright}\isanewline
\isacommand{qed}\isamarkupfalse%
%
\endisatagproof
{\isafoldproof}%
%
\isadelimproof
%
\endisadelimproof
%
\begin{isamarkuptext}%
Any generating set with cardinality at most the dimension is a basis.%
\end{isamarkuptext}%
\isamarkuptrue%
\isacommand{lemma}\isamarkupfalse%
\ {\isacharparenleft}\isakeyword{in}\ vectorspace{\isacharparenright}\ dim{\isacharunderscore}gen{\isacharunderscore}is{\isacharunderscore}basis{\isacharcolon}\isanewline
\ \ \isakeyword{fixes}\ A\isanewline
\ \ \isakeyword{assumes}\ fa{\isacharcolon}\ {\isachardoublequoteopen}finite\ A{\isachardoublequoteclose}\ \isakeyword{and}\ ca{\isacharcolon}\ {\isachardoublequoteopen}A{\isasymsubseteq}carrier\ V{\isachardoublequoteclose}\ \isakeyword{and}\ li{\isacharcolon}\ {\isachardoublequoteopen}gen{\isacharunderscore}set\ A{\isachardoublequoteclose}\ \isanewline
\ \ \ \ \isakeyword{and}\ d{\isacharcolon}\ {\isachardoublequoteopen}card\ A\ {\isasymle}\ dim{\isachardoublequoteclose}\isanewline
\ \ \isakeyword{shows}\ {\isachardoublequoteopen}basis\ A{\isachardoublequoteclose}\isanewline
%
\isadelimproof
%
\endisadelimproof
%
\isatagproof
\isacommand{proof}\isamarkupfalse%
\ {\isacharminus}\ \isanewline
\ \ \isacommand{have}\isamarkupfalse%
\ {\isadigit{0}}{\isacharcolon}\ {\isachardoublequoteopen}{\isasymAnd}S{\isachardot}\ finite\ S{\isasymand}\ S{\isasymsubseteq}carrier\ V\ {\isasymand}\ gen{\isacharunderscore}set\ S\ {\isasymLongrightarrow}\ card\ S\ {\isasymge}dim{\isachardoublequoteclose}\isanewline
\ \ \ \ \isacommand{by}\isamarkupfalse%
\ {\isacharparenleft}intro\ gen{\isacharunderscore}ge{\isacharunderscore}dim{\isacharcomma}\ auto{\isacharparenright}\isanewline
\isanewline
\ \ \isacommand{from}\isamarkupfalse%
\ {\isadigit{0}}\ assms\ \isacommand{have}\isamarkupfalse%
\ h{\isadigit{1}}{\isacharcolon}\ \ {\isachardoublequoteopen}minimal\ A\ {\isacharparenleft}{\isasymlambda}S{\isachardot}\ finite\ S\ {\isasymand}\ S{\isasymsubseteq}carrier\ V\ {\isasymand}\ gen{\isacharunderscore}set\ S{\isacharparenright}{\isachardoublequoteclose}\isanewline
\ \ \ \ \isacommand{apply}\isamarkupfalse%
\ {\isacharparenleft}unfold\ minimal{\isacharunderscore}def{\isacharparenright}\ \isanewline
\ \ \ \ \isacommand{apply}\isamarkupfalse%
\ auto\isanewline
\ \ \ \ \isacommand{by}\isamarkupfalse%
\ {\isacharparenleft}metis\ card{\isacharunderscore}seteq\ eq{\isacharunderscore}iff{\isacharparenright}\isanewline
\ \ \isanewline
\ \ \isacommand{from}\isamarkupfalse%
\ h{\isadigit{1}}\ \isacommand{have}\isamarkupfalse%
\ h{\isacharcolon}\ {\isachardoublequoteopen}{\isasymAnd}B{\isachardot}\ B\ {\isasymsubseteq}\ A\ {\isasymand}\ B\ {\isasymsubseteq}\ carrier\ V\ {\isasymand}\ LinearCombinations{\isachardot}module{\isachardot}gen{\isacharunderscore}set\ K\ V\ B\ {\isasymLongrightarrow}\ B\ {\isacharequal}\ A{\isachardoublequoteclose}\isanewline
\ \ \isacommand{proof}\isamarkupfalse%
\ {\isacharminus}\ \isanewline
\ \ \ \ \isacommand{fix}\isamarkupfalse%
\ B\isanewline
\ \ \ \ \isacommand{assume}\isamarkupfalse%
\ asm{\isacharcolon}\ {\isachardoublequoteopen}B\ {\isasymsubseteq}\ A\ {\isasymand}\ B\ {\isasymsubseteq}\ carrier\ V\ {\isasymand}\ LinearCombinations{\isachardot}module{\isachardot}gen{\isacharunderscore}set\ K\ V\ B{\isachardoublequoteclose}\ \isanewline
\ \ \ \ \isacommand{from}\isamarkupfalse%
\ asm\ h{\isadigit{1}}\ \isacommand{have}\isamarkupfalse%
\ {\isachardoublequoteopen}finite\ B{\isachardoublequoteclose}\ \isanewline
\ \ \ \ \ \ \isacommand{apply}\isamarkupfalse%
\ {\isacharparenleft}unfold\ minimal{\isacharunderscore}def{\isacharparenright}\ \isanewline
\ \ \ \ \ \ \ \isacommand{apply}\isamarkupfalse%
\ {\isacharparenleft}intro\ finite{\isacharunderscore}subset{\isacharbrackleft}\isakeyword{where}\ {\isacharquery}A{\isacharequal}{\isachardoublequoteopen}B{\isachardoublequoteclose}\ \isakeyword{and}\ {\isacharquery}B{\isacharequal}{\isachardoublequoteopen}A{\isachardoublequoteclose}{\isacharbrackright}{\isacharparenright}\ \isanewline
\ \ \ \ \ \ \ \isacommand{by}\isamarkupfalse%
\ auto\isanewline
\ \ \ \ \isacommand{from}\isamarkupfalse%
\ h{\isadigit{1}}\ asm\ this\ \isacommand{show}\isamarkupfalse%
\ {\isachardoublequoteopen}{\isacharquery}thesis\ B{\isachardoublequoteclose}\ \isacommand{apply}\isamarkupfalse%
\ {\isacharparenleft}unfold\ minimal{\isacharunderscore}def{\isacharparenright}\ \isacommand{by}\isamarkupfalse%
\ simp\isanewline
\ \ \isacommand{qed}\isamarkupfalse%
\isanewline
\ \ \isacommand{from}\isamarkupfalse%
\ h{\isadigit{1}}\ h\ \isacommand{have}\isamarkupfalse%
\ h{\isadigit{2}}{\isacharcolon}\ {\isachardoublequoteopen}minimal\ A\ {\isacharparenleft}{\isasymlambda}S{\isachardot}\ S{\isasymsubseteq}carrier\ V\ {\isasymand}\ gen{\isacharunderscore}set\ S{\isacharparenright}{\isachardoublequoteclose}\ \isanewline
\ \ \ \ \isacommand{apply}\isamarkupfalse%
\ {\isacharparenleft}unfold\ minimal{\isacharunderscore}def{\isacharparenright}\isanewline
\ \ \ \ \isacommand{by}\isamarkupfalse%
\ presburger\isanewline
\ \ \isacommand{from}\isamarkupfalse%
\ h{\isadigit{2}}\ \isacommand{show}\isamarkupfalse%
\ {\isacharquery}thesis\ \isacommand{by}\isamarkupfalse%
\ {\isacharparenleft}rule\ min{\isacharunderscore}gen{\isacharunderscore}is{\isacharunderscore}basis{\isacharparenright}\isanewline
\isacommand{qed}\isamarkupfalse%
%
\endisatagproof
{\isafoldproof}%
%
\isadelimproof
%
\endisadelimproof
%
\begin{isamarkuptext}%
$\beta$ is a basis iff for all $v\in V$, there exists a unique $(a_v)_{v\in S}$ such that
$\sum_{v\in S} a_v v=v$.%
\end{isamarkuptext}%
\isamarkuptrue%
\isacommand{lemma}\isamarkupfalse%
\ {\isacharparenleft}\isakeyword{in}\ vectorspace{\isacharparenright}\ basis{\isacharunderscore}criterion{\isacharcolon}\isanewline
\ \ \isakeyword{fixes}\ A\isanewline
\ \ \isakeyword{assumes}\ A{\isacharunderscore}fin{\isacharcolon}\ {\isachardoublequoteopen}finite\ A{\isachardoublequoteclose}\ \isakeyword{and}\ AinC{\isacharcolon}\ {\isachardoublequoteopen}A{\isasymsubseteq}carrier\ V{\isachardoublequoteclose}\isanewline
\ \ \isakeyword{shows}\ {\isachardoublequoteopen}basis\ A\ {\isacharless}{\isacharminus}{\isachargreater}\ {\isacharparenleft}{\isasymforall}v{\isachardot}\ v{\isasymin}\ carrier\ V\ {\isasymlongrightarrow}{\isacharparenleft}{\isasymexists}{\isacharbang}\ a{\isachardot}\ \ a{\isasymin}A\ {\isasymrightarrow}\isactrlsub E\ carrier\ K\ {\isasymand}\ lincomb\ a\ A\ {\isacharequal}\ v{\isacharparenright}{\isacharparenright}{\isachardoublequoteclose}\isanewline
%
\isadelimproof
%
\endisadelimproof
%
\isatagproof
\isacommand{proof}\isamarkupfalse%
\ {\isacharminus}\isanewline
\ \ \isacommand{have}\isamarkupfalse%
\ {\isadigit{1}}{\isacharcolon}\ {\isachardoublequoteopen}{\isasymnot}{\isacharparenleft}{\isasymforall}v{\isachardot}\ \ v{\isasymin}\ carrier\ V\ {\isasymlongrightarrow}{\isacharparenleft}{\isasymexists}{\isacharbang}\ a{\isachardot}\ \ a{\isasymin}A\ {\isasymrightarrow}\isactrlsub E\ carrier\ K\ {\isasymand}\ lincomb\ a\ A\ {\isacharequal}\ v{\isacharparenright}{\isacharparenright}\ {\isasymLongrightarrow}\ {\isasymnot}basis\ A{\isachardoublequoteclose}\isanewline
\ \ \isacommand{proof}\isamarkupfalse%
\ {\isacharminus}\ \isanewline
\ \ \ \ \isacommand{assume}\isamarkupfalse%
\ {\isachardoublequoteopen}{\isasymnot}{\isacharparenleft}{\isasymforall}v{\isachardot}\ \ v{\isasymin}\ carrier\ V\ {\isasymlongrightarrow}{\isacharparenleft}{\isasymexists}{\isacharbang}\ a{\isachardot}\ \ a{\isasymin}A\ {\isasymrightarrow}\isactrlsub E\ carrier\ K\ {\isasymand}\ lincomb\ a\ A\ {\isacharequal}\ v{\isacharparenright}{\isacharparenright}{\isachardoublequoteclose}\isanewline
\ \ \ \ \isacommand{then}\isamarkupfalse%
\ \isacommand{obtain}\isamarkupfalse%
\ v\ \isakeyword{where}\ v{\isacharcolon}\ {\isachardoublequoteopen}v{\isasymin}\ carrier\ V\ {\isasymand}\ {\isasymnot}{\isacharparenleft}{\isasymexists}{\isacharbang}\ a{\isachardot}\ \ a{\isasymin}A\ {\isasymrightarrow}\isactrlsub E\ carrier\ K\ {\isasymand}\ lincomb\ a\ A\ {\isacharequal}\ v{\isacharparenright}{\isachardoublequoteclose}\ \isacommand{by}\isamarkupfalse%
\ metis\isanewline
\ \ \ \ \isanewline
\ \ \ \ \isacommand{from}\isamarkupfalse%
\ v\ \isacommand{have}\isamarkupfalse%
\ vinC{\isacharcolon}\ {\isachardoublequoteopen}v{\isasymin}carrier\ V{\isachardoublequoteclose}\ \isacommand{by}\isamarkupfalse%
\ auto\isanewline
\ \ \ \ \isacommand{from}\isamarkupfalse%
\ v\ \isacommand{have}\isamarkupfalse%
\ {\isachardoublequoteopen}{\isasymnot}{\isacharparenleft}{\isasymexists}\ a{\isachardot}\ \ a{\isasymin}A\ {\isasymrightarrow}\isactrlsub E\ carrier\ K\ {\isasymand}\ lincomb\ a\ A\ {\isacharequal}\ v{\isacharparenright}\ {\isasymor}\ \ {\isacharparenleft}{\isasymexists}\ a\ b{\isachardot}\ \isanewline
\ \ \ \ \ \ a{\isasymin}A\ {\isasymrightarrow}\isactrlsub E\ carrier\ K\ {\isasymand}\ lincomb\ a\ A\ {\isacharequal}\ v\ {\isasymand}\ b{\isasymin}A\ {\isasymrightarrow}\isactrlsub E\ carrier\ K\ {\isasymand}\ lincomb\ b\ A\ {\isacharequal}\ v\ \isanewline
\ \ \ \ \ \ {\isasymand}\ a{\isasymnoteq}b{\isacharparenright}{\isachardoublequoteclose}\ \isacommand{by}\isamarkupfalse%
\ metis\isanewline
\ \ \ \ \isacommand{from}\isamarkupfalse%
\ this\ \isacommand{show}\isamarkupfalse%
\ {\isacharquery}thesis\isanewline
\ \ \ \ \isacommand{proof}\isamarkupfalse%
\ {\isacharparenleft}rule\ disjE{\isacharparenright}\isanewline
\ \ \ \ \ \ \isacommand{assume}\isamarkupfalse%
\ a{\isacharcolon}\ {\isachardoublequoteopen}{\isasymnot}{\isacharparenleft}{\isasymexists}\ a{\isachardot}\ \ a{\isasymin}A\ {\isasymrightarrow}\isactrlsub E\ carrier\ K\ {\isasymand}\ lincomb\ a\ A\ {\isacharequal}\ v{\isacharparenright}{\isachardoublequoteclose}\isanewline
\ \ \ \ \ \ \isacommand{from}\isamarkupfalse%
\ A{\isacharunderscore}fin\ AinC\ \isacommand{have}\isamarkupfalse%
\ {\isadigit{1}}{\isacharcolon}\ {\isachardoublequoteopen}{\isasymAnd}a{\isachardot}\ a{\isasymin}A\ {\isasymrightarrow}\ carrier\ K\ {\isasymLongrightarrow}\ lincomb\ a\ A\ {\isacharequal}\ lincomb\ {\isacharparenleft}restrict\ a\ A{\isacharparenright}\ A{\isachardoublequoteclose}\ \isanewline
\ \ \ \ \ \ \ \ \isacommand{apply}\isamarkupfalse%
\ {\isacharparenleft}unfold\ lincomb{\isacharunderscore}def\ restrict{\isacharunderscore}def{\isacharparenright}\isanewline
\ \ \ \ \ \ \ \ \isacommand{apply}\isamarkupfalse%
\ {\isacharparenleft}drule\ Pi{\isacharunderscore}implies{\isacharunderscore}Pi{\isadigit{2}}{\isacharparenright}\isanewline
\ \ \ \ \ \ \ \ \isacommand{by}\isamarkupfalse%
\ {\isacharparenleft}simp\ cong{\isacharcolon}\ finsum{\isacharunderscore}cong\ add{\isacharcolon}\ ring{\isacharunderscore}subset{\isacharunderscore}carrier\ Pi{\isacharunderscore}simp{\isacharparenright}\isanewline
\ \ \ \ \ \ \isacommand{have}\isamarkupfalse%
\ {\isadigit{2}}{\isacharcolon}\ {\isachardoublequoteopen}{\isasymAnd}a{\isachardot}\ a{\isasymin}A\ {\isasymrightarrow}\ carrier\ K\ {\isasymLongrightarrow}\ restrict\ a\ A{\isasymin}\ A{\isasymrightarrow}\isactrlsub E\ carrier\ K{\isachardoublequoteclose}\ \isacommand{by}\isamarkupfalse%
\ auto\isanewline
\ \ \ \ \ \ \isacommand{from}\isamarkupfalse%
\ a\ {\isadigit{1}}\ {\isadigit{2}}\ \isacommand{have}\isamarkupfalse%
\ {\isadigit{3}}{\isacharcolon}\ {\isachardoublequoteopen}{\isasymnot}{\isacharparenleft}{\isasymexists}\ a{\isachardot}\ \ a{\isasymin}A\ {\isasymrightarrow}\ carrier\ K\ {\isasymand}\ lincomb\ a\ A\ {\isacharequal}\ v{\isacharparenright}{\isachardoublequoteclose}\ \isacommand{by}\isamarkupfalse%
\ algebra\isanewline
\ \ \ \ \ \ \isacommand{from}\isamarkupfalse%
\ {\isadigit{3}}\ A{\isacharunderscore}fin\ AinC\ \isacommand{have}\isamarkupfalse%
\ {\isadigit{4}}{\isacharcolon}\ {\isachardoublequoteopen}v{\isasymnotin}span\ A{\isachardoublequoteclose}\ \isanewline
\ \ \ \ \ \ \ \ \isacommand{by}\isamarkupfalse%
\ {\isacharparenleft}subst\ finite{\isacharunderscore}span{\isacharcomma}\ auto{\isacharparenright}\isanewline
\ \ \ \ \ \ \isacommand{from}\isamarkupfalse%
\ {\isadigit{4}}\ AinC\ v\ \isacommand{show}\isamarkupfalse%
\ {\isachardoublequoteopen}{\isasymnot}{\isacharparenleft}basis\ A{\isacharparenright}{\isachardoublequoteclose}\ \isacommand{by}\isamarkupfalse%
\ {\isacharparenleft}unfold\ basis{\isacharunderscore}def{\isacharcomma}\ auto{\isacharparenright}\isanewline
\ \ \ \ \isacommand{next}\isamarkupfalse%
\isanewline
\ \ \ \ \ \ \isacommand{assume}\isamarkupfalse%
\ a{\isadigit{2}}{\isacharcolon}\ {\isachardoublequoteopen}{\isacharparenleft}{\isasymexists}\ a\ b{\isachardot}\ \isanewline
\ \ \ \ \ \ \ \ a{\isasymin}A\ {\isasymrightarrow}\isactrlsub E\ carrier\ K\ {\isasymand}\ lincomb\ a\ A\ {\isacharequal}\ v\ {\isasymand}\ b{\isasymin}A\ {\isasymrightarrow}\isactrlsub E\ carrier\ K\ {\isasymand}\ lincomb\ b\ A\ {\isacharequal}\ v\ \isanewline
\ \ \ \ \ \ \ \ {\isasymand}\ a{\isasymnoteq}b{\isacharparenright}{\isachardoublequoteclose}\isanewline
\ \ \ \ \ \ \isacommand{then}\isamarkupfalse%
\ \isacommand{obtain}\isamarkupfalse%
\ a\ b\ \isakeyword{where}\ ab{\isacharcolon}\ {\isachardoublequoteopen}a{\isasymin}A\ {\isasymrightarrow}\isactrlsub E\ carrier\ K\ {\isasymand}\ lincomb\ a\ A\ {\isacharequal}\ v\ {\isasymand}\ b{\isasymin}A\ {\isasymrightarrow}\isactrlsub E\ carrier\ K\ {\isasymand}\ lincomb\ b\ A\ {\isacharequal}\ v\ \isanewline
\ \ \ \ \ \ \ \ {\isasymand}\ a{\isasymnoteq}b{\isachardoublequoteclose}\ \isacommand{by}\isamarkupfalse%
\ metis\isanewline
\ \ \ \ \ \ \isacommand{from}\isamarkupfalse%
\ ab\ \isacommand{obtain}\isamarkupfalse%
\ w\ \isakeyword{where}\ w{\isacharcolon}\ {\isachardoublequoteopen}w{\isasymin}A\ {\isasymand}\ a\ w\ {\isasymnoteq}\ b\ w{\isachardoublequoteclose}\ \isacommand{apply}\isamarkupfalse%
\ {\isacharparenleft}unfold\ PiE{\isacharunderscore}def{\isacharcomma}\ auto{\isacharparenright}\isanewline
\ \ \ \ \ \ \ \ \isacommand{by}\isamarkupfalse%
\ {\isacharparenleft}metis\ extensionalityI{\isacharparenright}\isanewline
\ \ \ \ \ \ \isacommand{let}\isamarkupfalse%
\ {\isacharquery}c{\isacharequal}{\isachardoublequoteopen}{\isasymlambda}\ x{\isachardot}\ {\isacharparenleft}if\ x{\isasymin}A\ then\ {\isacharparenleft}{\isacharparenleft}a\ x{\isacharparenright}\ {\isasymominus}\isactrlbsub K\isactrlesub \ {\isacharparenleft}b\ x{\isacharparenright}{\isacharparenright}\ else\ undefined{\isacharparenright}{\isachardoublequoteclose}\isanewline
\ \ \ \ \ \ \isacommand{from}\isamarkupfalse%
\ ab\ \isacommand{have}\isamarkupfalse%
\ a{\isacharunderscore}fun{\isacharcolon}\ {\isachardoublequoteopen}a{\isasymin}A\ {\isasymrightarrow}\ carrier\ K{\isachardoublequoteclose}\ \isanewline
\ \ \ \ \ \ \ \ \ \ \ \ \ \ \ \isakeyword{and}\ b{\isacharunderscore}fun{\isacharcolon}\ {\isachardoublequoteopen}b{\isasymin}A\ {\isasymrightarrow}\ carrier\ K{\isachardoublequoteclose}\ \isanewline
\ \ \ \ \ \ \ \ \isacommand{by}\isamarkupfalse%
\ {\isacharparenleft}unfold\ PiE{\isacharunderscore}def{\isacharcomma}\ auto{\isacharparenright}\isanewline
\ \ \ \ \ \ \isacommand{from}\isamarkupfalse%
\ w\ a{\isacharunderscore}fun\ b{\isacharunderscore}fun\ \isacommand{have}\isamarkupfalse%
\ abinC{\isacharcolon}\ {\isachardoublequoteopen}a\ w\ {\isasymin}carrier\ K{\isachardoublequoteclose}\ {\isachardoublequoteopen}b\ w\ {\isasymin}carrier\ K{\isachardoublequoteclose}\ \isacommand{by}\isamarkupfalse%
\ auto\isanewline
\isanewline
\ \ \ \ \ \ \isacommand{from}\isamarkupfalse%
\ abinC\ w\ \isacommand{have}\isamarkupfalse%
\ nz{\isacharcolon}\ {\isachardoublequoteopen}a\ w\ {\isasymominus}\isactrlbsub K\isactrlesub \ b\ w\ {\isasymnoteq}\ {\isasymzero}\isactrlbsub K\isactrlesub {\isachardoublequoteclose}\ \isanewline
\ \ \ \ \ \ \ \ \isacommand{by}\isamarkupfalse%
\ auto\ \isanewline
\ \ \ \ \ \ \isacommand{from}\isamarkupfalse%
\ A{\isacharunderscore}fin\ AinC\ a{\isacharunderscore}fun\ b{\isacharunderscore}fun\ ab\ vinC\ \isacommand{have}\isamarkupfalse%
\ a{\isacharunderscore}b{\isacharcolon}\isanewline
\ \ \ \ \ \ {\isachardoublequoteopen}LinearCombinations{\isachardot}module{\isachardot}lincomb\ V\ {\isacharparenleft}{\isasymlambda}x{\isachardot}\ if\ x\ {\isasymin}\ A\ then\ a\ x\ {\isasymominus}\isactrlbsub K\isactrlesub \ b\ x\ else\ undefined{\isacharparenright}\ A\ {\isacharequal}\ {\isasymzero}\isactrlbsub V\isactrlesub {\isachardoublequoteclose}\isanewline
\ \ \ \ \ \ \ \ \isacommand{apply}\isamarkupfalse%
\ {\isacharparenleft}subst\ refl{\isacharparenright}\isanewline
\ \ \ \ \ \ \ \ \isacommand{apply}\isamarkupfalse%
\ {\isacharparenleft}drule\ Pi{\isacharunderscore}implies{\isacharunderscore}Pi{\isadigit{2}}{\isacharparenright}{\isacharplus}\isanewline
\ \ \ \ \ \ \ \ \isacommand{apply}\isamarkupfalse%
\ {\isacharparenleft}simp\ cong{\isacharcolon}\ lincomb{\isacharunderscore}cong\ add{\isacharcolon}\ Pi{\isacharunderscore}simp{\isacharparenright}\isanewline
\ \ \ \ \ \ \ \ \isacommand{apply}\isamarkupfalse%
\ {\isacharparenleft}unfold\ Pi{\isadigit{2}}{\isacharunderscore}def{\isacharparenright}\isanewline
\ \ \ \ \ \ \ \ \isacommand{apply}\isamarkupfalse%
\ {\isacharparenleft}subst\ lincomb{\isacharunderscore}diff{\isacharparenright}\isanewline
\ \ \ \ \ \ \ \ \ \ \ \ \isacommand{by}\isamarkupfalse%
\ {\isacharparenleft}simp{\isacharunderscore}all\ add{\isacharcolon}\ minus{\isacharunderscore}eq\ r{\isacharunderscore}neg{\isacharparenright}\ \isanewline
\ \ \ \ \ \ \isacommand{from}\isamarkupfalse%
\ A{\isacharunderscore}fin\ AinC\ ab\ w\ v\ nz\ a{\isacharunderscore}b\ \isacommand{have}\isamarkupfalse%
\ {\isachardoublequoteopen}lin{\isacharunderscore}dep\ A{\isachardoublequoteclose}\isanewline
\ \ \ \ \ \ \ \ \isacommand{apply}\isamarkupfalse%
\ {\isacharparenleft}intro\ lin{\isacharunderscore}dep{\isacharunderscore}crit{\isacharbrackleft}\isakeyword{where}\ {\isacharquery}A{\isacharequal}{\isachardoublequoteopen}A{\isachardoublequoteclose}\ \isakeyword{and}\ {\isacharquery}a{\isacharequal}{\isachardoublequoteopen}{\isacharquery}c{\isachardoublequoteclose}\ \isakeyword{and}\ {\isacharquery}v{\isacharequal}{\isachardoublequoteopen}w{\isachardoublequoteclose}{\isacharbrackright}{\isacharparenright}\isanewline
\ \ \ \ \ \ \ \ \ \ \ \ \ \isacommand{apply}\isamarkupfalse%
\ {\isacharparenleft}auto\ simp\ add{\isacharcolon}\ PiE{\isacharunderscore}def{\isacharparenright}\isanewline
\ \ \ \ \ \ \ \ \isacommand{by}\isamarkupfalse%
\ auto\isanewline
\ \ \ \ \ \ \isacommand{thus}\isamarkupfalse%
\ {\isachardoublequoteopen}{\isasymnot}basis\ A{\isachardoublequoteclose}\ \isacommand{by}\isamarkupfalse%
\ {\isacharparenleft}unfold\ basis{\isacharunderscore}def{\isacharcomma}\ auto{\isacharparenright}\isanewline
\ \ \ \ \isacommand{qed}\isamarkupfalse%
\isanewline
\ \ \isacommand{qed}\isamarkupfalse%
\isanewline
\ \ \isacommand{have}\isamarkupfalse%
\ {\isadigit{2}}{\isacharcolon}\ {\isachardoublequoteopen}{\isacharparenleft}{\isasymforall}v{\isachardot}\ v{\isasymin}\ carrier\ V\ {\isasymlongrightarrow}{\isacharparenleft}{\isasymexists}{\isacharbang}\ a{\isachardot}\ \ a{\isasymin}A\ {\isasymrightarrow}\isactrlsub E\ carrier\ K\ {\isasymand}\ lincomb\ a\ A\ {\isacharequal}\ v{\isacharparenright}{\isacharparenright}\ {\isasymLongrightarrow}\ basis\ A{\isachardoublequoteclose}\isanewline
\ \ \isacommand{proof}\isamarkupfalse%
\ {\isacharminus}\isanewline
\ \ \ \ \isacommand{assume}\isamarkupfalse%
\ b{\isadigit{1}}{\isacharcolon}\ {\isachardoublequoteopen}{\isacharparenleft}{\isasymforall}v{\isachardot}\ v{\isasymin}\ carrier\ V\ {\isasymlongrightarrow}{\isacharparenleft}{\isasymexists}{\isacharbang}\ a{\isachardot}\ \ a{\isasymin}A\ {\isasymrightarrow}\isactrlsub E\ carrier\ K\ {\isasymand}\ lincomb\ a\ A\ {\isacharequal}\ v{\isacharparenright}{\isacharparenright}{\isachardoublequoteclose}\ \isanewline
\ \ \ \ \ \ {\isacharparenleft}\isakeyword{is}\ {\isachardoublequoteopen}{\isacharparenleft}{\isasymforall}v{\isachardot}\ v{\isasymin}\ carrier\ V\ {\isasymlongrightarrow}{\isacharparenleft}{\isasymexists}{\isacharbang}\ a{\isachardot}\ \ {\isacharquery}Q\ a\ v{\isacharparenright}{\isacharparenright}{\isachardoublequoteclose}{\isacharparenright}\isanewline
\ \ \ \ \isacommand{from}\isamarkupfalse%
\ b{\isadigit{1}}\ \isacommand{have}\isamarkupfalse%
\ b{\isadigit{2}}{\isacharcolon}\ {\isachardoublequoteopen}{\isacharparenleft}{\isasymforall}v{\isachardot}\ \ v{\isasymin}\ carrier\ V\ {\isasymlongrightarrow}{\isacharparenleft}{\isasymexists}\ a{\isachardot}\ \ a{\isasymin}A\ {\isasymrightarrow}\ carrier\ K\ {\isasymand}\ lincomb\ a\ A\ {\isacharequal}\ v{\isacharparenright}{\isacharparenright}{\isachardoublequoteclose}\ \isanewline
\ \ \ \ \ \ \isacommand{apply}\isamarkupfalse%
\ {\isacharparenleft}unfold\ PiE{\isacharunderscore}def{\isacharparenright}\ \isanewline
\ \ \ \ \ \ \isacommand{by}\isamarkupfalse%
\ blast\ \isanewline
\ \ \ \ \isacommand{from}\isamarkupfalse%
\ A{\isacharunderscore}fin\ AinC\ b{\isadigit{2}}\ \isacommand{have}\isamarkupfalse%
\ {\isachardoublequoteopen}gen{\isacharunderscore}set\ A{\isachardoublequoteclose}\isanewline
\ \ \ \ \ \ \isacommand{apply}\isamarkupfalse%
\ {\isacharparenleft}unfold\ span{\isacharunderscore}def{\isacharparenright}\isanewline
\ \ \ \ \ \ \isacommand{by}\isamarkupfalse%
\ blast\isanewline
\ \ \ \ \isacommand{from}\isamarkupfalse%
\ b{\isadigit{1}}\ \isacommand{have}\isamarkupfalse%
\ A{\isacharunderscore}li{\isacharcolon}\ {\isachardoublequoteopen}lin{\isacharunderscore}indpt\ A{\isachardoublequoteclose}\isanewline
\ \ \ \ \isacommand{proof}\isamarkupfalse%
\ {\isacharminus}\isanewline
\ \ \ \ \ \ \isacommand{let}\isamarkupfalse%
\ {\isacharquery}z{\isacharequal}{\isachardoublequoteopen}{\isasymlambda}\ x{\isachardot}\ {\isacharparenleft}if\ {\isacharparenleft}x{\isasymin}A{\isacharparenright}\ then\ {\isasymzero}\isactrlbsub K\isactrlesub \ else\ undefined{\isacharparenright}{\isachardoublequoteclose}\ \isanewline
\ \ \ \ \ \ \isacommand{from}\isamarkupfalse%
\ A{\isacharunderscore}fin\ AinC\ \isacommand{have}\isamarkupfalse%
\ zero{\isacharcolon}\ {\isachardoublequoteopen}{\isacharquery}Q\ {\isacharquery}z\ {\isasymzero}\isactrlbsub V\isactrlesub {\isachardoublequoteclose}\ \isanewline
\ \ \ \ \ \ \ \ \isacommand{by}\isamarkupfalse%
\ {\isacharparenleft}unfold\ PiE{\isacharunderscore}def\ extensional{\isacharunderscore}def\ lincomb{\isacharunderscore}def{\isacharcomma}\ auto\ simp\ add{\isacharcolon}\ ring{\isacharunderscore}subset{\isacharunderscore}carrier{\isacharparenright}\isanewline
\ \ \ \ \ \ \ \ \isanewline
\ \ \ \ \ \ \isacommand{from}\isamarkupfalse%
\ A{\isacharunderscore}fin\ AinC\ \isacommand{show}\isamarkupfalse%
\ {\isacharquery}thesis\ \isanewline
\ \ \ \ \ \ \isacommand{proof}\isamarkupfalse%
\ {\isacharparenleft}rule\ finite{\isacharunderscore}lin{\isacharunderscore}indpt{\isadigit{2}}{\isacharparenright}\isanewline
\ \ \ \ \ \ \ \ \isacommand{fix}\isamarkupfalse%
\ a\isanewline
\ \ \ \ \ \ \ \ \isacommand{assume}\isamarkupfalse%
\ a{\isacharunderscore}fun{\isacharcolon}\ {\isachardoublequoteopen}a\ {\isasymin}\ A\ {\isasymrightarrow}\ carrier\ K{\isachardoublequoteclose}\ \isakeyword{and}\isanewline
\ \ \ \ \ \ \ \ \ \ lc{\isacharunderscore}a{\isacharcolon}\ {\isachardoublequoteopen}LinearCombinations{\isachardot}module{\isachardot}lincomb\ V\ a\ A\ {\isacharequal}\ {\isasymzero}\isactrlbsub V\isactrlesub {\isachardoublequoteclose}\isanewline
\ \ \ \ \ \ \ \ \isacommand{from}\isamarkupfalse%
\ a{\isacharunderscore}fun\ \isacommand{have}\isamarkupfalse%
\ a{\isacharunderscore}res{\isacharcolon}\ {\isachardoublequoteopen}restrict\ a\ A\ {\isasymin}\ A\ {\isasymrightarrow}\isactrlsub E\ carrier\ K{\isachardoublequoteclose}\ \isacommand{by}\isamarkupfalse%
\ auto\isanewline
\ \ \ \ \ \ \ \ \isacommand{from}\isamarkupfalse%
\ a{\isacharunderscore}fun\ A{\isacharunderscore}fin\ AinC\ lc{\isacharunderscore}a\ \isacommand{have}\isamarkupfalse%
\ \isanewline
\ \ \ \ \ \ \ \ \ \ lc{\isacharunderscore}a{\isacharunderscore}res{\isacharcolon}\ {\isachardoublequoteopen}LinearCombinations{\isachardot}module{\isachardot}lincomb\ V\ {\isacharparenleft}restrict\ a\ A{\isacharparenright}\ A\ {\isacharequal}\ {\isasymzero}\isactrlbsub V\isactrlesub {\isachardoublequoteclose}\isanewline
\ \ \ \ \ \ \ \ \ \ \isacommand{apply}\isamarkupfalse%
\ {\isacharparenleft}unfold\ lincomb{\isacharunderscore}def\ restrict{\isacharunderscore}def{\isacharparenright}\isanewline
\ \ \ \ \ \ \ \ \ \ \isacommand{by}\isamarkupfalse%
\ {\isacharparenleft}drule\ Pi{\isacharunderscore}implies{\isacharunderscore}Pi{\isadigit{2}}{\isacharcomma}\ simp\ cong{\isacharcolon}\ finsum{\isacharunderscore}cong\ add{\isacharcolon}\ Pi{\isacharunderscore}simp\ ring{\isacharunderscore}subset{\isacharunderscore}carrier{\isacharparenright}\isanewline
\ \ \ \ \ \ \ \ \isacommand{from}\isamarkupfalse%
\ a{\isacharunderscore}fun\ a{\isacharunderscore}res\ lc{\isacharunderscore}a{\isacharunderscore}res\ zero\ b{\isadigit{1}}\ \isacommand{have}\isamarkupfalse%
\ {\isachardoublequoteopen}restrict\ a\ A\ {\isacharequal}\ {\isacharquery}z{\isachardoublequoteclose}\ \isacommand{by}\isamarkupfalse%
\ auto\isanewline
\ \ \ \ \ \ \ \ \isacommand{from}\isamarkupfalse%
\ this\ \isacommand{show}\isamarkupfalse%
\ {\isachardoublequoteopen}{\isasymforall}v{\isasymin}A{\isachardot}\ a\ v\ {\isacharequal}\ {\isasymzero}\isactrlbsub K\isactrlesub {\isachardoublequoteclose}\ \isanewline
\ \ \ \ \ \ \ \ \ \ \isacommand{apply}\isamarkupfalse%
\ {\isacharparenleft}unfold\ restrict{\isacharunderscore}def{\isacharparenright}\isanewline
\ \ \ \ \ \ \ \ \ \ \isacommand{by}\isamarkupfalse%
\ meson\isanewline
\ \ \ \ \ \ \isacommand{qed}\isamarkupfalse%
\isanewline
\ \ \ \ \isacommand{qed}\isamarkupfalse%
\isanewline
\ \ \ \ \isacommand{have}\isamarkupfalse%
\ A{\isacharunderscore}gen{\isacharcolon}\ {\isachardoublequoteopen}gen{\isacharunderscore}set\ A{\isachardoublequoteclose}\ \isanewline
\ \ \ \ \isacommand{proof}\isamarkupfalse%
\ {\isacharminus}\ \isanewline
\ \ \ \ \ \ \isacommand{from}\isamarkupfalse%
\ AinC\ \isacommand{have}\isamarkupfalse%
\ dir{\isadigit{1}}{\isacharcolon}\ {\isachardoublequoteopen}span\ A{\isasymsubseteq}carrier\ V{\isachardoublequoteclose}\ \isacommand{by}\isamarkupfalse%
\ {\isacharparenleft}rule\ span{\isacharunderscore}is{\isacharunderscore}subset{\isadigit{2}}{\isacharparenright}\isanewline
\ \ \ \ \ \ \isacommand{have}\isamarkupfalse%
\ dir{\isadigit{2}}{\isacharcolon}\ {\isachardoublequoteopen}carrier\ V{\isasymsubseteq}span\ A{\isachardoublequoteclose}\isanewline
\ \ \ \ \ \ \isacommand{proof}\isamarkupfalse%
\ {\isacharparenleft}auto{\isacharparenright}\isanewline
\ \ \ \ \ \ \ \ \isacommand{fix}\isamarkupfalse%
\ v\isanewline
\ \ \ \ \ \ \ \ \isacommand{assume}\isamarkupfalse%
\ v{\isacharcolon}\ {\isachardoublequoteopen}v{\isasymin}carrier\ V{\isachardoublequoteclose}\isanewline
\ \ \ \ \ \ \ \ \isacommand{from}\isamarkupfalse%
\ v\ b{\isadigit{2}}\ \isacommand{obtain}\isamarkupfalse%
\ a\ \isakeyword{where}\ {\isachardoublequoteopen}a{\isasymin}A\ {\isasymrightarrow}\ carrier\ K\ {\isasymand}\ lincomb\ a\ A\ {\isacharequal}\ v{\isachardoublequoteclose}\ \isacommand{by}\isamarkupfalse%
\ auto\isanewline
\ \ \ \ \ \ \ \ \isacommand{from}\isamarkupfalse%
\ this\ A{\isacharunderscore}fin\ AinC\ \isacommand{show}\isamarkupfalse%
\ {\isachardoublequoteopen}v{\isasymin}span\ A{\isachardoublequoteclose}\ \isacommand{by}\isamarkupfalse%
\ {\isacharparenleft}subst\ finite{\isacharunderscore}span{\isacharcomma}\ auto{\isacharparenright}\isanewline
\ \ \ \ \ \ \isacommand{qed}\isamarkupfalse%
\isanewline
\ \ \ \ \ \ \isacommand{from}\isamarkupfalse%
\ dir{\isadigit{1}}\ dir{\isadigit{2}}\ \isacommand{show}\isamarkupfalse%
\ {\isacharquery}thesis\ \isacommand{by}\isamarkupfalse%
\ auto\isanewline
\ \ \ \ \isacommand{qed}\isamarkupfalse%
\isanewline
\ \ \ \ \isacommand{from}\isamarkupfalse%
\ A{\isacharunderscore}li\ A{\isacharunderscore}gen\ AinC\ \isacommand{show}\isamarkupfalse%
\ {\isachardoublequoteopen}basis\ A{\isachardoublequoteclose}\ \isacommand{by}\isamarkupfalse%
\ {\isacharparenleft}unfold\ basis{\isacharunderscore}def{\isacharcomma}\ auto{\isacharparenright}\isanewline
\ \ \isacommand{qed}\isamarkupfalse%
\isanewline
\ \ \isacommand{from}\isamarkupfalse%
\ {\isadigit{1}}\ {\isadigit{2}}\ \isacommand{show}\isamarkupfalse%
\ {\isacharquery}thesis\ \isacommand{by}\isamarkupfalse%
\ satx\isanewline
\isacommand{qed}\isamarkupfalse%
%
\endisatagproof
{\isafoldproof}%
%
\isadelimproof
%
\endisadelimproof
%
\isamarkupsubsection{The rank-nullity (dimension) theorem%
}
\isamarkuptrue%
%
\begin{isamarkuptext}%
If $V$ is finite-dimensional and $T:V\to W$ is a linear map, then $\text{dim}(\text{im}(T))+
\text{dim}(\text{ker}(T)) = \text{dim} V$.%
\end{isamarkuptext}%
\isamarkuptrue%
\isacommand{theorem}\isamarkupfalse%
\ {\isacharparenleft}\isakeyword{in}\ linear{\isacharunderscore}map{\isacharparenright}\ rank{\isacharunderscore}nullity{\isacharcolon}\ \isanewline
\ \ \isakeyword{assumes}\ fd{\isacharcolon}\ {\isachardoublequoteopen}V{\isachardot}fin{\isacharunderscore}dim{\isachardoublequoteclose}\isanewline
\ \ \isakeyword{shows}\ {\isachardoublequoteopen}{\isacharparenleft}vectorspace{\isachardot}dim\ K\ {\isacharparenleft}W{\isachardot}vs\ imT{\isacharparenright}{\isacharparenright}\ {\isacharplus}\ {\isacharparenleft}vectorspace{\isachardot}dim\ K\ {\isacharparenleft}V{\isachardot}vs\ kerT{\isacharparenright}{\isacharparenright}\ {\isacharequal}\ V{\isachardot}dim{\isachardoublequoteclose}\isanewline
%
\isadelimproof
%
\endisadelimproof
%
\isatagproof
\isacommand{proof}\isamarkupfalse%
\ {\isacharminus}\ \isanewline
\ \ %
\isamarkupcmt{First interpret kerT, imT as vectorspaces%
}
\isanewline
\ \ \isacommand{have}\isamarkupfalse%
\ subs{\isacharunderscore}ker{\isacharcolon}\ {\isachardoublequoteopen}subspace\ K\ kerT\ V{\isachardoublequoteclose}\ \isacommand{by}\isamarkupfalse%
\ {\isacharparenleft}intro\ kerT{\isacharunderscore}is{\isacharunderscore}subspace{\isacharparenright}\isanewline
\ \ \isacommand{from}\isamarkupfalse%
\ subs{\isacharunderscore}ker\ \isacommand{have}\isamarkupfalse%
\ vs{\isacharunderscore}ker{\isacharcolon}\ {\isachardoublequoteopen}vectorspace\ K\ {\isacharparenleft}V{\isachardot}vs\ kerT{\isacharparenright}{\isachardoublequoteclose}\ \isacommand{by}\isamarkupfalse%
\ {\isacharparenleft}rule\ V{\isachardot}subspace{\isacharunderscore}is{\isacharunderscore}vs{\isacharparenright}\isanewline
\ \ \isacommand{from}\isamarkupfalse%
\ vs{\isacharunderscore}ker\ \isacommand{interpret}\isamarkupfalse%
\ ker{\isacharcolon}\ vectorspace\ K\ {\isachardoublequoteopen}{\isacharparenleft}V{\isachardot}vs\ kerT{\isacharparenright}{\isachardoublequoteclose}\ \isacommand{by}\isamarkupfalse%
\ auto\isanewline
\ \ \isacommand{have}\isamarkupfalse%
\ kerInC{\isacharcolon}\ {\isachardoublequoteopen}kerT{\isasymsubseteq}carrier\ V{\isachardoublequoteclose}\ \isacommand{by}\isamarkupfalse%
\ {\isacharparenleft}unfold\ ker{\isacharunderscore}def{\isacharcomma}\ auto{\isacharparenright}\isanewline
\isanewline
\ \ \isacommand{have}\isamarkupfalse%
\ subs{\isacharunderscore}im{\isacharcolon}\ {\isachardoublequoteopen}subspace\ K\ imT\ W{\isachardoublequoteclose}\ \isacommand{by}\isamarkupfalse%
\ {\isacharparenleft}intro\ imT{\isacharunderscore}is{\isacharunderscore}subspace{\isacharparenright}\ \ \isanewline
\ \ \isacommand{from}\isamarkupfalse%
\ subs{\isacharunderscore}im\ \isacommand{have}\isamarkupfalse%
\ vs{\isacharunderscore}im{\isacharcolon}\ {\isachardoublequoteopen}vectorspace\ K\ {\isacharparenleft}W{\isachardot}vs\ imT{\isacharparenright}{\isachardoublequoteclose}\ \isacommand{by}\isamarkupfalse%
\ {\isacharparenleft}rule\ W{\isachardot}subspace{\isacharunderscore}is{\isacharunderscore}vs{\isacharparenright}\isanewline
\ \ \isacommand{from}\isamarkupfalse%
\ vs{\isacharunderscore}im\ \isacommand{interpret}\isamarkupfalse%
\ im{\isacharcolon}\ vectorspace\ K\ {\isachardoublequoteopen}{\isacharparenleft}W{\isachardot}vs\ imT{\isacharparenright}{\isachardoublequoteclose}\ \isacommand{by}\isamarkupfalse%
\ auto\isanewline
\ \ \isacommand{have}\isamarkupfalse%
\ imInC{\isacharcolon}\ {\isachardoublequoteopen}imT{\isasymsubseteq}carrier\ W{\isachardoublequoteclose}\ \isacommand{by}\isamarkupfalse%
\ {\isacharparenleft}unfold\ im{\isacharunderscore}def{\isacharcomma}\ auto{\isacharparenright}\isanewline
\ \ \isanewline
\ \ \isacommand{have}\isamarkupfalse%
\ zero{\isacharunderscore}same{\isacharbrackleft}simp{\isacharbrackright}{\isacharcolon}\ {\isachardoublequoteopen}{\isasymzero}\isactrlbsub V{\isachardot}vs\ kerT\isactrlesub \ {\isacharequal}\ {\isasymzero}\isactrlbsub V\isactrlesub {\isachardoublequoteclose}\ \isacommand{apply}\isamarkupfalse%
\ {\isacharparenleft}unfold\ ker{\isacharunderscore}def{\isacharparenright}\ \isacommand{by}\isamarkupfalse%
\ auto\isanewline
\ \ %
\isamarkupcmt{Show ker T has a finite basis. This is not obvious. Show that any linearly independent set 
has size at most that of V. There exists a maximal linearly independent set, which is the basis.%
}
\isanewline
\ \ \isacommand{have}\isamarkupfalse%
\ every{\isacharunderscore}li{\isacharunderscore}small{\isacharcolon}\ {\isachardoublequoteopen}{\isasymAnd}A{\isachardot}\ {\isacharparenleft}A\ {\isasymsubseteq}\ kerT{\isacharparenright}{\isasymand}\ ker{\isachardot}lin{\isacharunderscore}indpt\ A\ {\isasymLongrightarrow}\ \isanewline
\ \ \ \ finite\ A\ {\isasymand}\ card\ A\ {\isasymle}\ V{\isachardot}dim{\isachardoublequoteclose}\isanewline
\ \ \isacommand{proof}\isamarkupfalse%
\ {\isacharminus}\ \isanewline
\ \ \ \ \isacommand{fix}\isamarkupfalse%
\ A\isanewline
\ \ \ \ \isacommand{assume}\isamarkupfalse%
\ eli{\isacharunderscore}asm{\isacharcolon}\ {\isachardoublequoteopen}{\isacharparenleft}A\ {\isasymsubseteq}\ kerT{\isacharparenright}{\isasymand}\ ker{\isachardot}lin{\isacharunderscore}indpt\ A{\isachardoublequoteclose}\isanewline
\ \ \ \ \isanewline
\ \ \ \ \isacommand{note}\isamarkupfalse%
\ V{\isachardot}module{\isachardot}span{\isacharunderscore}li{\isacharunderscore}not{\isacharunderscore}depend{\isacharparenleft}{\isadigit{2}}{\isacharparenright}{\isacharbrackleft}\isakeyword{where}\ {\isacharquery}N{\isacharequal}{\isachardoublequoteopen}kerT{\isachardoublequoteclose}\ \isakeyword{and}\ {\isacharquery}S{\isacharequal}{\isachardoublequoteopen}A{\isachardoublequoteclose}{\isacharbrackright}\ \isanewline
\ \ \ \ \isacommand{from}\isamarkupfalse%
\ this\ subs{\isacharunderscore}ker\ fd\ eli{\isacharunderscore}asm\ kerInC\ \isacommand{show}\isamarkupfalse%
\ {\isachardoublequoteopen}{\isacharquery}thesis\ A{\isachardoublequoteclose}\ \isanewline
\ \ \ \ \ \ \isacommand{apply}\isamarkupfalse%
\ {\isacharparenleft}intro\ conjI{\isacharparenright}\ \isanewline
\ \ \ \ \ \ \ \isacommand{by}\isamarkupfalse%
\ {\isacharparenleft}auto\ intro{\isacharbang}{\isacharcolon}\ V{\isachardot}li{\isacharunderscore}le{\isacharunderscore}dim{\isacharparenright}\isanewline
\ \ \isacommand{qed}\isamarkupfalse%
\isanewline
\ \ \isacommand{from}\isamarkupfalse%
\ every{\isacharunderscore}li{\isacharunderscore}small\ \isacommand{have}\isamarkupfalse%
\ exA{\isacharcolon}\ \isanewline
\ \ \ \ {\isachardoublequoteopen}{\isasymexists}A{\isachardot}\ finite\ A\ {\isasymand}\ maximal\ A\ {\isacharparenleft}{\isasymlambda}S{\isachardot}\ S{\isasymsubseteq}carrier\ {\isacharparenleft}V{\isachardot}vs\ kerT{\isacharparenright}\ {\isasymand}\ ker{\isachardot}lin{\isacharunderscore}indpt\ S{\isacharparenright}{\isachardoublequoteclose}\isanewline
\ \ \ \ \isacommand{apply}\isamarkupfalse%
\ {\isacharparenleft}intro\ maximal{\isacharunderscore}exists{\isacharbrackleft}\isakeyword{where}\ {\isacharquery}N{\isacharequal}{\isachardoublequoteopen}V{\isachardot}dim{\isachardoublequoteclose}\ \isakeyword{and}\ {\isacharquery}B{\isacharequal}{\isachardoublequoteopen}{\isacharbraceleft}{\isacharbraceright}{\isachardoublequoteclose}{\isacharbrackright}{\isacharparenright}\isanewline
\ \ \ \ \ \isacommand{apply}\isamarkupfalse%
\ auto\isanewline
\ \ \ \ \isacommand{by}\isamarkupfalse%
\ {\isacharparenleft}unfold\ ker{\isachardot}lin{\isacharunderscore}dep{\isacharunderscore}def{\isacharcomma}\ auto{\isacharparenright}\isanewline
\ \ \isacommand{from}\isamarkupfalse%
\ exA\ \isacommand{obtain}\isamarkupfalse%
\ A\ \isakeyword{where}\ A{\isacharcolon}{\isachardoublequoteopen}\ finite\ A\ {\isasymand}\ maximal\ A\ {\isacharparenleft}{\isasymlambda}S{\isachardot}\ S{\isasymsubseteq}carrier\ {\isacharparenleft}V{\isachardot}vs\ kerT{\isacharparenright}\ {\isasymand}\ ker{\isachardot}lin{\isacharunderscore}indpt\ S{\isacharparenright}{\isachardoublequoteclose}\ \isanewline
\ \ \ \ \isacommand{by}\isamarkupfalse%
\ blast\isanewline
\ \ \isacommand{hence}\isamarkupfalse%
\ finA{\isacharcolon}\ {\isachardoublequoteopen}finite\ A{\isachardoublequoteclose}\ \isakeyword{and}\ Ainker{\isacharcolon}\ {\isachardoublequoteopen}A{\isasymsubseteq}carrier\ {\isacharparenleft}V{\isachardot}vs\ kerT{\isacharparenright}{\isachardoublequoteclose}\ \isakeyword{and}\ AinC{\isacharcolon}\ {\isachardoublequoteopen}A{\isasymsubseteq}carrier\ V{\isachardoublequoteclose}\isanewline
\ \ \ \ \isacommand{by}\isamarkupfalse%
\ {\isacharparenleft}unfold\ maximal{\isacharunderscore}def\ ker{\isacharunderscore}def{\isacharcomma}\ auto{\isacharparenright}\isanewline
\ \ %
\isamarkupcmt{We obtain the basis A of kerT. It is also linearly independent when considered in V rather
than kerT%
}
\isanewline
\ \ \isacommand{from}\isamarkupfalse%
\ A\ \isacommand{have}\isamarkupfalse%
\ Abasis{\isacharcolon}\ {\isachardoublequoteopen}ker{\isachardot}basis\ A{\isachardoublequoteclose}\ \isanewline
\ \ \ \ \isacommand{by}\isamarkupfalse%
\ {\isacharparenleft}intro\ ker{\isachardot}max{\isacharunderscore}li{\isacharunderscore}is{\isacharunderscore}basis{\isacharcomma}\ auto{\isacharparenright}\ \isanewline
\ \ \isacommand{from}\isamarkupfalse%
\ subs{\isacharunderscore}ker\ Abasis\ \isacommand{have}\isamarkupfalse%
\ spanA{\isacharcolon}\ {\isachardoublequoteopen}V{\isachardot}module{\isachardot}span\ A\ {\isacharequal}\ kerT{\isachardoublequoteclose}\isanewline
\ \ \ \ \isacommand{apply}\isamarkupfalse%
\ {\isacharparenleft}unfold\ ker{\isachardot}basis{\isacharunderscore}def{\isacharparenright}\isanewline
\ \ \ \ \isacommand{by}\isamarkupfalse%
\ {\isacharparenleft}subst\ sym{\isacharbrackleft}OF\ V{\isachardot}module{\isachardot}span{\isacharunderscore}li{\isacharunderscore}not{\isacharunderscore}depend{\isacharparenleft}{\isadigit{1}}{\isacharparenright}{\isacharbrackleft}\isakeyword{where}\ {\isacharquery}N{\isacharequal}{\isachardoublequoteopen}kerT{\isachardoublequoteclose}{\isacharbrackright}{\isacharbrackright}{\isacharcomma}\ auto{\isacharparenright}\isanewline
\ \ \isacommand{from}\isamarkupfalse%
\ Abasis\ \isacommand{have}\isamarkupfalse%
\ Akerli{\isacharcolon}\ {\isachardoublequoteopen}ker{\isachardot}lin{\isacharunderscore}indpt\ A{\isachardoublequoteclose}\ \isanewline
\ \ \ \ \isacommand{apply}\isamarkupfalse%
\ {\isacharparenleft}unfold\ ker{\isachardot}basis{\isacharunderscore}def{\isacharparenright}\ \isanewline
\ \ \ \ \isacommand{by}\isamarkupfalse%
\ auto\isanewline
\ \ \isacommand{from}\isamarkupfalse%
\ subs{\isacharunderscore}ker\ Ainker\ Akerli\ \isacommand{have}\isamarkupfalse%
\ Ali{\isacharcolon}\ {\isachardoublequoteopen}V{\isachardot}module{\isachardot}lin{\isacharunderscore}indpt\ A{\isachardoublequoteclose}\ \isanewline
\ \ \ \ \isacommand{by}\isamarkupfalse%
\ {\isacharparenleft}auto\ simp\ add{\isacharcolon}\ V{\isachardot}module{\isachardot}span{\isacharunderscore}li{\isacharunderscore}not{\isacharunderscore}depend{\isacharparenleft}{\isadigit{2}}{\isacharparenright}{\isacharparenright}%
\begin{isamarkuptxt}%
Use the replacement theorem to find C such that $A\cup C$ is a basis of V.%
\end{isamarkuptxt}%
\isamarkuptrue%
\ \ \isacommand{from}\isamarkupfalse%
\ fd\ \isacommand{obtain}\isamarkupfalse%
\ B\ \isakeyword{where}\ B{\isacharcolon}\ {\isachardoublequoteopen}finite\ B{\isasymand}\ V{\isachardot}basis\ B{\isachardoublequoteclose}\ \isacommand{by}\isamarkupfalse%
\ {\isacharparenleft}metis\ V{\isachardot}finite{\isacharunderscore}basis{\isacharunderscore}exists{\isacharparenright}\isanewline
\ \ \isacommand{from}\isamarkupfalse%
\ B\ \isacommand{have}\isamarkupfalse%
\ Bfin{\isacharcolon}\ {\isachardoublequoteopen}finite\ B{\isachardoublequoteclose}\ \isakeyword{and}\ Bbasis{\isacharcolon}{\isachardoublequoteopen}V{\isachardot}basis\ B{\isachardoublequoteclose}\ \isacommand{by}\isamarkupfalse%
\ auto\isanewline
\ \ \isacommand{from}\isamarkupfalse%
\ B\ \isacommand{have}\isamarkupfalse%
\ Bcard{\isacharcolon}\ {\isachardoublequoteopen}V{\isachardot}dim\ {\isacharequal}\ card\ B{\isachardoublequoteclose}\ \isacommand{by}\isamarkupfalse%
\ {\isacharparenleft}intro\ V{\isachardot}dim{\isacharunderscore}basis{\isacharcomma}\ auto{\isacharparenright}\ \isanewline
\ \ \isacommand{from}\isamarkupfalse%
\ Bbasis\ \isacommand{have}\isamarkupfalse%
\ {\isadigit{6}}{\isadigit{2}}{\isacharcolon}\ {\isachardoublequoteopen}V{\isachardot}module{\isachardot}span\ B\ {\isacharequal}\ carrier\ V{\isachardoublequoteclose}\ \isanewline
\ \ \ \ \isacommand{by}\isamarkupfalse%
\ {\isacharparenleft}unfold\ V{\isachardot}basis{\isacharunderscore}def{\isacharcomma}\ auto{\isacharparenright}\isanewline
\ \ \isacommand{from}\isamarkupfalse%
\ A\ Abasis\ Ali\ B\ vs{\isacharunderscore}ker\ \isacommand{have}\isamarkupfalse%
\ {\isachardoublequoteopen}{\isasymexists}C{\isachardot}\ finite\ C\ {\isasymand}\ C{\isasymsubseteq}carrier\ V\ {\isasymand}\ C{\isasymsubseteq}\ V{\isachardot}module{\isachardot}span\ B\ {\isasymand}\ C{\isasyminter}A{\isacharequal}{\isacharbraceleft}{\isacharbraceright}\ \isanewline
\ \ \ \ \ \ {\isasymand}\ int\ {\isacharparenleft}card\ C{\isacharparenright}\ {\isasymle}\ {\isacharparenleft}int\ {\isacharparenleft}card\ B{\isacharparenright}{\isacharparenright}\ {\isacharminus}\ {\isacharparenleft}int\ {\isacharparenleft}card\ A{\isacharparenright}{\isacharparenright}\ {\isasymand}\ {\isacharparenleft}V{\isachardot}module{\isachardot}span\ {\isacharparenleft}A\ {\isasymunion}\ C{\isacharparenright}\ {\isacharequal}\ V{\isachardot}module{\isachardot}span\ B{\isacharparenright}{\isachardoublequoteclose}\isanewline
\ \ \ \ \isacommand{apply}\isamarkupfalse%
\ {\isacharparenleft}intro\ V{\isachardot}replacement{\isacharparenright}\isanewline
\ \ \ \ \isacommand{apply}\isamarkupfalse%
\ {\isacharparenleft}unfold\ vectorspace{\isachardot}basis{\isacharunderscore}def\ V{\isachardot}basis{\isacharunderscore}def{\isacharparenright}\isanewline
\ \ \ \ \ \ \isacommand{by}\isamarkupfalse%
\ {\isacharparenleft}unfold\ ker{\isacharunderscore}def{\isacharcomma}\ auto{\isacharparenright}%
\begin{isamarkuptxt}%
From replacement we got $|C|\leq |B|-|A|$. Equality must actually hold, because no generating set
can be smaller than $B$. Now $A\cup C$ is a maximal generating set, hence a basis; its cardinality
equals the dimension.%
\end{isamarkuptxt}%
\isamarkuptrue%
%
\begin{isamarkuptxt}%
We claim that $T(C)$ is basis for $\text{im}(T)$.%
\end{isamarkuptxt}%
\isamarkuptrue%
\ \ \isacommand{then}\isamarkupfalse%
\ \isacommand{obtain}\isamarkupfalse%
\ C\ \isakeyword{where}\ C{\isacharcolon}\ {\isachardoublequoteopen}finite\ C\ {\isasymand}\ C{\isasymsubseteq}carrier\ V\ {\isasymand}\ C{\isasymsubseteq}\ V{\isachardot}module{\isachardot}span\ B\ {\isasymand}\ C{\isasyminter}A{\isacharequal}{\isacharbraceleft}{\isacharbraceright}\ \isanewline
\ \ \ \ {\isasymand}\ int\ {\isacharparenleft}card\ C{\isacharparenright}\ {\isasymle}\ {\isacharparenleft}int\ {\isacharparenleft}card\ B{\isacharparenright}{\isacharparenright}\ {\isacharminus}\ {\isacharparenleft}int\ {\isacharparenleft}card\ A{\isacharparenright}{\isacharparenright}\ {\isasymand}\ {\isacharparenleft}V{\isachardot}module{\isachardot}span\ {\isacharparenleft}A\ {\isasymunion}\ C{\isacharparenright}\ {\isacharequal}\ V{\isachardot}module{\isachardot}span\ B{\isacharparenright}{\isachardoublequoteclose}\ \isacommand{by}\isamarkupfalse%
\ auto\isanewline
\ \ \isacommand{hence}\isamarkupfalse%
\ Cfin{\isacharcolon}\ {\isachardoublequoteopen}finite\ C{\isachardoublequoteclose}\ \isakeyword{and}\ CinC{\isacharcolon}\ {\isachardoublequoteopen}C{\isasymsubseteq}carrier\ V{\isachardoublequoteclose}\ \isakeyword{and}\ CinspanB{\isacharcolon}\ {\isachardoublequoteopen}\ C{\isasymsubseteq}V{\isachardot}module{\isachardot}span\ B{\isachardoublequoteclose}\ \isakeyword{and}\ CAdis{\isacharcolon}\ {\isachardoublequoteopen}C{\isasyminter}A{\isacharequal}{\isacharbraceleft}{\isacharbraceright}{\isachardoublequoteclose}\ \isanewline
\ \ \ \ \isakeyword{and}\ Ccard{\isacharcolon}\ {\isachardoublequoteopen}int\ {\isacharparenleft}card\ C{\isacharparenright}\ {\isasymle}\ {\isacharparenleft}int\ {\isacharparenleft}card\ B{\isacharparenright}{\isacharparenright}\ {\isacharminus}\ {\isacharparenleft}int\ {\isacharparenleft}card\ A{\isacharparenright}{\isacharparenright}{\isachardoublequoteclose}\isanewline
\ \ \ \ \isakeyword{and}\ ACspanB{\isacharcolon}\ {\isachardoublequoteopen}{\isacharparenleft}V{\isachardot}module{\isachardot}span\ {\isacharparenleft}A\ {\isasymunion}\ C{\isacharparenright}\ {\isacharequal}\ V{\isachardot}module{\isachardot}span\ B{\isacharparenright}{\isachardoublequoteclose}\ \isacommand{by}\isamarkupfalse%
\ auto\isanewline
\ \ \isacommand{from}\isamarkupfalse%
\ C\ \isacommand{have}\isamarkupfalse%
\ cardLe{\isacharcolon}\ {\isachardoublequoteopen}card\ A\ {\isacharplus}\ card\ C\ {\isasymle}\ card\ B{\isachardoublequoteclose}\ \isacommand{by}\isamarkupfalse%
\ auto\isanewline
\ \ \isacommand{from}\isamarkupfalse%
\ B\ C\ \isacommand{have}\isamarkupfalse%
\ ACgen{\isacharcolon}\ {\isachardoublequoteopen}V{\isachardot}module{\isachardot}gen{\isacharunderscore}set\ {\isacharparenleft}A{\isasymunion}C{\isacharparenright}{\isachardoublequoteclose}\ \isacommand{apply}\isamarkupfalse%
\ {\isacharparenleft}unfold\ V{\isachardot}basis{\isacharunderscore}def{\isacharparenright}\ \isacommand{by}\isamarkupfalse%
\ auto\isanewline
\ \ \isacommand{from}\isamarkupfalse%
\ finA\ C\ ACgen\ AinC\ B\ \isacommand{have}\isamarkupfalse%
\ cardGe{\isacharcolon}\ {\isachardoublequoteopen}card\ {\isacharparenleft}A{\isasymunion}C{\isacharparenright}\ {\isasymge}\ card\ B{\isachardoublequoteclose}\ \isacommand{by}\isamarkupfalse%
\ {\isacharparenleft}intro\ V{\isachardot}li{\isacharunderscore}smaller{\isacharunderscore}than{\isacharunderscore}gen{\isacharcomma}\ unfold\ V{\isachardot}basis{\isacharunderscore}def{\isacharcomma}\ auto{\isacharparenright}\isanewline
\ \ \isacommand{from}\isamarkupfalse%
\ finA\ C\ \isacommand{have}\isamarkupfalse%
\ cardUn{\isacharcolon}\ {\isachardoublequoteopen}card\ {\isacharparenleft}A{\isasymunion}C{\isacharparenright}{\isasymle}\ \ card\ A\ {\isacharplus}\ card\ C{\isachardoublequoteclose}\isanewline
\ \ \ \ \isacommand{by}\isamarkupfalse%
\ {\isacharparenleft}metis\ Int{\isacharunderscore}commute\ card{\isacharunderscore}Un{\isacharunderscore}disjoint\ le{\isacharunderscore}refl{\isacharparenright}\isanewline
\ \ \isacommand{from}\isamarkupfalse%
\ cardLe\ cardUn\ cardGe\ Bcard\ \isacommand{have}\isamarkupfalse%
\ cardEq{\isacharcolon}\ \isanewline
\ \ \ \ {\isachardoublequoteopen}card\ {\isacharparenleft}A{\isasymunion}C{\isacharparenright}\ {\isacharequal}\ card\ A\ {\isacharplus}\ card\ C{\isachardoublequoteclose}\ \isanewline
\ \ \ \ {\isachardoublequoteopen}card\ {\isacharparenleft}A{\isasymunion}C{\isacharparenright}\ {\isacharequal}\ card\ B{\isachardoublequoteclose}\ \isanewline
\ \ \ \ {\isachardoublequoteopen}card\ {\isacharparenleft}A{\isasymunion}C{\isacharparenright}\ {\isacharequal}\ V{\isachardot}dim{\isachardoublequoteclose}\ \isanewline
\ \ \ \ \isacommand{by}\isamarkupfalse%
\ auto\isanewline
\ \ \isacommand{from}\isamarkupfalse%
\ Abasis\ C\ cardEq\ \isacommand{have}\isamarkupfalse%
\ disj{\isacharcolon}\ {\isachardoublequoteopen}A{\isasyminter}C{\isacharequal}{\isacharbraceleft}{\isacharbraceright}{\isachardoublequoteclose}\ \isacommand{by}\isamarkupfalse%
\ auto\isanewline
\ \ \isacommand{from}\isamarkupfalse%
\ finA\ AinC\ C\ cardEq\ {\isadigit{6}}{\isadigit{2}}\ \isacommand{have}\isamarkupfalse%
\ ACfin{\isacharcolon}\ {\isachardoublequoteopen}finite\ {\isacharparenleft}A{\isasymunion}C{\isacharparenright}{\isachardoublequoteclose}\ \isakeyword{and}\ ACbasis{\isacharcolon}\ {\isachardoublequoteopen}V{\isachardot}basis\ {\isacharparenleft}A{\isasymunion}C{\isacharparenright}{\isachardoublequoteclose}\ \isanewline
\ \ \ \ \isacommand{by}\isamarkupfalse%
\ {\isacharparenleft}auto\ intro{\isacharbang}{\isacharcolon}\ V{\isachardot}dim{\isacharunderscore}gen{\isacharunderscore}is{\isacharunderscore}basis{\isacharparenright}\ \isanewline
\ \ \isacommand{have}\isamarkupfalse%
\ lm{\isacharcolon}\ {\isachardoublequoteopen}linear{\isacharunderscore}map\ K\ V\ W\ T{\isachardoublequoteclose}\isacommand{{\isachardot}{\isachardot}}\isamarkupfalse%
%
\begin{isamarkuptxt}%
Let $C'$ be the image of $C$ under $T$. We will show $C'$ is a basis for $\text{im}(T)$.%
\end{isamarkuptxt}%
\isamarkuptrue%
\ \ \isacommand{let}\isamarkupfalse%
\ {\isacharquery}C{\isacharprime}\ {\isacharequal}\ {\isachardoublequoteopen}T{\isacharbackquote}C{\isachardoublequoteclose}\isanewline
\ \ \isacommand{from}\isamarkupfalse%
\ Cfin\ \isacommand{have}\isamarkupfalse%
\ C{\isacharprime}fin{\isacharcolon}\ {\isachardoublequoteopen}finite\ {\isacharquery}C{\isacharprime}{\isachardoublequoteclose}\ \isacommand{by}\isamarkupfalse%
\ auto\isanewline
\ \ \isacommand{from}\isamarkupfalse%
\ AinC\ C\ \isacommand{have}\isamarkupfalse%
\ cim{\isacharcolon}\ {\isachardoublequoteopen}{\isacharquery}C{\isacharprime}{\isasymsubseteq}imT{\isachardoublequoteclose}\ \isacommand{by}\isamarkupfalse%
\ {\isacharparenleft}unfold\ im{\isacharunderscore}def{\isacharcomma}\ auto{\isacharparenright}%
\begin{isamarkuptxt}%
"There is a subtle detail: we first have to show $T$ is injective on $C$.%
\end{isamarkuptxt}%
\isamarkuptrue%
%
\begin{isamarkuptxt}%
We establish that no nontrivial linear combination of $C$ can have image 0 under $T$, 
because that would mean it is a linear combination of $A$, giving that $A\cup C$ is linearly dependent, 
contradiction. We use this result in 2 ways: (1) if $T$ is not injective on $C$, then we obtain $v$, $w\in C$ 
such that $v-w$ is in the kernel, contradiction, (2) if $T(C)$ is linearly dependent, 
taking the inverse image of that linear combination gives a linear combination of $C$ in the kernel, 
contradiction. Hence $T$ is injective on $C$ and $T(C)$ is linearly independent.%
\end{isamarkuptxt}%
\isamarkuptrue%
\ \ \isacommand{have}\isamarkupfalse%
\ lc{\isacharunderscore}in{\isacharunderscore}ker{\isacharcolon}\ {\isachardoublequoteopen}{\isasymAnd}d\ D\ v{\isachardot}\ {\isasymlbrakk}D{\isasymsubseteq}C{\isacharsemicolon}\ d{\isasymin}D{\isasymrightarrow}carrier\ K{\isacharsemicolon}\ T\ {\isacharparenleft}V{\isachardot}module{\isachardot}lincomb\ d\ D{\isacharparenright}\ {\isacharequal}\ {\isasymzero}\isactrlbsub W\isactrlesub {\isacharsemicolon}\ \isanewline
\ \ \ \ v{\isasymin}D{\isacharsemicolon}\ d\ v\ {\isasymnoteq}{\isasymzero}\isactrlbsub K\isactrlesub {\isasymrbrakk}{\isasymLongrightarrow}False{\isachardoublequoteclose}\isanewline
\ \ \isacommand{proof}\isamarkupfalse%
\ {\isacharminus}\isanewline
\ \ \ \ \isacommand{fix}\isamarkupfalse%
\ d\ D\ v\isanewline
\ \ \ \ \isacommand{assume}\isamarkupfalse%
\ D{\isacharcolon}\ {\isachardoublequoteopen}D{\isasymsubseteq}C{\isachardoublequoteclose}\ \isakeyword{and}\ d{\isacharcolon}\ {\isachardoublequoteopen}d{\isasymin}D{\isasymrightarrow}carrier\ K{\isachardoublequoteclose}\ \isakeyword{and}\ T{\isadigit{0}}{\isacharcolon}\ {\isachardoublequoteopen}T\ {\isacharparenleft}V{\isachardot}module{\isachardot}lincomb\ d\ D{\isacharparenright}\ {\isacharequal}\ {\isasymzero}\isactrlbsub W\isactrlesub {\isachardoublequoteclose}\ \isanewline
\ \ \ \ \ \ \isakeyword{and}\ v{\isacharcolon}\ {\isachardoublequoteopen}v{\isasymin}D{\isachardoublequoteclose}\ \isakeyword{and}\ dvnz{\isacharcolon}\ {\isachardoublequoteopen}d\ v\ {\isasymnoteq}{\isasymzero}\isactrlbsub K\isactrlesub {\isachardoublequoteclose}\isanewline
\ \ \ \ \isacommand{from}\isamarkupfalse%
\ D\ Cfin\ \isacommand{have}\isamarkupfalse%
\ Dfin{\isacharcolon}\ {\isachardoublequoteopen}finite\ D{\isachardoublequoteclose}\ \ \isacommand{by}\isamarkupfalse%
\ {\isacharparenleft}auto\ intro{\isacharcolon}\ finite{\isacharunderscore}subset{\isacharparenright}\isanewline
\ \ \ \ \isacommand{from}\isamarkupfalse%
\ D\ CinC\ \isacommand{have}\isamarkupfalse%
\ DinC{\isacharcolon}\ {\isachardoublequoteopen}D{\isasymsubseteq}carrier\ V{\isachardoublequoteclose}\ \isacommand{by}\isamarkupfalse%
\ auto\isanewline
\ \ \ \ \isacommand{from}\isamarkupfalse%
\ T{\isadigit{0}}\ d\ Dfin\ DinC\ \isacommand{have}\isamarkupfalse%
\ lc{\isacharunderscore}d{\isacharcolon}\ {\isachardoublequoteopen}V{\isachardot}module{\isachardot}lincomb\ d\ D{\isasymin}kerT{\isachardoublequoteclose}\ \isanewline
\ \ \ \ \ \ \isacommand{by}\isamarkupfalse%
\ {\isacharparenleft}unfold\ ker{\isacharunderscore}def{\isacharcomma}\ auto{\isacharparenright}\isanewline
\ \ \ \ \isacommand{from}\isamarkupfalse%
\ lc{\isacharunderscore}d\ spanA\ AinC\ \isacommand{have}\isamarkupfalse%
\ \ {\isachardoublequoteopen}{\isasymexists}a{\isacharprime}\ A{\isacharprime}{\isachardot}\ A{\isacharprime}{\isasymsubseteq}A\ {\isasymand}\ a{\isacharprime}{\isasymin}A{\isacharprime}{\isasymrightarrow}carrier\ K\ {\isasymand}\isanewline
\ \ \ \ \ \ \ V{\isachardot}module{\isachardot}lincomb\ a{\isacharprime}\ A{\isacharprime}{\isacharequal}\ V{\isachardot}module{\isachardot}lincomb\ d\ D{\isachardoublequoteclose}\ \isanewline
\ \ \ \ \ \ \isacommand{by}\isamarkupfalse%
\ {\isacharparenleft}intro\ V{\isachardot}module{\isachardot}in{\isacharunderscore}span{\isacharcomma}\ auto{\isacharparenright}\isanewline
\ \ \ \ \isacommand{then}\isamarkupfalse%
\ \isacommand{obtain}\isamarkupfalse%
\ a{\isacharprime}\ A{\isacharprime}\ \isakeyword{where}\ a{\isacharprime}{\isacharcolon}\ {\isachardoublequoteopen}A{\isacharprime}{\isasymsubseteq}A\ {\isasymand}\ a{\isacharprime}{\isasymin}A{\isacharprime}{\isasymrightarrow}carrier\ K\ {\isasymand}\isanewline
\ \ \ \ \ \ V{\isachardot}module{\isachardot}lincomb\ d\ D\ {\isacharequal}\ V{\isachardot}module{\isachardot}lincomb\ a{\isacharprime}\ A{\isacharprime}{\isachardoublequoteclose}\ \isanewline
\ \ \ \ \ \ \isacommand{by}\isamarkupfalse%
\ metis\isanewline
\ \ \ \ \isacommand{hence}\isamarkupfalse%
\ \ A{\isacharprime}sub{\isacharcolon}\ {\isachardoublequoteopen}A{\isacharprime}{\isasymsubseteq}A{\isachardoublequoteclose}\ \isakeyword{and}\ a{\isacharprime}fun{\isacharcolon}\ {\isachardoublequoteopen}a{\isacharprime}{\isasymin}A{\isacharprime}{\isasymrightarrow}carrier\ K{\isachardoublequoteclose}\ \isanewline
\ \ \ \ \ \ \isakeyword{and}\ a{\isacharprime}{\isacharunderscore}lc{\isacharcolon}{\isachardoublequoteopen}V{\isachardot}module{\isachardot}lincomb\ d\ D\ {\isacharequal}\ V{\isachardot}module{\isachardot}lincomb\ a{\isacharprime}\ A{\isacharprime}{\isachardoublequoteclose}\ \ \isacommand{by}\isamarkupfalse%
\ auto\isanewline
\ \ \ \ \isacommand{from}\isamarkupfalse%
\ a{\isacharprime}\ finA\ Dfin\ \isacommand{have}\isamarkupfalse%
\ A{\isacharprime}fin{\isacharcolon}\ {\isachardoublequoteopen}finite\ {\isacharparenleft}A{\isacharprime}{\isacharparenright}{\isachardoublequoteclose}\ \ \isacommand{by}\isamarkupfalse%
\ {\isacharparenleft}auto\ intro{\isacharcolon}\ finite{\isacharunderscore}subset{\isacharparenright}\isanewline
\ \ \ \ \isacommand{from}\isamarkupfalse%
\ AinC\ A{\isacharprime}sub\ \isacommand{have}\isamarkupfalse%
\ A{\isacharprime}inC{\isacharcolon}\ {\isachardoublequoteopen}A{\isacharprime}{\isasymsubseteq}carrier\ V{\isachardoublequoteclose}\ \isacommand{by}\isamarkupfalse%
\ auto\isanewline
\ \ \ \ \isacommand{let}\isamarkupfalse%
\ {\isacharquery}e\ {\isacharequal}\ {\isachardoublequoteopen}{\isacharparenleft}{\isasymlambda}v{\isachardot}\ if\ v\ {\isasymin}\ A{\isacharprime}\ then\ a{\isacharprime}\ v\ else\ {\isasymominus}\isactrlbsub K\isactrlesub {\isasymone}\isactrlbsub K\isactrlesub {\isasymotimes}\isactrlbsub K\isactrlesub \ d\ v{\isacharparenright}{\isachardoublequoteclose}\isanewline
\ \ \ \ \isacommand{from}\isamarkupfalse%
\ a{\isacharprime}fun\ d\ \isacommand{have}\isamarkupfalse%
\ e{\isacharunderscore}fun{\isacharcolon}\ {\isachardoublequoteopen}{\isacharquery}e\ {\isasymin}\ A{\isacharprime}\ {\isasymunion}\ D\ {\isasymrightarrow}\ carrier\ K{\isachardoublequoteclose}\ \isanewline
\ \ \ \ \ \ \isacommand{apply}\isamarkupfalse%
\ {\isacharparenleft}unfold\ Pi{\isacharunderscore}def{\isacharparenright}\ \isanewline
\ \ \ \ \ \ \isacommand{by}\isamarkupfalse%
\ auto\isanewline
\ \ \ \ \isacommand{from}\isamarkupfalse%
\isanewline
\ \ \ \ \ \ A{\isacharprime}fin\ Dfin\ \isanewline
\ \ \ \ \ \ A{\isacharprime}inC\ DinC\ \isanewline
\ \ \ \ \ \ a{\isacharprime}fun\ d\ e{\isacharunderscore}fun\ \isanewline
\ \ \ \ \ \ disj\ D\ A{\isacharprime}sub\ \isanewline
\ \ \ \ \isacommand{have}\isamarkupfalse%
\ lccomp{\isadigit{1}}{\isacharcolon}\isanewline
\ \ \ \ \ \ {\isachardoublequoteopen}V{\isachardot}module{\isachardot}lincomb\ a{\isacharprime}\ A{\isacharprime}\ {\isasymoplus}\isactrlbsub V\isactrlesub \ {\isasymominus}\isactrlbsub K\isactrlesub {\isasymone}\isactrlbsub K\isactrlesub {\isasymodot}\isactrlbsub V\isactrlesub \ V{\isachardot}module{\isachardot}lincomb\ d\ D\ {\isacharequal}\ \isanewline
\ \ \ \ \ \ \ \ V{\isachardot}module{\isachardot}lincomb\ {\isacharparenleft}{\isasymlambda}v{\isachardot}\ if\ v{\isasymin}A{\isacharprime}\ then\ a{\isacharprime}\ v\ else\ {\isasymominus}\isactrlbsub K\isactrlesub {\isasymone}\isactrlbsub K\isactrlesub {\isasymotimes}\isactrlbsub K\isactrlesub \ d\ v{\isacharparenright}\ {\isacharparenleft}A{\isacharprime}{\isasymunion}D{\isacharparenright}{\isachardoublequoteclose}\isanewline
\ \ \ \ \ \ \isacommand{apply}\isamarkupfalse%
\ {\isacharparenleft}subst\ sym{\isacharbrackleft}OF\ V{\isachardot}module{\isachardot}lincomb{\isacharunderscore}smult{\isacharbrackright}{\isacharparenright}\isanewline
\ \ \ \ \ \ \ \ \ \ \isacommand{apply}\isamarkupfalse%
\ {\isacharparenleft}simp{\isacharunderscore}all{\isacharparenright}\isanewline
\ \ \ \ \ \ \isacommand{apply}\isamarkupfalse%
\ {\isacharparenleft}subst\ V{\isachardot}module{\isachardot}lincomb{\isacharunderscore}union{\isadigit{2}}{\isacharparenright}\isanewline
\ \ \ \ \ \ \ \ \ \ \ \isacommand{by}\isamarkupfalse%
\ {\isacharparenleft}auto{\isacharparenright}\isanewline
\ \ \ \ \isacommand{from}\isamarkupfalse%
\isanewline
\ \ \ \ \ \ A{\isacharprime}fin\ \isanewline
\ \ \ \ \ \ A{\isacharprime}inC\ \isanewline
\ \ \ \ \ \ a{\isacharprime}fun\ \isanewline
\ \ \ \ \isacommand{have}\isamarkupfalse%
\ lccomp{\isadigit{2}}{\isacharcolon}\ \isanewline
\ \ \ \ \ \ {\isachardoublequoteopen}V{\isachardot}module{\isachardot}lincomb\ a{\isacharprime}\ A{\isacharprime}\ {\isasymoplus}\isactrlbsub V\isactrlesub \ {\isasymominus}\isactrlbsub K\isactrlesub {\isasymone}\isactrlbsub K\isactrlesub {\isasymodot}\isactrlbsub V\isactrlesub \ V{\isachardot}module{\isachardot}lincomb\ d\ D\ {\isacharequal}\ \isanewline
\ \ \ \ \ \ {\isasymzero}\isactrlbsub V\isactrlesub {\isachardoublequoteclose}\ \isanewline
\ \ \ \ \ \ \isacommand{by}\isamarkupfalse%
\ {\isacharparenleft}simp\ add{\isacharcolon}\ a{\isacharprime}{\isacharunderscore}lc\ \isanewline
\ \ \ \ \ \ \ \ V{\isachardot}module{\isachardot}smult{\isacharunderscore}minus{\isacharunderscore}{\isadigit{1}}\ \ V{\isachardot}module{\isachardot}M{\isachardot}r{\isacharunderscore}neg{\isacharparenright}\isanewline
\ \ \ \ \isacommand{from}\isamarkupfalse%
\ lccomp{\isadigit{1}}\ lccomp{\isadigit{2}}\ \isacommand{have}\isamarkupfalse%
\ lc{\isadigit{0}}{\isacharcolon}\ {\isachardoublequoteopen}V{\isachardot}module{\isachardot}lincomb\ {\isacharparenleft}{\isasymlambda}v{\isachardot}\ if\ v{\isasymin}A{\isacharprime}\ then\ a{\isacharprime}\ v\ else\ {\isasymominus}\isactrlbsub K\isactrlesub {\isasymone}\isactrlbsub K\isactrlesub {\isasymotimes}\isactrlbsub K\isactrlesub \ d\ v{\isacharparenright}\ {\isacharparenleft}A{\isacharprime}{\isasymunion}D{\isacharparenright}\isanewline
\ \ \ \ \ \ {\isacharequal}{\isasymzero}\isactrlbsub V\isactrlesub {\isachardoublequoteclose}\ \isacommand{by}\isamarkupfalse%
\ auto\isanewline
\ \ \ \ \isacommand{from}\isamarkupfalse%
\ disj\ a{\isacharprime}\ v\ D\ \isacommand{have}\isamarkupfalse%
\ v{\isacharunderscore}nin{\isacharcolon}\ {\isachardoublequoteopen}v{\isasymnotin}A{\isacharprime}{\isachardoublequoteclose}\ \isacommand{by}\isamarkupfalse%
\ auto\isanewline
\ \ \ \ \isacommand{from}\isamarkupfalse%
\ A{\isacharprime}fin\ Dfin\ \isanewline
\ \ \ \ \ \ A{\isacharprime}inC\ DinC\ \isanewline
\ \ \ \ \ \ e{\isacharunderscore}fun\ d\ \isanewline
\ \ \ \ \ \ A{\isacharprime}sub\ D\ disj\ \isanewline
\ \ \ \ \ \ v\ dvnz\ \isanewline
\ \ \ \ \ \ lc{\isadigit{0}}\isanewline
\ \ \ \ \isacommand{have}\isamarkupfalse%
\ AC{\isacharunderscore}ld{\isacharcolon}\ {\isachardoublequoteopen}V{\isachardot}module{\isachardot}lin{\isacharunderscore}dep\ {\isacharparenleft}A{\isasymunion}C{\isacharparenright}{\isachardoublequoteclose}\ \isanewline
\ \ \ \ \ \ \isacommand{apply}\isamarkupfalse%
\ {\isacharparenleft}intro\ V{\isachardot}module{\isachardot}lin{\isacharunderscore}dep{\isacharunderscore}crit{\isacharbrackleft}\isakeyword{where}\ {\isacharquery}A{\isacharequal}{\isachardoublequoteopen}A{\isacharprime}{\isasymunion}D{\isachardoublequoteclose}\ \isakeyword{and}\ \isanewline
\ \ \ \ \ \ \ \ {\isacharquery}S{\isacharequal}{\isachardoublequoteopen}A{\isasymunion}C{\isachardoublequoteclose}\ \isakeyword{and}\ {\isacharquery}a{\isacharequal}{\isachardoublequoteopen}{\isasymlambda}v{\isachardot}\ if\ v{\isasymin}A{\isacharprime}\ then\ a{\isacharprime}\ v\ else\ {\isasymominus}\isactrlbsub K\isactrlesub {\isasymone}\isactrlbsub K\isactrlesub {\isasymotimes}\isactrlbsub K\isactrlesub \ d\ v{\isachardoublequoteclose}\ \isakeyword{and}\ {\isacharquery}v{\isacharequal}{\isachardoublequoteopen}v{\isachardoublequoteclose}{\isacharbrackright}{\isacharparenright}\isanewline
\ \ \ \ \ \ \ \ \ \ \ \isacommand{by}\isamarkupfalse%
\ {\isacharparenleft}auto\ dest{\isacharcolon}\ integral{\isacharparenright}\isanewline
\ \ \ \ \isacommand{from}\isamarkupfalse%
\ AC{\isacharunderscore}ld\ ACbasis\ \isacommand{show}\isamarkupfalse%
\ False\ \isacommand{by}\isamarkupfalse%
\ {\isacharparenleft}unfold\ V{\isachardot}basis{\isacharunderscore}def{\isacharcomma}\ auto{\isacharparenright}\isanewline
\ \ \isacommand{qed}\isamarkupfalse%
\isanewline
\ \ \isacommand{have}\isamarkupfalse%
\ C{\isacharprime}{\isacharunderscore}card{\isacharcolon}\ {\isachardoublequoteopen}inj{\isacharunderscore}on\ T\ C{\isachardoublequoteclose}\ {\isachardoublequoteopen}card\ C\ {\isacharequal}\ card\ {\isacharquery}C{\isacharprime}{\isachardoublequoteclose}\isanewline
\ \ \isacommand{proof}\isamarkupfalse%
\ {\isacharminus}\isanewline
\ \ \ \ \isacommand{show}\isamarkupfalse%
\ {\isachardoublequoteopen}inj{\isacharunderscore}on\ T\ C{\isachardoublequoteclose}\isanewline
\ \ \ \ \isacommand{proof}\isamarkupfalse%
\ {\isacharparenleft}rule\ ccontr{\isacharparenright}\isanewline
\ \ \ \ \ \ \isacommand{assume}\isamarkupfalse%
\ {\isachardoublequoteopen}{\isasymnot}inj{\isacharunderscore}on\ T\ C{\isachardoublequoteclose}\isanewline
\ \ \ \ \ \ \isacommand{then}\isamarkupfalse%
\ \isacommand{obtain}\isamarkupfalse%
\ v\ w\ \isakeyword{where}\ {\isachardoublequoteopen}v{\isasymin}C{\isachardoublequoteclose}\ {\isachardoublequoteopen}w{\isasymin}C{\isachardoublequoteclose}\ {\isachardoublequoteopen}v{\isasymnoteq}w{\isachardoublequoteclose}\ {\isachardoublequoteopen}T\ v\ {\isacharequal}\ T\ w{\isachardoublequoteclose}\ \isacommand{by}\isamarkupfalse%
\ {\isacharparenleft}unfold\ inj{\isacharunderscore}on{\isacharunderscore}def{\isacharcomma}\ auto{\isacharparenright}\ \isanewline
\ \ \ \ \ \ \isacommand{from}\isamarkupfalse%
\ this\ CinC\ \isacommand{show}\isamarkupfalse%
\ False\ \isanewline
\ \ \ \ \ \ \ \ \isacommand{apply}\isamarkupfalse%
\ {\isacharparenleft}intro\ lc{\isacharunderscore}in{\isacharunderscore}ker{\isacharbrackleft}\isakeyword{where}\ {\isacharquery}D{\isacharequal}{\isachardoublequoteopen}{\isacharbraceleft}v{\isacharcomma}w{\isacharbraceright}{\isachardoublequoteclose}\ \isakeyword{and}\ {\isacharquery}d{\isacharequal}{\isachardoublequoteopen}{\isasymlambda}x{\isachardot}\ if\ x{\isacharequal}v\ then\ {\isasymone}\isactrlbsub K\isactrlesub \ else\ {\isasymominus}\isactrlbsub K\isactrlesub {\isasymone}\isactrlbsub K\isactrlesub {\isachardoublequoteclose}\isanewline
\ \ \ \ \ \ \ \ \ \ \isakeyword{and}\ {\isacharquery}v{\isacharequal}{\isachardoublequoteopen}v{\isachardoublequoteclose}{\isacharbrackright}{\isacharparenright}\isanewline
\ \ \ \ \ \ \ \ \ \ \ \ \isacommand{by}\isamarkupfalse%
\ {\isacharparenleft}auto\ simp\ add{\isacharcolon}\ V{\isachardot}module{\isachardot}lincomb{\isacharunderscore}def\ hom{\isacharunderscore}sum\ ring{\isacharunderscore}subset{\isacharunderscore}carrier\ \isanewline
\ \ \ \ \ \ \ \ \ \ \ \ \ \ W{\isachardot}module{\isachardot}smult{\isacharunderscore}minus{\isacharunderscore}{\isadigit{1}}\ r{\isacharunderscore}neg\ T{\isacharunderscore}im{\isacharparenright}\isanewline
\ \ \ \ \isacommand{qed}\isamarkupfalse%
\isanewline
\ \ \ \ \isacommand{from}\isamarkupfalse%
\ this\ Cfin\ \isacommand{show}\isamarkupfalse%
\ {\isachardoublequoteopen}card\ C\ {\isacharequal}\ card\ {\isacharquery}C{\isacharprime}{\isachardoublequoteclose}\isanewline
\ \ \ \ \ \ \isacommand{by}\isamarkupfalse%
\ {\isacharparenleft}metis\ card{\isacharunderscore}image{\isacharparenright}\ \isanewline
\ \ \isacommand{qed}\isamarkupfalse%
\isanewline
\ \ \isacommand{let}\isamarkupfalse%
\ {\isacharquery}f{\isacharequal}{\isachardoublequoteopen}the{\isacharunderscore}inv{\isacharunderscore}into\ C\ T{\isachardoublequoteclose}\isanewline
\ \ \isacommand{have}\isamarkupfalse%
\ f{\isacharcolon}\ {\isachardoublequoteopen}{\isasymAnd}x{\isachardot}\ x{\isasymin}C\ {\isasymLongrightarrow}\ {\isacharquery}f\ {\isacharparenleft}T\ x{\isacharparenright}\ {\isacharequal}\ x{\isachardoublequoteclose}\ {\isachardoublequoteopen}{\isasymAnd}y{\isachardot}\ y{\isasymin}{\isacharquery}C{\isacharprime}\ {\isasymLongrightarrow}\ T\ {\isacharparenleft}{\isacharquery}f\ y{\isacharparenright}\ {\isacharequal}\ y{\isachardoublequoteclose}\isanewline
\ \ \ \ \isacommand{apply}\isamarkupfalse%
\ {\isacharparenleft}insert\ C{\isacharprime}{\isacharunderscore}card{\isacharparenleft}{\isadigit{1}}{\isacharparenright}{\isacharparenright}\isanewline
\ \ \ \ \ \isacommand{apply}\isamarkupfalse%
\ {\isacharparenleft}metis\ the{\isacharunderscore}inv{\isacharunderscore}into{\isacharunderscore}f{\isacharunderscore}f{\isacharparenright}\isanewline
\ \ \ \ \isacommand{by}\isamarkupfalse%
\ {\isacharparenleft}metis\ f{\isacharunderscore}the{\isacharunderscore}inv{\isacharunderscore}into{\isacharunderscore}f{\isacharparenright}\isanewline
\isanewline
\ \ \isacommand{have}\isamarkupfalse%
\ C{\isacharprime}{\isacharunderscore}li{\isacharcolon}\ {\isachardoublequoteopen}im{\isachardot}lin{\isacharunderscore}indpt\ {\isacharquery}C{\isacharprime}{\isachardoublequoteclose}\isanewline
\ \ \isacommand{proof}\isamarkupfalse%
\ {\isacharparenleft}rule\ ccontr{\isacharparenright}\isanewline
\ \ \ \ \isacommand{assume}\isamarkupfalse%
\ Cld{\isacharcolon}\ {\isachardoublequoteopen}{\isasymnot}im{\isachardot}lin{\isacharunderscore}indpt\ {\isacharquery}C{\isacharprime}{\isachardoublequoteclose}\isanewline
\ \ \ \ \isacommand{from}\isamarkupfalse%
\ Cld\ cim\ subs{\isacharunderscore}im\ \isacommand{have}\isamarkupfalse%
\ CldW{\isacharcolon}\ {\isachardoublequoteopen}W{\isachardot}module{\isachardot}lin{\isacharunderscore}dep\ {\isacharquery}C{\isacharprime}{\isachardoublequoteclose}\ \isanewline
\ \ \ \ \ \ \isacommand{apply}\isamarkupfalse%
\ {\isacharparenleft}subst\ sym{\isacharbrackleft}OF\ W{\isachardot}module{\isachardot}span{\isacharunderscore}li{\isacharunderscore}not{\isacharunderscore}depend{\isacharparenleft}{\isadigit{2}}{\isacharparenright}{\isacharbrackleft}\isakeyword{where}\ {\isacharquery}S{\isacharequal}{\isachardoublequoteopen}T{\isacharbackquote}C{\isachardoublequoteclose}\ \isakeyword{and}\ {\isacharquery}N{\isacharequal}{\isachardoublequoteopen}imT{\isachardoublequoteclose}{\isacharbrackright}{\isacharbrackright}{\isacharparenright}\ \isanewline
\ \ \ \ \ \ \ \ \isacommand{by}\isamarkupfalse%
\ auto\isanewline
\ \ \ \ \isacommand{from}\isamarkupfalse%
\ C\ CldW\ \isacommand{have}\isamarkupfalse%
\ {\isachardoublequoteopen}{\isasymexists}c{\isacharprime}\ v{\isacharprime}{\isachardot}\ {\isacharparenleft}c{\isacharprime}{\isasymin}\ {\isacharparenleft}{\isacharquery}C{\isacharprime}{\isasymrightarrow}carrier\ K{\isacharparenright}{\isacharparenright}\ {\isasymand}\ {\isacharparenleft}W{\isachardot}module{\isachardot}lincomb\ c{\isacharprime}\ {\isacharquery}C{\isacharprime}\ {\isacharequal}\ {\isasymzero}\isactrlbsub W\isactrlesub {\isacharparenright}\ \isanewline
\ \ \ \ \ \ {\isasymand}\ {\isacharparenleft}v{\isacharprime}{\isasymin}{\isacharquery}C{\isacharprime}{\isacharparenright}\ {\isasymand}\ {\isacharparenleft}c{\isacharprime}\ v{\isacharprime}{\isasymnoteq}\ {\isasymzero}\isactrlbsub K\isactrlesub {\isacharparenright}{\isachardoublequoteclose}\ \isacommand{by}\isamarkupfalse%
\ {\isacharparenleft}intro\ W{\isachardot}module{\isachardot}finite{\isacharunderscore}lin{\isacharunderscore}dep{\isacharcomma}\ auto{\isacharparenright}\isanewline
\ \ \ \ \isacommand{then}\isamarkupfalse%
\ \isacommand{obtain}\isamarkupfalse%
\ c{\isacharprime}\ v{\isacharprime}\ \isakeyword{where}\ c{\isacharprime}{\isacharcolon}\ {\isachardoublequoteopen}{\isacharparenleft}c{\isacharprime}{\isasymin}\ {\isacharparenleft}{\isacharquery}C{\isacharprime}{\isasymrightarrow}carrier\ K{\isacharparenright}{\isacharparenright}\ {\isasymand}\ {\isacharparenleft}W{\isachardot}module{\isachardot}lincomb\ c{\isacharprime}\ {\isacharquery}C{\isacharprime}\ {\isacharequal}\ {\isasymzero}\isactrlbsub W\isactrlesub {\isacharparenright}\ \isanewline
\ \ \ \ \ \ {\isasymand}\ {\isacharparenleft}v{\isacharprime}{\isasymin}{\isacharquery}C{\isacharprime}{\isacharparenright}\ {\isasymand}\ {\isacharparenleft}c{\isacharprime}\ v{\isacharprime}{\isasymnoteq}\ {\isasymzero}\isactrlbsub K\isactrlesub {\isacharparenright}{\isachardoublequoteclose}\ \isacommand{by}\isamarkupfalse%
\ auto\isanewline
\ \ \ \ \isacommand{hence}\isamarkupfalse%
\ c{\isacharprime}fun{\isacharcolon}\ {\isachardoublequoteopen}{\isacharparenleft}c{\isacharprime}{\isasymin}\ {\isacharparenleft}{\isacharquery}C{\isacharprime}{\isasymrightarrow}carrier\ K{\isacharparenright}{\isacharparenright}{\isachardoublequoteclose}\ \isakeyword{and}\ c{\isacharprime}lc{\isacharcolon}\ {\isachardoublequoteopen}{\isacharparenleft}W{\isachardot}module{\isachardot}lincomb\ c{\isacharprime}\ {\isacharquery}C{\isacharprime}\ {\isacharequal}\ {\isasymzero}\isactrlbsub W\isactrlesub {\isacharparenright}{\isachardoublequoteclose}\ \isakeyword{and}\ \isanewline
\ \ \ \ \ \ v{\isacharprime}{\isacharcolon}{\isachardoublequoteopen}{\isacharparenleft}v{\isacharprime}{\isasymin}{\isacharquery}C{\isacharprime}{\isacharparenright}{\isachardoublequoteclose}\ \isakeyword{and}\ cvnz{\isacharcolon}\ {\isachardoublequoteopen}{\isacharparenleft}c{\isacharprime}\ v{\isacharprime}{\isasymnoteq}\ {\isasymzero}\isactrlbsub K\isactrlesub {\isacharparenright}{\isachardoublequoteclose}\ \isacommand{by}\isamarkupfalse%
\ auto%
\begin{isamarkuptxt}%
We take the inverse image of $C'$ under $T$ to get a linear combination of $C$ that is 
in the kernel and hence a linear combination of $A$. This contradicts $A\cup C$ being linearly
independent.%
\end{isamarkuptxt}%
\isamarkuptrue%
\ \ \ \ \isacommand{let}\isamarkupfalse%
\ {\isacharquery}c{\isacharequal}{\isachardoublequoteopen}{\isasymlambda}v{\isachardot}\ c{\isacharprime}\ {\isacharparenleft}T\ v{\isacharparenright}{\isachardoublequoteclose}\isanewline
\ \ \ \ \isacommand{from}\isamarkupfalse%
\ c{\isacharprime}fun\ \isacommand{have}\isamarkupfalse%
\ c{\isacharunderscore}fun{\isacharcolon}\ {\isachardoublequoteopen}{\isacharquery}c{\isasymin}\ C{\isasymrightarrow}carrier\ K{\isachardoublequoteclose}\ \isacommand{by}\isamarkupfalse%
\ auto\isanewline
\ \ \ \ \isacommand{from}\isamarkupfalse%
\ Cfin\ \isanewline
\ \ \ \ \ \ c{\isacharunderscore}fun\ c{\isacharprime}fun\ \isanewline
\ \ \ \ \ \ C{\isacharprime}{\isacharunderscore}card\ \isanewline
\ \ \ \ \ \ CinC\ \isanewline
\ \ \ \ \ \ f\ \ \isanewline
\ \ \ \ \ \ c{\isacharprime}lc\ \isanewline
\ \ \ \ \isacommand{have}\isamarkupfalse%
\ T{\isacharunderscore}lc{\isacharunderscore}{\isadigit{0}}{\isacharcolon}\ {\isachardoublequoteopen}T\ {\isacharparenleft}V{\isachardot}module{\isachardot}lincomb\ {\isacharquery}c\ C{\isacharparenright}\ {\isacharequal}\ {\isasymzero}\isactrlbsub W\isactrlesub {\isachardoublequoteclose}\isanewline
\ \ \ \ \ \ \isacommand{apply}\isamarkupfalse%
\ {\isacharparenleft}unfold\ V{\isachardot}module{\isachardot}lincomb{\isacharunderscore}def\ W{\isachardot}module{\isachardot}lincomb{\isacharunderscore}def{\isacharparenright}\isanewline
\ \ \ \ \ \ \isacommand{apply}\isamarkupfalse%
\ {\isacharparenleft}subst\ hom{\isacharunderscore}sum{\isacharcomma}\ auto{\isacharparenright}\isanewline
\ \ \ \ \ \ \isacommand{apply}\isamarkupfalse%
\ {\isacharparenleft}drule\ Pi{\isacharunderscore}implies{\isacharunderscore}Pi{\isadigit{2}}{\isacharparenright}{\isacharplus}\isanewline
\ \ \ \ \ \ \isacommand{apply}\isamarkupfalse%
\ {\isacharparenleft}auto\ cong{\isacharcolon}\ finsum{\isacharunderscore}cong\ simp\ add{\isacharcolon}\ T{\isacharunderscore}smult\ ring{\isacharunderscore}subset{\isacharunderscore}carrier\ Pi{\isacharunderscore}simp{\isacharparenright}\isanewline
\ \ \ \ \ \ \isacommand{apply}\isamarkupfalse%
\ {\isacharparenleft}subst\ finsum{\isacharunderscore}reindex{\isacharbrackleft}\isakeyword{where}\ {\isacharquery}f{\isacharequal}{\isachardoublequoteopen}{\isasymlambda}w{\isachardot}\ c{\isacharprime}\ w\ {\isasymodot}\isactrlbsub W\isactrlesub \ w{\isachardoublequoteclose}\ \isakeyword{and}\ {\isacharquery}h{\isacharequal}{\isachardoublequoteopen}T{\isachardoublequoteclose}\ \isakeyword{and}\ {\isacharquery}A{\isacharequal}{\isachardoublequoteopen}C{\isachardoublequoteclose}{\isacharcomma}\ THEN\ sym{\isacharbrackright}{\isacharparenright}\isanewline
\ \ \ \ \ \ \ \ \ \isacommand{by}\isamarkupfalse%
\ {\isacharparenleft}auto\ simp\ add{\isacharcolon}\ Pi{\isacharunderscore}simp{\isacharparenright}\isanewline
\ \ \ \ \isacommand{from}\isamarkupfalse%
\ f\ c{\isacharprime}fun\ cvnz\ v{\isacharprime}\ T{\isacharunderscore}lc{\isacharunderscore}{\isadigit{0}}\ \isacommand{show}\isamarkupfalse%
\ False\isanewline
\ \ \ \ \ \ \isacommand{by}\isamarkupfalse%
\ {\isacharparenleft}intro\ lc{\isacharunderscore}in{\isacharunderscore}ker{\isacharbrackleft}\isakeyword{where}\ {\isacharquery}D{\isacharequal}{\isachardoublequoteopen}C{\isachardoublequoteclose}\ \isakeyword{and}\ {\isacharquery}d{\isacharequal}{\isachardoublequoteopen}{\isacharquery}c{\isachardoublequoteclose}\ \isakeyword{and}\ {\isacharquery}v{\isacharequal}{\isachardoublequoteopen}{\isacharquery}f\ v{\isacharprime}{\isachardoublequoteclose}{\isacharbrackright}{\isacharcomma}\ auto{\isacharparenright}\isanewline
\ \ \isacommand{qed}\isamarkupfalse%
\isanewline
\ \ \isacommand{have}\isamarkupfalse%
\ C{\isacharprime}{\isacharunderscore}gen{\isacharcolon}\ {\isachardoublequoteopen}im{\isachardot}gen{\isacharunderscore}set\ {\isacharquery}C{\isacharprime}{\isachardoublequoteclose}\isanewline
\ \ \isacommand{proof}\isamarkupfalse%
\ {\isacharminus}\ \isanewline
\ \ \ \ \isacommand{have}\isamarkupfalse%
\ C{\isacharprime}{\isacharunderscore}span{\isacharcolon}\ {\isachardoublequoteopen}span\ {\isacharquery}C{\isacharprime}\ {\isacharequal}\ imT{\isachardoublequoteclose}\isanewline
\ \ \ \ \isacommand{proof}\isamarkupfalse%
\ {\isacharparenleft}rule\ equalityI{\isacharparenright}\isanewline
\ \ \ \ \ \ \isacommand{from}\isamarkupfalse%
\ cim\ subs{\isacharunderscore}im\ \isacommand{show}\isamarkupfalse%
\ {\isachardoublequoteopen}W{\isachardot}module{\isachardot}span\ {\isacharquery}C{\isacharprime}\ {\isasymsubseteq}\ imT{\isachardoublequoteclose}\isanewline
\ \ \ \ \ \ \ \ \isacommand{by}\isamarkupfalse%
\ {\isacharparenleft}intro\ span{\isacharunderscore}is{\isacharunderscore}subset{\isacharcomma}\ unfold\ subspace{\isacharunderscore}def{\isacharcomma}\ auto{\isacharparenright}\isanewline
\ \ \ \ \isacommand{next}\isamarkupfalse%
\isanewline
\ \ \ \ \ \ \isacommand{show}\isamarkupfalse%
\ {\isachardoublequoteopen}imT{\isasymsubseteq}W{\isachardot}module{\isachardot}span\ {\isacharquery}C{\isacharprime}{\isachardoublequoteclose}\isanewline
\ \ \ \ \ \ \isacommand{proof}\isamarkupfalse%
\ {\isacharparenleft}auto{\isacharparenright}\isanewline
\ \ \ \ \ \ \ \ \isacommand{fix}\isamarkupfalse%
\ w\isanewline
\ \ \ \ \ \ \ \ \isacommand{assume}\isamarkupfalse%
\ w{\isacharcolon}\ {\isachardoublequoteopen}w{\isasymin}imT{\isachardoublequoteclose}\isanewline
\ \ \ \ \ \ \ \ \isacommand{from}\isamarkupfalse%
\ this\ finA\ Cfin\ AinC\ CinC\ \isacommand{obtain}\isamarkupfalse%
\ v\ \isakeyword{where}\ v{\isacharunderscore}inC{\isacharcolon}\ {\isachardoublequoteopen}v{\isasymin}carrier\ V{\isachardoublequoteclose}\ \isakeyword{and}\ w{\isacharunderscore}eq{\isacharunderscore}T{\isacharunderscore}v{\isacharcolon}\ {\isachardoublequoteopen}w{\isacharequal}\ T\ v{\isachardoublequoteclose}\ \isanewline
\ \ \ \ \ \ \ \ \ \ \isacommand{by}\isamarkupfalse%
\ {\isacharparenleft}unfold\ im{\isacharunderscore}def\ image{\isacharunderscore}def{\isacharcomma}\ auto{\isacharparenright}\isanewline
\ \ \ \ \ \ \ \ \isacommand{from}\isamarkupfalse%
\ finA\ Cfin\ AinC\ CinC\ v{\isacharunderscore}inC\ ACgen\ \isacommand{have}\isamarkupfalse%
\ {\isachardoublequoteopen}{\isasymexists}a{\isachardot}\ \ a\ {\isasymin}\ A{\isasymunion}C\ {\isasymrightarrow}\ carrier\ K{\isasymand}\ V{\isachardot}module{\isachardot}lincomb\ a\ {\isacharparenleft}A{\isasymunion}C{\isacharparenright}\ {\isacharequal}\ v{\isachardoublequoteclose}\isanewline
\ \ \ \ \ \ \ \ \ \ \isacommand{by}\isamarkupfalse%
\ {\isacharparenleft}intro\ V{\isachardot}module{\isachardot}finite{\isacharunderscore}in{\isacharunderscore}span{\isacharcomma}\ auto{\isacharparenright}\isanewline
\ \ \ \ \ \ \ \ \isacommand{then}\isamarkupfalse%
\ \isacommand{obtain}\isamarkupfalse%
\ a\ \isakeyword{where}\ \isanewline
\ \ \ \ \ \ \ \ \ \ a{\isacharunderscore}fun{\isacharcolon}\ {\isachardoublequoteopen}a\ {\isasymin}\ A{\isasymunion}C\ {\isasymrightarrow}\ carrier\ K{\isachardoublequoteclose}\ \isakeyword{and}\isanewline
\ \ \ \ \ \ \ \ \ \ lc{\isacharunderscore}a{\isacharunderscore}v{\isacharcolon}\ {\isachardoublequoteopen}v{\isacharequal}\ V{\isachardot}module{\isachardot}lincomb\ a\ {\isacharparenleft}A{\isasymunion}C{\isacharparenright}{\isachardoublequoteclose}\isanewline
\ \ \ \ \ \ \ \ \ \ \isacommand{by}\isamarkupfalse%
\ auto\isanewline
\ \ \ \ \ \ \ \ \isacommand{let}\isamarkupfalse%
\ {\isacharquery}a{\isacharprime}{\isacharequal}{\isachardoublequoteopen}{\isasymlambda}v{\isachardot}\ a\ {\isacharparenleft}{\isacharquery}f\ v{\isacharparenright}{\isachardoublequoteclose}\isanewline
\ \ \ \ \ \ \ \ \isacommand{from}\isamarkupfalse%
\ finA\ Cfin\ AinC\ CinC\ a{\isacharunderscore}fun\ disj\ Ainker\ f\ C{\isacharprime}{\isacharunderscore}card\ \isacommand{have}\isamarkupfalse%
\ Tv{\isacharcolon}\ {\isachardoublequoteopen}T\ v\ {\isacharequal}\ W{\isachardot}module{\isachardot}lincomb\ {\isacharquery}a{\isacharprime}\ {\isacharquery}C{\isacharprime}{\isachardoublequoteclose}\isanewline
\ \ \ \ \ \ \ \ \ \ \isacommand{apply}\isamarkupfalse%
\ {\isacharparenleft}subst\ lc{\isacharunderscore}a{\isacharunderscore}v{\isacharparenright}\isanewline
\ \ \ \ \ \ \ \ \ \ \isacommand{apply}\isamarkupfalse%
\ {\isacharparenleft}subst\ V{\isachardot}module{\isachardot}lincomb{\isacharunderscore}union{\isacharcomma}\ simp{\isacharunderscore}all{\isacharparenright}\ \isanewline
\ \ \ \ \ \ \ \ \ \ \isacommand{apply}\isamarkupfalse%
\ {\isacharparenleft}unfold\ lincomb{\isacharunderscore}def\ V{\isachardot}module{\isachardot}lincomb{\isacharunderscore}def{\isacharparenright}\isanewline
\ \ \ \ \ \ \ \ \ \ \isacommand{apply}\isamarkupfalse%
\ {\isacharparenleft}subst\ hom{\isacharunderscore}sum{\isacharcomma}\ auto{\isacharparenright}\ \isanewline
\ \ \ \ \ \ \ \ \ \ \isacommand{apply}\isamarkupfalse%
\ {\isacharparenleft}drule\ Pi{\isacharunderscore}implies{\isacharunderscore}Pi{\isadigit{2}}{\isacharparenright}{\isacharplus}\isanewline
\ \ \ \ \ \ \ \ \ \ \isacommand{apply}\isamarkupfalse%
\ {\isacharparenleft}simp\ add{\isacharcolon}\ \ Pi{\isacharunderscore}simp\ subsetD\isanewline
\ \ \ \ \ \ \ \ \ \ \ \ hom{\isacharunderscore}sum\ \isanewline
\ \ \ \ \ \ \ \ \ \ \ \ T{\isacharunderscore}ker\ \isanewline
\ \ \ \ \ \ \ \ \ \ \ \ {\isacharparenright}\ \isanewline
\ \ \ \ \ \ \ \ \ \ \isacommand{apply}\isamarkupfalse%
\ {\isacharparenleft}subst\ finsum{\isacharunderscore}reindex{\isacharbrackleft}\isakeyword{where}\ {\isacharquery}h{\isacharequal}{\isachardoublequoteopen}T{\isachardoublequoteclose}\ \isakeyword{and}\ {\isacharquery}f{\isacharequal}{\isachardoublequoteopen}{\isasymlambda}v{\isachardot}\ {\isacharquery}a{\isacharprime}\ v{\isasymodot}\isactrlbsub W\isactrlesub \ v{\isachardoublequoteclose}{\isacharbrackright}{\isacharcomma}\ auto{\isacharparenright}\isanewline
\ \ \ \ \ \ \ \ \ \ \ \isacommand{by}\isamarkupfalse%
\ {\isacharparenleft}auto\ cong{\isacharcolon}\ finsum{\isacharunderscore}cong\ simp\ add{\isacharcolon}\ \ Pi{\isacharunderscore}simp\ ring{\isacharunderscore}subset{\isacharunderscore}carrier{\isacharparenright}\isanewline
\ \ \ \ \ \ \ \ \isacommand{from}\isamarkupfalse%
\ a{\isacharunderscore}fun\ f\ \isacommand{have}\isamarkupfalse%
\ a{\isacharprime}{\isacharunderscore}fun{\isacharcolon}\ {\isachardoublequoteopen}{\isacharquery}a{\isacharprime}{\isasymin}{\isacharquery}C{\isacharprime}\ {\isasymrightarrow}\ carrier\ K{\isachardoublequoteclose}\ \isacommand{by}\isamarkupfalse%
\ auto\isanewline
\ \ \ \ \ \ \ \ \isacommand{from}\isamarkupfalse%
\ C{\isacharprime}fin\ CinC\ this\ w{\isacharunderscore}eq{\isacharunderscore}T{\isacharunderscore}v\ a{\isacharprime}{\isacharunderscore}fun\ Tv\ \isacommand{show}\isamarkupfalse%
\ {\isachardoublequoteopen}w\ {\isasymin}\ LinearCombinations{\isachardot}module{\isachardot}span\ K\ W\ {\isacharparenleft}T\ {\isacharbackquote}\ C{\isacharparenright}{\isachardoublequoteclose}\ \isanewline
\ \ \ \ \ \ \ \ \ \ \isacommand{by}\isamarkupfalse%
\ {\isacharparenleft}subst\ finite{\isacharunderscore}span{\isacharcomma}\ auto{\isacharparenright}\isanewline
\ \ \ \ \ \ \isacommand{qed}\isamarkupfalse%
\isanewline
\ \ \ \ \isacommand{qed}\isamarkupfalse%
\isanewline
\ \ \ \ \isacommand{from}\isamarkupfalse%
\ this\ subs{\isacharunderscore}im\ CinC\ \isacommand{show}\isamarkupfalse%
\ {\isacharquery}thesis\ \isanewline
\ \ \ \ \ \ \isacommand{apply}\isamarkupfalse%
\ {\isacharparenleft}subst\ span{\isacharunderscore}li{\isacharunderscore}not{\isacharunderscore}depend{\isacharparenleft}{\isadigit{1}}{\isacharparenright}{\isacharparenright}\isanewline
\ \ \ \ \ \ \ \ \isacommand{by}\isamarkupfalse%
\ {\isacharparenleft}unfold\ im{\isacharunderscore}def\ subspace{\isacharunderscore}def{\isacharcomma}\ auto{\isacharparenright}\isanewline
\ \ \isacommand{qed}\isamarkupfalse%
\isanewline
\ \ \isacommand{from}\isamarkupfalse%
\ C{\isacharprime}{\isacharunderscore}li\ C{\isacharprime}{\isacharunderscore}gen\ C\ cim\ \isacommand{have}\isamarkupfalse%
\ C{\isacharprime}{\isacharunderscore}basis{\isacharcolon}\ {\isachardoublequoteopen}im{\isachardot}basis\ \ {\isacharparenleft}T{\isacharbackquote}C{\isacharparenright}{\isachardoublequoteclose}\ \isanewline
\ \ \ \ \isacommand{by}\isamarkupfalse%
\ {\isacharparenleft}unfold\ im{\isachardot}basis{\isacharunderscore}def{\isacharcomma}\ auto{\isacharparenright}\isanewline
\ \ \isacommand{have}\isamarkupfalse%
\ C{\isacharunderscore}card{\isacharunderscore}im{\isacharcolon}\ {\isachardoublequoteopen}card\ C\ {\isacharequal}\ {\isacharparenleft}vectorspace{\isachardot}dim\ K\ {\isacharparenleft}W{\isachardot}vs\ imT{\isacharparenright}{\isacharparenright}{\isachardoublequoteclose}\isanewline
\ \ \isacommand{proof}\isamarkupfalse%
\ {\isacharminus}\ \isanewline
\ \ \ \ \isacommand{from}\isamarkupfalse%
\ C{\isacharprime}fin\ C{\isacharprime}{\isacharunderscore}card\ C{\isacharprime}{\isacharunderscore}basis\ \isacommand{have}\isamarkupfalse%
\ {\isachardoublequoteopen}vectorspace{\isachardot}dim\ K\ {\isacharparenleft}W{\isachardot}vs\ imT{\isacharparenright}\ {\isacharequal}\ card\ {\isacharquery}C{\isacharprime}{\isachardoublequoteclose}\ \isanewline
\ \ \ \ \ \ \isacommand{apply}\isamarkupfalse%
\ {\isacharparenleft}intro\ im{\isachardot}dim{\isacharunderscore}basis{\isacharparenright}\ \isanewline
\ \ \ \ \ \ \ \isacommand{by}\isamarkupfalse%
\ auto\isanewline
\ \ \ \ \isacommand{from}\isamarkupfalse%
\ C{\isacharprime}{\isacharunderscore}card\ this\ \isacommand{show}\isamarkupfalse%
\ {\isacharquery}thesis\ \isacommand{by}\isamarkupfalse%
\ auto\isanewline
\ \ \isacommand{qed}\isamarkupfalse%
\isanewline
\ \ \isacommand{from}\isamarkupfalse%
\ finA\ Abasis\ \isacommand{have}\isamarkupfalse%
\ A{\isacharunderscore}card{\isacharunderscore}ker{\isacharcolon}\ {\isachardoublequoteopen}ker{\isachardot}dim\ {\isacharequal}\ card\ A{\isachardoublequoteclose}\ \isacommand{by}\isamarkupfalse%
\ {\isacharparenleft}rule\ ker{\isachardot}dim{\isacharunderscore}basis{\isacharparenright}\isanewline
\ \ \isacommand{from}\isamarkupfalse%
\ C{\isacharunderscore}card{\isacharunderscore}im\ A{\isacharunderscore}card{\isacharunderscore}ker\ cardEq\ \isacommand{show}\isamarkupfalse%
\ {\isacharquery}thesis\ \isacommand{by}\isamarkupfalse%
\ auto\isanewline
\isacommand{qed}\isamarkupfalse%
%
\endisatagproof
{\isafoldproof}%
%
\isadelimproof
\isanewline
%
\endisadelimproof
\isanewline
\isanewline
%
\isadelimtheory
\isanewline
%
\endisadelimtheory
%
\isatagtheory
\isacommand{end}\isamarkupfalse%
%
\endisatagtheory
{\isafoldtheory}%
%
\isadelimtheory
%
\endisadelimtheory
\end{isabellebody}%
%%% Local Variables:
%%% mode: latex
%%% TeX-master: "root"
%%% End:


%%% Local Variables:
%%% mode: latex
%%% TeX-master: "root"
%%% End:


% optional bibliography
\bibliographystyle{alpha}
\bibliography{root}

\end{document}

%%% Local Variables:
%%% mode: latex
%%% TeX-master: t
%%% End:
